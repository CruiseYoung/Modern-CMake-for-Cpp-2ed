本章我们将构建的软件并不打算非常复杂——我们将创建一个简单的计算器,它可以实现两个数字的相加(图15.1)。这将是一个控制台应用程序,具有文本用户界面,利用第三方库和独立的计算库,这些库可以用于其他项目。尽管这个项目可能没有重要的实际应用,但其简单性非常适合演示本书讨论的各种技术应用。

\myGraphic{0.6}{content/chapter15/images/1.png}{图15.1:我们的项目在支持鼠标的终端中执行的文本用户界面}

通常,项目要么生成面向用户的可执行文件,要么为开发者生成库。项目同时产生这两者的情况较少,尽管这种情况确实存在。例如,一些应用程序附带了独立的SDK或库以帮助开发插件。另一个例子是附带了使用示例的库。我们的项目属于后者,展示了库的功能。

我们将通过回顾章节列表,回忆每个章节的内容,并选择我们将用来构建应用程序的技术和工具来开始规划:

\begin{itemize}
\item
第1章,CMake入门:

本章提供了关于CMake的基本细节,包括安装和用于构建项目的命令行使用。它还涵盖了关于项目文件的基本信息,如它们的作用、典型的命名约定和特殊性。

\item
第2章,CMake语言:

我们介绍了编写正确的CMake列表文件和脚本所需的工具,涵盖了代码基础,如注释、命令调用和参数。我们解释了变量、列表和控制结构,引入了几个有用的命令。这个基础在我们的项目中至关重要。

\item
第3章,在流行的IDE中使用CMake:

我们讨论了三个IDE——CLion、VS Code和Visual Studio IDE,突出了它们的优点。在最终项目中,选择IDE(或不选择)由你决定。一旦决定,你可以在这个项目中使用Dev容器开始,只需几个步骤就可以构建Docker镜像(或者直接从Docker Hub获取)。在容器中运行镜像可以确保开发环境与生产环境相匹配。

\item
第4章,设置你的第一个CMake项目:

配置项目至关重要,因为它决定了将生效的CMake策略、命名、版本控制和编程语言。我们将使用本章来影响构建过程的基本行为。

我们还将遵循建立的项目分区和结构来确定目录和文件的布局,并利用系统发现变量以适应不同的构建环境。工具链配置是另一个关键方面,它允许我们强制使用特定的C++版本和编译器支持的标准。按照章节的建议,我们将禁用源内构建以保持工作区清洁。

\item
第5章,使用目标:

在这里,我们了解到每个现代CMake项目都广泛使用目标。我们当然也会应用目标来定义一些库和可执行文件(用于测试和生产),这将使项目保持组织并确保我们遵循DRY(不要重复自己)的原则。对目标属性和传递使用要求(传播属性)的了解将使我们能够使配置接近目标定义。

\item
第6章,使用生成器表达式:

生成器表达式在我们的项目中大量使用。我们将力求使这些表达式尽可能简单。项目将包含自定义命令以生成Valgrind和覆盖率报告的文件。此外,我们还将使用目标钩子,特别是PRE\_BUILD,来清理覆盖率检测过程产生的.gcda文件。

\item
第7章,使用CMake编译C++源代码:

没有C++项目的编译是不可能的。基础知识相当简单,但CMake允许我们以许多方式调整这个过程:扩展目标源代码、配置优化器并提供调试信息。对于这个项目,默认的编译标志就可以了,但我们会稍微玩弄一下预处理器:

\begin{itemize}
\item
我们将构建元数据(项目版本、构建时间和Git提交SHA)存储在编译后的可执行文件中并向用户展示。

\item
我们将启用预编译头文件。在如此小的项目中,这并不是真正必要的,但它将帮助我们练习这个概念。
\end{itemize}

Unity构建将不需要——这个项目不会大到使添加它们变得值得。

\item
第8章,链接可执行文件和库:

我们将获得默认情况下对任何项目都有用的链接的一般信息。此外,由于这个项目包含一个库,我们将明确引用以下特定构建指令:

\begin{itemize}
\item
用于测试和开发的静态库

\item
用于发布的共享库
\end{itemize}

本章还概述了如何隔离main()函数以用于测试目的,我们将采用这种做法。

\item
第9章,在CMake中管理依赖关系:

为了增强项目的吸引力,我们将引入一个外部依赖:一个基于文本的用户界面库。第9章探讨了管理依赖关系的各种方法。选择将很简单:FetchContent实用模块通常推荐且最方便。

\item
第10章,使用C++20模块:

尽管我们已经探讨了使用C++20模块以及支持此功能的环境要求(CMake 3.28,最新编译器),但其广泛支持仍然不足。为了确保项目的可访问性,我们暂时不会引入模块。

\item
第11章:测试框架

实施适当的自动化测试对于确保我们的解决方案质量随时间保持一致至关重要。我们将集成CTest并组织项目以方便测试,并应用之前提到的main()函数分离方法。

本章将讨论两种测试框架:Catch2和GTest与GMock;我们将使用后者。为了获取覆盖率的详细信息,我们将使用LCOV生成HTML报告。

\item
第12章:程序分析工具

对于静态分析,我们可以从一系列工具中选择:Clang-Tidy、Cpplint、Cppcheck、include-what-you-use(IWYU)和link-what-you-use(LWYU)。我们将选择Cppcheck,因为Clang-Tidy与使用GCC构建的预编译头文件兼容性较差。

动态分析将使用Valgrind的Memcheck工具,并配合Memcheck-cover包装器来生成HTML报告。此外,在构建过程中,我们的源代码将自动通过ClangFormat进行格式化。

\item
第13章:文档生成

提供文档对于我们项目中的库来说是必不可少的。CMake支持使用Doxygen自动化生成文档。我们将采用这种方法,并在设计中加入doxygen-awesome-css主题以更新样式。

\item
第14章:安装与打包

最后,我们将配置解决方案的安装和打包,并准备文件形成包,包括目标定义。我们将安装这些内容及构建目标产生的工件到合适的目录中,通过包含GNUInstallDirs模块实现。我们还将配置一些组件以模块化解决方案,并为CPack做好准备。
\end{itemize}

专业的项目通常会附带一些文本文件:README、LICENSE、INSTALL等。我们将在章节末尾简要介绍这些文件。

为了简化流程,我们不会实现自定义逻辑来检查所有必需的工具和依赖项是否可用。我们将依赖于CMake来显示其诊断信息并告诉用户缺少什么。如果你的项目获得了重要的关注,你可能需要考虑添加这些机制以改善用户体验。

有了清晰的计划后,让我们讨论如何实际地构建项目结构,包括逻辑目标和目录结构。


































