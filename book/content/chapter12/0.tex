编写高质量的代码并非易事,即使对于经验丰富的开发者来说也是如此。通过在解决方案中包含测试,我们降低了在主代码中犯基本错误的可能性。但这还不足以避免更复杂的问题。每一款软件都包含如此多的细节,跟踪它们全部变成了一份全职工作。各种约定和特定的设计实践由负责维护产品的团队建立。

有些问题与一致的编码风格相关:我们的代码应该使用80列还是120列?我们应该允许使用std::bind还是坚持使用lambda函数?使用C风格数组是否可以接受?应该将小函数写在一行中吗?我们应该总是使用auto,还是只在它提高可读性时使用?理想情况下,我们应该避免使用公认的通常不正确的语句:无限循环、使用标准库保留的标识符、无意中的数据丢失、不必要的if语句,以及任何不是“最佳实践”的东西(更多信息请参见“进一步阅读”部分)。

另一个需要考虑的方面是代码现代化。随着C++的发展,它引入了新特性。跟踪我们可以更新到最新标准的所有位置可能具有挑战性。此外,手动执行此操作既耗时又增加了引入错误的风险,特别是在大型代码库中。最后,我们应该检查事物在运行时的操作情况:运行程序并检查其内存。内存使用后是否正确释放?我们是否正在访问正确初始化的数据?还是代码试图访问不存在的指针?

手动管理所有这些挑战和问题既耗时又容易出错。幸运的是,我们可以使用自动化工具来检查和执行规则,纠正错误,并使我们的代码保持最新。是时候探索程序分析工具了。我们的代码将在每次构建时受到严格审查,以确保其符合行业标准。

在本章中,将包含以下主要内容:

\begin{itemize}
\item
执行格式化

\item
使用静态检查器

\item
使用Valgrind进行动态分析
\end{itemize}