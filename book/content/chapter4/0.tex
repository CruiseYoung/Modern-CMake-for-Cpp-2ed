我们现在已经收集了足够的信息,可以开始讨论CMake的核心功能:构建项目。在CMake中,一个项目包含了所有源文件和必要的配置,以管理将我们的解决方案变为现实的过程。配置过程从执行所有检查开始:验证目标平台是否受支持,确保所有必需的依赖项和工具的存在,并确认提供的编译器与所需特性兼容。

完成初步检查后,CMake继续生成一个定制的构建系统,该构建系统针对所选的构建工具。然后,执行构建系统,这意味着编译源文件,并将它们与各自的依赖项链接在一起,以创建输出工件。

生成的工件可以通过不同的方式分发给消费者。它们可以直接与用户共享作为二进制包,允许用户使用包管理器将它们安装到自己的系统中。另一种方式是将它们作为单一的可执行安装程序分发。此外,最终用户还可以通过访问开源仓库中的项目自行创建工件。在这种情况下,用户可以使用CMake在自己的机器上编译项目,并随后安装它们。

充分利用CMake项目可以显著提高开发体验和生成的代码的整体质量。通过利用CMake的力量,许多日常任务可以自动化,例如在构建后执行测试和运行代码覆盖检查器、格式化器、验证器、校验器和其他工具。这种自动化不仅节省了时间,还确保了一致性,并在整个开发过程中推广了代码质量。

要充分发挥CMake项目的潜力,我们首先需要做出一些关键决策:如何正确配置整个项目,以及如何分割项目和设置源树,以便所有文件都能整齐地组织在正确的目录中。从一开始就建立一个连贯的结构和组织,CMake项目就可以有效地管理和扩展。

接下来,我们将查看项目的构建环境。我们将了解我们正在使用的体系结构、可用的工具、它们支持的功能以及我们正在使用的语言标准。为了确保一切同步,我们将编译一个测试C++文件,并查看我们选择的编译器是否符合我们为项目设定的标准要求。这一切都是为了确保我们的项目、我们使用的工具以及我们选择的标准之间的无缝配合。

在本章中,我们将涵盖以下主要主题:

\begin{itemize}
\item
理解基本指令和命令

\item
分割你的项目

\item
考虑项目结构

\item
设置环境范围

\item
配置工具链

\item
禁用源内构建
\end{itemize}