

众所周知,随着项目的增长,在项目列表文件和源代码中找到所需内容变得越来越困难。因此,从项目开始就保持项目整洁非常重要。

设想一个场景,你需要交付一些重要且时间紧迫的更改,但这些更改不适合项目中的任一目录。现在,你需要额外推送一个清理提交来重新组织文件层次结构,以便你的更改可以整齐地融入。或者更糟,你决定随意放置它们,并添加一个TODO,以后再处理这个问题。

在一年时间里,这些问题累积起来,技术债务增加,维护代码的成本也随之上升。当实时系统出现严重故障需要快速修复,或者不熟悉代码库的人需要引入偶尔的更改时,这种情况变得极其麻烦。

因此,我们需要一个好的项目结构。但这意味着什么呢?我们可以从软件开发的其他领域,如系统设计,借鉴一些规则。项目应该具有以下特点:

\begin{itemize}
\item
易于导航和扩展

\item
边界清晰(项目特定文件应包含在项目目录中)

\item
单个目标遵循层次树结构
\end{itemize}

没有一个确定的解决方案,但在网上可用的各种项目结构模板中,我建议使用以下这个简单且可扩展的模板:

\myGraphic{0.7}{content/chapter4/images/1.png}{图4.1:项目结构的一个示例}

此项目为以下组件定义了目录:

\begin{itemize}
\item
cmake:共享宏和函数,find\_modules 和一次性脚本

\item
src: 二进制文件和库的源文件和头文件

\item
test: 自动化测试的源代码
\end{itemize}

在此结构中,CMakeLists.txt文件应存在于以下目录中:顶级项目目录、test和src及其所有子目录。主列表文件不应自行声明任何构建步骤,而应配置项目的一般方面,并通过add\_subdirectory()命令将构建责任委托给嵌套的列表文件。如有需要,这些列表文件可以进一步将工作委托给更深的层次。

\begin{myNotic}{Note}
一些开发者建议将可执行文件与库分开,并创建两个顶级目录而不是一个:src和lib。CMake对这两种工件的处理方式相同,在这个层次上的分离其实并不重要。如果你更喜欢这种模型,可以自由遵循。
\end{myNotic}

对于较大的项目,src目录中有多个子目录会非常方便。但如果你只构建一个可执行文件或库,可以省略它们,直接在src中存储源文件。无论如何,记得在那里添加一个CMakeLists.txt文件,并执行任何嵌套的列表文件。对于一个简单目标的文件树可能如下所示:

\myGraphic{0.7}{content/chapter4/images/2.png}{图4.2:可执行文件的目录结构}

在图4.1中,我们看到src目录根目录下有一个CMakeLists.txt文件——它将配置关键的项目设置,并包含来自嵌套目录的所有列表文件。app1目录(如图4.2所示)包含另一个CMakeLists.txt文件以及.cpp实现文件:class\_a.cpp和class\_b.cpp。还有一个main.cpp文件,其中包含可执行文件的入口点。

CMakeLists.txt文件应定义一个目标,使用这些源文件构建一个可执行文件——我们将在下一章学习如何做到这一点。

我们的头文件放置在include目录中,可以用来为其他C++翻译单元声明符号。接下来,我们有一个仅对此可执行文件特定的库的lib3目录(项目中其他地方使用的库或对外部提供的库应位于src目录中)。这种结构提供了极大的灵活性,并允许轻松扩展项目。随着我们继续添加更多类,我们可以方便地将它们分组到库中以提高编译速度。让我们看看一个库的样子:

\myGraphic{0.7}{content/chapter4/images/3.png}{图4.3:库的目录结构}

库应遵循与可执行文件相同的结构,但有细微的区别:在include目录中添加了一个可选的lib1目录。当库打算超出项目范围供外部使用时,会包含此目录。它包含其他项目在编译期间将使用的公共头文件。我们将在第7章“使用CMake编译C++源代码”中构建自己的库时再次讨论这个主题。

我们已经讨论了文件如何在目录结构中布局。现在,是时候看看单个CMakeLists.txt文件如何组合成一个项目,以及它们在更大场景中的角色。

\myGraphic{0.7}{content/chapter4/images/4.png}{图4.4:CMake如何将列表文件合并到单个项目中}

在前面的图中,每个框代表一个位于每个目录中的CMakeLists.txt列表文件,而其中的斜体标签代表每个文件执行的操作(从上到下)。让我们再次从CMake的角度分析这个项目(有关详细信息,请查看ch04/05-structure目录中的示例):

\begin{enumerate}
\item
执行从项目的根开始——即源树顶部的CMakeLists.txt列表文件。该文件将设置所需的最低CMake版本和相应的策略,设置项目名称、支持的语言和全局变量,并包含cmake目录中的文件,以便它们的内容全局可用。

\item
下一步是调用add\_subdirectory(src bin)命令进入src目录的范围(我们希望将编译后的工件放在<binary\_tree>/bin而不是/bin中)。

\item
CMake读取src/CMakeLists.txt文件,并发现其唯一目的是添加四个嵌套子目录:app1、app2、lib1和lib2。

\item
CMake进入app1的变量范围,并了解到另一个嵌套库lib3,它有自己的CMakeLists.txt文件;然后进入lib3的范围。你可能已经注意到,这是对目录结构的深度优先遍历。

\item
lib3库添加了一个同名静态库目标。CMake返回到app1的父范围。

\item
app1子目录添加了一个依赖于lib3的可执行文件。CMake返回到src的父范围。

\item
CMake将继续进入剩余的嵌套范围并执行它们的列表文件,直到所有add\_subdirectory()调用完成。

\item
CMake返回到顶级范围并执行剩余的命令add\_subdirectory(test)。每次,CMake进入新的范围并执行相应列表文件中的命令。

\item
收集并检查所有目标的正确性。CMake现在有了生成构建系统所需的所有必要信息。
\end{enumerate}

重要的是要注意,前面的步骤按照我们在列表文件中编写命令的确切顺序发生。在某些情况下,这个顺序是重要的,而在其他情况下,可能没有那么关键。我们将在下一章,第5章“处理目标”中深入探讨这个话题,以理解其含义。

那么,何时是创建包含项目所有元素的目录的正确时机呢?我们应该一开始就做——为未来创建所需的一切并保持目录为空——还是等到我们实际上有了需要放入自己类别的文件再说?这是一个选择——我们可以遵循极限编程(XP)的规则YAGNI(你不会需要它),或者我们可以尝试使我们的项目未来无忧,并为新来的开发者打下良好的基础。

尝试在这两种方法之间找到良好的平衡——如果你怀疑你的项目可能有一天需要extern目录,那么就添加它(你的版本控制系统可能需要一个空的.keep文件来将目录签入仓库)。

另一种有效的做法是通过创建一个README文件来指导他人放置他们的外部依赖,该文件概述了推荐的结构。这对于将来将要处理项目的经验不足的程序员特别有益。你可能自己已经观察到了:开发者不愿意创建目录,尤其是在项目的根目录下。如果我们提供了一个好的项目结构,其他人就会倾向于遵循它。

有些项目几乎可以在任何环境中构建,而有些项目对它们的要求非常特殊。顶级列表文件是确定适当行动方案的最佳位置。让我们看看如何做到这一点。




