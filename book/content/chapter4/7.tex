
第1章中,我们讨论了源内构建,以及如何建议总是指定构建路径在源码之外。这不仅允许更清晰的构建树和更简单的.gitignore文件,而且还能降低意外覆盖或删除源文件的风险。

若要提前停止构建,可以使用以下检查:

\filename{ch04/09-in-source/CMakeLists.txt}

\begin{cmake}
cmake_minimum_required(VERSION 3.26.0)
project(NoInSource CXX)
if(PROJECT_SOURCE_DIR STREQUAL PROJECT_BINARY_DIR)
    message(FATAL_ERROR "In-source builds are not allowed")
endif()
message("Build successful!")
\end{cmake}

如果想要了解更多关于STR前缀和变量引用的信息,请复习第2章。

需要注意的是,无论前面的代码中做了什么,CMake似乎仍然会创建一个CMakeFiles/目录和一个CMakeCache.txt文件。

\begin{myNotic}{Note}
可能会在线上找到建议使用未记录的变量,来确保用户无论如何都不能在源目录中写入。不建议依赖未记录的变量来限制在源目录中写入,它们可能不会在所有版本中都有效,并且可能会在没有警告的情况下被移除或修改。
\end{myNotic}

如果担心用户将这些文件留在源目录中,将它们添加到.gitignore(或等效的文件中),并更改消息,请求手动清理。
































