第1章中,已经了解了一个简单的项目定义。这是一个带有CMakeLists.txt文件的目录,其中包含几个配置语言处理器的命令:

\filename{chapter01/01-hello/CMakeLists.txt}

\begin{cmake}
cmake_minimum_required(VERSION 3.26)
project(Hello)
add_executable(Hello hello.cpp)
\end{cmake}

在同一章节的“项目文件”部分,我们了解了一些基本命令。

现在,让我们在这里深入介绍它们。

\mySubsubsection{4.2.1.}{指定最低CMake版本}

项目文件和脚本的顶部使用cmake\_minimum\_required()命令,这个命令不仅验证系统是否具有正确的CMake版本,还会隐式触发另一个命令cmake\_policy(VERSION),后者指定项目要使用的策略。这些策略定义了命令在CMake中的行为方式,它们随着CMake的发展,以及对支持的语言和CMake本身的改进而引入。

为了保持语言的清晰和简单,CMake团队在出现向后不兼容的更改时引入了策略。每个策略都启用了与该更改相关的新行为。这些策略确保项目可以适应CMake不断发展的特性和功能,同时保持与旧代码库的兼容性。

通过调用cmake\_minimum\_required(),我们告诉CMake需要应用在参数中提供的版本配置的默认策略。当CMake升级时,不必担心它会破坏项目,因为新版本的新策略不会启用。

策略可能会影响CMake的每一个方面,包括project()等其他重要命令。因此,在CMakeLists.txt文件开始时设置正在使用的版本非常重要;否则,将得到警告和错误。每个CMake版本都引入了许多策略。然而,除非在将旧项目升级到最新CMake版本时遇到挑战,否则没有必要深入细节。这种情况下,建议参考官方文档中关于策略的全面信息和指导:\url{https://cmake.org/cmake/help/latest/manual/cmake-policies.7.html}。

\mySubsubsection{4.2.2.}{定义语言和元数据}

建议在cmake\_minimum\_required()之后立即放置project()命令,这样做将确保在配置项目时使用正确的策略。可以使用以下两种形式之一:

\begin{shell}
project(<PROJECT-NAME> [<language-name>...])
\end{shell}

或者:

\begin{shell}
project(<PROJECT-NAME>
        [VERSION <major>[.<minor>[.<patch>[.<tweak>]]]]
        [DESCRIPTION <project-description-string>]
        [HOMEPAGE_URL <url-string>]
        [LANGUAGES <language-name>...])
\end{shell}

需要指定<PROJECT-NAME>,但其他参数可选。调用此命令将隐式设置以下变量:

\begin{shell}
PROJECT_NAME
CMAKE_PROJECT_NAME (only in the top-level CMakeLists.txt)
PROJECT_IS_TOP_LEVEL, <PROJECT-NAME>_IS_TOP_LEVEL
PROJECT_SOURCE_DIR, <PROJECT-NAME>_SOURCE_DIR
PROJECT_BINARY_DIR, <PROJECT-NAME>_BINARY_DIR
\end{shell}

支持哪些语言?相当多。而且可以在同一时间使用多种语言!以下是可以用来配置项目的语言关键字列表:

\begin{itemize}
\item
ASM, ASM\_NASM, ASM\_MASM, ASMMARMASM, ASM-ATT: 汇编语言

\item
C: C

\item
CXX: C++

\item
CUDA: Nvidia的统一计算设备架构

\item
OBJC: Objective-C

\item
OBJCXX: Objective-C++

\item
Fortran: Fortran

\item
HIP: 用于Nvidia和AMD平台的异构(计算)接口便携性

\item
ISPC: 隐式SPMD程序编译器的语言

\item
CSharp: C\#

\item
Java: Java (需要额外步骤,请参阅官方文档)
\end{itemize}

CMake默认启用C和C++,因此可能只想为C++项目明确指定CXX。为什么?因为project()命令将检测和测试您选择的语言的可用编译器,所以声明所需的编译器,将在配置阶段跳过对未使用语言的检查,从而节省时间。

指定VERSION关键字将自动设置可以用来配置包的变量,或者在编译期间在头文件中暴露的变量:

\begin{shell}
PROJECT_VERSION, <PROJECT-NAME>_VERSION
CMAKE_PROJECT_VERSION (only in the top-level CMakeLists.txt)
PROJECT_VERSION_MAJOR, <PROJECT-NAME>_VERSION_MAJOR
PROJECT_VERSION_MINOR, <PROJECT-NAME>_VERSION_MINOR
PROJECT_VERSION_PATCH, <PROJECT-NAME>_VERSION_PATCH
PROJECT_VERSION_TWEAK, <PROJECT-NAME>_VERSION_TWEAK
\end{shell}

还可以设置DESCRIPTION和HOMEPAGE\_URL,这将为了类似目的设置以下变量:

\begin{shell}
PROJECT_DESCRIPTION, <PROJECT-NAME>_DESCRIPTION
PROJECT_HOMEPAGE_URL, <PROJECT-NAME>_HOMEPAGE_URL
\end{shell}

cmake\_minimum\_required()和project()命令会创建一个基本的项目列表文件,并初始化一个空项目。虽然对于小型、单文件项目来说,结构可能不是一个大问题,但随着代码库的扩展,它变得至关重要。如何为此做准备呢?


