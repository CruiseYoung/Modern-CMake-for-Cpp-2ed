你可能认为一旦我们成功地将源代码编译成二进制文件,作为构建工程师的角色就完成了。然而,事实并非完全如此。虽然二进制文件确实包含了一个CPU执行所需的所有代码,但这些代码可以以复杂的方式分布在多个文件中。我们不希望CPU在不同的文件中搜索单个代码片段。相反,我们的目标是将这些单独的单元合并成一个文件。为了实现这一点,我们使用一个称为链接的过程。

快速浏览显示,CMake的链接命令并不多,其中target\_link\_libraries()是主要的。那么,为什么还要为单个命令写一整个章节呢?不幸的是,在计算机科学中,几乎没有什么是简单的,链接也不例外:为了得到正确的结果,我们需要了解整个故事——我们需要知道链接器是如何工作的,并且要掌握基础知识。我们将讨论目标文件的内部结构,如何工作的重定位和引用解析机制,以及它们的用途。我们将讨论最终可执行文件与其组件的不同之处,以及系统在将程序加载到内存时如何构建进程映像。

然后,我们将向您介绍各种类型的库:静态库、共享库和共享模块。尽管它们都被称为“库”,但它们非常不同。创建一个链接良好的可执行文件依赖于正确的配置和处理特定细节,例如位置无关代码(PIC)。

我们将学习链接的另一个难题——唯一定义规则(ODR)。准确地定义符号的数量是至关重要的。管理重复的符号可能特别具有挑战性,尤其是对于共享库。此外,我们将探讨为什么链接器有时无法定位外部符号,即使可执行文件已正确链接到相关库。

最后,我们将发现如何高效地使用链接器,为我们的解决方案在特定框架内进行测试做好准备。

在本章中,将包含以下内容:

\begin{itemize}
\item
掌握链接的基本知识

\item
构建不同类型的库

\item
解决ODR问题

\item
链接和未解析符号的顺序

\item
为测试分离main()
\end{itemize}











































