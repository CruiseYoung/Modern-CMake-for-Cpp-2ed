
Phil Karlton,Netscape的主要怪人和技术愿景家,说下面的话时是正确的:

\begin{myTip}{Tip}
 “计算机科学中有两件难事:缓存失效和给予命名。”
\end{myTip}

名称之所以难以处理,有几个原因。它们必须精确而简单,简洁而有表现力。这不仅赋予了它们意义,还使程序员能够掌握底层原始实现背后的概念。C++和许多其他语言增加了另一个规定:大多数名称必须唯一。

这一要求以ODR的形式体现:在单个翻译单元(单个.cpp文件)的范围内,即使某个名称(无论是变量、函数、类类型、枚举、概念还是模板)多次声明,也必须精确地定义它一次。"声明"是引入符号,而"定义"则提供其所有细节,如变量的值或函数的主体。

链接期间,此规则扩展到整个程序,包含在代码中实际使用的所有非内联函数和变量。考虑以下由三个源文件组成的示例:

\filename{ch08/02-odr-fail/shared.h}

\begin{cpp}
int i;
\end{cpp}

\filename{ch08/02-odr-fail/one.cpp}

\begin{cpp}
#include <iostream>
#include "shared.h"

int main() {
    std::cout << i << std::endl;
}
\end{cpp}

\filename{ch08/02-odr-fail/two.cpp}

\begin{cpp}
#include "shared.h
\end{cpp}

还包括一个列表文件:

\filename{ch08/02-odr-fail/CMakeLists.txt}

\begin{cmake}
cmake_minimum_required(VERSION 3.26)
project(ODR CXX)
set(CMAKE_CXX_STANDARD 20)
add_executable(odr one.cpp two.cpp)
\end{cmake}

这个例子非常简单——创建了一个shared.h头文件,定义了变量i,其在两个单独的翻译单元中使用:

\begin{itemize}
\item
one.cpp 只是将i打印到屏幕上

\item
two.cpp 只包含头文件
\end{itemize}

但当尝试构建示例时,链接器产生了以下错误:

\begin{shell}
/usr/bin/ld:
CMakeFiles/odr.dir/two.cpp.o:(.bss+0x0): multiple definition of 'i';
CMakeFiles/odr.dir/one.cpp.o:(.bss+0x0): first defined here
collect2: error: ld returned 1 exit status
\end{shell}

符号不能定义多次,但有一个重要的例外。类型、模板和extern内联函数可以在多个翻译单元中有重复的定义,但前提是这些定义必须完全相同(它们有完全相同的标记序列)。

为了证明这一点,用一个类型的定义替换变量的定义:

\filename{ch08/03-odr-success/shared.h}

\begin{cpp}
struct shared {
    static inline int i = 1;
};
\end{cpp}

然后,这样使用:

\filename{ch08/03-odr-success/one.cpp}

\begin{cpp}
#include <iostream>
#include "shared.h"
int main() {
    std::cout << shared::i << std::endl;
}
\end{cpp}

其他两个文件,two.cpp和CMakeLists.txt,保持与02-odr-fail示例相同。这样的更改会链接成功:

\begin{shell}
[ 33%] Building CXX object CMakeFiles/odr.dir/one.cpp.o
[ 66%] Building CXX object CMakeFiles/odr.dir/two.cpp.o
[100%] Linking CXX executable odr
[100%] Built target odr
\end{shell}

另外,可以将变量标记为翻译单元局部(不会导出到对象文件之外)。为此,将使用static关键字(这个关键字具有上下文相关性,所以不要将它与类中的static关键字混淆):

\filename{ch08/04-odr-success/shared.h}

\begin{cpp}
static int i;
\end{cpp}

如果尝试链接这个示例,会看到它是有效的,所以静态变量对于每个翻译单元是分开存储的。因此,对一个的修改不会影响另一个。

ODR规则对于静态库和对象文件完全相同,但在使用共享库构建代码时,情况就没那么明了——让我们来看一下。

\mySubsubsection{8.4.1.}{解决动态链接中的重复符号问题}

链接器将在此处允许重复的符号。以下示例中,我们将创建两个共享库A和B,每个库都有一个duplicated()函数和两个唯一的a()和b()函数:

\filename{ch08/05-dynamic/a.cpp}

\begin{cpp}
#include <iostream>
void a() {
    std::cout << "A" << std::endl;
}
void duplicated() {
    std::cout << "duplicated A" << std::endl;
}
\end{cpp}

第二个实现文件几乎与第一个完全相同:

\filename{ch08/05-dynamic/b.cpp}

\begin{cpp}
#include <iostream>
void b() {
    std::cout << "B" << std::endl;
}
void duplicated() {
    std::cout << "duplicated B" << std::endl;
}
\end{cpp}

现在,使用每个函数来看看会发生什么(简单起见,将本地声明为extern):

\filename{ch08/05-dynamic/main.cpp}

\begin{cpp}
extern void a();
extern void b();
extern void duplicated();
int main() {
    a();
    b();
    duplicated();
}
\end{cpp}

前面的代码将运行每个库中的唯一函数,然后调用在两个动态库中定义的具有相同签名的函数。你认为会发生什么?链接顺序会有关系吗?让我们针对以下两种情况进行测试:

\begin{itemize}
\item
main\_1 目标将首先与a库链接

\item
main\_2 目标将首先与b库链接
\end{itemize}

列表文件如下所示:

\filename{ch08/05-dynamic/CMakeLists.txt}

\begin{cmake}
cmake_minimum_required(VERSION 3.26)
project(Dynamic CXX)
add_library(a SHARED a.cpp)
add_library(b SHARED b.cpp)
add_executable(main_1 main.cpp)
target_link_libraries(main_1 a b)
add_executable(main_2 main.cpp)
target_link_libraries(main_2 b a)
\end{cmake}

构建并运行两个可执行文件后,将看到以下输出:

\begin{shell}
root@ce492a7cd64b:/root/examples/ch08/05-dynamic# b/main_1
A
B
duplicated A

root@ce492a7cd64b:/root/examples/ch08/05-dynamic# b/main_2
A
B
duplicated B
\end{shell}

啊哈!显然,链接库的顺序对链接器有关系。如果不警惕,这可能会导致混淆。与人们可能认为的不同,命名冲突在实践中并不少见。

如果定义了本地可见的符号,其将优先于从DLL中可用的符号。在main.cpp中定义duplicated()函数将覆盖两个目标的行为。

在从库中导出名称时,总是要非常小心,因为迟早会遇到命名冲突。

\mySubsubsection{8.4.2.}{使用命名空间——不要依赖链接器}

C++命名空间是为了避免此类奇怪的问题,并更有效地处理ODR而发明的。最佳实践是将你的库代码包装在以库命名的命名空间中。

这种策略有助于防止由于重复符号引起的复杂问题。项目中,我们可能会遇到一个共享库链接到另一个库的情况,形成一个长链。在复杂的配置中,这种情况很常见。然而,理解仅仅将一个库链接到另一个库,并不会引入任何类型的命名空间继承至关重要。这个链中的每个链接的符号都保留在编译时的原始命名空间中。

虽然链接器的复杂性很吸引人,偶尔也是必要的,但另一个紧迫的问题经常出现:正确定义的符号会神秘消失。让我们在下一节中深入探讨这个问题。



