
链接器的行为有时可能显得反复无常,无缘无故地抛出抱怨。这对于不熟悉这个工具复杂性的初学者程序员来说,常常是一个特别棘手的挑战。可以理解的是,他们通常会尽可能长时间地避免处理构建配置。但是,当他们需要做出更改时——也许是要整合他们开发的库——那么问题就会全面爆发。

考虑以下情况:一个相对直接的依赖链,其中主可执行文件依赖于一个“外部”库。而这个外部库又依赖于包含必要变量int b的“嵌套”库。突然之间,一个神秘的错误信息出现在程序员面前:

\begin{shell}
outer.cpp:(.text+0x1f): undefined reference to 'b'
\end{shell}

此类错误并不特别罕见。通常,它们表示链接器中忘记了一个库。然而,在这个场景中,库似乎已经正确地添加到了target\_link\_libraries()命令中:

\filename{ch08/06-unresolved/CMakeLists.txt}

\begin{cmake}
cmake_minimum_required(VERSION 3.26)
project(Order CXX)
add_library(outer outer.cpp)
add_library(nested nested.cpp)
add_executable(main main.cpp)
target_link_libraries(main nested outer)
\end{cmake}

那是什么问题!?很少有错误能像这样调试和理解时让人如此愤怒。我们看到的是链接顺序不正确。让我们深入源代码来找出原因:

\filename{ch08/06-unresolved/main.cpp}

\begin{cmake}
#include <iostream>
extern int a;
int main() {
    std::cout << a << std::endl;
}
\end{cmake}

代码看起来足够简单——我们将打印外部变量a,该变量可以在outer库中找到。我们提前使用extern关键字声明它。以下是该库的源代码:

\filename{ch08/06-unresolved/outer.cpp}

\begin{cmake}
extern int b;
int a = b;
\end{cmake}

这也相当简单——outer依赖于嵌套库提供外部变量b,该变量被赋值给变量a。让我们看看nested的源代码,以确认我们没有遗漏定义:

\filename{ch08/06-unresolved/nested.cpp}

\begin{cmake}
int b = 123;
\end{cmake}

确实,我们为b提供了定义,并且由于它没有被static关键字标记为局部变量,因此它从nested目标正确导出。如我们之前所见,此目标在CMakeLists.txt中与主可执行文件链接:

\begin{cmake}
target_link_libraries(main nested outer)
\end{cmake}

那么,对’b’的未定义引用错误从何而来?解析未定义的符号是这样工作的——链接器从左到右处理二进制文件。

当链接器遍历二进制文件时,它将执行以下操作:

\begin{enumerate}
\item
收集此二进制文件导出的所有未定义符号,并存储起来以备后用。

\item
尝试用此二进制文件中定义的符号解析未定义的符号(来自迄今为止处理的所有二进制文件)。

\item
对下一个二进制文件重复此过程。
\end{enumerate}

如果在整个操作完成后仍有任何符号未定义,链接将失败。这就是我们示例中的情况(CMake将可执行文件目标的对象文件放在库的前面):

\begin{enumerate}
\item
链接器处理了main.o,发现对a变量的未定义引用,并将其收集起来以备将来解析。

\item
链接器处理了libnested.a,没有发现未定义的引用,也没有需要解析的。

\item
链接器处理了libouter.a,发现对b变量的未定义引用,并解析了对a变量的引用。
\end{enumerate}

我们正确地解析了对a变量的引用,但未解析对b变量的引用。要纠正这一点,我们需要反转链接顺序,使nested在outer之后:

\begin{cmake}
arget_link_libraries(main outer nested)
\end{cmake}

有时,我们会遇到循环引用,其中翻译单元相互定义符号,没有一种有效的顺序可以满足所有引用。解决这个问题的唯一方法是对某些目标进行两次处理:

\begin{cmake}
target_link_libraries(main nested outer nested)
\end{cmake}

这是一种常见做法,但使用起来略显不雅。如果您有幸使用CMake 3.24或更高版本,您可以使用\$<LINK\_GROUP>生成器表达式与RESCAN功能,该功能添加了链接器特定的标志,如-{}-start-group或-{}-end-group,以确保评估所有符号:

\begin{cmake}
target_link_libraries(main "$<LINK_GROUP:RESCAN,nested,outer>")
\end{cmake}

请记住,这种机制引入了额外的处理步骤,应当只在必要时使用。需要循环引用的情况非常罕见(并且是合理的)。遇到这个问题通常表明设计不佳。它在Linux、BSD、SunOS以及使用GNU工具链的Windows上得到支持。

我们现在准备处理ODR问题。我们还可能遇到哪些其他问题?链接过程中神秘地缺失符号。让我们来看看这是关于什么的。

\mySubsubsection{8.5.1.}{处理未引用的符号}

当创建库时,尤其是静态库,它们基本上是由多个对象文件捆绑在一起的归档文件。我们提到过,一些归档工具可能会创建符号索引以加速链接过程。这些索引提供了每个符号与定义它们的对象文件之间的映射。当解析一个符号时,包含该符号的对象文件会被合并到最终的二进制文件中(一些链接器通过只包含文件的具体部分来进一步优化)。如果静态库中的对象文件没有引用任何符号,那么该对象文件可能会被完全忽略。因此,只有实际使用的静态库部分才会出现在最终的二进制文件中。

然而,在某些情况下,您可能需要一些未引用的符号:

\begin{itemize}
\item
静态初始化:如果您的库有需要在main()之前初始化的全局对象(即,它们的构造函数需要被执行),并且这些对象在其它地方没有被直接引用;链接器可能会将它们从最终二进制文件中排除。

\item
插件架构:如果您正在开发一个插件系统(带有模块库),其中代码需要在运行时被识别和加载,而不需要直接引用。

\item
静态库中的未使用代码:如果您正在开发一个包含实用函数或代码的静态库,这些代码并非总是被直接引用,但您仍然希望它们出现在最终二进制文件中。

\item
模板实例化:对于重度依赖模板的库;如果未明确提及,一些模板实例化可能会在链接过程中被忽略。

\item
链接问题:特别是对于复杂的构建系统或详尽的代码库,链接可能会产生不可预测的结果,其中某些符号或代码部分似乎缺失。
\end{itemize}

在这些情况下,强制在链接过程中包含所有对象文件可能是有益的。这通常通过一种称为全归档链接的模式来实现。

特定的编译器链接标志如下:

\begin{itemize}
\item
GCC使用 -{}-whole-archive

\item
Clang使用 -{}-force-load

\item
MSVC使用 /WHOLEARCHIVE
\end{itemize}

为了实现这一点,我们可以使用target\_link\_options()命令:

\begin{cmake}
target_link_options(tgt INTERFACE
    -Wl,--whole-archive $<TARGET_FILE:lib1> -Wl,--no-whole-archive
)
\end{cmake}

然而,这个命令是特定于链接器的,因此结合生成器表达式来检测不同的编译器并提供相应的标志是必要的。幸运的是,CMake 3.24引入了一个新的生成器表达式用于此目的:

\begin{cmake}
target_link_libraries(tgt INTERFACE
    "$<LINK_LIBRARY:WHOLE_ARCHIVE,lib1>"
)
\end{cmake}

使用这种方法可以确保tgt目标包含lib1库的所有对象文件。

尽管如此,还需要考虑一些潜在的缺点:

\begin{itemize}
\item
增加二进制文件大小:这个标志可能会大幅增加您的最终二进制文件大小,因为它包含了指定库的所有对象文件,无论它们是否被使用。

\item
符号冲突的可能性:引入所有符号可能会导致与其他符号冲突,从而引发链接错误。

\item
维护负担:过度依赖此类标志可能会掩盖代码设计或结构中的潜在问题。
\end{itemize}

了解了如何解决常见的链接挑战后,我们现在可以继续准备我们的项目进行测试了。




