
链接器的行为有时可能显得反复无常,无缘无故地抛出错误。这对于不熟悉这个工具复杂性的初级开发者来说,常常是一个特别棘手的挑战。他们通常会尽可能长时间地避免处理构建配置,但当需要做出更改时——也许是要整合他们开发的库——问题就会全面爆发。

考虑以下情况:一个相对直接的依赖链,其中主可执行文件依赖于一个“外部”库。而这个外部库又依赖于包含必要变量int b的“嵌套”库。突然之间,一个神秘的错误信息出现在程序员面前:

\begin{shell}
outer.cpp:(.text+0x1f): undefined reference to 'b'
\end{shell}

此类错误很常见。通常,表示链接器中忘记了一个库。然而,在这个场景中,库似乎已经正确地添加到了target\_link\_libraries()命令中:

\filename{ch08/06-unresolved/CMakeLists.txt}

\begin{cmake}
cmake_minimum_required(VERSION 3.26)
project(Order CXX)
add_library(outer outer.cpp)
add_library(nested nested.cpp)
add_executable(main main.cpp)
target_link_libraries(main nested outer)
\end{cmake}

问题是什么呢!?很少有错误能像这样调试和理解时让人如此愤怒。我们看到的是链接顺序不正确:

\filename{ch08/06-unresolved/main.cpp}

\begin{cmake}
#include <iostream>
extern int a;
int main() {
    std::cout << a << std::endl;
}
\end{cmake}

代码看起来很简单——将打印外部变量a,该变量可以在outer库中找到。提前使用extern关键字声明,以下是该库的源码:

\filename{ch08/06-unresolved/outer.cpp}

\begin{cmake}
extern int b;
int a = b;
\end{cmake}

这相当简单——outer依赖于嵌套库提供外部变量b,该变量赋值给变量a。来看看nested的源代码,以确认没有遗漏定义:

\filename{ch08/06-unresolved/nested.cpp}

\begin{cmake}
int b = 123;
\end{cmake}

确实,我们为b提供了定义,并且由于它没有使用static关键字标记为局部变量,因此会从nested目标正确导出。如我们之前所见,此目标在CMakeLists.txt中与主可执行文件链接:

\begin{cmake}
target_link_libraries(main nested outer)
\end{cmake}

那么,对’b’的未定义引用错误从何而来?解析未定义的符号是这样工作的——链接器从左到右处理二进制文件。

当链接器遍历二进制文件时,将执行以下操作:

\begin{enumerate}
\item
收集此二进制文件导出的所有未定义符号,并存储起来以备后用。

\item
尝试用此二进制文件中定义的符号解析未定义的符号(来自迄今为止处理的所有二进制文件)。

\item
对下一个二进制文件重复此过程。
\end{enumerate}

如果在整个操作完成后仍有符号未定义,链接将失败。这就是我们示例中的情况(CMake将可执行文件目标的对象文件放在库的前面):

\begin{enumerate}
\item
链接器处理了main.o,发现对a变量的未定义引用,并将其收集起来以备将来解析。

\item
链接器处理了libnested.a,没有发现未定义的引用,也没有需要解析的。

\item
链接器处理了libouter.a,发现对b变量的未定义引用,并解析了对a变量的引用。
\end{enumerate}

我们正确地解析了对a变量的引用,但未解析对b变量的引用。要纠正这一点,我们需要反转链接顺序,使nested在outer之后:

\begin{cmake}
arget_link_libraries(main outer nested)
\end{cmake}

有时,会遇到循环引用,其中翻译单元相互定义符号,没有一种有效的顺序可以满足所有引用。解决这个问题的唯一方法是对某些目标进行两次处理:

\begin{cmake}
target_link_libraries(main nested outer nested)
\end{cmake}

这是一种常见做法,但使用起来略显不雅。如果有幸使用CMake 3.24或更高版本,可以使用\$<LINK\_GROUP>生成器表达式与RESCAN功能,该功能添加了链接器特定的标志,如-{}-start-group或-{}-end-group,以确保能找到所有的符号:

\begin{cmake}
target_link_libraries(main "$<LINK_GROUP:RESCAN,nested,outer>")
\end{cmake}

这种机制引入了额外的处理步骤,应当只在必要时使用。需要循环引用的情况非常罕见(并且是合理的)。遇到这个问题通常表明设计不佳,其在Linux、BSD、SunOS,以及使用GNU工具链的Windows上得到支持。

现在准备处理ODR问题。我们还可能遇到哪些其他问题?链接过程中神秘地缺失符号。来看看这是关于什么的。

\mySubsubsection{8.5.1.}{处理未引用的符号}

当创建库时,尤其是静态库,基本上是由多个对象文件捆绑在一起的归档文件。我们提到过,一些归档工具可能会创建符号索引以加速链接过程。这些索引提供了每个符号与定义它们的对象文件之间的映射。当解析一个符号时,包含该符号的对象文件会合并到最终的二进制文件中(一些链接器通过只包含文件的具体部分来进一步优化)。如果静态库中没有从对象文件引用符号,则可能会完全忽略该对象文件。因此,只有实际使用的静态库部分,才会出现在最终的二进制文件中。

然而,在某些情况下一些未引用的符号也需要包含进二进制文件:

\begin{itemize}
\item
静态初始化:如果库有需要在main()之前初始化的全局对象(即构造函数需要执行),并且这些对象在其它地方没有直接引用;链接器可能会将其从最终二进制文件中排除。

\item
插件架构:如果正在开发一个插件系统(带有模块库),其中代码需要在运行时进行识别和加载,而不需要直接引用。

\item
静态库中的未使用代码:如果正在开发一个包含实用函数或代码的静态库,这些代码并非总是直接引用,但仍然希望它们出现在最终二进制文件中。

\item
模板实例化:对于重度依赖模板的库;如果未明确提及,一些模板实例化可能会在链接过程中忽略。

\item
链接问题:特别是对于复杂的构建系统或详尽的代码库,链接可能会产生不可预测的结果,其中某些符号或代码部分似乎缺失。
\end{itemize}

这时,强制在链接过程中包含所有对象文件很有用,可通过一种称为“全归档链接”的模式来实现。

特定的编译器链接标志如下:

\begin{itemize}
\item
GCC使用 -{}-whole-archive

\item
Clang使用 -{}-force-load

\item
MSVC使用 /WHOLEARCHIVE
\end{itemize}

为了实现这一点,可以使用target\_link\_options()命令:

\begin{cmake}
target_link_options(tgt INTERFACE
    -Wl,--whole-archive $<TARGET_FILE:lib1> -Wl,--no-whole-archive
)
\end{cmake}

然而,这个命令是特定于链接器的,因此结合生成器表达式,来检测不同的编译器,并提供必要的标志。幸运的是,CMake 3.24引入了一个新的生成器表达式用于此目的:

\begin{cmake}
target_link_libraries(tgt INTERFACE
    "$<LINK_LIBRARY:WHOLE_ARCHIVE,lib1>"
)
\end{cmake}

使用这种方法可以确保tgt目标包含lib1库的所有对象文件。

尽管如此,还需要考虑一些潜在的问题:

\begin{itemize}
\item
增加二进制文件大小:这个标志可能会大幅增加最终二进制文件大小,其包含了指定库的所有对象文件,无论是否使用。

\item
符号冲突的可能性:引入所有符号可能会导致与其他符号冲突,从而引发链接错误。

\item
维护负担:过度依赖此类标志会掩盖代码设计或结构中的潜在问题。
\end{itemize}

了解了如何解决常见的链接挑战后,我们现在可以继续准备项目的测试了。




