
在编译源代码后,通常希望避免针对同一平台的重新编译,甚至将编译输出与外部项目共享。可以将最初生成的单个对象文件分发出去,但这会带来挑战。分发多个文件并将它们逐个集成到构建系统中可能很麻烦,特别是在处理大量文件时。更有效的方法是将所有对象文件合并为一个单元以供共享。CMake大大简化了这个任务。我们可以使用简单的add\_library()命令(与target\_link\_libraries()命令配对)生成这些库。

按照约定,所有库都有一个公共前缀,即lib,并使用特定于系统的扩展名来表示它们是什么类型的库:

\begin{itemize}
\item
在类Unix系统中,静态库具有.a扩展名,在Windows上为.lib。

\item
在某些类Unix系统(如Linux)上,共享库(和模块)具有.so扩展名,而在其他系统(如macOS)上为.dylib。在Windows上,它们的扩展名为.dll。

\item
共享模块通常使用与共享库相同的扩展名,但不总是如此。在macOS上,它们可以使用.so,特别是当模块是从另一个Unix平台移植过来时。
\end{itemize}

构建库(静态、共享或共享模块)的过程通常称为“链接”,可以在ch08/01-libraries项目的构建输出中看到:

\begin{shell}
[ 33%] Linking CXX static library libmy_static.a
[ 66%] Linking CXX shared library libmy_shared.so
[100%] Linking CXX shared module libmy_module.so
[100%] Built target module_gui
\end{shell}

然而,前面提到的库并非都需要使用链接器来创建。对于某些库,该过程可能会跳过某些步骤,如重定位和引用解析。

让我们深入探讨每种库类型,了解它们各自的工作原理。

\mySubsubsection{8.3.1.}{静态库}

静态库本质上是存储在存档中的原始对象文件的集合。有时,它们会通过索引来加速链接过程。在类Unix系统中,这样的存档可以通过ar工具创建,并使用ranlib进行索引。

在构建过程中,只有静态库中必要的符号才会被导入最终的可执行文件中,优化其大小和内存使用。这种选择性集成确保了可执行文件是自包含的,消除了在运行时对外部文件的需求。

要创建静态库,我们可以简单地使用前面章节中已经看到的命令:

\begin{shell}
add_library(<name> [<source>...])
\end{shell}

这个简写代码默认会生成一个静态库。这可以通过将BUILD\_SHARED\_LIBS变量设置为ON来覆盖。如果我们想无论如何都要构建一个静态库,可以提供一个显式关键字:

\begin{shell}
add_library(<name> STATIC [<source>...])
\end{shell}

使用静态库可能并不总是理想的选择,特别是当我们希望在同一台机器上运行的多个应用程序之间共享编译后的代码时。

\mySubsubsection{8.3.2.}{共享库}

共享库与静态库有显著不同。它们是使用链接器构建的,链接器完成了链接的两个阶段。这产生了一个包含节头、节和节头表的完整文件,如图8.1所示。

共享库,通常称为共享对象,可以被多个不同的应用程序同时使用。当第一个程序使用共享库时,操作系统会将该库的一个实例加载到内存中。随后,操作系统为其他程序提供相同的地址,这要归功于复杂的虚拟内存机制。然而,对于每个使用该库的进程,库的.data和.bss段是分别实例化的。这确保了每个进程都可以调整其变量而不影响其他进程。

感谢这种方法,系统整体内存使用得到了优化。如果我们使用的是广泛认可的库,可能不需要将其包含在我们的程序中,因为它可能已经存在于目标机器上。但是,如果它没有预安装,用户需要在运行应用程序之前手动安装它。如果安装的库版本与预期版本不同,这可能导致潜在问题。这类问题被称为“依赖地狱”。更多细节可以在本章的进一步阅读部分找到。

我们可以通过显式使用SHARED关键字来构建共享库:

\begin{shell}
add_library(<name> SHARED [<source>...])
\end{shell}

由于共享库在程序初始化期间加载到操作系统的内存中,因此执行程序与磁盘上的实际库文件之间没有关联。相反,链接是间接完成的。在类Unix系统中,这是通过共享对象名(SONAME)实现的,可以将其理解为库的“逻辑名称”。

这允许在库版本控制上具有灵活性,并确保对库的向后兼容性更改不会立即破坏依赖的应用程序。

我们可以使用生成器表达式查询产生的SONAME文件的一些路径属性(确保将target替换为你的目标名称):

\begin{itemize}
\item
\$<TARGET\_SONAME\_FILE:target> 返回完整路径 (.so.3)。

\item
\$<TARGET\_SONAME\_FILE\_NAME:target> 只返回文件名。

\item
\$<TARGET\_SONAME\_FILE\_DIR:target> 只返回相应文件夹路径。
\end{itemize}

这些在本书后面将介绍的高级场景中非常有用,包括:

\begin{itemize}
\item
在打包和安装过程中正确使用生成的库。

\item
编写自定义的CMake规则进行依赖管理。

\item
在测试过程中利用SONAME。

\item
在构建后命令中复制或重命名生成的库。
\end{itemize}

您可能对其他特定于操作系统的工件有类似的需求;为此,CMake提供了两个生成器表达式家族,它们提供了与SONAME相同的后缀。对于Windows,我们有:

\begin{itemize}
\item
\$<TARGET\_LINKER\_FILE:target>返回与生成的动态链接库(DLL)关联的.lib导入库的完整路径。请注意,.lib扩展名与静态Windows库相同,但它们的应用并不相同。

\item
\$<TARGET\_RUNTIME\_DLLS:target> 返回目标在运行时依赖的DLL列表。

\item
\$<TARGET\_PDB\_FILE:target> 返回.pdb程序数据库文件的完整路径(用于调试目的)。
\end{itemize}

由于共享库在程序初始化期间加载到操作系统的内存中,因此它们适用于提前知道程序将使用哪些库的情况。那么在运行时确定这些的情况又如何呢?

\mySubsubsection{8.3.3.}{共享模块}

共享模块或模块库是共享库的一种变体,设计用于在运行时作为插件加载。与在程序启动时自动加载的标准共享库不同,共享模块仅在程序明确请求时加载。这可以通过以下系统调用来完成:

\begin{itemize}
\item
在Windows上使用LoadLibrary

\item
在Linux和macOS上使用dlopen(),然后是dlsym()
\end{itemize}

这种方法的主要原因是内存节约。许多软件应用程序具有在整个进程生命周期中未被使用的高级功能。每次都将这些功能加载到内存中将是低效的。

或者,我们可能希望提供一种途径,以便通过可以单独销售、交付和加载的专业功能来扩展主程序。

要构建共享模块,我们需要使用MODULE关键字:

\begin{shell}
add_library(<name> MODULE [<source>...])
\end{shell}

你不应该尝试将模块与可执行文件链接,因为模块被设计为与将使用它的可执行文件分开部署。

\mySubsubsection{8.3.4.}{位置无关代码(PIC)}

由于使用了虚拟内存,今天的程序本质上是某种程度上位置无关的。这项技术抽象了物理地址。当调用一个函数时,CPU使用内存管理单元(MMU)将虚拟地址(每个进程从0开始)转换为相应的物理地址(在分配时确定)。有趣的是,这些映射并不总是遵循特定的顺序。

编译库引入了不确定性:不清楚哪些进程可能会使用库,或者它在虚拟内存中的位置。我们也无法预测符号的地址或它们相对于库的机器代码的位置。为了处理这个问题,我们需要另一层间接寻址。

PIC被引入用来将符号(如对函数和全局变量的引用)映射到它们的运行时地址。PIC在二进制文件中引入了一个新的节:全局偏移表(GOT)。在链接期间,计算了GOT节相对于.text节(程序代码)的相对位置。所有符号引用将通过一个偏移量指向GOT中的占位符。

当程序被加载时,GOT节转换为一个内存段。随着时间的推移,这个段累积了符号的运行时地址。这种方法称为“懒加载”,确保加载器仅在需要时填充特定的GOT条目。

共享库和模块的所有源代码必须在使用PIC标志激活的情况下编译。通过将POSITION\_INDEPENDENT\_CODE目标属性设置为ON,我们将告诉CMake适当地添加编译器特定的标志,例如为GCC或Clang添加-fPIC。

这个属性对于共享库是自动启用的。然而,如果共享库依赖于另一个目标,例如静态或对象库,你也必须将这个属性应用于依赖目标:

\begin{shell}
set_target_properties(dependency
                      PROPERTIES POSITION_INDEPENDENT_CODE ON)
\end{shell}

忽略这一步将导致CMake中的冲突,因为它会检查这个属性的一致性。您可以在第5章《与目标合作》的“处理冲突的传播属性”部分找到对此的更深入探讨。

我们的下一个讨论点转向符号。特别是,下一节将探讨名称冲突的挑战,这可能导致模糊和定义不一致。











