Having your project documented with a clean, fresh design is important. After all, if we put all this work into writing high-quality documentation for our cutting-edge project, it is imperative that the user perceives it as such. Although Doxygen is feature-rich, it isn’t renowned for adhering to the latest visual trends. However, revamping its appearance doesn’t require substantial effort.

Luckily, a developer named jothepro created a theme called doxygen-awesome-css, which offers a modern, customizable design. This theme is presented in the following screenshot:

\myGraphic{0.8}{content/chapter13/images/4.png}{Figure 13.4: HTML documentation in doxygen-awesome-css theme}

The theme doesn’t require any additional dependencies and can be easily fetched from its GitHub page at \url{https://github.com/jothepro/doxygen-awesome-css}.

\begin{myNotic}{Note}
While some online sources recommend using a combination of applications, like transforming Doxygen’s output with Sphinx via Breathe and Exhale extensions, this method can be complex and dependency-heavy (requiring Python, for example). A simpler approach is usually more practical, particularly for teams where not all members are deeply familiar with CMake.
\end{myNotic}

We can efficiently implement this theme with an automated process. Let’s see how we can extend our Doxygen.cmake file to use it by adding a new macro:

\filename{ch13/02-doxygen-nice/cmake/Doxygen.cmake (fragment)}

\begin{cmake}
macro(UseDoxygenAwesomeCss)
    include(FetchContent)
    FetchContent_Declare(doxygen-awesome-css
        GIT_REPOSITORY
            https://github.com/jothepro/doxygen-awesome-css.git
        GIT_TAG
            V2.3.1
    )
    FetchContent_MakeAvailable(doxygen-awesome-css)
    set(DOXYGEN_GENERATE_TREEVIEW YES)
    set(DOXYGEN_HAVE_DOT YES)
    set(DOXYGEN_DOT_IMAGE_FORMAT svg)
    set(DOXYGEN_DOT_TRANSPARENT YES)
    set(DOXYGEN_HTML_EXTRA_STYLESHEET
        ${doxygen-awesome-css_SOURCE_DIR}/doxygen-awesome.css)
endmacro()
\end{cmake}

We already know all of these commands from previous chapters of the book, but let’s reiterate what happens for perfect clarity:

\begin{enumerate}
\item
Fetching doxygen-awesome-css from Git using the FetchContent module

\item
Configuring extra options for Doxygen (these are specifically recommended by the theme’s README file)

\item
Copying the theme’s css file to Doxygen’s output directory
\end{enumerate}

As you can imagine, it’s best to call this macro in the Doxygen function right before doxygen\_add\_docs(), like this:

\filename{ch13/02-doxygen-nice/cmake/Doxygen.cmake (fragment)}

\begin{cmake}
function(Doxygen input output)
# ...
UseDoxygenAwesomeCss()
doxygen_add_docs (...)
endfunction()

macro(UseDoxygenAwesomeCss)
# ...
endmacro()
\end{cmake}

Remember, all variables in macros are set in the scope of the calling function.
We can now enjoy a modern style in our generated HTML documentation and share it proudly with the world. However, our theme offers some JavaScript modules to enhance the experience.

How do we include them?









































































