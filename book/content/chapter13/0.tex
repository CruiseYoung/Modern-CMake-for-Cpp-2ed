高质量的代码不仅仅是编写良好、可运行且经过测试的——还应该有详尽的文档。文档使我们能够分享可能丢失的信息,描绘更大的蓝图,提供上下文,揭示意图,并且最终——对外部用户和维护者进行介绍。

你还记得上次加入一个新项目,在目录和文件的迷宫中迷失数小时的经历吗?这种情况是可以避免的。真正优秀的文档,可以让一个完全的新手在几秒钟内找到他们所需的代码行。遗憾的是,缺少文档的问题常常会被忽视。这并不奇怪——它需要相当多的技巧,而我们中的许多人并不擅长于此。此外,文档和代码很快就会过时。除非实施严格的更新和审查过程,否则很容易忘记文档也需要关注。

一些团队(为了节省时间或因为经理鼓励他们这样做)遵循编写自文档化代码的实践,通过为文件名、函数、变量等选择有意义、可读的标识符,希望避免编写文档的麻烦。即使是最优秀的函数签名也不能确保传达所有必要的信息——例如,int removeDuplicates(); 是描述性的,但它没有揭示返回了什么。可能是找到的重复项的数量,剩余项目的数量,或其他东西——都不清楚。虽然良好的命名习惯绝对正确,但它不能取代认真编写文档的行为。记住:世上没有免费的午餐。

为了使事情变得更容易,专业人士使用自动文档生成器来分析源文件中的代码和注释,以生成各种格式的全面文档。将这样的生成器添加到 CMake 项目中非常简单——来看看如何操作!

本章中,将包含以下内容:

\begin{itemize}
\item
将 Doxygen 添加到项目

\item
生成具有现代外观的文档

\item
使用自定义 HTML 增强输出
\end{itemize}