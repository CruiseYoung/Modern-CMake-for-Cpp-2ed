
有了好的实际例子,理论就更容易掌握。显然,我们想要编写一些 CMake 代码并尝试一下。然而,由于生成器表达式直到配置完成之后才进行计算,不能使用配置时的命令,如 message() 来验证。为了调试生成器表达式需要使用一些特殊的技巧,可以使用以下方法:

\begin{itemize}
\item
将其写入文件(file() 命令的特定版本支持生成器表达式): file(GENERATE OUTPUT filename CONTENT "\$<...>")

\item
添加一个自定义目标,并显式构建: add\_custom\_target(gendbg COMMAND \$\{CMAKE\_COMMAND\} -E echo "\$<...>")
\end{itemize}

我推荐第一个选项进行简单的练习。记住,我们无法在这些命令中使用所有的表达式,因为有些是特定于目标的。

了解了这一点,再来看一下生成器表达式的某些用途。

\mySubsubsection{6.6.1.}{构建配置}

在第1章中,讨论了构建类型,指定正在构建的配置——Debug、Release 等。可能有些情况下,希望根据正在进行的构建类型进行不同的操作。一个简单且直接的方法是使用\$<CONFIG>生成器表达式:

\begin{shell}
target_compile_options(tgt $<$<CONFIG:DEBUG>:-ginline-points>)
\end{shell}

前面的例子检查配置是否等于 DEBUG;如果是这样,嵌套表达式计算为 1。然后外部的简写 if 表达式变为真,-ginline-points 调试标志就添加到选项中。

了解这种形式很重要,这样就能够理解其他项目中的此类表达式,但我建议为了更好的可读性,可以使用更详细的\$<IF:...>.

\mySubsubsection{6.6.2.}{系统特定的单行命令}

生成器表达式还可以用来将冗长的 if 命令压缩成整洁的单行命令。

假设我们有以下代码:

\begin{cmake}
if (${CMAKE_SYSTEM_NAME} STREQUAL "Linux")
    target_compile_definitions(myProject PRIVATE LINUX=1)
endif()
\end{cmake}

它告诉编译器,如果这是目标系统,则向参数中添加 -DLINUX=1。虽然这并不算太长,但可以用一个相当简单的表达式替换:

\begin{cmake}
target_compile_definitions(myProject PRIVATE
                           $<$<CMAKE_SYSTEM_NAME:LINUX>:LINUX=1>)
\end{cmake}

这样的代码运行良好,但可以在生成器表达式中打包的内容有限,直到变得难以阅读。此外,许多 CMake 用户会推迟学习生成器表达式,并难以跟踪发生的事情。幸运的是,在完成本章后,我们就不会遇到这样的问题。

\mySubsubsection{6.6.3.}{带有编译器特定标志的接口库}

如第5章所讨论,接口库可以用来提供与编译器匹配的标志:

\begin{cmake}
add_library(enable_rtti INTERFACE)
target_compile_options(enable_rtti INTERFACE
    $<$<OR:$<COMPILER_ID:GNU>,$<COMPILER_ID:Clang>>:-rtti>
)
\end{cmake}

即使如此简单的例子中,当嵌套太多生成器表达式时,理解表达式也会变得难以理解。但有时这是实现期望效果的唯一方式。以下是例子的解释:

\begin{itemize}
\item
检查 COMPILER\_ID 是否为 GNU;如果是,将 OR 计算为 1。

\item
如果不是,检查 COMPILER\_ID 是否为 Clang,并将 OR 计算为 1。否则,将 OR 评估为 0。

\item
如果 OR 计算为 1,将 -rtti 添加到 enable\_rtti 编译选项中。否则,不做任何事情。
\end{itemize}

接下来,可以将库和可执行文件与 enable\_rtti 接口库链接起来。如果编译器支持,CMake 将添加 -rtti 标志。RTTI 代表运行时类型信息,在 C++ 中与 typeid 等关键字一起使用,以在运行时确定对象的类;除非代码使用这个功能,否则不需要启用该标志。

\mySubsubsection{6.6.4.}{嵌套生成器表达式}

有时,会尝试在生成器表达式中嵌套元素,但会发生什么并不明显。我们可以通过生成一个测试,将其输出到调试文件来调试表达式。

看看会发生什么:

\filename{ch06/01-nesting/CMakeLists.txt}

\begin{cmake}
set(myvar "small text")
set(myvar2 "small text >")

file(GENERATE OUTPUT nesting CONTENT "
    1 $<PLATFORM_ID>
    2 $<UPPER_CASE:$<PLATFORM_ID>>
    3 $<UPPER_CASE:hello world>
    4 $<UPPER_CASE:${myvar}>
    5 $<UPPER_CASE:${myvar2}>
")
\end{cmake}

在构建此项目后,可以使用 Unix 的 cat 命令读取生成的 nesting 文件:

\begin{shell}
# cat nesting

    1 Linux
    2 LINUX
    3 HELLO WORLD
    4 SMALL TEXT
    5 SMALL text>
\end{shell}

这就是每行的工作内容:

\begin{enumerate}
\item
PLATFORM\_ID 的输出值是 LINUX。

\item
嵌套值的输出将被正确地转换为大写的 LINUX。

\item
可以转换普通字符串。

\item
可以转换配置阶段的变量内容。

\item
变量将首先插值,然后闭合的尖括号(>)将解释为生成器表达式的一部分,只有字符串的部分大写。
\end{enumerate}

注意,变量的内容可能会影响生成器表达式的扩展行为。如果需要在变量中使用尖括号,请使用\$<ANGLE-R>。

\mySubsubsection{6.6.5.}{布尔表达式与 BOOL 运算符的计算差异}

当涉及到将布尔类型计算为字符串时,生成器表达式可能会有些令人困惑。了解它们与常规条件表达式的区别至关重要,尤其是从 IF 关键字开始:

\filename{ch06/02-boolean/CMakeLists.txt}

\begin{cmake}
cmake_minimum_required(VERSION 3.26)
project(Boolean CXX)

file(GENERATE OUTPUT boolean CONTENT "
    1 $<0:TRUE>
    2 $<0:TRUE,FALSE> (won't work)
    3 $<1:TRUE,FALSE>
    4 $<IF:0,TRUE,FALSE>
    5 $<IF:0,TRUE,>
")
\end{cmake}

使用 Linux 的 cat 命令读取生成的文件:

\begin{enumerate}
\item
这是一个布尔展开,其中 BOOL 是 0;因此,TRUE 字符串不会写入。

\item
这是一个典型的错误——作者原本打算根据 BOOL 值打印 TRUE 或 FALSE,但也是一个布尔假的展开,两个参数当作一个参数处理,因此不会输出。

\item
这是相同错误的一个反转值——它是一个布尔真的展开,两者都在同一行写入。

\item
这是一个以 IF 开头的正确条件表达式——输出 FALSE,因为第一个参数是 0。

\item
这是条件表达式的正确用法,但当不需要为布尔假提供值时,应该使用第一行中的方式。
\end{enumerate}

生成器表达式因其复杂的语法而声名狼藉,本例中提到的差异甚至可能会让经验丰富的构建者感到困惑。如果有所疑虑,将这样的表达式复制到另一个文件中,并通过添加缩进和空格来分析,以更好地理解。

了解了生成器表达式的工作示例,从而为我们使用它们做好了准备。接下来的章节将讨论许多与生成器表达式相关的主题。






