

可以执行逻辑运算,确定是否应该展开生成式表达式的值。这是一个很棒的功能,但因为历史原因,其语法可能不一致且难以阅读。它有两种形式,第一种形式支持正面和负面路径:

\begin{shell}
$<IF:condition,true_string,false_string>
\end{shell}

IF 表达式依赖于生成式表达式的嵌套:可以将任何一个参数替换为另一个表达式,并产生相当复杂的计算(甚至可以在一个 IF 条件中嵌套另一个)。这种形式需要正好三个参数,所以不能省略任何东西。可以通过如下形式表示在不满足condition条件时不返回任何值:

\begin{shell}
$<IF:condition,true_string,>
\end{shell}

有一个简写版本可以省略 IF 关键字和逗号:

\begin{shell}
$<condition:true_string>
\end{shell}

其打破了将表达式名称作为第一个标记提供的惯例。我猜这里的意图是为了缩短表达式,并避免输入那些宝贵的几个字符,但结果可能真的很难理性化。以下是从 CMake 文档中取出的一例:

\begin{shell}
$<$<AND:$<COMPILE_LANGUAGE:CXX>,$<CXX_COMPILER_ID:AppleClang,Clang>>:COMPILING_CXX_WITH_CLANG>
\end{shell}

这个表达式只有在用 Clang 编译器编译的 C++ 代码中才返回 COMPILING\_CXX\_WITH\_CLANG(其他情况下返回空字符串)。我希望这个语法能够与常规 IF 命令的条件保持一致,但遗憾的是情况并非如此。现在,如果在某个地方看到了第二种形式,知道它是怎么工作的就好;但为了可读性,应该避免在自己的项目中使用。

\mySubsubsection{6.4.1.}{计算布尔值}

生成器表达式计算为两种类型之一——布尔值或字符串。布尔值用 1(真)和 0(假)表示。没有专用的数值类型;除了布尔值之外的类型都只是字符串。

需要记住的是,作为条件表达式中的条件传递的嵌套表达式,明确要求计算为布尔值。

布尔类型可以隐式转换为字符串,但是字符串转换为布尔类型需要使用明确的 BOOL 运算符(稍后解释)。

有三类表达式可计算为布尔值:逻辑运算符、比较表达式和查询。

\mySamllsection{逻辑运算符}

有四个逻辑运算符:

\begin{itemize}
\item
\$<NOT:arg>: 否定布尔参数。

\item
\$<AND:arg1,arg2,arg3...>: 如果所有参数都为真,则返回真。

\item
\$<OR:arg1,arg2,arg3...>: 如果任一参数为真,则返回真。

\item
\$<BOOL:string\_arg>: 这将字符串参数从字符串转换为布尔类型。
\end{itemize}

使用\$<BOOL>的字符串转换在以下条件都不满足时,将计算为布尔真(1):

\begin{itemize}
\item
字符串为空。

\item
字符串是 0、FALSE、OFF、N、NO、IGNORE 或 NOTFOUND 的不区分大小写的等价物。

\item
字符串以 -NOTFOUND 后缀结尾(区分大小写)。
\end{itemize}

\mySamllsection{比较}

如果满足条件,比较将计算为 1,否则为 0。以下是一些可能有用的常见操作:

\begin{itemize}
\item
\$<STREQUAL:arg1,arg2>: 这以区分大小写的方式比较字符串。

\item
\$<EQUAL:arg1,arg2>: 这将字符串转换为数字并比较相等性。

\item
\$<IN\_LIST:arg,list>: 这检查 arg 元素是否在 list 列表中(区分大小写)。

\item
\$<VERSION\_EQUAL:v1,v2>,\$<VERSION\_LESS:v1,v2>, \$<VERSION\_GREATER:v1,v2>,\$<VERSION\_LESS\_EQUAL:v1,v2> 和 \$<VERSION\_GREATER\_EQUAL:v1,v2> 以逐组件的方式比较版本。

\item
\$<PATH\_EQUAL:path1,path2>: 这比较两个路径的词法表示,不进行标准化(自 CMake 3.24 起)。
\end{itemize}

\mySamllsection{查询}

查询直接从变量返回布尔值,或者作为操作的结果。

最简单的查询之一是:

\begin{shell}
$<TARGET_EXISTS:arg>
\end{shell}

如果目标在配置阶段定义,将返回真。

现在,知道如何应用条件展开,使用逻辑运算符、比较和基本查询来计算为布尔值。但生成器表达式还有更多功能,特别是在查询的上下文中:可以在 IF 条件展开中使用,或者作为参数独立地传递给命令。

是时候在适当的上下文中介绍它们了。



