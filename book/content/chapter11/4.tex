编写单元测试从技术上讲,可以不使用任何类型的框架。我们所要做的就是创建想要测试的类的实例,执行其一个方法,并检查新的状态或返回的值是否符合我们的预期。然后,报告结果并删除测试的对象。

将使用以下结构:

\begin{shell}
- CMakeLists.txt
- src
  |- CMakeLists.txt
  |- calc.cpp
  |- calc.h
  |- main.cpp
- test
  |- CMakeLists.txt
  |- calc_test.cpp
\end{shell}

从main.cpp开始,看到其使用了Calc类:

\filename{ch11/01-no-framework/src/main.cpp}

\begin{cpp}
#include <iostream>
#include "calc.h"
using namespace std;
int main() {
    Calc c;
    cout << "2 + 2 = " << c.Sum(2, 2) << endl;
    cout << "3 * 3 = " << c.Multiply(3, 3) << endl;
}
\end{cpp}

没什么太花哨的——main.cpp只是包含了calc.h头文件,并调用了Calc对象的两个方法。让我们快速地看一下SUT(测试系统)Calc的接口:

\filename{ch11/01-no-framework/src/calc.h}

\begin{cpp}
#pragma once
class Calc {
    public:
    int Sum(int a, int b);
    int Multiply(int a, int b);
};
\end{cpp}

这个接口尽可能简单。这里使用\#pragma once——工作原理与常见的预处理器包含保护完全一样,尽管它不是官方标准的一部分。

\begin{myNotic}{Note}
包含保护是头文件中的一行简短的代码,用于防止在同一个父文件中多次包含。
\end{myNotic}

看看类的实现:

\filename{ch11/01-no-framework/src/calc.cpp}

\begin{cpp}
#include "calc.h"
int Calc::Sum(int a, int b) {
    return a + b;
}
int Calc::Multiply(int a, int b) {
    return a * a; // a mistake!
}
\end{cpp}

哎呀!我们引入了一个错误!Multiply忽略了b参数,并返回了a的平方。这应该正确编写的单元测试检测到。所以,让我们来写一些:

\filename{ch11/01-no-framework/test/calc\_test.cpp}

\begin{cpp}
#include "calc.h"
#include <cstdlib>
void SumAddsTwoIntegers() {
    Calc sut;
    if (4 != sut.Sum(2, 2))
        std::exit(1);
}
void MultiplyMultipliesTwoIntegers() {
    Calc sut;
    if(3 != sut.Multiply(1, 3))
    std::exit(1);
}
\end{cpp}

从calc\_test.cpp文件开始编写两个测试方法,每个测试的SUT方法一个。如果从调用方法返回的值不符合预期,每个函数将调用std::exit(1)。这里可以使用assert()、abort()或terminate(),但这样会在ctest的输出中得到一个不太明确的子进程终止消息,而不是更易读的失败消息。

是时候创建一个测试运行器了。我们的将尽可能简单,以避免引入大量工作。看看为了运行两个测试而编写的main()函数:

\filename{ch11/01-no-framework/test/unit\_tests.cpp}

\begin{cpp}
#include <string>
void SumAddsTwoIntegers();
void MultiplyMultipliesTwoIntegers();
int main(int argc, char *argv[]) {
    if (argc < 2 || argv[1] == std::string("1"))
        SumAddsTwoIntegers();
    if (argc < 2 || argv[1] == std::string("2"))
        MultiplyMultipliesTwoIntegers();
}
\end{cpp}

以下是发生的情况的分解:

\begin{enumerate}
\item
声明两个外部函数,这些函数将从一个翻译单元中链接。

\item
如果没有提供参数,执行两个测试(argv[]中的第一个元素始终是程序名称)。

\item
如果第一个参数是测试的标识符,执行它。

\item
如果任何测试失败,将内部调用exit()并返回1退出代码。

\item
如果没有执行测试或所有测试都通过,将隐式返回0退出代码。
\end{enumerate}

运行第一个测试,执行:

\begin{shell}
./unit_tests 1
\end{shell}

运行第二个测试,执行:

\begin{shell}
./unit_tests 2
\end{shell}

我们尽可能简化了代码,但它仍然难以阅读。可能需要维护这一部分的人,在添加了更多测试后,都不会感到轻松。这个功能相当原始——调试这样的测试将很困难。尽管如此,来看看如何使用CTest来使用它:

\filename{ch11/01-no-framework/CMakeLists.txt}

\begin{cmake}
cmake_minimum_required(VERSION 3.26.0)
project(NoFrameworkTests CXX)
include(CTest)
add_subdirectory(src bin)
add_subdirectory(test)
\end{cmake}

从常规的头文件开始,并include(CTest)——启用了CTest,并且应该在顶级CMakeLists.txt中始终这样做。接下来,在src和test每个子目录中包含两个嵌套的列表文件。指定的bin值表示我们希望将src子目录的二进制输出放置在<build\_tree>/bin中。否则,二进制文件最终会出现在<build\_tree>/src中,这可能会使用户感到困惑,因为构建工件不是源文件。

对于src目录,列表文件是直接的,并包含一个简单的main目标:

\filename{ch11/01-no-framework/src/CMakeLists.txt}

\begin{cmake}
add_executable(main main.cpp calc.cpp)
\end{cmake}

还需要为test目录创建一个列表文件:

\filename{ch11/01-no-framework/test/CMakeLists.txt}

\begin{cmake}
add_executable(unit_tests
               unit_tests.cpp
               calc_test.cpp
               ../src/calc.cpp)
target_include_directories(unit_tests PRIVATE ../src)
add_test(NAME SumAddsTwoInts COMMAND unit_tests 1)
add_test(NAME MultiplyMultipliesTwoInts COMMAND unit_tests 2)
\end{cmake}

现在定义了第二个unit\_tests目标,使用了src/calc.cpp实现文件及其相应的头文件。最后,明确添加了两个测试:

\begin{itemize}
\item
SumAddsTwoInts

\item
MultiplyMultipliesTwoInts
\end{itemize}

每个测试都将其ID作为参数传递给add\_test()命令。CTest将简单地接受COMMAND关键字之后提供的内容,并在子shell中执行,收集输出和退出代码。不要过于依赖add\_test()方法;在后面的单元测试框架部分,将了解处理测试用例的更好方法。

要运行测试,请在构建树中执行ctest:

\begin{shell}
# ctest
Test project /tmp/b
    Start 1: SumAddsTwoInts
1/2 Test #1: SumAddsTwoInts ................... Passed 0.00 sec
    Start 2: MultiplyMultipliesTwoInts
2/2 Test #2: MultiplyMultipliesTwoInts ........***Failed 0.00 sec
50% tests passed, 1 tests failed out of 2
Total Test time (real) = 0.00 sec
The following tests FAILED:
          2 - MultiplyMultipliesTwoInts (Failed)
Errors while running CTest
Output from these tests are in: /tmp/b/Testing/Temporary/LastTest.log
Use "--rerun-failed --output-on-failure" to re-run the failed cases
verbosely
\end{shell}

CTest执行了两个测试,并报告其中一个失败了——Calc::Multiply返回的值不符合预期,非常好。现在知道我们的代码有一个错误,需要有人修复它。

\begin{myNotic}{Note}
迄今为止的大多数示例中,我们并不一定采用了第4章中描述的项目结构,这样做是为了保持简洁。这一章讨论了更高级的概念,使用完整结构是合理的。在项目(无论多小)中,最好从一开始就遵循这种结构。正如一位智者所说:“踏上这条路时,如果不保持警惕,就不知道会去到哪里。”
\end{myNotic}

现在已经很清楚了,为项目从头开始构建测试框架并不是一个好主意。即使是最基本的示例也难以阅读,有很多开销,并且没有增加任何价值。然而,在我们采用单元测试框架之前,需要重新考虑项目的结构。






































