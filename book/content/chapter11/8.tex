表面上,与恰当测试相关的复杂性似乎如此之大,以至于它们不值得付出努力。有多少代码在外运行却没有任何测试,这真是令人惊讶,主要是测试软件是一项令人生畏的任务。如果是手动进行,则更是如此。不幸的是,如果没有严格的自动化测试,代码中的问题都是不完整或根本不可见的。未经测试的代码可能更快编写(但并非总是如此);然而,要阅读、重构和修复这些代码,绝对要慢得多。

本章中,概述了从一开始就进行测试工作的几个关键原因,其中最引人注目的是心理健康和良好的夜间睡眠。没有一个开发者躺在床上想:等不及几个小时后被打扰,去处理一些生产环境中的火灾和修复错误。但认真地说,在将错误部署到生产环境之前捕捉它们,对你(和公司)可能是一根救命稻草。

当涉及到测试工具时,CMake在这里真正展现了它的强大。CTest在检测故障测试方面能发挥奇迹:隔离、混洗、重复和超时。所有这些技术都非常有用,可以通过一个方便的命令行标志获得。我们了解到如何使用CTest来列出测试,过滤,并控制测试用例的输出,但最重要的是,现在知道如何使用标准解决方案的真正力量。任何用CMake构建的项目都可以进行完全相同的测试,而无需探究其内部细节。

接下来,我们结构化了项目,简化了测试过程,并在生产代码和测试运行器之间重用相同的对象文件。编写自己的测试运行器很有趣,但也许我们应专注于我们程序实际应解决的问题,并花时间接受一个主流的第三方测试框架。

说到这,我们学习了Catch2和GoogleTest的基础知识。进一步深入研究了GMock库的细节,并理解了测试替身,如何工作以实现真正的单元测试。最后,使用LCOV设置了一些报告。毕竟,没有什么比硬数据更能证明,我们的解决方案确实是经过全面测试的。

下一章中,将讨论更多有用的工具,以提高代码的质量,并找出潜在的问题。