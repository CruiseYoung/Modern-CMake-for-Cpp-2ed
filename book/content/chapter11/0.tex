经验丰富的专业人士知道,测试必须自动化。这是几年前有人向他们解释的,或者通过艰难的方式学到的。对于没有经验的程序员来说,这一做法并不那么明显,看起来像是额外的、不必要的劳动,并不会带来太多价值。这是可以理解的:当某人刚开始编写代码时,还没有创建真正复杂的解决方案,也没有在大型的代码库上工作。很可能,是自己项目的唯一开发者。这些早期项目很少需要超过几个月就能完成,因此很难看到代码在较长时间内是如何恶化的。

所有这些因素都导致人们认为编写测试是浪费时间和精力。编程新手可能会告诉自己,每次进行构建和运行流程时,实际上确实在测试代码。毕竟,已经手动确认了代码能够正常工作,并做到了预期效果。所以,是时候继续下一个任务了,对吧?自动化测试确保新的更改不会无意中破坏程序。本章中,将学习为什么测试很重要,以及如何使用CTest来协调测试执行。CTest可以查询可用的测试,过滤执行,随机排序,重复执行,并设置时间限制。我们将探讨如何使用这些功能,控制CTest的输出,并处理测试失败。

接下来,将修改项目的结构以适应测试,并创建自己的测试运行器。

介绍了基本原理之后,将继续添加流行的测试框架:Catch2和GoogleTest(也称为GTest),以及其模拟库。最后,将介绍使用LCOV进行详细的测试覆盖率报告。

本章中,将包含以下内容:

\begin{itemize}
\item
为什么自动化测试值得麻烦?

\item
使用CTest在CMake中标准化测试

\item
为CTest创建最基本的单元测试

\item
单元测试框架

\item
生成测试覆盖率报告
\end{itemize}




















