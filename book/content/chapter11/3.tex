
最终,自动化测试不过是运行一个可执行文件,将系统置于测试状态(SUT),执行想要测试的操作,并检查结果是否符合预期。可以视为一种结构化的方式来完成“GIVEN\_<CONDITION>\_WHEN\_<SCENARIO>\_THEN\_<EXPECTED-OUTCOME>”,并验证这对于SUT是否成立。一些资源建议按照这种非常方式命名你的测试函数:例如,GIVEN\_4\_and\_2\_WHEN\_Sum\_THEN\_returns\_6。

实现和执行这些测试的方法有很多,取决于选择的框架、如何将其与SUT连接,以及相应的设置。对于一个首次接触项目的用户来说,即使是测试二进制文件这样的小细节,也会影响他们的体验。由于没有标准的命名约定,开发者可能会将他们的测试可执行文件命名为test\_my\_app,另一个可能会选择unit\_tests,第三个可能会选择更不直观的名称,或者完全跳过测试。弄清楚要运行哪个文件、使用哪个框架、传递哪些参数,以及如何收集结果都是用户想要避免的麻烦。

CMake通过一个单独的ctest命令行工具解决了这个问题。通过项目作者通过列表文件配置,提供了一种标准化的运行测试方式。这个统一的界面适用于使用CMake构建的每个项目。遵循这个标准,将享受到其他好处:将项目集成到持续集成/持续部署(CI/CD)中变得更加容易,测试在IDE(如Visual Studio或CLion)中显示得更方便。最重要的是,只需最小努力就能获得一个健壮的测试运行工具。

那么,如何在已经配置的项目中使用CTest运行测试呢?需要选择以下三种操作模式之一:

\begin{itemize}
\item
仪表盘模式

\item
测试模式

\item
构建并测试模式
\end{itemize}

仪表盘模式会将测试结果发送到一个名为CDash的单独工具,也来自Kitware。CDash收集并展示软件质量测试结果,在一个易于导航的仪表盘中。这个主题对于非常大的项目很有用,但超出了本书的范围。

测试模式的命令行如下:

\begin{shell}
ctest [<options>]
\end{shell}

此模式下,应在使用CMake构建项目后,在构建树中运行CTest。有许多选项可用,但在深入讨论之前,需要解决一个小的不便:必须在构建树中运行ctest二进制文件,并且只有在项目构建之后才能运行。

为了简化操作,CTest提供了构建并测试模式。首先探讨这个模式,这样稍后可以全神贯注于测试模式。

\mySubsubsection{11.3.1.}{构建并测试模式}

要使用此模式,需要执行ctest后跟–{}-build-and-test:

\begin{shell}
ctest --build-and-test <source-tree> <build-tree>
      --build-generator <generator> [<options>...]
      [--build-options <opts>...]
      [--test-command <command> [<args>...]]
\end{shell}

这是测试模式的一个简单包装,接受构建配置选项和–{}-test-command参数后的测试命令。重要的是,除非在–{}-test-command后包含ctest关键字,否则不会运行任何测试:

\begin{shell}
ctest --build-and-test project/source-tree /tmp/build-tree --buildgenerator "Unix Makefiles" --test-command ctest
\end{shell}

此命令中,指定源和构建路径,并选择一个构建生成器。所有这三个都是必需的,并遵循第1章中详细描述的cmake命令的规则。

可以添加更多参数,这些参数通常分为以下三个类别:配置控制、构建过程或测试设置。

配置阶段的参数如下:

\begin{itemize}
\item
-{}-build-options ——为cmake配置包含选项。在–{}-test-command之前放置,必须放在最后。

\item
-{}-build-two-config —— 对CMake运行两次配置阶段。

\item
-{}-build-nocmake —— 跳过配置阶段。

\item
-{}-build-generator-platform —— 提供生成器特定的平台。

\item
-{}-build-generator-toolset —— 提供生成器特定的工具集。

\item
-{}-build-makeprogram —— 为基于Make或Ninja的生成器指定make可执行文件。
\end{itemize}

构建阶段的参数如下:

\begin{itemize}
\item
-{}-build-target —— 指定要构建的目标。

\item
-{}-build-noclean —— 构建clean目标之前进行构建。

\item
-{}-build-project —— 命名正在构建的项目。
\end{itemize}

测试阶段的参数如下:

\begin{itemize}
\item
-{}-test-timeout —— 设置测试的时间限制,以秒为单位。
\end{itemize}

现在可以配置测试模式,方法是在-{}-test-command cmake后添加参数,或者直接运行测试模式。

\mySubsubsection{11.3.2.}{测试模式}

构建项目后,可以在构建目录中使用ctest命令运行测试。如果使用构建并测试模式,就会一起完成。在没有其他标志的情况下运行ctest,通常足以满足大多数情况。如果所有测试都成功,ctest将返回0的退出代码(在类Unix系统中),可以在CI/CD中验证,以防止将故障更改合并到生产分支。

编写好的测试可能与编写生产代码本身一样具有挑战性,将系统置于特定的状态(SUT),运行单个测试,然后销毁SUT实例。这个过程相当复杂,可能会产生各种问题:跨测试污染、时间并发干扰、资源争用、死锁导致的冻结执行,以及其他许多问题。

CTest提供了多种选项来缓解这些问题,可以控制哪些测试运行、执行顺序、生成的输出、时间限制和重复率,以及其他方面。接下来的部分将提供必要的上下文和最有用选项的简要概述。

\mySamllsection{查询测试}

我们可能需要做的第一件事是了解哪些测试实际上是为项目编写的。CTest提供了-N选项,禁用执行并只打印列表:

\begin{shell}
# ctest -N
Test project /tmp/b
  Test #1: SumAddsTwoInts
  Test #2: MultiplyMultipliesTwoInts
Total Tests: 2
\end{shell}

使用-N与下一节中描述的过滤器一起使用,以检查当应用过滤器时哪些测试会执行。

如果需要一个可以自动化工具消费的JSON格式,可以执行ctest并使用-{}-show-only=json-v1。

CTest还提供了一个使用LABELS关键字来分组测试的机制,列出所有可用的标签(无需实际执行测试),请使用-{}-print-labels。这个选项在手动定义测试时非常有用,例如在列表文件中使用add\_test()命令,然后就可以通过测试属性指定个别标签:

\begin{cmake}
set_tests_properties(<name> PROPERTIES LABELS "<label>")
\end{cmake}

然而,来自各种框架的自动化测试方法可能不支持这种标签。

\mySamllsection{过滤测试}

有时可能只想运行特定测试而不是整个套件。例如,正在调试一个失败的单个测试,没有必要运行其他所有测试。还可以使用这种机制来为大型项目跨多台机器分发测试。

这些标志将根据提供的正则表达式(regex)过滤测试:

\begin{itemize}
\item
-R <r>, -{}-tests-regex <r> - 只运行<r>与匹配测试名称的测试

\item
-E <r>, -{}-exclude-regex <r> - 跳过与<r>匹配测试名称的测试

\item
-L <r>, -{}-label-regex <r> - 只运行与<r>匹配标签的测试

\item
 -LE <r>, -{}-label-exclude <regex> - 跳过与<r>匹配标签的测试
\end{itemize}

高级场景可以通过使用-{}-tests-information选项(或更短的-I形式)来实现。此选项用逗号分隔的格式<start>,<end>,<step>,<test-IDs>的范围,可以省略其他字段但保留逗号。<test-IDs>选项是一个逗号分隔的测试序号的列表。例如:

\begin{itemize}
\item
-I 3,, 跳过测试1和2(执行从第三个测试开始)

\item
-I ,2, 只运行第一个和第二个测试

\item
-I 2,,3 每行运行第三个测试,从第二行开始

\item
-I ,0,,3,9,7 只运行第三个、第九个和第七个测试
\end{itemize}

还可以将这些范围指定在一个文件中,以在分布式方式上在多台机器上执行非常庞大的测试套件。当与-R一起使用-I时,只有满足两个条件的测试才会运行。如果想运行满足任一条件的测试,请使用-U选项。如前所述,可以使用-N选项来检查过滤的结果。

\mySamllsection{打乱测试}

编写单元测试可能会遇到一些意想不到的问题,其中一个令人惊讶的问题是测试耦合,即一个测试通过不完全设置或清除SUT的状态来影响另一个测试。换句话说,第一个执行的测试可能会“泄漏”其状态,并污染第二个测试。这种耦合是坏消息,因为它引入了测试之间的未知、隐式关系。

更糟糕的是,这种错误往往隐藏在测试场景的复杂性中。可能会在随机失败的情况下检测到,但相反的情况也同样可能:一个不正确的状态,可能会导致测试在没有错误的情况下通过。这些错误的测试会给人一种安全感的错觉,这比完全没有测试还要糟糕。认为代码正确测试的假设可能,会鼓励更大胆的行动,导致意外的结果。

发现这类问题的一个方法是,独立运行每个测试。通常,当直接从测试框架执行测试运行器而没有CTest时,这并不是情况。要运行单个测试,需要向测试可执行文件传递一个特定于框架的参数。这允许检测那些在套件中通过,但在单独执行时失败的测试。

另一方面,CTest通过隐式地在子CTest实例中执行每个测试用例,有效地消除了所有基于内存的测试交叉污染。甚至可以更进一步,添加-{}-force-new-ctest-process选项来强制使用单独的进程。

不幸的是,如果测试使用了外部、争用的资源,如GPU、数据库或文件,那么仅凭这一点可能无法奏效。可以采取的另一项预防措施是随机化测试执行的顺序,引入这种变化通常足以最终检测出假性通过的测试。CTest支持这一策略,可通过-{}-schedule-random选项进行打乱。

\mySamllsection{处理失败}

这里有一个著名的约翰·C·麦克斯韦的名言:“尽早失败,经常失败,但总是向前失败。”向前失败意味着从我们的错误中学习,这就是我们运行单元测试(以及在生活的其他领域)时想要做的事情。除非在运行测试时附带调试器,否则很难发现自己犯了什么错误,因为CTest会保持简洁,只列出失败的测试,而不会实际打印它们的输出。

测试用例或SUT打印到标准输出的消息可能非常有价值,以确定确切出了什么问题。要看到它们,可以运行ctest并使用-{}-output-on-failure,或设置CTEST\_OUTPUT\_ON\_FAILURE环境变量也会有同样的效果。

根据解决方案的大小,测试失败后停止执行可能有意义。这可以通过向ctest提供-{}-stop-on-failure参数来实现。

CTest会存储失败的测试名称。为了节省长时间测试套件的时间,可以专注于这些失败的测试,跳过运行通过测试,直到问题解决。这一特性是通过-{}-rerun-failed选项启用的(其他过滤器都将忽略)。解决所有问题后,记得运行所有测试,以确保在此期间没有引入回归。

当CTest没有检测到任何测试时,可能有两种情况:要么测试不存在,要么项目存在问题。默认情况下,ctest会打印一个警告消息并返回0的退出代码,以避免混淆。大多数用户都有足够的上下文来理解遇到了哪种情况,以及下一步该做什么。然而,在某些环境中,ctest总是作为自动化流水线的一部分执行。这时,可能需要明确指出,缺乏测试应视为错误(并返回非零退出代码),可以通过-{}-no-tests=error参数来进行配置。对于相反的行为(无警告),请使用-{}-no-tests=ignore选项。

\mySamllsection{重复测试}

职业生涯中,迟早会遇到那些大多数时间都运行正确的测试,我想强调的是“大多数”。偶尔,这些测试会因为环境原因而失败:例如,由于错误地模拟时间、事件循环问题、异步执行的处理不当、并行性、哈希冲突,以及其他在每次运行中都不会发生的非常复杂的场景。这些不可靠的测试称为\textbf{易碎测试}。

这种不一致似乎是一个不太重要的问题。我们可能会说,测试不是一个真正的生产环境,这就是为什么它们有时会失败的原因。这种说法确实有一定的道理:测试不是为了复现每一个细节。测试是一种模拟,是对可能发生的事情的近似,这通常就足够了。如果在下次运行时会通过,那么重新运行测试会有什么伤害呢?实际上,这确实有伤害。

有三个主要关注点,如下所述:

\begin{itemize}
\item
如果代码库中有足够多的易碎测试,将成为平滑交付代码变更的严重障碍。当急于回家在周五下午,或者急于向客户交付一个严重问题的关键修复时,这尤其令人沮丧。

\item
不能真正确定易碎测试是否因为测试环境的不完善而失败。可能是相反的情况:它们失败是因为复现了一个已经在生产环境中发生的罕见场景。这还没有明显到足以发出警报……但已经足够了。

\item
不是测试易碎,而是你的代码!环境有时会出问题——作为开发者,我们以确定性的方式处理这些问题。如果SUT以这种方式行为,那就是一个严重的错误的迹象——代码可能会从未初始化的内存中读取。
\end{itemize}

没有完美的方法来解决所有上述情况——可能的原因太多了。然而,我们可以通过多次运行它们并使用–repeat <mode>:<\#>option选项来增加识别易碎测试的机会。有三种模式可供选择:

\begin{itemize}
\item
until-fail —— 运行测试<\#>次;所有运行都必须通过。

\item
until-pass —— 最多运行测试<\#>次;它必须至少通过一次。这适用于处理已知易碎但过于复杂和重要而无法调试或禁用的测试。

\item
after-timeout —— 最多运行测试<\#>次,但仅在测试超时时才重试。在繁忙的测试环境中使用。
\end{itemize}

一个普遍的建议是尽快调试易碎测试,或者如果不能信任产生一致的结果,就尽快消除它们。

\mySamllsection{控制输出}

将每一条信息都打印到屏幕上会变得非常繁忙。CTest减少了噪音,并将执行的测试输出收集到日志文件中,只在常规运行中提供最有用的信息。当事情出错测试失败时,可以期望一个总结,以及如果启用了-{}-output-on-failure,可能还会有一些日志,如之前所述。

从经验中了解,“足够的信息”的确足够,直到它不再足够。有时,可能还想看到通过的测试的输出,以检查是否真的在正常工作(而不是悄悄地停止而没有错误)。要获取更详细的输出,可以添加-V选项(或-{}-verbose,如果想在自动化流水线中明确表示)。如果这还不够,可能会需要使用-VV或-{}-extra-verbose。对于极其深入的调试,可以使用–debug(但要做好准备,可能会出现带有所有细节的文本墙)。

如果寻找相反的情况,CTest也提供了“禅模式”,通过-Q或-{}-quiet启用。那时不会有任何输出(可以停止担心,学会爱上这个bug)。这个选项似乎除了让人困惑之外没有其他用途,但要注意输出仍然会存储在测试文件中(默认情况下存储在./Testing/Temporary)。自动化流水线可以检查退出代码是否为非零值,并收集日志文件进行进一步处理,而不会在主要输出中添加可能使不熟悉产品的开发者困惑的细节。

要将日志存储在特定路径,请使用-O <file>, -{}-output-log <file>选项。如果输出过长,有两个限制选项可以限制每个测试的字节数:-{}-test-output-size-passed 和-{}-test-output-size-failed <size>。

\mySamllsection{其他选项}

还有一些其他选项可以满足日常测试需求:

\begin{itemize}
\item
-C <cfg>, -{}-build-config <cfg> —— 指定要测试的配置。调试配置通常具有调试符号,使事情更容易理解,因为重优化选项可能会影响SUT的行为,所以发布版本也应该测试。这个选项仅适用于多配置生成器。

\item
-j <jobs>, -{}-parallel <jobs> —— 设置并行执行的测试数量。这对于加快长时间测试的执行非常有用。在繁忙的环境中(共享的测试运行器上),这可能会由于调度而产生不利影响。这可以通过下一个选项稍作缓解。

\item
-{}-test-load <level> —— 以一种方式安排并行测试,使得CPU负载不超过值(尽最大努力)。

\item
-{}-timeout <seconds> —— 指定单个测试的默认时间限制。
\end{itemize}

现在,已经了解了在许多不同场景下如何执行ctest,接下来让我们学习如何添加一个简单的测试。



