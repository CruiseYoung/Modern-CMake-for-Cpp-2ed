C++ 具备一些有限的内省能力,但无法提供像 Java 那样强大的反射特性。这可能是为什么在 C++ 代码中编写测试和单元测试框架比在其他功能更丰富的环境中更具挑战性的原因。这种有限方法的一个结果是,程序员需要更深入地参与编写可测试的代码。我们需要仔细设计我们的接口,并考虑实际方面。例如,我们如何避免编译代码两次,并在测试和生产之间重用工件?

对于较小的项目来说,编译时间可能不是大问题,但随着项目的发展,缩短编译循环的需求仍然存在。在前面的例子中,我们将所有 SUT 源文件包含在单元测试可执行文件中,除了 main.cpp 文件。如果你仔细观察,你会注意到该文件中的一些代码没有被测试(即 main() 函数本身的内容)。编译代码两次会引入一个轻微的风险,即产生的工件可能不会完全相同。随着时间的推移,这些差异可能会逐渐增加,特别是在添加编译标志和预处理器指令时,如果贡献者急于完成、经验不足或不熟悉项目,这可能会带来风险。

这个问题有多种解决方案,但最直接的方法是将整个解决方案构建为一个库,并与单元测试链接。你可能会想知道然后如何运行它。答案是创建一个引导可执行文件,它与库链接并执行其代码。

首先,将你当前的 main() 函数重命名为 run() 或 start\_program()。然后,创建一个只包含新的 main() 函数的实现文件(bootstrap.cpp)。这个函数充当适配器:它的唯一作用是提供一个入口点并调用 run(),传递任何命令行参数。将所有东西链接在一起后,你最终会得到一个可测试的项目。

通过重命名 main(),你现在可以将 SUT 与测试链接,并测试其 main 功能。否则,你会违反第 8 章讨论的“单一定义规则”(ODR),因为测试运行器也需要它自己的 main() 函数。正如我们在第 8 章的“为测试分离 main()”部分所承诺的,我们将在这里详细讨论这个主题。

请注意,测试框架可能默认提供它自己的 main() 函数,因此编写一个可能是不必要的。通常,它会自动检测所有链接的测试并根据你的配置运行它们。

这种方法产生的工件可以分为以下目标:

\begin{itemize}
\item
一个包含生产代码的 sut 库

\item
一个引导程序,其中包含调用 sut 中 run() 的 main() 包装器

\item
一个单元测试,其中包含运行所有 sut 测试的 main() 包装器
\end{itemize}

下面的图表显示了目标之间的符号关系:

\myGraphic{0.9}{content/chapter11/images/1.png}{图 11.1:在测试和生产可执行文件之间共享工件}

我们最终得到六个实现文件,它们将分别产生各自的(.o)对象文件,如下所示:

\begin{itemize}
\item
calc.cpp: 将要进行单元测试的 Calc 类。这被称为单元测试对象(UUT),因为 UUT 是 SUT 的一个特化。

\item
run.cpp: 原始入口点重命名为 run(),现在可以对其进行测试。

\item
bootstrap.cpp: 新的 main() 入口点,调用 run()。

\item
calc\_test.cpp: 测试 Calc 类。

\item
run\_test.cpp: 新的 run() 测试可以放在这里。

\item
unit\_tests.o: 单元测试的入口点,扩展为调用 run() 的测试。
\end{itemize}

我们即将构建的库不一定是静态或共享库。通过选择对象库,我们可以避免不必要的归档或链接。从技术上讲,使用动态链接 SUT 可以节省一些时间,但我们经常发现自己同时修改两个目标:测试和 SUT,这抵消了节省的任何时间。

让我们看看之前名为 main.cpp 的文件是如何变化的:

\filename{ch11/02-structured/src/run.cpp}

\begin{cpp}
#include <iostream>
#include "calc.h"
using namespace std;
int run() {
    Calc c;
    cout << "2 + 2 = " << c.Sum(2, 2) << endl;
    cout << "3 * 3 = " << c.Multiply(3, 3) << endl;
    return 0;
}
\end{cpp}

变化很小:文件和函数被重命名,我们添加了一个返回语句,因为编译器不会隐式地为 main() 之外的函数添加返回语句。

新的 main() 函数如下所示:

\filename{ch11/02-structured/src/bootstrap.cpp}

\begin{cpp}
int run(); // declaration
int main() {
    run();
}
\end{cpp}

保持简单,我们声明链接器将提供来自另一个翻译单元的 run() 函数,并调用它。

接下来是 src 列表文件:

\filename{ch11/02-structured/src/CMakeLists.txt}

\begin{cmake}
add_library(sut STATIC calc.cpp run.cpp)
target_include_directories(sut PUBLIC .)
add_executable(bootstrap bootstrap.cpp)
target_link_libraries(bootstrap PRIVATE sut)
\end{cmake}

首先,我们创建一个 SUT 库,并将 . 标记为 PUBLIC 包含目录,这样它就会被传播到所有与 SUT 链接的目标(即 bootstrap 和 unit\_tests)。请注意,包含目录相对于列表文件,允许我们使用点(.)来引用当前的 <source\_tree>/src 目录。

现在是我们更新 unit\_tests 目标的时候了。我们将替换对 …/src/calc.cpp 文件的直接引用,改为对 sut 的链接引用,用于 unit\_tests 目标。我们还将为 run\_test.cpp 文件中的主函数添加一个新的测试。为了简洁,我们将省略对此的讨论,但如果你感兴趣,可以查看本书仓库中的示例。

同时,以下是整个测试列表文件:

\filename{ch11/02-structured/test/CMakeLists.txt}

\begin{cmake}
add_executable(unit_tests
               unit_tests.cpp
               calc_test.cpp
               run_test.cpp)
target_link_libraries(unit_tests PRIVATE sut)
add_test(NAME SumAddsTwoInts COMMAND unit_tests 1)
add_test(NAME MultiplyMultipliesTwoInts COMMAND unit_tests 2)
add_test(NAME RunOutputsCorrectEquations COMMAND unit_tests 3)
\end{cmake}

完成了!我们按需注册了新的测试。通过遵循这种做法,你可以确保你的测试是在将要用于生产中的机器代码上执行的。

\begin{myNotic}{Note}
我们在这里使用的目标名称,sut 和 bootstrap,是为了从测试的角度非常清楚地表明它们是关于什么的。在实际项目中,你应该选择与生产代码上下文(而不是测试)匹配的名称。例如,对于一个 FooApp,将你的目标命名为 foo 而不是 bootstrap,将 lib\_foo 而不是 sut。
\end{myNotic}

现在我们知道了如何在适当的目标中构建一个可测试的项目,让我们将焦点转移到测试框架本身。我们不想手动将每个测试案例添加到我们的列表文件中,对吧?








































