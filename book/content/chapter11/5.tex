C++ 具备有限的内省能力,但无法提供像 Java 那样强大的反射特性。这可能是为什么在 C++ 代码中编写测试和单元测试框架,比在其他功能更丰富的环境中更具挑战性的原因。这种有限方法的一个结果是,开发者需要更深入地参与编写可测试的代码。需要仔细设计接口,并考虑实际应用。例如,如何避免编译代码两次,并在测试和生产之间重用工件?

对于较小的项目来说,编译时间可能不是大问题,但随着项目的发展,缩短编译循环的需求仍然存在。前面的例子中,将所有 SUT 源文件包含在单元测试可执行文件中,除了 main.cpp 文件。如果仔细观察,会注意到该文件中的一些代码没有测试(即 main() 函数本身的内容)。编译代码两次会引入一个轻微的风险,即产生的工件可能不会完全相同。随着时间的推移,这些差异可能会逐渐增加,特别是在添加编译标志和预处理器指令时,如果贡献者急于完成、经验不足或不熟悉项目,这可能会带来风险。

这个问题有多种解决方案,但最直接的方法是将整个解决方案构建为一个库,并与单元测试链接。可能会有人想知道然后如何运行。那么就创建一个引导可执行文件,它与库链接并执行其代码。

首先,将当前的 main() 函数重命名为 run() 或 start\_program()。然后,创建一个只包含新的 main() 函数的实现文件(bootstrap.cpp)。这个函数充当适配器:唯一作用是提供一个入口点并调用 run(),传递命令行参数。将所有东西链接在一起后,最终会得到一个可测试的项目。

通过重命名 main(),现在可以将 SUT 与测试链接,并测试其 main 功能。否则,会违反第 8 章讨论的“单一定义规则”(ODR),测试运行器也需要自己的 main() 函数。

注意,测试框架默认提供它自己的 main() 函数,它会自动检测所有链接的测试,并根据配置运行它们。

这种方法产生的工件可以分为以下目标:

\begin{itemize}
\item
包含生产代码的 sut 库

\item
引导程序,其中包含调用 sut 中 run() 的 main() 包装器

\item
单元测试,其中包含运行所有 sut 测试的 main() 包装器
\end{itemize}

下面的图表显示了目标之间的符号关系:

\myGraphic{0.9}{content/chapter11/images/1.png}{图 11.1:测试和生产可执行文件之间共享工件}

最终得到六个实现文件,分别产生各自的(.o)对象文件:

\begin{itemize}
\item
calc.cpp: 将要进行单元测试的 Calc 类。这称为单元测试对象(UUT),因为 UUT 是 SUT 的一个特化。

\item
run.cpp: 原始入口点重命名为 run(),现在可以对其进行测试。

\item
bootstrap.cpp: 新的 main() 入口点,调用 run()。

\item
calc\_test.cpp: 测试 Calc 类。

\item
run\_test.cpp: 新的 run() 测试可以放在这里。

\item
unit\_tests.o: 单元测试的入口点,扩展为调用 run() 的测试。
\end{itemize}

我们即将构建的库不一定是静态或共享库。通过选择对象库,可以避免不必要的归档或链接。从技术上讲,使用动态链接 SUT 可以节省一些时间,但经常发现自己同时修改两个目标:测试和 SUT,这抵消了节省的时间。

让我们看看之前名为 main.cpp 的文件是如何变化的:

\filename{ch11/02-structured/src/run.cpp}

\begin{cpp}
#include <iostream>
#include "calc.h"
using namespace std;
int run() {
    Calc c;
    cout << "2 + 2 = " << c.Sum(2, 2) << endl;
    cout << "3 * 3 = " << c.Multiply(3, 3) << endl;
    return 0;
}
\end{cpp}

变化很小:文件和函数重命名,添加了一个返回语句,因为编译器不会隐式地为 main() 之外的函数添加返回语句。

新的 main() 函数如下所示:

\filename{ch11/02-structured/src/bootstrap.cpp}

\begin{cpp}
int run(); // declaration
int main() {
    run();
}
\end{cpp}

保持简单,我们声明链接器将提供来自另一个翻译单元的 run() 函数,并调用它。

接下来是 src 下的列表文件:

\filename{ch11/02-structured/src/CMakeLists.txt}

\begin{cmake}
add_library(sut STATIC calc.cpp run.cpp)
target_include_directories(sut PUBLIC .)
add_executable(bootstrap bootstrap.cpp)
target_link_libraries(bootstrap PRIVATE sut)
\end{cmake}

首先,创建一个 SUT 库,并将 . 标记为 PUBLIC 包含目录,这样它就会传播到所有与 SUT 链接的目标(即 bootstrap 和 unit\_tests)。请注意,包含目录相对于列表文件,允许使用点(.)来引用当前的 <source\_tree>/src 目录。

现在是更新 unit\_tests 目标的时候了。我们将替换对 …/src/calc.cpp 文件的直接引用,改为对 sut 的链接引用,用于 unit\_tests 目标,还将为 run\_test.cpp 文件中的主函数添加一个新的测试。为了简洁,我们将省略对此的讨论,但如果感兴趣,可以查看本书仓库中的示例。

同时,以下是整个测试列表文件:

\filename{ch11/02-structured/test/CMakeLists.txt}

\begin{cmake}
add_executable(unit_tests
               unit_tests.cpp
               calc_test.cpp
               run_test.cpp)
target_link_libraries(unit_tests PRIVATE sut)
add_test(NAME SumAddsTwoInts COMMAND unit_tests 1)
add_test(NAME MultiplyMultipliesTwoInts COMMAND unit_tests 2)
add_test(NAME RunOutputsCorrectEquations COMMAND unit_tests 3)
\end{cmake}

完成了!我们按需注册了新的测试。通过遵循这种做法,可以确保测试是在用于生产中的机器代码上执行的。

\begin{myNotic}{Note}
这里使用的目标名称,sut 和 bootstrap,是为了从测试的角度清楚地表明它们是关于什么的。在实际项目中,应该选择与生产代码上下文(而不是测试)匹配的名称。例如,对于一个 FooApp,将目标命名为 foo 而不是 bootstrap,将 lib\_foo 而不是 sut。
\end{myNotic}

现在,我们知道了如何在适当的目标中构建一个可测试的项目,再将焦点转移到测试框架本身。我们不想手动将每个测试案例添加到列表文件中,对吧?








































