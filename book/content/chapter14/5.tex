
我们将从解释术语“组件”可能引起的混淆开始。考虑find\_package()的完整签名:

\begin{shell}
find_package(<PackageName>
            [version] [EXACT] [QUIET] [MODULE] [REQUIRED]
            [[COMPONENTS] [components...]]
            [OPTIONAL_COMPONENTS components...]
            [NO_POLICY_SCOPE]
)
\end{shell}

重要的是不要将这里提到的组件,与install()命令中使用的COMPONENT关键字混淆。尽管它们名称相同,但是不同的概念,必须分开理解。我们将在以下子节中进一步探讨这一点。

\mySubsubsection{14.5.1.}{如何在find\_package()中使用组件}

当调用带有COMPONENTS或OPTIONAL\_COMPONENTS列表的find\_package()时,告诉CMake我们只对提供这些组件的包感兴趣。然而,理解验证这一要求是包的责任至关重要。如果包提供商没有在配置文件中实现必要的检查,则不会按预期进行。

请求的组件通过<package>\_FIND\_COMPONENTS变量传递给配置文件(包括可选和不可选的)。对于每个不可选组件,都会设置一个<package>\_FIND\_REQUIRED\_<component>变量。包作者可以编写宏来扫描这个列表,并验证所有必需组件的提供情况。check\_required\_components()函数就为此目的服务。当找到必要的组件时,配置文件应设置<package>\_<component>\_FOUND变量。文件末尾的一个宏将验证是否设置了所有必需的变量。

\mySubsubsection{14.5.2.}{如何在install()命令中使用组件}

并非在所有情况下都需要安装所有生成的工件。例如,一个项目可能为了开发而安装静态库和公共头文件,但默认情况下,可能只需要为运行时安装一个共享库。为了启用这种双重行为,可以使用COMPONENT关键字将工件分组在install()命令下的一个通用名称下。有兴趣限制安装到特定组件的用户可以通过执行以下区分大小写的命令来实现:

\begin{shell}
cmake --install <build tree>
      --component=<component1 name> --component=<component2 name>
\end{shell}

为工件分配COMPONENT关键字并不会自动将其从默认安装中排除。要实现这种排除,必须添加EXCLUDE\_FROM\_ALL关键字。

让我们在代码示例中探讨这个概念:

\filename{ch14/13-components/CMakeLists.txt (片段)}

\begin{cmake}
install(TARGETS calc EXPORT CalcTargets
    ARCHIVE
        COMPONENT lib
    FILE_SET HEADERS
        COMPONENT headers
)
install(EXPORT CalcTargets
    DESTINATION ${CMAKE_INSTALL_LIBDIR}/calc/cmake
    NAMESPACE Calc::
    COMPONENT lib
)
install(CODE "MESSAGE(\"Installing 'extra' component\")"
    COMPONENT extra
    EXCLUDE_FROM_ALL
)
\end{cmake}

前面的安装命令定义了以下组件:

\begin{itemize}
\item
lib: 这包含静态库和目标导出文件。默认安装。

\item
headers: 这包含C++头文件。也默认安装。

\item
extra: 这执行一段代码以打印消息。不默认安装。
\end{itemize}

让我们重申:

\begin{itemize}
\item
cmake -{}-install 没有 -{}-component 参数将安装lib和headers组件。

\item
cmake -{}-install -{}-component headers 将只安装公共头文件。

\item
cmake -{}-install -{}-component extra 将打印一条消息,其他命令不会打印(EXCLUDE\_FROM\_ALL关键字阻止了这一点)。
\end{itemize}

如果没有为安装的工件指定COMPONENT关键字,它默认为未指定,由CMAKE\_INSTALL\_DEFAULT\_COMPONENT\_NAME变量定义。

\begin{myNotic}{Note}
由于无法从cmake命令行列出所有可用组件,因此记录包的所有组件对于用户来说可能非常有帮助。安装"README"文件是放置这些信息的绝佳位置。
\end{myNotic}

如果cmake使用-{}-component参数调用一个不存在的组件,命令将成功完成,不会有警告或错误,但不会安装任何东西。

将我们的安装划分为组件,使得用户可以选择性地安装包的某一部分。现在来管理版本化共享库的符号链接,这是一个优化你的安装过程的有用功能。

\mySubsubsection{14.5.3.}{管理版本化共享库的符号链接}

有些安装的目标平台可以使用符号链接,来帮助链接器发现共享库的当前安装版本。如通常创建 lib<name>.so 符号链接链接到 lib<name>.so.1 实际库文件(可以使用set_property()指令给动态库目标指定版本,这时该目标包含有数字后缀的实际库文件和无数字后缀的链接到实际库文件的符号链接),之后就可以通过向链接器统一传递 -l<name> 参数(不需要指定共享库的版本后缀)来链接这样的库。

CMake 的 install(TARGETS <target> LIBRARY) 在安装时会同时安装符号链接和实际库文件。也可以只安装实际库文件,将符号链接的安装移到另一个 install() 命令中。这通过在这个块中添加 NAMELINK\_SKIP 来实现:

\begin{shell}
install(TARGETS <target> LIBRARY
        COMPONENT cmp NAMELINK_SKIP)
\end{shell}

以上指令在执行安装时,只会安装 <target> 中的实际库文件,且 cmp 组件中只包含实际库文件。

可以使用有 NAMELINK\_ONLY 关键词的 install() 命令只安装符号链接文件:

\begin{shell}
install(TARGETS <target> LIBRARY
        COMPONENT lnk NAMELINK_ONLY)
\end{shell}

以上指令在执行安装时,只会安装 <target> 中的符号链接,且 lnk 组件中只包含符号链接。

以上的效果也可以通过使用 NAMELINK\_COMPONENT 关键词来达到。以下命令执行后,会同时安装 <target> 中的实际库文件和符号链接,但是 cmp 组件中只包含实际库文件, lnk 组件中只包含符号链接:

\begin{shell}
install(TARGETS <target> LIBRARY
        COMPONENT cmp NAMELINK_COMPONENT lnk)
\end{shell}

现在我们已经配置了自动安装过程,可以使用随 CMake 一起提供的 CPack 工具来为用户提供预构建的工件。








































