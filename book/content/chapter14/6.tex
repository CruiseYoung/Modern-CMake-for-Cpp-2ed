虽然从源代码构建项目有其好处,但对于最终用户特别是非开发人员来说,这可能会既耗时又复杂。一种更方便的分发方法是使用二进制包,其中包含了编译后的制品和其他必要的静态文件。CMake 支持使用名为 cpack 的命令行工具来生成此类包。

要生成一个包,需要为目标平台和包类型选择一个合适的包生成器。不要将包生成器与像 Unix Makefiles 或 Visual Studio 这样的构建系统生成器混淆。

下表列出了可用的包生成器:

% Please add the following required packages to your document preamble:
% \usepackage{longtable}
% Note: It may be necessary to compile the document several times to get a multi-page table to line up properly
\begin{longtable}{|l|l|l|}
\hline
\textbf{生成器名称} & \textbf{生成的文件类型}              & \textbf{平台}     \\ \hline
\endfirsthead
%
\endhead
%
Archive &
\begin{tabular}[c]{@{}l@{}}7Z, 7zip - (.7z)\\ TBZ2 (.tar.bz2)\\ TGZ (.tar.gz)\\ TXZ (.tar.xz)\\ TZ (.tar.Z)\\ TZST (.tar.zst)\\ ZIP (.zip)\end{tabular} &
跨平台 \\ \hline
Bundle                  & macOs Bundle (.bundle)                    & macOS                 \\ \hline
Cygwin                  & Cygwin packages                           & Cygwin                \\ \hline
DEB                     & Debian packages (.deb)                    & Linux                 \\ \hline
External                & 第三方打包程序使用的 JSON (.json) 文件 & 跨平台        \\ \hline
FreeBSD                 & PKG (.pkg)                                & *BSD, Linux, macOS    \\ \hline
IFW                     & QT 安装程序二进制文件                       & Linux, Windows, macOS \\ \hline
NSIS                    & Binary (.exe)                             & Windows               \\ \hline
NuGet                   & NuGet 包 (.nupkg)                    & Windows               \\ \hline
productbuild            & PKG (.pkg)                                & macOS                 \\ \hline
RPM                     & RPM (.rpm)                                & Linux                 \\ \hline
WIX                     & Microsoft Installer (.msi)                & Windows               \\ \hline
\end{longtable}

\begin{center}
表 14.3: 可用的包生成器
\end{center}

大多数这些生成器都有广泛的配置选项。本书不打算深入探讨所有细节,相应信息可以在“扩展阅读”部分找到更多信息。

为了使用 CPack,需要使用必要的 install() 命令来配置项目的安装,并构建项目。CPack 会根据构建树中的 CPackConfig.cmake 文件准备二进制包。虽然可以手动创建这个文件,但在项目的列表文件中使用 include(CPack) 更加简便,它会在构建树中生成配置文件并提供所需的默认值。

让我们扩展 13-components 示例以供 CPack 使用:

\filename{ch14/14-cpack/CMakeLists.txt (fragment)}

\begin{cmake}
cmake_minimum_required(VERSION 3.26)
project(CPackPackage VERSION 1.2.3 LANGUAGES CXX)
include(GNUInstallDirs)
add_subdirectory(src bin)
install(...)
install(...)
install(...)
set(CPACK_PACKAGE_VENDOR "Rafal Swidzinski")
set(CPACK_PACKAGE_CONTACT "email@example.com")
set(CPACK_PACKAGE_DESCRIPTION "Simple Calculator")
include(CPack)
\end{cmake}

CPack 模块从 project() 命令中提取以下变量:

\begin{itemize}
\item
CPACK\_PACKAGE\_NAME

\item
CPACK\_PACKAGE\_VERSION

\item
CPACK\_PACKAGE\_FILE\_NAME
\end{itemize}

The CPACK\_PACKAGE\_FILE\_NAME 存储了包名的结构:

\begin{shell}
$CPACK_PACKAGE_NAME-$CPACK_PACKAGE_VERSION-$CPACK_SYSTEM_NAME
\end{shell}

CPACK\_SYSTEM\_NAME 是目标操作系统的名称,例如 Linux 或 win32。例如,在 Debian 上执行 ZIP 生成器时,CPack 将生成一个名为 CPackPackage-1.2.3-Linux.zip 的文件。

要在构建项目后生成包,请转到项目的构建树并运行:

\begin{shell}
cpack [<options>]
\end{shell}

CPack 从 CPackConfig.cmake 文件读取选项,可以覆盖这些设置:

\begin{itemize}
\item
-G <generators>: 以分号分隔的包生成器列表。默认值可以在 CPackConfig.cmake 中的 CPACK\_GENERATOR 变量中指定。

\item
-C <configs>: 以分号分隔的构建配置列表(debug, release),用于生成包(对于多配置构建系统生成器是必需的)。

\item
-D <var>=<value>: 此选项覆盖 CPackConfig.cmake 文件中设置的变量。

\item
-{}-config <config-file>: 此选项使用指定的配置文件代替默认的 CPackConfig.cmake 文件。
cmake.

\item
-{}-verbose, -V: 此选项提供详细的输出。

\item
-P <packageName>: 此选项覆盖包名。

\item
-R <packageVersion>: 此选项覆盖包版本。

\item
-{}-vendor <vendorName>: 此选项覆盖包供应商。

\item
-B <packageDirectory>: 此选项指定 cpack 的输出目录(默认情况下,这将是当前工作目录)。
\end{itemize}

让我们尝试为我们的 14-cpack 示例项目生成包。我们将使用 ZIP、7Z 和 Debian 包生成器:

\begin{shell}
cpack -G "ZIP;7Z;DEB" -B packages
\end{shell}

应该会得到以下这些包:

\begin{itemize}
\item
CPackPackage-1.2.3-Linux.7z

\item
CPackPackage-1.2.3-Linux.deb

\item
CPackPackage-1.2.3-Linux.zip
\end{itemize}

这些二进制包已准备好发布在项目网站上、GitHub Release页面中,或作为最终用户的包存储库。
