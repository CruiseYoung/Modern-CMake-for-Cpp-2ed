
在第1章中,我们指出CMake提供了一个命令行模式,用于在系统上安装构建的项目:

\begin{shell}
cmake --install <dir> [<options>]
\end{shell}

这里,<dir>是指向生成的构建树路径(必需的)。包括:

\begin{itemize}
\item
-{}-config <cfg>: 这用于为多配置生成器选择构建配置。

\item
-{}-component <comp>: 这将安装限制在给定的组件内。

\item
-{}-default-directory-permissions <permissions>: 这为安装的目录设置默认权限(格式为<u=rwx,g=rx,o=rx>)。

\item
-{}-install-prefix <prefix>: 这指定非默认的安装路径(在CMAKE\_INSTALL\_PREFIX变量中)。在类Unix系统中默认为/usr/local,在Windows中默认为C:/Program Files/\$\{PROJECT\_NAME\}。在CMake 3.21之前,需要使用一个不太明确的选项: -{}-prefix <prefix>.

\item
-v, -{}-verbose: 这增加了输出信息的详细程度(也可以通过设置VERBOSE环境变量实现)。
\end{itemize}

安装通常涉及将生成的工件和必要的依赖项复制到系统目录。使用CMake为所有CMake项目引入了一个方便的安装标准:

\begin{itemize}
\item
为不同类型的工件提供特定于平台的安装路径(遵循GNU编码标准)。

\item
通过生成目标导出文件来增强安装过程,允许其他项目直接重用项目目标。

\item
通过配置文件创建可发现的包,包装目标导出文件和作者定义的特定于包的CMake宏和函数。
\end{itemize}

这些功能非常强大,节省了大量时间并简化了以这种方式准备的项目使用。执行基本安装的第一步是将构建的工件复制到目标目录。这就引出了install()命令及其各种模式:

\begin{itemize}
\item
install(TARGETS): 这安装输出工件,如库和可执行文件。

\item
install(FILES|PROGRAMS): 这安装单个文件并设置它们的权限。这些文件不需要是逻辑目标的一部分。

\item
install(DIRECTORY): 这安装整个目录。

\item
install(SCRIPT|CODE): 这在安装过程中运行CMake脚本或代码片段。

\item
install(EXPORT): 这生成并安装目标导出文件。

\item
install(RUNTIME\_DEPENDENCY\_SET <set-name> [...]): 这安装项目中定义的运行时依赖项集。

\item
install(IMPORTED\_RUNTIME\_ARTIFACTS <target>... [...]): 这查询导入的目标以获取运行时工件并安装它们。
\end{itemize}

将这些命令添加到列表文件中,会在构建树中生成一个cmake\_install.cmake文件。虽然可以手动用cmake -P调用这个脚本,但这并不推荐。该文件旨在由CMake在执行cmake -{}-install时内部使用。

每个install()模式都有完整的一套选项,其中一些在模式之间共享:

\begin{itemize}
\item
DESTINATION: 这指定安装路径。相对路径会以CMAKE\_INSTALL\_PREFIX为前缀,而绝对路径会按原样使用(并且不受cpack支持)。

\item
PERMISSIONS: 这在支持的平台设置文件权限。可用的值包括OWNER\_READ, OWNER\_WRITE, OWNER\_EXECUTE, GROUP\_READ, GROUP\_WRITE, GROUP\_ EXECUTE, WORLD\_READ, WORLD\_WRITE, WORLD\_EXECUTE, SETUID 和 SETGID。安装时创建的默认目录权限可以通过 CMAKE\_INSTALL\_DEFAULT\_DIRECTORY\_PERMISSIONS 变量来设置。

\item
CONFIGURATIONS: 这指定配置(Debug,Release)。跟随此关键字的选项仅当当前构建配置在列表中时才适用。

\item
OPTIONAL: 避免安装文件不存在时出现错误。
\end{itemize}

两个共享选项,COMPONENT和EXCLUDE\_FROM\_ALL,用于特定于组件的安装。这些将在本章后面的“定义组件”部分讨论。现在,来看看第一个安装模式:install(TARGETS)。

\mySubsubsection{14.3.1.}{安装逻辑目标}

由add\_library()和add\_executable()定义的目标可以通过install(TARGETS)命令轻松安装,将构建系统生成的工件复制到适当的目标目录,并为设置合适的文件权限。此模式的通用签名为:

\begin{shell}
install(TARGETS <target>... [EXPORT <export-name>]
        [<output-artifact-configuration> ...]
        [INCLUDES DESTINATION [<dir> ...]]
)
\end{shell}

指定了初始模式标识符,即TARGETS之后,必须提供想要安装的目标列表。在这里,可以选择性地使用EXPORT选项将它们分配给一个命名的导出,这可以在export(EXPORT)和install(EXPORT)中使用,以生成目标导出文件。然后,必须配置输出工件的安装(按类型分组)。可选地,可以提供一个目录列表,这些目录将添加到每个目标在其INTERFACE\_INCLUDE\_DIRECTORIES属性中的目标导出文件中。

[<output-artifact-configuration>...] 提供了一个配置块列表。单个块的完整语法如下:

\begin{shell}
<TYPE> [DESTINATION <dir>]
       [PERMISSIONS permissions...]
       [CONFIGURATIONS [Debug|Release|...]]
       [COMPONENT <component>]
       [NAMELINK_COMPONENT <component>]
       [OPTIONAL] [EXCLUDE_FROM_ALL]
       [NAMELINK_ONLY|NAMELINK_SKIP]
\end{shell}

命令要求每个输出工件块以<TYPE>开头(这是唯一必需的元素)。CMake识别以下几种类型:

\begin{itemize}
\item
ARCHIVE: 静态库(.a)和Windows系统上的DLL导入库(.lib)。

\item
LIBRARY: 共享库(.so),但不包括DLL。

\item
RUNTIME: 可执行文件和DLL。

\item
OBJECTS: 来自OBJECT库的对象文件。

\item
FRAMEWORK: 设置了FRAMEWORK属性的静态和共享库(从ARCHIVE和LIBRARY中排除),是macOS特定的。

\item
BUNDLE: 标记有MACOSX\_BUNDLE的可执行文件(也不属于RUNTIME)。

\item
FILE\_SET <set>: 指定给目标的文件集<set>中的文件。可以是C++头文件或C++模块头文件(自CMake 3.23起)。

\item
PUBLIC\_HEADER, PRIVATE\_HEADER, RESOURCE: 在目标属性中指定相同名称的文件(在Apple平台上,应该设置在FRAMEWORK或BUNDLE目标上)。
\end{itemize}

CMake文档声称,如果只配置一个工件类型(例如,LIBRARY),只有这个类型会安装。对于CMake版本3.26.0,所有工件都会安装。这可以通过为所有不需要的工件类型指定 EXCLUDE\_FROM\_ALL来解决。

\begin{myNotic}{Note}
单个install(TARGETS)命令可以有多个工件配置块。但是请注意,每个调用中只能指定每种类型的一个。也就是说,如果想要为Debug和Release配置配置不同的ARCHIVE工件的目标,必须进行两次单独的install(TARGETS … ARCHIVE)调用。
\end{myNotic}

也可以省略类型名称,并为所有工件指定选项。安装将针对这些目标产生的每个文件执行,而不管其类型:

\begin{cmake}
install(TARGETS executable, static_lib1
    DESTINATION /tmp
)
\end{cmake}

许多情况下,由于内置的默认值,需要显式提供DESTINATION,但在处理不同平台时需要记住一些注意事项。

\mySamllsection{利用不同平台上的默认目标目录}

当CMake安装你的项目的文件时,会将它们复制到系统中的特定目录。不同类型的文件属于不同的目录。这个目录由以下公式确定:

\begin{cmake}
${CMAKE_INSTALL_PREFIX} + ${DESTINATION}
\end{cmake}

如前一部分所述,可以为安装显式提供DESTINATION组件,或者让CMake根据工件的类型使用内置的默认值:

% Please add the following required packages to your document preamble:
% \usepackage{longtable}
% Note: It may be necessary to compile the document several times to get a multi-page table to line up properly
\begin{longtable}{|l|l|l|}
\hline
\textbf{工件类型}                                    & \textbf{内置默认值} & \textbf{安装目录变量} \\ \hline
\endfirsthead
%
\endhead
%
RUNTIME                                                   & bin                       & CMAKE\_INSTALL\_BINDIR              \\ \hline
\begin{tabular}[c]{@{}l@{}}LIBRARY\\ ARCHIVE\end{tabular} & lib                       & CMAKE\_INSTALL\_LIBDIR              \\ \hline
\begin{tabular}[c]{@{}l@{}}PUBLIC\_HEADER\\ PRIVATE\_HEADER\\ FILE\_SET (类型 HEADERS)\end{tabular} & include & CMAKE\_INSTALL\_INCLUDEDIR \\ \hline
\end{longtable}

\begin{center}
表14.1:每种工件类型的默认目标位置
\end{center}

虽然默认路径很有用,但并不总是合适。例如,CMake 将库的默认 DESTINATION 设置为 lib。对于类 Unix 系统,库的完整路径为 /usr/local/lib,而在 Windows 上则类似于 C:\verb|\|Program Files (x86)\verb|\|<project-name>\verb|\|lib。然而,这对于支持多架构的 Debian 来说并不理想,当 INSTALL\_PREFIX 是 /usr 时,需要一个特定于架构的路径(例如,i386-linux-gnu)。为每个平台确定正确的路径对于类 Unix 系统来说是一个常见的挑战。为了解决这个问题,请遵循 GNU 编码标准,该链接添加在本章末尾的进一步阅读部分。

我们可以通过设置 CMAKE\_INSTALL\_<DIRTYPE>\_DIR 变量来覆盖每个值的默认目标。不是开发一个算法来检测平台,并将适当的路径分配给安装目录变量,而是使用 CMake 的 GNUInstallDirs 实用模块。该模块通过相应地设置安装目录变量来处理大多数平台,只需在 install() 命令之前使用 include() 包含它即可。

需要自定义配置的用户可以通过命令行参数覆盖安装目录变量:

\begin{shell}
-DCMAKE_INSTALL_BINDIR=/path/in/the/system
\end{shell}

然而,安装库的公共头文件仍然存在挑战。

\mySamllsection{处理公共头文件}

CMake 中管理公共头文件时,最佳实践是将它们存储在一个指示其来源,并引入命名空间的目录中,例如 /usr/local/include/calc。这允许它们在 C++ 项目中使用包含指令来使用:

\begin{cpp}
#include <calc/basic.h>
\end{cpp}

大多数预处理器将尖括号指令解释为请求扫描标准系统目录,可以使用 GNUInstallDirs 模块来自动填充安装路径的 DESTINATION 部分,确保头文件最终位于 include 目录中。

自从 CMake 3.23.0 起,可以使用 target\_sources() 命令和 FILE\_SET 关键字明确地将头文件添加到适当的目标以进行安装。这种方法更为可取,其负责了头文件的重新定位。以下是语法:

\begin{shell}
target_sources(<target>
    [<PUBLIC|PRIVATE|INTERFACE>
        [FILE_SET <name> TYPE <type> [BASE_DIR <dir>] FILES]
        <files>...
    ]...
)
\end{shell}

假设头文件位于 src/include/calc 目录中:

\filename{ch14/02-install-targets/src/CMakeLists.txt (片段)}

\begin{cmake}
add_library(calc STATIC basic.cpp)
target_include_directories(calc INTERFACE include)
target_sources(calc PUBLIC FILE_SET HEADERS
                           BASE_DIRS include
                           FILES include/calc/basic.h
)
\end{cmake}

前面的代码片段定义了一个名为 HEADERS 的新目标文件集。我们在这里使用了一个特殊情况:如果文件集的名称与可用类型之一匹配,CMake 会假定文件集是此类类型,从而消除了明确定义类型的需要。如果使用不同的名称,请记住使用适当的 TYPE <TYPE> 关键字来定义 FILE\_SET 的类型。

定义了文件集后,我们可以在安装命令中使用它:

\filename{ch14/02-install-targets/src/CMakeLists.txt (续)}

\begin{cmake}
...

include(GNUInstallDirs)
install(TARGETS calc ARCHIVE FILE_SET HEADERS)
\end{cmake}

包含了 GNUInstallDirs 模块并配置了 calc 静态库,及其头文件的安装。以安装模式运行 cmake 按预期工作:

\begin{shell}
# cmake -S <source-tree> -B <build-tree>
# cmake --build <build-tree>
# cmake --install <build-tree>
-- Install configuration: ""
-- Installing: /usr/local/lib/libcalc.a
-- Installing: /usr/local/include/calc/basic.h
\end{shell}

对 FILE\_SET HEADERS 关键字的支持是相对较新的更新,但并非所有环境都会提供较新版本的 CMake。

如果还停留在 3.23 之前的版本,需要以分号分隔的列表形式指定公共头文件(在库目标的 PUBLIC\_HEADER 属性中),并手动处理重新定位:

\filename{ch14/03-install-targets-legacy/src/CMakeLists.txt (片段)}

\begin{cmake}
add_library(calc STATIC basic.cpp)
target_include_directories(calc INTERFACE include)
set_target_properties(calc PROPERTIES
    PUBLIC_HEADER src/include/calc/basic.h
)
\end{cmake}

还需要更改目标目录,以便在包含路径中包含库名称:

\filename{ch14/02-install-targets-legacy/src/CMakeLists.txt (续)}

\begin{cmake}
...
include(GNUInstallDirs)
install(TARGETS calc
    ARCHIVE
    PUBLIC_HEADER
    DESTINATION ${CMAKE_INSTALL_INCLUDEDIR}/calc
)
\end{cmake}

因为指定在 PUBLIC\_HEADER 属性中的文件不会保留目录结构,所以需要在路径中插入 /calc。即使嵌套在不同的基本目录中,也都会安装到同一个目标位置。因为这个缺点,才进行了 FILE\_SET 的开发。

现在,了解了如何处理大多数安装情况,但是应该如何处理更高级的场景呢?

\mySubsubsection{14.3.2.}{低层安装}

现代 CMake 正在远离直接操作文件。理想情况下,应该将文件添加到逻辑目标中,使用它作为更高层次的抽象来代表所有底层资产:源文件、头文件、资源、配置等。主要优点是代码的简洁性;通常,将文件添加到目标只需要更改不超过一行。

不幸的是,将每个安装的文件添加到目标并不可能或方便的。这种情况下,有三个选项可用:install(FILES)、install(PROGRAMS) 和 install(DIRECTORY)。

\mySamllsection{使用 install(FILES) 和 install(PROGRAMS) 安装}

FILES 和 PROGRAMS 模式非常相似,可以用来安装各种资产,包括公共头文件、文档、shell 脚本、配置,以及像图像、音频文件和数据集这样的运行时资源。

以下是命令签名:

\begin{shell}
install(<FILES|PROGRAMS> files...
        TYPE <type> | DESTINATION <dir>
        [PERMISSIONS permissions...]
        [CONFIGURATIONS [Debug|Release|...]]
        [COMPONENT <component>]
        [RENAME <name>] [OPTIONAL] [EXCLUDE_FROM_ALL]
)
\end{shell}

FILES 和 PROGRAMS 之间的主要区别在于复制文件时设置的默认文件权限。install(PROGRAMS) 为所有用户设置 EXECUTE,而 install(FILES) 则不设置(尽管两者都会设置 OWNER\_WRITE, OWNER\_READ, GROUP\_READ 和 WORLD\_READ)。

可以使用可选的 PERMISSIONS 关键字来修改这种行为,然后选择领先的关键字(FILES 或 PROGRAMS)作为安装内容的指示器。我们已经介绍了 PERMISSIONS、CONFIGURATIONS 和 OPTIONAL 的用法。COMPONENT 和 EXCLUDE\_FROM\_ALL 将在后面的定义组件部分讨论。

初始关键字之后,需要列出想要安装的所有文件。CMake 支持相对路径和绝对路径,以及生成器表达式。请记住,如果文件路径以生成器表达式开头,则必须是绝对路径。

下一个必需的关键字是 TYPE 或 DESTINATION。可以选择明确提供 DESTINATION 路径,或者要求 CMake 为特定 TYPE 文件查找它。与 install(TARGETS) 不同,在此上下文中,TYPE 并不选择要安装的提供的文件子集。尽管如此,安装路径的计算遵循相同的模式(其中 + 符号表示平台特定的路径分隔符):

\begin{cmake}
${CMAKE_INSTALL_PREFIX} + ${DESTINATION}
\end{cmake}

同样,每种 TYPE 都将具有内置的默认值:

% Please add the following required packages to your document preamble:
% \usepackage{longtable}
% Note: It may be necessary to compile the document several times to get a multi-page table to line up properly
\begin{longtable}{|l|l|l|}
\hline
\textbf{文件类型} & \textbf{内置默认值} & \textbf{安装目录变量} \\ \hline
\endfirsthead
%
\endhead
%
BIN           & bin                       & CMAKE\_INSTALL\_BINDIR                   \\ \hline
SBIN          & sbin                      & CMAKE\_INSTALL\_SBINDIR                  \\ \hline
LIB           & lib                       & CMAKE\_INSTALL\_LIBDIR                   \\ \hline
INCLUDE       & include                   & CMAKE\_INSTALL\_INCLUDEDIR               \\ \hline
SYSCONF       & etc                       & CMAKE\_INSTALL\_SYSCONFDIR               \\ \hline
SHAREDSTATE   & com                       & CMAKE\_INSTALL\_SHARESTATEDIR            \\ \hline
LOCALSTATE    & var                       & CMAKE\_INSTALL\_LOCALSTATEDIR            \\ \hline
RUNSTATE      & \$LOCALSTATE/run          & CMAKE\_INSTALL\_RUNSTATEDIR              \\ \hline
DATA          & \$DATAROOT                & CMAKE\_INSTALL\_DATADIR                  \\ \hline
INFO          & \$DATAROOT/info           & CMAKE\_INSTALL\_INFODIR                  \\ \hline
LOCALE        & \$DATAROOT/locale         & CMAKE\_INSTALL\_LOCALEDIR                \\ \hline
MAN           & \$DATAROOT/man            & CMAKE\_INSTALL\_MANDIR                   \\ \hline
DOC           & \$DATAROOT/doc            & CMAKE\_INSTALL\_DOCDIR                   \\ \hline
\end{longtable}

\begin{center}
表14.2:每种文件类型的内置默认值
\end{center}

这里的行为遵循了之前描述的“为不同平台利用默认目标目录”子节中的相同原则:如果未为这种文件TYPE设置安装目录变量,CMake将提供一个内置的默认路径。为了可移植性,可以使用GNUInstallDirs模块。

表中的某些内置猜测带有安装目录变量的前缀:

\begin{itemize}
\item
\$OCALSTATE是CMAKE\_INSTALL\_LOCALSTATEDIR,默认为var。

\item
\$DATAROOT是CMAKE\_INSTALL\_DATAROOTDIR,默认为share。
\end{itemize}

与install(TARGETS)一样,GNUInstallDirs模块将为不同平台提供特定的安装目录变量。来看一个例子:

\filename{ch14/04-install-files/CMakeLists.txt}

\begin{cmake}
cmake_minimum_required(VERSION 3.26)
project(InstallFiles CXX)
include(GNUInstallDirs)
install(FILES
    src/include/calc/basic.h
    src/include/calc/nested/calc_extended.h
    DESTINATION ${CMAKE_INSTALL_INCLUDEDIR}/calc
)
\end{cmake}

,CMake将两个仅包含头文件的库(basic.h和nested/calc\_extended.h),安装到系统include目录中的项目特定子目录中。

\begin{myNotic}{Note}
从GNUInstallDirs源码中了解,CMAKE\_INSTALL\_INCLUDEDIR对所有支持平台的值相同。然而,仍然推荐使用,以提高可读性和与更动态变量的一致性。例如,CMAKE\_INSTALL\_LIBDIR根据架构和发行版而变化——lib、lib64或lib/。
\end{myNotic}

从CMake 3.20开始,可以使用RENAME关键字与install(FILES)和install(PROGRAMS)命令一起使用。这个关键字后面必须跟着一个新的文件名,并且只有在命令安装单个文件时才有效。

本节中的示例演示了将文件安装到适当目录的便利性。然而,存在一个问题——观察安装输出:

\begin{shell}
# cmake -S <source-tree> -B <build-tree>
# cmake --build <build-tree>
# cmake --install <build-tree>
-- Install configuration: ""
-- Installing: /usr/local/include/calc/basic.h
-- Installing: /usr/local/include/calc/calc_extended.h
\end{shell}

两个文件都安装到了同一个目录中,而不管它们的原始嵌套结构如何。有时,这并不理想。在下一节中,我们将探讨如何处理这种情况。

\mySamllsection{处理整个目录}

如果向安装命令添加单个文件不合适,可以处理整个目录。install(DIRECTORY)模式就是为了这个目的设计的,指定的目录原样复制到所选的目的地。

\begin{shell}
install(DIRECTORY dirs...
        TYPE <type> | DESTINATION <dir>
        [FILE_PERMISSIONS permissions...]
        [DIRECTORY_PERMISSIONS permissions...]
        [USE_SOURCE_PERMISSIONS] [OPTIONAL] [MESSAGE_NEVER]
        [CONFIGURATIONS [Debug|Release|...]]
        [COMPONENT <component>] [EXCLUDE_FROM_ALL]
        [FILES_MATCHING]
        [[PATTERN <pattern> | REGEX <regex>] [EXCLUDE]
        [PERMISSIONS permissions...]] [...]
)
\end{shell}

许多选项与install(FILES)和install(PROGRAMS)中的选项类似,并按相同方式工作。一个关键的细节是,如果DIRECTORY关键字后面的路径不以/结尾,路径的最后一个目录将附加到目的地。例如:

\begin{cmake}
install(DIRECTORY aaa DESTINATION /xxx)
\end{cmake}

这个命令会创建一个目录,/xxx/aaa,并将aaa的内容复制到其中。相比之下,以下命令会将aaa的内容直接复制到/xxx:

\begin{cmake}
install(DIRECTORY aaa/ DESTINATION /xxx)
\end{cmake}

install(DIRECTORY)还引入了其他在文件上不可用的机制:

\begin{itemize}
\item
静默输出

\item
扩展权限控制

\item
文件/目录过滤
\end{itemize}

从输出静默选项MESSAGE\_NEVER开始,禁用了安装过程中的输出诊断。当有大量文件需要安装,并且打印所有文件会过于嘈杂时,这个选项非常有用。

关于权限,install(DIRECTORY)支持三个选项:

\begin{itemize}
\item
USE\_SOURCE\_PERMISSIONS设置安装文件的原有文件权限,只有在FILE\_PERMISSIONS未设置时才有效。

\item
FILE\_PERMISSIONS允许指定为安装的文件和目录设置的权限,默认权限是OWNER\_WRITE、OWNER\_READ、GROUP\_READ和WORLD\_READ。

\item
DIRECTORY\_PERMISSIONS与FILE\_PERMISSIONS类似,但它将为所有用户设置EXECUTE权限(这是因为Unix-like系统中目录的EXECUTE权限表示列出其内容的权限)。
\end{itemize}

注意,CMake在那些不支持权限选项的平台上将忽略这些选项。通过在每个过滤表达式后添加PERMISSIONS关键字并指定一个权限列表,可以实现更细致的权限控制。匹配这些的文件或目录将获得指定的权限。

现在,让我们看看过滤器或“通配”表达式。控制源目录中哪些文件/目录将安装,并遵循以下语法:

\begin{shell}
PATTERN <pat> | REGEX <reg> [EXCLUDE] [PERMISSIONS <perm>]
\end{shell}

有两种匹配方法可供选择:

\begin{itemize}
\item
使用PATTERN,这是更简单的选项,可以提供一个带有?占位符(匹配任何字符)和*通配符(匹配任何字符串)的模式。只有以结尾的路径才会匹配。

\item
REGEX选项更高级,支持正则表达式。允许匹配路径的部分,尽管\^{}和\$锚点仍然可以表示路径的开始和结束。
\end{itemize}

可选地,第一个过滤器之前可以设置FILES\_MATCHING关键字,指定过滤器将应用于文件而不是目录。

请记住两个事项:

\begin{itemize}
\item
FILES\_MATCHING需要一个包容性过滤器。可以排除一些文件,但如果不同时包含一些文件,则不会有文件被复制。然而,所有目录都将创建,无论如何过滤。

\item
所有子目录默认都会包含,只能过滤掉。
\end{itemize}

对于每个过滤方法,可以选择使用EXCLUDE命令来排除匹配的路径(这仅在未使用FILES\_MATCHING时有效)。

为所有匹配的路径设置特定权限,可以通过在过滤器后添加PERMISSIONS关键字并指定一个权限列表来实现。通过一个例子来探索这一点,以不同的方式安装三个目录。有一些用于运行时使用的静态数据文件:

\begin{shell}
data
- data.csv
\end{shell}

还需要一些位于src目录中的公共头文件,其中还有其他不相关的文件:

\begin{shell}
src
- include
  - calc
    - basic.h
    - ignored
      - empty.file
    - nested
      - calc_extended.h
\end{shell}

最后,需要两个不同嵌套级别的配置文件。为了增加趣味性,/etc/calc/的内容只能被文件所有者访问:

\begin{shell}
etc
- calc
  - nested.conf
- sample.conf
\end{shell}

要安装包含静态数据文件的目录,从项目中最基本的install(DIRECTORY)命令开始:

\filename{ch14/05-install-directories/CMakeLists.txt (片段)}

\begin{cmake}
cmake_minimum_required(VERSION 3.26)
project(InstallDirectories CXX)
install(DIRECTORY data/ DESTINATION share/calc)
\end{cmake}

这个命令将简单地取我们数据目录的所有内容,并将其放入\$\{CMAKE\_INSTALL\_PREFIX\}和share/calc。我们的源路径以/符号结束,以表明不想复制数据目录本身,只想复制其内容。

第二个情况正好相反:不添加源路径的尾随/,因为目录应该包含。这是因为依赖于INCLUDE文件类型的系统特定路径,这是由GNUInstallDirs提供的(注意INCLUDE和EXCLUDE关键字代表不相关的概念):

\filename{ch14/05-install-directories/CMakeLists.txt (片段)}

\begin{cmake}
...
include(GNUInstallDirs)
install(DIRECTORY src/include/calc TYPE INCLUDE
    PATTERN "ignored" EXCLUDE
    PATTERN "calc_extended.h" EXCLUDE
)
\end{cmake}

此外,还从这次操作中排除了两个路径:整个ignored目录和所有以calc\_extended.h结尾的文件(记住PATTERN是如何工作的)。

第三个案例安装了一些默认配置文件并设置了其权限:

\filename{ch14/05-install-directories/CMakeLists.txt (片段)}

\begin{cmake}
install(DIRECTORY etc/ TYPE SYSCONF
    DIRECTORY_PERMISSIONS
        OWNER_READ OWNER_WRITE OWNER_EXECUTE
    PATTERN "nested.conf"
        PERMISSIONS OWNER_READ OWNER_WRITE
)
\end{cmake}

避免在SYSCONF路径中添加etc(GNUInstallDirs已经提供了这个路径)以避免重复。我们设置了两个权限规则:子目录只能由所有者编辑和列出,以nested.conf结尾的文件只能由所有者编辑。

安装目录覆盖了各种用例,但对于其他高级场景(如安装后配置),可能需要外部工具。我们如何集成它们?

\mySubsubsection{14.3.3.}{安装过程中调用脚本}

如果曾在类Unix系统上安装过共享库,可能会记得需要指导动态链接器扫描可信目录,并使用ldconfig构建其缓存(请参阅扩展阅读部分以获取参考文献)。为了实现完全自动化的安装,CMake提供了install(SCRIPT)和install(CODE)模式。这是完整的语法:

\begin{shell}
install([[SCRIPT <file>] [CODE <code>]]
        [ALL_COMPONENTS | COMPONENT <component>]
        [EXCLUDE_FROM_ALL] [...]
)
\end{shell}

在SCRIPT和CODE模式之间进行选择,并提供必要的参数——在安装过程中要运行的CMake脚本路径或要执行的CMake代码片段。为了说明,将修改02-install-targets示例以构建共享库:

\filename{ch14/06-install-code/src/CMakeLists.txt}

\begin{cmake}
add_library(calc SHARED basic.cpp)
target_include_directories(calc INTERFACE include)
target_sources(calc PUBLIC FILE_SET HEADERS
                           BASE_DIRS include
                           FILES include/calc/basic.h
)
\end{cmake}

安装脚本中将工件类型从ARCHIVE改为LIBRARY,并在之后添加运行ldconfig的逻辑:

\filename{ch14/06-install-code/CMakeLists.txt (片段)}

\begin{cmake}
install(TARGETS calc LIBRARY FILE_SET HEADERS))
if (UNIX)
    install(CODE "execute_process(COMMAND ldconfig)")
endif()
\end{cmake}

if()条件确保命令适用于操作系统(ldconfig不应在Windows或macOS上执行),提供的代码必须在CMake中语法正确(错误仅在安装过程中出现)。

运行安装命令后,通过打印缓存库来确认其成功:

\begin{shell}
# cmake -S <source-tree> -B <build-tree>
# cmake --build <build-tree>
# cmake --install <build-tree>
-- Install configuration: ""
-- Installing: /usr/local/lib/libcalc.so
-- Installing: /usr/local/include/calc/basic.h
# ldconfig -p | grep libcalc
        libcalc.so (libc6,x86-64) => /usr/local/lib/libcalc.so
\end{shell}

SCRIPT和CODE模式都支持生成器表达式,为这个命令增加了灵活性。可以用于各种目的:打印用户消息、验证成功安装、广泛配置、文件签名等。

接下来,让我们深入探讨CMake安装中的运行时依赖管理,这是CMake的最新特性之一。

\mySubsubsection{14.3.4.}{安装运行时依赖项}

已经涵盖了所有可安装工件及其相应的命令,最后一个要讨论的主题是运行时依赖。执行程序和共享库通常依赖于其他必须在系统中存在的库,并在程序初始化时动态加载。从CMake 3.21版本开始,CMake可以为每个目标构建这些所需库的列表,并通过引用二进制文件的适当部分在构建时捕获它们的位置。这个列表随后可以用来在系统中安装这些运行时工件,以供将来使用。

对于项目中的目标,这可以通过以下两个步骤实现:

\begin{shell}
install(TARGETS ... RUNTIME_DEPENDENCY_SET <set-name>)
install(RUNTIME_DEPENDENCY_SET <set-name> <arg>...)
\end{shell}

或者,通过一个具有相同效果的单一命令来完成:

\begin{shell}
install(TARGETS ... RUNTIME_DEPENDENCIES <arg>...)
\end{shell}

如果目标是通过导入而不是在项目中定义的,运行时依赖项可以按照以下方式安装:

\begin{shell}
install(IMPORTED_RUNTIME_ARTIFACTS <target>...)
\end{shell}

上述代码段可以通过添加RUNTIME\_DEPENDENCY\_SET <set-name> 参数来扩展,以创建一个可以稍后用于install(RUNTIME\_DEPENDENCY\_SET)命令的命名引用。

如果项目可以从这个功能中受益,我建议查看install()命令的官方文档以了解更多。

现在,了解了可以在系统上,以各种方式安装文件的所有不同方法,接下来来探讨如何将它们转换为其他CMake项目本地的可用包。










