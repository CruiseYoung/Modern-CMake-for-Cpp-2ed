Navigating the intricacies of writing cross-platform installation scripts can be daunting, but CMake significantly simplifies this task. Although it requires some initial setup, CMake streamlines the process, integrating seamlessly with the concepts and techniques we’ve explored throughout this book.

We began by understanding how to export CMake targets from projects, enabling their use in other projects without installation. This was followed by insights into installing projects that are already configured for export. Delving into installation basics, we focused on a crucial aspect: installing CMake targets. We now have a grasp of how CMake allocates different destinations for various artifact types and the special considerations for public headers. We also examined other modes of the install() command, encompassing the installation of files, programs, and directories, and executing scripts during installation.

Our journey then led us to CMake’s reusable packages. We explored how to make project targets relocatable, facilitating user-defined installation locations. This included creating fully defined packages consumable via find\_package(), entailing the preparation of target export files, config files, and version files. Acknowledging diverse user needs, we learned how to group artifacts and actions into installation components, distinguishing them from the components of CMake packages. Our exploration culminated in an introduction to CPack. We discovered how to prepare basic binary packages, offering an efficient method to distribute pre-compiled software. While mastering the nuances of installation and packaging in CMake is an ongoing journey, this chapter lays a robust foundation. It equips us to handle common scenarios and delve deeper with confidence.

In the next chapter, we’ll apply our accumulated knowledge by crafting a cohesive, professional project, showcasing practical applications of these CMake techniques.


















