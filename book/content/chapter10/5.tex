本章中,我们深入探讨了 C++20 模块,了解了它们与 CMake 模块的区别,并代表了 C++ 在简化编译和解决与冗余头文件编译,及处理预处理器宏相关挑战方面的进步。

我们通过一个简单的示例演示了如何编写和导入 C++20 模块。然后,探讨了如何为 C++20 模块设置 CMake。由于这个特性是实验性的,需要设置特定的变量,提供了一系列条件语句来确保项目能够正确配置正在使用的 CMake 版本。关于必要的工具,我们强调构建系统必须支持动态依赖,目前的选择是 Ninja 1.11 或更新版本。对于编译器支持,Clang 16 和 Visual Studio 2022 17.4 (19.34) 中的 MSVC 适合完全支持 C++20 模块,而 GCC 的支持仍在等待中。此外,还指导各位通过配置 CMake 来使用选定的工具链,包括选择构建系统生成器和设置编译器版本。配置和构建项目后,可以运行程序来查看 C++20 模块的实际应用。

下一章中,我们将了解自动化测试的重要性及其应用,以及 CMake 对测试框架的支持。