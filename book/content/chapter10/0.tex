C++20 引入了一个新特性:模块,可以用模块文件替代了头文件中的纯文本符号声明,该模块文件将预编译为二进制格式,减少了构建时间。

我们将讨论 CMake 中 C++20 模块的内容,从 C++20 模块作为一个概念开始:相对于标准头文件的优点,以及如何简化源码单元的管理。尽管简化构建过程令人兴奋,但本章强调了其采纳的道路既困难又漫长。

理论部分结束后,我们将继续讨论在项目中实现模块的实际方面:将讨论在早期版本的 CMake 中启用它们的实验性支持,以及在 CMake 3.28 中的完整发布。

通过 C++20 模块的旅程不仅仅是为了理解一个新特性 —— 是关于重新思考在大型 C++ 项目中组件如何交互。本章结束时,不仅会了解模块的理论知识,还能通过示例获得实践经验,增强利用该特性实现更好的项目成果。

本章中,包含以下内容:

\begin{itemize}
\item
C++20 模块是什么?

\item
使用 C++20 模块支持的编写项目

\item
配置工具链
\end{itemize}














