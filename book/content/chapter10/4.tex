
根据Kitware网站上的博客文章(见“进一步阅读”部分),CMake最早在3.25版本就支持模块功能。尽管3.28版本正式支持此功能,但这并不是我们要享受模块便利性的唯一拼图。

下一个要求集中在构建系统上:它需要支持动态依赖。截至目前,你只有以下两个选择:

\begin{itemize}
\item
Ninja 1.11及更新版本(Ninja和Ninja Multi-Config)

\item
Visual Studio 17 2022及更新版本
\end{itemize}

同样,你的编译器需要以特定格式生成映射源依赖的文件,以供CMake使用。这种格式在Kitware开发者撰写的一篇论文中有描述,这篇论文被称为p1589r5。该论文已提交给所有主流编译器以供实施。目前,只有以下三种编译器实现了所需的格式:

\begin{itemize}
\item
Clang 16

\item
Visual Studio 2022 17.4 (19.34)中的MSVC

\item
GCC 14(针对开发分支,2023年9月20日之后)及更新版本
\end{itemize}

假设你的环境中有所有必要的工具(你可以使用我们为本书提供的Docker镜像),并且你的CMake项目已准备好构建,剩下的就是配置CMake以使用所需的工具链。你可能还记得,从第一章中,你可以使用-G命令行参数选择构建系统生成器:

\begin{shell}
cmake -B <build tree> -S <source tree> -G "Ninja"
\end{shell}

此命令将配置项目以使用Ninja构建系统。下一步是设置编译器。如果你的默认编译器不支持模块,并且你已安装了另一个编译器来尝试,你可以通过定义全局变量CMAKE\_CXX\_COMPILER来实现,如下所示:

\begin{shell}
cmake -B <build tree> -S <source tree> -G "Ninja" -D CMAKE_CXX_ COMPILER=clang++-18
\end{shell}

我们在这个例子中选择Clang 18,因为它是撰写本文时(包含在Docker镜像中)可用的最新版本。成功配置后(你可能会看到一些关于实验性功能的警告),你需要构建项目:

\begin{shell}
cmake --build <build tree>
\end{shell}

和往常一样,确保用适当的路径替换占位符<build tree>和<source tree>。如果一切顺利,你可以运行你的程序,观察模块功能按预期工作:

\begin{shell}
$ ./main
Addition 2 + 2 = 4
\end{shell}

就这样,C++20模块在实际中得以应用。

\begin{myNotic}{Note}
“进一步阅读”部分包括来自Kitware的博客文章和关于C++编译器的源依赖格式的提案,提供了更多关于C++20模块的实施和使用的深入见解。
\end{myNotic}
























