This chapter provides an in-depth look at using IDEs to optimize the programming process, particularly focusing on IDEs that deeply integrate with CMake. It offers a comprehensive guide for both beginners and experienced professionals, detailing the benefits of using an IDE and how to select one that best fits individual or organizational needs.

We started with a discussion on the importance of IDEs in enhancing development speed and code quality, explaining what an IDE is and how it simplifies the various steps involved in software development by integrating tools like code editors, compilers, and debuggers. This was followed by a short reminder about toolchains, where we explained the necessity of their installation if they aren’t present in the system, and we presented a short list of the most common choices.

We discussed how to start with CLion and its unique features, and we took an advanced look at its debugging capabilities. VS Code, a free, cross-platform IDE by Microsoft, is recognized for its vast extension ecosystem and support for numerous programming languages. We guided you through the initial setup and its key extension installations, and we introduced an advanced feature called Dev Containers. The VS IDE, exclusive to Windows, provides a refined, feature-rich environment tailored to various user needs. The setup process, key features, and the advanced Hot Reload debugging feature were also covered.

Each IDE section provided insights into why you might choose a particular IDE, the steps to get started, and a look at an advanced feature that sets the IDE apart. We also emphasized the concept of remote development support, highlighting its growing importance in the industry.

In summary, this chapter serves as a foundational guide for programmers seeking to understand and choose an IDE, offering a clear overview of the top options, their unique benefits, and how to effectively use them in conjunction with CMake to enhance coding efficiency and project management. In the next chapter, we’ll learn the basics of project setup using CMake.



