C++源代码的编译看起来是一个相当直接的过程。我们从经典的Hello World示例开始。

以下代码位于ch01/01-hello/hello.cpp,这是C++语言中的Hello World:

\begin{cpp}
#include <iostream>
int main() {
    std::cout << "Hello World!" << std::endl;
    return 0;
}
\end{cpp}

要产生一个可执行文件,我们当然需要一个C++编译器。CMake不包含编译器,所以你需要自己选择并安装一个。流行的选择包括:

\begin{itemize}
\item
Microsoft Visual C++编译器

\item
GNU编译器集合

\item
Clang/LLVM
\end{itemize}
大多数读者都熟悉编译器,因为它是学习C++时不可或缺的,所以我们不会深入讨论选择和安装。本书中的示例将使用GNU GCC,因为它是一个成熟的开源软件编译器,可以在许多平台上免费获得。

假设我们已经安装了编译器,运行它对于大多数供应商和系统来说都是类似的。我们应该以文件名作为参数调用它:

\begin{shell}
$ g++ hello.cpp -o hello
\end{shell}

我们的代码是正确的,所以编译器会默默地产生一个可执行的二进制文件,我们的机器可以理解。我们可以通过调用它的名字来运行它:

\begin{shell}
$ ./hello
Hello World!
\end{shell}

运行一个命令来构建你的程序很简单;然而,随着我们的项目增长,你很快就会明白,将所有内容放在一个文件中是不可能的。干净的代码实践建议,源代码文件应该保持小而有序。手动编译每个文件可能会变得乏味和脆弱。必须有更好的方法。

\mySubsubsection{1.2.1}{什么是CMake?}

假设我们通过编写一个脚本来自动化构建,该脚本遍历我们的项目树并编译所有内容。为了避免不必要的编译,我们的脚本将检测源代码自上次运行脚本以来是否已修改。现在,我们希望有一种方便的方式来管理传递给编译器的每个文件的参数——最好是基于可配置的标准。此外,我们的脚本应该知道如何将所有编译文件链接成一个单一的二进制文件,甚至更好地构建整个解决方案,这些解决方案可以作为更大的项目中的模块重复使用和集成。

软件构建是一个非常多样化的过程,可以涵盖多个不同的方面:

\begin{itemize}
\item
编译可执行文件和库

\item
管理依赖关系

\item
测试

\item
安装

\item
打包

\item
生成文档

\item
再测试
\end{itemize}

创建一个真正模块化和强大的C++构建工具,以适应各种目的,需要很长时间。Kitware的Bill Hoffman在20多年前实现了CMake的第一个版本。正如你可能已经猜到的,它非常成功。今天,它具有许多功能和广泛的社区支持。CMake正在积极开发中,并已成为C和C++程序员行业标准。

自动化构建代码的问题比CMake还要古老,因此自然而然地,有很多选择:GNU Make、Autotools、SCons、Ninja、Premake等。但是CMake为什么能胜出?

CMake有几件事我觉得很重要:

\begin{itemize}
\item
它专注于支持现代编译器和工具链。

\item
CMake真正跨平台——它支持为Windows、Linux、macOS和Cygwin构建。

\item
它为流行的IDE生成项目文件:Microsoft Visual Studio、Xcode和Eclipse CDT。此外,它是其他项目的项目模型,如CLion。

\item
CMake在正确的抽象层次上操作——它允许你将文件分组为可重用的目标和项目。

\item
有大量使用CMake构建的项目,提供了一种简单的方式将它们集成到你的项目中。

\item
CMake将测试、打包和安装视为构建过程的固有部分。

\item
旧的、不常用的功能会被废弃,以保持CMake精简。
\end{itemize}

CMake提供了一致、流畅的体验。无论是使用IDE还是直接从命令行构建软件,最重要的是它也照顾到构建后的阶段。

你的CI/CD管道可以轻松使用相同的CMake配置和构建项目,即使所有先前环境都不同,也可以使用单一标准。

\mySubsubsection{1.2.2}{它是如何工作的?}

你可能认为CMake是一个工具,在一端读取源代码,在另一端生成二进制文件——尽管这在原则上是对的,但这并不是全部。

CMake不能独立构建任何东西——它依赖于系统中的其他工具来执行实际的编译、链接等任务。你可以把它看作是构建过程的指挥家:它知道需要完成哪些步骤,最终目标是什么,以及如何找到合适的工人和材料。

这个过程有三个阶段:

\begin{itemize}
\item
配置

\item
生成

\item
构建
\end{itemize}

让我们详细探讨它们。

\mySamllsection{配置阶段}

这个阶段是关于读取存储在目录中的项目详情,称为源代码树,并为生成阶段准备一个输出目录或构建树。

CMake首先检查项目是否以前配置过,并从CMakeCache.txt文件中读取缓存的配置变量。首次运行时,情况并非如此,所以它会创建一个空的构建树,并收集它正在工作的环境的所有详细信息:例如,架构是什么,可用的编译器是什么,以及是否安装了链接器和打包器。此外,它还会检查一个简单的测试程序是否可以正确编译。

接下来,CMakeLists.txt项目配置文件被解析并执行(是的,CMake项目使用CMake的编码语言进行配置)。这个文件是CMake项目的最小配置(源文件可以在稍后添加)。它告诉CMake关于项目结构、其目标和依赖项(库和其他CMake包)。

在这个过程中,CMake将收集的信息存储在构建树中,例如系统详情、项目配置、日志和临时文件,这些信息用于下一阶段。具体来说,CMakeCache.txt文件用于存储更稳定的信息(例如编译器和工具的路径),这在整个构建序列再次执行时可以节省时间。

\mySamllsection{生成阶段}

在阅读项目配置后,CMake将为它正在工作的确切环境生成一个构建系统。构建系统只是为其他构建工具(例如,GNU Make的Makefiles或Ninja以及Visual Studio的IDE项目文件)定制的配置文件。在这个阶段,CMake仍然可以通过评估生成器表达式对构建配置进行一些最后的调整。

\begin{myTip}{Tip}
生成阶段在配置阶段之后自动执行。因此,本书和其他资源有时会交替使用这两个阶段,当提到“配置”或“生成”构建系统时。要明确只运行配置阶段,你可以使用cmake-gui工具。
\end{myTip}

\mySamllsection{构建阶段}

为了生成项目中指定的最终产品(如可执行文件和库),CMake必须运行适当的构建工具。这可以通过直接调用、通过IDE或使用适当的CMake命令来执行。反过来,这些构建工具将执行一系列步骤,使用编译器、链接器、静态和动态分析工具、测试框架、报告工具以及你能想到的任何其他工具来生成目标产品。

这种解决方案的美丽之处在于能够根据需要为每个平台生成构建系统(也就是说,使用相同的项目文件):

\myGraphic{0.9}{content/chapter1/images/1.png}{图1.1:CMake的各个阶段}

你还记得我们在“理解基础知识”部分中的hello.cpp应用程序吗?使用CMake构建它真的非常简单。我们只需要在源文件所在的同一目录下创建以下CMakeLists.txt文件。

\filename{ch01/01-hello/CMakeLists.txt}

\begin{cmake}
cmake_minimum_required(VERSION 3.26)
project(Hello)
add_executable(Hello hello.cpp)
\end{cmake}

创建此文件后,在该目录下执行以下命令:

\begin{shell}
cmake -B <build tree>
cmake --build <build tree>
\end{shell}

请注意,是一个占位符,应该用一个临时目录的路径替换,该目录将包含生成的文件。

以下是在Ubuntu系统上运行的Docker(Docker是一个可以在其他系统上运行的虚拟机;我们将在“在不同平台上安装CMake”部分讨论它)的输出。

第一个命令生成构建系统:

\begin{shell}
~/examples/ch01/01-hello# cmake -B ~/build_tree
-- The C compiler identification is GNU 11.3.0
-- The CXX compiler identification is GNU 11.3.0
-- Detecting C compiler ABI info
-- Detecting C compiler ABI info - done
-- Check for working C compiler: /usr/bin/cc - skipped
-- Detecting C compile features
-- Detecting C compile features - done
-- Detecting CXX compiler ABI info
-- Detecting CXX compiler ABI info - done
-- Check for working CXX compiler: /usr/bin/c++ - skipped
-- Detecting CXX compile features
-- Detecting CXX compile features - done
-- Configuring done (1.0s)
-- Generating done (0.1s)
-- Build files have been written to: /root/build_tree
\end{shell}

第二个命令实际上构建项目:

\begin{shell}
~/examples/ch01/01-hello# cmake --build ~/build_tree
Scanning dependencies of target Hello
[ 50%] Building CXX object CMakeFiles/Hello.dir/hello.cpp.o
[100%] Linking CXX executable Hello
[100%] Built target Hello
\end{shell}

最后,运行编译后的程序:

\begin{shell}
~/examples/ch01/01-hello# ~/build_tree/Hello
Hello World!
\end{shell}

在这里,我们在构建树目录中生成了一个构建系统。接着,我们执行了构建阶段,并产生了一个我们可以运行的最终二进制文件。

现在你知道结果是什么样子了,我敢肯定你会有很多问题:这个过程的先决条件是什么?这些命令是什么意思?为什么我们需要两个?我如何编写自己的项目文件?别担心——这些问题将在接下来的部分中得到解答。

\begin{myNotic}{Note}
这本书将为您提供与当前版本的CMake(撰写时为3.26)相关的最重要信息。为了为您提供最佳建议,我明确避免了任何废弃和不推荐使用的功能,并强烈建议至少使用CMake 3.15版本,这被认为是现代的CMake。如果您需要更多信息,您可以在网上找到最新的完整文档,网址为 \url{https://cmake.org/cmake/help/}。
\end{myNotic}
















