CMake 项目由大量的文件和目录构成。来大致了解一下每个组成部分的作用,以便进行调整。

这些文件和目录可以分为几个类别:

\begin{itemize}
\item
项目文件,作为开发者会随着项目的成长修改这些文件。

\item
CMake 生成的文件,包含 CMake 语言命令,但并不供开发者编辑。何手动更改都将被覆盖。

\item
高级用户(即:非项目开发者)使用的文件,以便根据个人需求自定义 CMake 构建项目的方式。

\item
一些临时文件,在特定上下文中提供有价值的信息。
\end{itemize}

本节还会建议哪些文件加入到版本控制系统 (VCS) 的忽略文件中。

\mySubsubsection{1.5.1}{源代码目录}

这是项目的存放目录(也称为项目根目录),包含了所有的 C++ 源文件和 CMake 项目文件。

以下是关于此目录的一些信息:

\begin{itemize}
\item
需要一个 CMakeLists.txt 配置文件。

\item
此目录的路径通过用户在生成构建系统时,传递给 cmake 命令的 -S 参数。

\item
避免在 CMake 代码中,硬编码源码目录的绝对路径——软件用户将会在另一个路径中存储项目。
\end{itemize}

在这个目录中初始化一个仓库是个好主意,可以使用 Git 这样的版本控制系统。

\mySubsubsection{1.5.2}{构建目录}

CMake 根据用户的指定,在此目录中创建构建系统,以及构建过程中产生的一切:项目制品、临时配置、缓存、构建日志和原生构建工具(如 GNU Make)的输出。这个目录的其他名称,包括构建根目录和二进制目录。

需要记住的是:

\begin{itemize}
\item
构建配置(构建系统)和构建制品(如二进制文件、可执行文件和库,以及用于最终链接的对象文件和归档文件)将在此处创建。

\item
CMake 推荐将此目录置于源代码目录之外(即外源构建),可以避免污染项目(即内源构建)。

\item
通过 -B 参数传递给 cmake 命令来指定。

\item
这个目录不是生成文件的最终目的地。相反,建议项目包含一个安装阶段,将最终制品复制到系统中的适当位置,并移除所有用于构建的临时文件。
\end{itemize}

不要将此目录加入到版本控制系统 (VCS) 中——每个用户都会为自己选择一个。如果充分的理由进行内源构建,请确保将此目录加入到版本控制系统的忽略文件中(如 .gitignore)。

\mySubsubsection{1.5.3}{列表文件}

包含 CMake 语言的文件称为列表文件 (listfiles),可以通过调用 include() 和 find\_package() 或间接地通过 add\_subdirectory() 来相互包含。CMake 对这些文件的命名没有强制规定,但按照惯例通常使用 .cmake 扩展名。

\mySamllsection{项目文件}

CMake 项目通过一个 CMakeLists.txt 列表文件进行配置(注意,由于历史原因,该文件具有非常规的扩展名)。此文件位于每个项目的源代码树的顶部,并且是在配置阶段首先执行。

顶层的 CMakeLists.txt 文件应至少包含以下两个命令:

\begin{itemize}
\item
cmake\_minimum\_required(VERSION <x.xx>): 设置期望的 CMake 版本,并告诉 CMake 如何处理遗留行为。

\item
project(<name> <OPTIONS>): 命名项目(提供的名称将存储在 PROJECT\_NAME 变量中),并指定配置项目的选项。
\end{itemize}

随着软件的增长,可能希望将其划分为较小的单元,这些单元可以单独配置和管理。CMake 支持通过子目录及其各自的 CMakeLists.txt 文件来实现这一点。

项目结构可能如下所示:

\begin{shell}
myProject/CMakeLists.txt
myProject/api/CMakeLists.txt
myProject/api/api.h
myProject/api/api.cpp
\end{shell}

一个简单的顶层 CMakeLists.txt 文件可以用来整合整个项目:

\begin{cmake}
cmake_minimum_required(VERSION 3.26)
project(app)
message("Top level CMakeLists.txt")
add_subdirectory(api)
\end{cmake}

项目的主要方面都在顶级文件中覆盖:管理依赖项、声明要求和检测环境。我们还使用 add\_subdirectory(api) 命令来包含来自 api 子目录的另一个 CMakeLists.txt 文件,以执行与应用的 API 部分相关的特定步骤。

\mySamllsection{缓存文件}

从列表文件中生成的缓存变量,将在首次运行配置阶段时被存储在 CMakeCache.txt 文件中。此文件位于构建树的根目录,并具有相当简单的格式(为简洁起见,省略了部分内容):

\begin{shell}
# This is the CMakeCache file.
# For build in directory: /root/build tree
# It was generated by CMake: /usr/local/bin/cmake
# The syntax for the file is as follows:
# KEY:TYPE=VALUE
# KEY is the name of a variable in the cache.
# TYPE is a hint to GUIs for the type of VALUE, DO NOT EDIT
  #TYPE!.
# VALUE is the current value for the KEY.
########################
# EXTERNAL cache entries
########################

# Flags used by the CXX compiler during DEBUG builds.
CMAKE_CXX_FLAGS_DEBUG:STRING=/MDd /Zi /Ob0 /Od /RTC1

# ... more variables here ...
########################
# INTERNAL cache entries
########################

# Minor version of cmake used to create the current loaded cache
CMAKE_CACHE_MINOR_VERSION:INTERNAL=19
# ... more variables here ...
\end{shell}

从注释中可以看出,EXTERNAL 部分的缓存条目是供用户修改的,而 INTERNAL 部分则由 CMake 管理。

以下是几个需要牢记的点:

\begin{itemize}
\item
可以手动管理此文件,也可以通过调用 cmake,或通过 ccmake 或 cmake-gui 来管理。

\item
通过删除此文件可以将项目重置为其默认配置;文件将从列表文件中重新生成。
\end{itemize}

可以从列表文件读取和写入缓存变量,变量引用的计算会有些复杂。

\mySamllsection{包定义文件}

CMake 生态系统的一个大部分是项目可以依赖的外部包,以无缝、跨平台的方式提供库和工具。希望提供 CMake 支持的包作者会将 CMake 包配置文件一起发布。

以下是几个注意的细节:

\begin{itemize}
\item
配置文件(原始拼写)包含有关如何使用库的二进制文件、头文件和辅助工具的信息。有时,会暴露在项目中使用的 CMake 宏和函数。

\item
配置文件命名为 <PackageName>-config.cmake 或 <PackageName>Config.cmake.

\item
使用 find\_package() 命令来包含包。
\end{itemize}

需要特定版本的包,CMake 会将此与相关的<PackageName>-config-version.cmake或<PackageName>ConfigVersion.cmake。

如果供应商没有为包提供配置文件,配置会与 CMake 本身捆绑,或者可以在项目中通过 Find-模块 提供。

\mySamllsection{生成的文件}

在生成阶段,许多文件由 cmake 可执行文件在构建树中生成。CMake 将它们用作 cmake 安装动作、CTest 和 CPack 的配置。

可能会看到:

\begin{itemize}
\item
cmake\_install.cmake

\item
CTestTestfile.cmake

\item
CPackConfig.cmake
\end{itemize}

如果正在实现源内构建,将它们添加到版本控制系统的忽略文件中吧。

\mySubsubsection{1.5.4}{JSON 和 YAML 文件}

CMake 使用的其他格式包括 JavaScript 对象表示法(JSON)和另一种标记语言(YAML)。这些文件作为与外部工具(如 IDE)通信的接口,或者提供易于生成和解析的配置。

\mySamllsection{预设文件}

当需要具体指定缓存变量、选择的生成器、构建树路径等时,项目的高级配置可能会成为一个相对繁忙的任务——尤其是有不止一种方式来构建项目时。这时预设就派上用场了——不需要通过命令行手动配置这些值,只需提供一个存储所有细节的文件,并与项目一起发布。自从 CMake 3.25 版本以来,预设还允许配置工作流程,将阶段(配置、构建、测试和打包)绑定到一个命名步骤列表中执行。

用户可以通过 GUI 选择预设,或者使用命令 -{}-list-presets 并使用 -{}-preset=<preset> 选项为构建系统选择预设。

预设存储在两个文件中:

\begin{itemize}
\item
CMakePresets.json: 这是供项目作者提供官方预设的文件。

\item
CMakeUserPresets.json: 这是专供想要根据个人喜好自定义项目配置的用户使用的文件(可以将其添加到版本控制系统忽略文件中)。
\end{itemize}

预设不是项目必需的,只在高级场景中变得有用。

\mySamllsection{基于文件的 API}

CMake 3.14 引入了一个 API,允许外部工具查询构建系统信息:生成文件的路径、缓存条目、工具链等。这里,只提到这个非常高级的话题,以避免在文档中遇到“基于文件的 API ”这个术语时产生混淆。这个名字表明了它的工作方式:一个包含查询的 JSON 文件,必须放置在构建树内的一个特殊路径中。CMake 在生成构建系统时,会读取这个文件,并将响应写入另一个文件,以便外部应用程序解析。

基于文件的 API 是为了替换一种在 CMake 3.26 版本中废弃的机制,称为服务器模式(或 cmake-server)。

\mySamllsection{配置日志}

自从 3.26 版本以来,CMake 将为配置阶段的深度调试提供一个结构化的日志文件,位置在:

\begin{shell}
<build tree>/CMakeFiles/CMakeConfigureLog.yaml
\end{shell}

这是一个不需要特别注意的功能。

\mySubsubsection{1.5.5}{Git 中忽略文件}

有许多版本控制系统;其中最受欢迎的是 Git。每当我们开始一个新项目时,确保只将必要的文件添加到仓库是很好的做法。如果指定了不需要的文件在 .gitignore 文件中,项目卫生更容易维护。例如,可能会排除生成的、特定于用户的或临时的文件。

Git 在形成新提交时会自动跳过。以下是我项目中使用的文件:

\filename{ch01/01-hello/.gitignore}

\begin{shell}
CMakeUserPresets.json
# If in-source builds are used, exclude their output like so:
build_debug/
build_release/

# Generated and user files
**/CMakeCache.txt
**/CMakeUserPresets.json
**/CTestTestfile.cmake
**/CPackConfig.cmake
**/cmake_install.cmake
**/install_manifest.txt
**/compile_commands.json
\end{shell}

现在掌握了一张项目文件的导航图。有些文件非常重要,将经常使用它们——其他的则不那么重要。虽然学习它们可能看起来像是浪费时间,但知道在哪里找不到答案可能非常有价值。无论如何,本章的最后一个问题是:还可以用 CMake 创建哪些自包含的单元?





























