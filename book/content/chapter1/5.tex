CMake 项目由大量的文件和目录构成。让我们大致了解一下每个组成部分的作用,以便开始对它们进行调整。这些文件和目录可以分为几个主要类别:

\begin{itemize}
\item
当然,我们将有项目文件,作为开发者的我们,会随着项目的成长准备和修改这些文件。

\item
会有 CMake 为自己生成的文件,尽管它们将包含 CMake 语言命令,但它们并不是供开发者编辑的。在那里所做的任何手动更改都将被 CMake 覆盖。

\item
有些文件是为高级用户(即:非项目开发者)准备的,以便他们根据个人需求自定义 CMake 构建项目的方式。

\item
有些文件是为高级用户(即:非项目开发者)准备的,以便他们根据个人需求自定义 CMake 构建项目的方式。
\end{itemize}

本节还将建议您可以将哪些文件加入到版本控制系统 (VCS) 的忽略文件中。


\mySubsubsection{1.5.1}{源代码目录}

这是项目的存放目录(也称为项目根目录)。它包含了所有的 C++ 源文件和 CMake 项目文件。

以下是关于此目录的一些重要点:

\begin{itemize}
\item
需要一个 CMakeLists.txt 配置文件。

\item
此目录的路径通过用户在生成构建系统时传递给 cmake 命令的 -S 参数给出。

\item
避免在 CMake 代码中硬编码源代码目录的绝对路径——您的软件用户将会在另一个路径中存储项目。
\end{itemize}

在这个目录中初始化一个仓库是个好主意,可以使用 Git 这样的版本控制系统。

\mySubsubsection{1.5.2}{构建目录}

CMake 根据用户的指定在此目录中创建构建系统以及构建过程中产生的一切:项目的制品、临时配置、缓存、构建日志和原生构建工具(如 GNU Make)的输出。这个目录的其他名称包括构建根目录和二进制目录。

需要记住的关键点:

\begin{itemize}
\item
构建配置(构建系统)和构建制品(如二进制文件、可执行文件和库,以及用于最终链接的对象文件和归档文件)将在此处创建。

\item
CMake 推荐将此目录置于源代码目录之外(即外源构建)。这样可以防止污染项目(即内源构建)。

\item
它通过 -B 参数传递给 cmake 命令来指定。

\item
这个目录不是生成文件的最终目的地。相反,建议您的项目包含一个安装阶段,将最终制品复制到系统中的适当位置,并移除所有用于构建的临时文件。
\end{itemize}

不要将此目录加入到您的版本控制系统 (VCS) 中——每个用户都会为自己选择一个。如果您有充分的理由进行内源构建,请确保将此目录加入到版本控制系统的忽略文件中(如 .gitignore)。

\mySubsubsection{1.5.3}{列表文件}

包含 CMake 语言的文件被称为列表文件 (listfiles),可以通过调用 include() 和 find\_package() 或间接地通过 add\_subdirectory() 来相互包含。CMake 对这些文件的命名没有强制规定,但按照惯例,它们通常带有 .cmake 扩展名。

\mySamllsection{项目文件}

CMake 项目通过一个 CMakeLists.txt 列表文件进行配置(请注意,由于历史原因,该文件具有非常规的扩展名)。此文件位于每个项目的源代码树的顶部,并且是在配置阶段首先被执行的文件。

顶级的 CMakeLists.txt 文件应至少包含以下两个命令:

\begin{itemize}
\item
cmake\_minimum\_required(VERSION <x.xx>): 设置期望的 CMake 版本,并告诉 CMake 如何处理遗留行为。

\item
project(<name> <OPTIONS>): 命名项目(提供的名称将存储在 PROJECT\_NAME 变量中),并指定配置项目的选项(更多细节请参见第 2 章,《CMake 语言》)。
\end{itemize}

随着软件的增长,您可能希望将其划分为较小的单元,这些单元可以单独配置和管理。CMake 支持通过子目录及其各自的 CMakeLists.txt 文件来实现这一点。您的项目结构可能如下所示:

\begin{shell}
myProject/CMakeLists.txt
myProject/api/CMakeLists.txt
myProject/api/api.h
myProject/api/api.cpp
\end{shell}

一个非常简单的顶级 CMakeLists.txt 文件可以用来整合整个项目:

\begin{cmake}
cmake_minimum_required(VERSION 3.26)
project(app)
message("Top level CMakeLists.txt")
add_subdirectory(api)
\end{cmake}

项目的主要方面都在顶级文件中覆盖:管理依赖项、声明要求和检测环境。我们还使用 add\_subdirectory(api) 命令来包含来自 api 子目录的另一个 CMakeLists.txt 文件,以执行与应用的 API 部分相关的特定步骤。

\mySamllsection{缓存文件}

从列表文件中生成的缓存变量将在首次运行配置阶段时被存储在 CMakeCache.txt 文件中。此文件位于构建树的根目录,并具有相当简单的格式(为简洁起见,省略了一些行):

\begin{shell}
# This is the CMakeCache file.
# For build in directory: /root/build tree
# It was generated by CMake: /usr/local/bin/cmake
# The syntax for the file is as follows:
# KEY:TYPE=VALUE
# KEY is the name of a variable in the cache.
# TYPE is a hint to GUIs for the type of VALUE, DO NOT EDIT
  #TYPE!.
# VALUE is the current value for the KEY.
########################
# EXTERNAL cache entries
########################

# Flags used by the CXX compiler during DEBUG builds.
CMAKE_CXX_FLAGS_DEBUG:STRING=/MDd /Zi /Ob0 /Od /RTC1

# ... more variables here ...
########################
# INTERNAL cache entries
########################

# Minor version of cmake used to create the current loaded cache
CMAKE_CACHE_MINOR_VERSION:INTERNAL=19
# ... more variables here ...
\end{shell}

从头部注释中可以看出,这种格式相当不言自明。EXTERNAL 部分的缓存条目是供用户修改的,而 INTERNAL 部分则由 CMake 管理。

以下是几个需要牢记的关键点:

\begin{itemize}
\item
您可以手动管理此文件,也可以通过调用 cmake(参见本章“掌握命令行”部分的缓存选项),或通过 ccmake 或 cmake-gui 来管理。

\item
通过删除此文件可以将项目重置为其默认配置;文件将从列表文件中重新生成。
\end{itemize}

可以从列表文件读取和写入缓存变量。有时,变量引用的评估会有些复杂;我们将在第 2 章《CMake 语言》中更详细地介绍这一点。

\mySamllsection{包定义文件}

CMake 生态系统的一个大部分是项目可以依赖的外部包。它们以无缝、跨平台的方式提供库和工具。希望提供 CMake 支持的包作者会将 CMake 包配置文件一起发布。

我们将在第 14 章《安装和打包》中学习如何编写这些文件。同时,以下是几个需要记住的有趣细节:

\begin{itemize}
\item
配置文件(原始拼写)包含有关如何使用库的二进制文件、头文件和辅助工具的信息。有时,它们会暴露可以在您的项目中使用的 CMake 宏和函数。

\item
配置文件命名为 <PackageName>-config.cmake 或 <PackageName>Config.cmake.

\item
使用 find\_package() 命令来包含包。
\end{itemize}

如果需要特定版本的包,CMake 会将此与相关的<PackageName>-config-version.cmake或<PackageName>ConfigVersion.cmake。

如果供应商没有为包提供配置文件,有时,配置会与 CMake 本身捆绑,或者可以在项目中通过 Find-module(原始拼写)提供。

\mySamllsection{生成的文件}

在生成阶段,许多文件由 cmake 可执行文件在构建树中生成。因此,它们不应手动编辑。CMake 将它们用作 cmake 安装动作、CTest 和 CPack 的配置。

您可能会遇到的文件有:

\begin{itemize}
\item
cmake\_install.cmake

\item
CTestTestfile.cmake

\item
CPackConfig.cmake
\end{itemize}

如果您正在实现源内构建,将它们添加到版本控制系统的忽略文件中可能是一个好主意。

\mySubsubsection{1.5.4}{JSON 和 YAML 文件}

CMake 使用的其他格式包括 JavaScript 对象表示法(JSON)和另一种标记语言(YAML)。这些文件被引入作为与外部工具(如 IDEs)通信的接口,或者提供易于生成和解析的配置。

\mySamllsection{预设文件}

当我们需要具体指定缓存变量、选择的生成器、构建树路径等等时,项目的高级配置可能会成为一个相对繁忙的任务——尤其是当我们有不止一种方式来构建项目时。这时预设就派上用场了——我们不需要通过命令行手动配置这些值,只需提供一个存储所有细节的文件,并与项目一起发布。自从 CMake 3.25 版本以来,预设还允许我们配置工作流程,将阶段(配置、构建、测试和打包)绑定到一个命名步骤列表中执行。

正如本章“掌握命令行”部分所提到的,用户可以通过 GUI 选择预设,或者使用命令 -{}-list-presets 并使用 -{}-preset=<preset> 选项为构建系统选择预设。

预设存储在两个文件中:

\begin{itemize}
\item
CMakePresets.json: 这是供项目作者提供官方预设的文件。

\item
CMakeUserPresets.json:这是专供想要根据个人喜好自定义项目配置的用户使用的文件(您可以将其添加到您的版本控制系统忽略文件中)。
\end{itemize}

预设不是项目必需的,只在高级场景中变得有用。详情请参见第 16 章《编写 CMake 预设》。

\mySamllsection{基于文件的 API}

CMake 3.14 引入了一个 API,允许外部工具查询构建系统信息:生成文件的路径、缓存条目、工具链等等。我们只提到这个非常高级的话题,以避免如果您在文档中遇到基于文件的 API 这个术语时产生混淆。这个名字表明了它的工作方式:一个包含查询的 JSON 文件必须放置在构建树内的一个特殊路径中。CMake 在生成构建系统时会读取这个文件,并将响应写入另一个文件,以便外部应用程序可以解析。

基于文件的 API 是为了替换一种在 CMake 3.26 版本中被废弃的机制,称为服务器模式(或 cmake-server)。

\mySamllsection{配置日志}

自从 3.26 版本以来,CMake 将为配置阶段的深度调试提供一个结构化的日志文件,位置在:

\begin{shell}
<build tree>/CMakeFiles/CMakeConfigureLog.yaml
\end{shell}

这是一个通常不需要特别注意的功能——直到你需要它。

\mySubsubsection{1.5.5}{在 Git 中忽略文件}

有许多版本控制系统;其中最受欢迎的是 Git。每当我们开始一个新项目时,确保只将必要的文件添加到仓库是很好的做法。如果我们指定了不需要的文件在 .gitignore 文件中,项目卫生更容易维护。例如,我们可能会排除生成的、特定于用户的或临时的文件。

Git 在形成新提交时会自动跳过它们。以下是我项目中使用的文件:

\filename{ch01/01-hello/.gitignore}

\begin{shell}
CMakeUserPresets.json
# If in-source builds are used, exclude their output like so:
build_debug/
build_release/

# Generated and user files
**/CMakeCache.txt
**/CMakeUserPresets.json
**/CTestTestfile.cmake
**/CPackConfig.cmake
**/cmake_install.cmake
**/install_manifest.txt
**/compile_commands.json
\end{shell}

现在您掌握了一张项目文件的大海图。有些文件非常重要,您将经常使用它们——其他的则不那么重要。虽然学习它们可能看起来像是浪费时间,但知道在哪里找不到答案可能非常有价值。无论如何,本章的最后一个问题是:您还可以用 CMake 创建哪些其他自包含的单元?





























