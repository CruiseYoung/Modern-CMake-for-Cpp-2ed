
软件开发有一种神奇之处。我们不仅仅是在创造一个能够付诸生命的运行机制,我们通常还在创造解决方案功能背后的想法。

为了将我们的想法变为现实,我们按照以下循环工作:设计、编码和测试。我们发明变化,我们用编译器能理解的语言表达它们,我们检查它们是否如预期那样工作。要从我们的源代码中创建正确、高质量的软件,我们需要小心翼翼地执行重复的、容易出错的任务:调用正确的命令、检查语法、链接二进制文件、运行测试、报告问题等等。

每次都记住每个步骤需要付出巨大的努力。相反,我们希望专注于实际的编码,将其他所有事情委托给自动化工具。理想情况下,这个过程会在我们更改代码后,通过一个按钮启动。它应该是智能的、快速的、可扩展的,并且在不同操作系统和环境中的工作方式相同。它应该得到多个集成开发环境(IDEs)的支持。更进一步,我们可以将这个过程简化为持续集成(CI)管道,每次向共享仓库提交更改时,都会构建和测试我们的软件。

CMake是满足许多此类需求的答案;然而,要正确配置和使用它需要一些工作。CMake并不是复杂性的来源;复杂性来自于我们这里要处理的主题。别担心,我们将非常系统地学习整个过程。在你意识到之前,你将成为一个软件构建的大师。

我知道你急于开始编写自己的CMake项目,这正是我们将在本书大部分内容中做的事情。但是,由于你将主要为了用户(包括你自己)创建项目,因此了解他们的视角对你来说很重要。

所以,让我们从这一点开始:成为一个CMake高级用户。我们将介绍一些基础知识:这个工具是什么,它的工作原理以及如何安装它。然后,我们将深入探讨命令行和操作模式。最后,我们将总结项目文件的不同用途,并解释如何在完全不创建项目的情况下使用CMake。

本章中,我们将包括以下主题:

\begin{itemize}
\item
理解基础知识

\item
在不同平台上安装CMake

\item
掌握命令行

\item
导航项目文件

\item
发现脚本和模块
\end{itemize}
























