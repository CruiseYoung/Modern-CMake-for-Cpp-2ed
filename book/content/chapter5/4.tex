理解目标,是编写清晰、现代的CMake项目的关键。本章中,不仅讨论了什么是目标,以及如何定义三种不同类型的目标:可执行文件、库和自定义目标。还解释了目标如何通过依赖关系图相互依赖,并了解了如何使用Graphviz模块可视化。有了这种基本的理解,就能够了解目标的关键特性——属性。我们不仅介绍了一些在目标上设置常规属性的命令,还解决了传播属性,也称为“目标传递的使用要求”谜团。

这是一个难以攻克的问题,因为需要理解的不仅是如何控制传播哪些属性,还有这种传播如何影响后续的目标。此外,还发现了如何确保从多源属性的兼容性。

然后,简要讨论了伪目标:导入的目标、别名目标和接口库。所有这些在以后的项目中都会派上用场,特别是当了解如何将它们与传播属性连接起来。接着,讨论了生成的构建目标,以及对配置阶段的影响。之后,花了一些时间研究一种与目标相似,但又不完全是目标的机制:自定义命令。提到了如何生成其他目标(编译、翻译等)使用的文件,以及其钩子功能:在构建目标时执行的步骤。

有了如此坚实的基础,就可以进行下一个主题了——将C++源代码编译成可执行文件和库。