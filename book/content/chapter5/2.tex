如果曾经使用过GNU Make,一定已经接触过目标(target)这个概念了。本质上,它是一个构建系统遵循的方式,用于将一组文件编译成另一个文件。可以是一个.cpp实现文件编译成一个.o对象文件,或者是一组.o文件打包成一个.a静态库。在构建系统中,目标有着众多的组合和可能性。

然而,CMake在更高的抽象层次上工作,它理解大多数语言如何将源文件构建为可执行文件,可以节省时间并跳过定义这些目标的中间步骤。所以,不需要像使用GNU Make那样编写显式的命令来编译C++对象文件。只需要一个add\_executable()命令,其第一个参数是可执行目标的名字,后续的参数是所有源文件构成的列表:

\begin{cmake}
add_executable(app1 a.cpp b.cpp c.cpp)
\end{cmake}

前面的章节中使用过这个命令,并且已经了解了在实践中,如何使用可执行目标——在生成步骤中,CMake将创建一个构建系统,并用适当的配方填充它,以编译每个源文件,并将它们链接成一个单一的二进制可执行文件。

CMake中,可以使用以下三个命令创建目标:

\begin{itemize}
\item
add\_executable()

\item
add\_library()

\item
add\_custom\_target()
\end{itemize}

构建可执行文件或库之前,CMake会进行一次检查,以确定生成过程中使用的源文件是否有更新,这种机制帮助CMake只重新构建发生变动的工件。通过比较时间戳,CMake有效地识别出哪些目标需要重新构建,从而减少了不必要的编译。

所有创建目标的命令都需要将目标名称作为第一个参数提供,可以在与目标进行交互的其他命令中引用该目标名称,如target\_link\_libraries(),target\_sources()或target\_include\_directories(),稍后会介绍这些命令。

现在,先来看看可以定义哪些目标。

\mySubsubsection{5.2.1.}{定义可执行目标}

定义可执行目标的命令add\_executable()(在前面的章节中已经使用过):

\begin{shell}
add_executable(<name> [WIN32] [MACOSX_BUNDLE]
                [EXCLUDE_FROM_ALL]
                [source1] [source2 ...])
\end{shell}

如果为Windows编译,通过添加可选参数WIN32关键字,将生成一个不会显示默认控制台窗口(通常在这里看到输出流到std::cout)的可执行文件。相反,应用程序将生成自己的GUI。

下一个可选参数MACOSX\_BUNDLE在某种程度上类似;它使得为macOS/iOS生成的应用程序可以在Finder中打开,作为GUI应用程序启动。

当使用EXCLUDE\_FROM\_ALL关键字时,该目标不会在常规默认构建中构建。这样的目标只有使用下面的命令明确指出要构建时,才会执行构建:

\begin{shell}
cmake --build -t <target>
\end{shell}

最后,需要提供编译目标的源文件列表。支持以下的扩展名:

\begin{itemize}
\item
对于C语言: c, m

\item
对于C++语言: C, M, c++, cc, cpp, cxx, m, mm, mpp, CPP, ixx, cppm, ccm, cxxm, c++m
\end{itemize}

我们没有将头文件添加到源文件列表中。头文件有两种方法指定:使用以头文件所在路径做为参数的target\_include\_directories()命令,或者使用target\_sources()命令的FILE\_SET功能(在CMake 3.23中添加)。这对于可执行文件来说是一个重要的话题,但由于它与目标正交且复杂,我们将在第7章再详细讨论。

\mySubsubsection{5.2.2.}{定义库目标}

定义库与定义可执行文件非常相似,但不需要定义GUI方面的关键字。以下是该命令的签名:

\begin{shell}
add_library(<name> [STATIC | SHARED | MODULE]
            [EXCLUDE_FROM_ALL]
            [<source>...])
\end{shell}

关于名称、排除所有和源的规则与可执行目标完全一致。唯一的区别在于STATIC、SHARED和MODULE关键字。如果有使用库的经验,就知道这些定义了CMake将生成哪种工件:静态链接库、共享(动态库)或模块。这又是一个相当广泛的主题,我们将在第8章中深入讨论。

\mySubsubsection{5.2.3.}{自定义目标}

自定义目标与可执行文件或库略有不同,通过执行明确给出的命令行来扩展构建功能,可用于:

\begin{itemize}
\item
计算其他二进制文件的校验和。

\item
运行sanitizer执行代码扫描并收集结果。

\item
将编译报告发送到指标通道。
\end{itemize}

从这张列表中可以猜到,自定义目标只在相当高级的项目中有用,所以我们只了解基础知识,然后继续讨论更重要的主题。

要定义一个自定义目标,请使用以下语法(为了简洁,省略了一些选项):

\begin{shell}
add_custom_target(Name [ALL] [COMMAND command2 [args2...] ...])
\end{shell}

自定义目标有一些需要考虑的缺点。由于涉及shell命令,可能是系统特定的,会限制可移植性。此外,自定义目标可能没有为CMake提供直接的方式,来确定生成的具体工件或副产品(如果有的话)。

自定义目标也不像可执行文件和库会进行陈旧性检查(不验证源文件是否比二进制文件更新),因为默认情况下它们不会添加到依赖关系图中(自定义目标在默认构建中不会生成,可以指定ALL关键字,使得每次构建都执行生成自定义目标,ALL与EXCLUDE\_FROM\_ALL具有完全相反的定义)。

来看看这个依赖关系图什么。

\mySubsubsection{5.2.4.}{依赖关系图}

成熟的应用程序通常由许多组件构建而成,特别是内部库。从结构的角度来看,划分项目是有用的。当相关的事物包装在一个逻辑实体中时,就可以与其他目标链接:另一个库或一个可执行文件。这对于多个目标使用同一个库的情况尤其方便。图5.1描述了一个示例依赖关系图:

\myGraphic{0.5}{content/chapter5/images/1.png}{图5.1:BankApp项目中构建依赖的顺序}

这个项目有两个库,两个可执行文件和一个自定义目标。用例是提供一个带有GUI的银行应用程序(GuiApp),以及一个作为自动化脚本一部分使用的命令行版本(TerminalApp)。两个可执行文件都依赖于相同的Calculations库,但只有一个需要Drawing库。为了确保应用程序二进制文件的完整性,我们还将计算一个校验和,并通过单独的安全渠道分发它。CMake在编写此类解决方案的列表文件时非常灵活:

\filename{ch05/01-targets/CMakeLists.txt}

\begin{cmake}
cmake_minimum_required(VERSION 3.26)
project(BankApp CXX)

add_executable(terminal_app terminal_app.cpp)
add_executable(gui_app gui_app.cpp)
target_link_libraries(terminal_app calculations)
target_link_libraries(gui_app calculations drawing)

add_library(calculations calculations.cpp)
add_library(drawing drawing.cpp)

add_custom_target(checksum ALL
    COMMAND sh -c "cksum terminal_app>terminal.ck"
    COMMAND sh -c "cksum gui_app>gui.ck"
    BYPRODUCTS terminal.ck gui.ck
    COMMENT "Checking the sums..."
)
\end{cmake}

通过使用target\_link\_libraries()命令将库与可执行文件链接。若不连接,会因未定义的符号,构建可执行文件失败。这里,发现在声明库之前就调用了这个命令吗?当CMake配置项目时,会收集关于目标和属性信息——名称、依赖关系、源文件和其他信息。

解析了所有文件之后,CMake将尝试构建一个依赖关系图。像所有有效的依赖关系图一样,它们是有向无环图(DAG)。所以有一个清晰的哪个目标依赖于哪个目标,并且这样的依赖关系不会形成循环。

当在构建模式下执行cmake时,生成的构建系统将检查定义的顶层目标,并递归地构建依赖关系。来看一下啊图5.1中的例子:

\begin{enumerate}
\item
从顶层开始,构建组1中的两个库。

\item
当Calculations和Drawing库完成后,构建组2 – GuiApp和TerminalApp。

\item
构建校验和目标;运行指定的命令行以生成校验和(cksum是一个Unix校验和工具,这个例子不会在其他平台上构建)。
\end{enumerate}

然而,这里有一个小问题——前面的解决方案不能保证校验和目标会在可执行文件之后构建。CMake不知道校验和依赖于可执行二进制文件的存在,所以它可以自由地首先开始构建它。为了解决这个问题,我们可以在文件末尾放置add\_dependencies()命令:

\begin{cmake}
add_dependencies(checksum terminal_app gui_app)
\end{cmake}

这将确保CMake理解校验和目标与可执行文件之间的关系。

那很好,但target\_link\_libraries()和add\_dependencies()之间有什么区别呢?target\_link\_libraries()旨在与实际库一起使用,并允许你控制属性传播。第二个命令仅用于顶级目标,以设置它们的构建顺序。

随着项目复杂性的增加,依赖树变得难以理解。要如何简化这个过程呢?

\mySubsubsection{5.2.5.}{可视化依赖关系}

即使是小项目,也可能难以推理并与其他开发者分享。一个整洁的图表将大有帮助。毕竟,一图胜千言。我们可以自己动手绘制图表,就像图5.1那样。但这很繁琐,并且每次项目更改时都需要更新。幸运的是,CMake有一个很棒的模块,可以生成dot/graphviz格式的依赖关系图,支持内部和外部依赖!

要使用它,可以简单地执行以下命令:

\begin{shell}
cmake --graphviz=test.dot .
\end{shell}

该模块将生成一个文本文件,可以将其导入到Graphviz可视化软件中,该软件可以渲染图像或生成PDF或SVG文件,可以作为软件文档的一部分存储。每个人都喜欢优秀的文档,但几乎没有人喜欢创建它——现在,不需要了!

自定义目标默认情况下是不可见的,需要创建一个特殊的配置文件,CMakeGraphVizOptions.cmake,就可以自定义图表了。使用set(GRAPHVIZ\_CUSTOM\_TARGETS TRUE)命令在图表中启用自定义目标:

\filename{ch05/01-targets/CMakeGraphVizOptions.cmake}

\begin{cmake}
set(GRAPHVIZ_CUSTOM_TARGETS TRUE)
\end{cmake}

其他选项允许添加图表名称、标题、节点前缀,并配置应包含或排除在输出中的目标(按名称或类型)。访问官方CMake文档以获取CMakeGraphVizOptions模块的完整描述。

甚至可以直接在浏览器中运行Graphviz,地址是:\url{https://dreampuf.github.io/GraphvizOnline/}。

所需要做的就是将test.dot文件的内容复制粘贴到左边的窗口中,项目就会可视化(如图5.2)。非常方便,不是吗?

\myGraphic{0.9}{content/chapter5/images/2.png}{图5.2:在Graphviz中可视化BankApp示例}

使用这种方法,可以快速看到所有明确定义的目标。

现在理解了目标的概念,我们知道如何定义不同类型的目标,包括可执行文件、库和自定义目标,以及如何创建依赖关系图并打印它。让我们使用这些信息进行更深入的研究,看看如何配置它们。

\mySubsubsection{5.2.6.}{设置目标的属性}

目标具有类似于C++对象字段的属性。其中一些属性为了修改而设计的,而有些是只读的。CMake定义了大量“已知属性”(请参阅扩展阅读部分),这些属性取决于目标的类型(可执行文件、库或自定义),也可以添加自己的属性。使用以下命令来操作目标的属性:

\begin{shell}
get_target_property(<var> <target> <property-name>)
set_target_properties(<target1> <target2> ...
                      PROPERTIES <prop1-name> <value1>
                      <prop2-name> <value2> ...)
\end{shell}

要在屏幕上输出目标属性,首先需要将其存储在<var>变量中,然后将其消息传递给用户。读取属性必须逐个进行;设置目标属性允许我们同时为多个目标指定多个属性。

\begin{myNotic}{Note}
属性的概念不仅适用于目标;CMake支持为其他范围设置属性:GLOBAL、DIRECTORY、SOURCE、INSTALL、TEST和CACHE。为了操作所有类型的属性,有通用的get\_property()和set\_property()命令。在某些项目中,会看到这些低层命令用来精确地完成set\_target\_properties()命令所做的事情:

\begin{shell}
set_property(TARGET <target> PROPERTY <name> <value>)
\end{shell}
\end{myNotic}

通常,尽可能多地使用高级命令是更好的选择。有时,CMake提供了简写命令。例如,add\_dependencies( )是向MANUALLY\_ADDED\_DEPENDENCIES目标属性追加依赖关系的简写。这种情况下,可以使用get\_target\_property()精确地查询,就像其他属性一样。然而,不能使用set\_target\_properties()来更改MANUALLY\_ADDED\_DEPENDENCIES(是只读的),因为CMake坚持使用add\_dependencies()命令来限制操作仅为追加。

在接下来的章节中讨论编译和链接时,我们将介绍更多设置属性的命令。

现在,让我们专注于一个目标的属性如何传递给另一个目标。

\mySamllsection{传递目标的使用要求}

命名是困难的,有时会得到一个难以理解的标签。不幸的是,在线CMake文档中,会遇到“传递使用要求”这样一个令人费解的标题。让我们解开这个奇怪的名字,并提供一个更容易理解的术语。

从中间的术语开始:使用。正如我们之前讨论的,一个目标可能依赖于另一个。CMake文档有时将这种依赖称为“使用”,即一个目标使用另一个目标。

在某些情况下,使用的目标会为自己设置特定的属性或依赖项,这反过来构成了使用它的其他目标的依赖:链接某些库、包含目录或要求特定的编译器特性。

解谜的最后一部分,“传递”这个词描述的行为正确(也许可以更简单一些)。CMake将一些使用目标的属性/要求添加到使用目标的属性中。

所以,一些属性可以隐式地在目标之间传递(或简单地传播),因此更容易表达依赖关系。

简化这个整个概念,就像源目标(被使用的目标)和目标目标(使用其他目标的目标)之间的传播属性。

来看一个具体的例子,来理解其为什么存在,以及是如何工作的:

\begin{shell}
target_compile_definitions(<source> <INTERFACE|PUBLIC|PRIVATE> [items1...])
\end{shell}

这个目标命令将填充一个<source>目标的COMPILE\_DEFINITIONS属性。编译定义就是传递给编译器的-Dname=definition标志,用于配置C++预处理器定义(将在第7章中讨论)。这里有趣的部分是第二个参数,需要指定三个值中的一个,INTERFACE、PUBLIC或PRIVATE,以控制属性应该传递给哪个目标。现在,不要将这些与C++访问修饰符混淆——这是一个全新的概念。

传播关键字的工作方式如下:

\begin{itemize}
\item
PRIVATE设置源目标的属性。

\item
INTERFACE设置使用目标的目标属性。

\item
PUBLIC设置源目标和使用目标的目标属性。
\end{itemize}

当属性不应传递给其他目标时,将其设置为PRIVATE。当需要这样的传递时,选择PUBLIC。如果处于一个源目标在其实现(.cpp文件)中就不使用该属性,而在头文件中使用,并且这些属性传递给使用目标的目标,则应使用INTERFACE关键字。

这是如何工作的?为了管理这些属性,CMake提供了一些命令,例如之前提到的target\_compile\_definitions()。当指定PRIVATE或PUBLIC关键字时,CMake将在目标的属性中存储提供的值,COMPILE\_DEFINITIONS。此外,关键字是INTERFACE或PUBLIC,将在具有INTERFACE\_前缀的属性中存储值——INTERFACE\_COMPILE\_DEFINITIONS。配置阶段,CMake将读取源目标的接口属性,并将其内容附加到目标目标。就这样传播属性,或CMake所说的传递目标的使用要求。

使用set\_target\_properties()命令管理的属性可以在\url{https://cmake.org/cmake/help/latest/manual/cmake-properties.7.html}找到,在目标属性部分(并非所有目标属性都可传递)。这里是最重要的几个:

\begin{itemize}
\item
COMPILE\_DEFINITIONS

\item
COMPILE\_FEATURES

\item
COMPILE\_OPTIONS

\item
INCLUDE\_DIRECTORIES

\item
LINK\_DEPENDS

\item
LINK\_DIRECTORIES

\item
LINK\_LIBRARIES

\item
LINK\_OPTIONS

\item
POSITION\_INDEPENDENT\_CODE

\item
PRECOMPILE\_HEADERS

\item
SOURCES
\end{itemize}

我们将在接下来的页面中讨论大多数这些选项,但所有这些选项当然都在CMake手册中有详细描述。在以下链接中找到它们的详细描述:\url{https://cmake.org/cmake/help/latest/prop_tgt/.html}

接下来的问题是,如何远传播这些属性。属性只设置在第一个目标上,还是发送到依赖图的最顶层?这可以决定。

为了在目标之间创建依赖关系,使用target\_link\_libraries()命令。这个命令需要一个传播关键字:

\begin{shell}
target_link_libraries(<target>
                    <PRIVATE|PUBLIC|INTERFACE> <item>...
                    [<PRIVATE|PUBLIC|INTERFACE> <item>...]...)
\end{shell}

这个签名也指定了传播关键字,控制源目标的属性如何在目标目标中存储。图5.3展示了生成阶段(配置阶段完成后)传播属性会发生什么情况:

\myGraphic{0.9}{content/chapter5/images/3.png}{图5.3:属性是如何在目标间进行传递的}

传播关键字的工作方式:

\begin{itemize}
\item
PRIVATE将源值添加到源目标的私有属性。

\item
INTERFACE将源值添加到源目标的接口属性。

\item
PUBLIC将值添加到源目标的两个属性。
\end{itemize}

INTERFACE属性仅用于将属性进一步传播到链中的下一个目标,而源目标在其构建过程中不会使用。

我们之前使用的基本target\_link\_libraries(…)命令隐式指定了PUBLIC关键字。

如果正确地为源目标设置了传播关键字,属性将自动放置在目标目标上——除非有冲突…

\mySamllsection{处理冲突的传播属性}

当一个目标依赖于多个其他目标时,可能存在传播属性之间直接冲突的情况。例如,一个使用的目标将POSITION\_INDEPENDENT\_CODE属性设置为true,而另一个设置为false。CMake将这种冲突理解为错误,并打印出类似于这样的错误信息:

\begin{shell}
CMake Error: The INTERFACE_POSITION_INDEPENDENT_CODE property of "source_ target" does not agree with the value of POSITION_INDEPENDENT_CODE already determined for "destination_target".
\end{shell}

开发者需要明确知道这种冲突,并且需要去解决它。CMake有一些自己的属性,这些属性必须在源目标和目标目标之间“一致”。

有时,这可能很重要——例如,在多个目标中使用相同的库,然后将它们链接到一个可执行文件。如果这些源目标使用的是同一库的不同版本,可能会遇到问题。

为了确保只使用特定版本的库,可以创建一个自定义接口属性,INTERFACE\_LIB\_VERSION,并在其中存储版本。这还不够解决问题,因为CMake默认不会传播自定义属性(这个机制只适用于内置目标属性),必须明确地将自定义属性添加到“兼容”属性列表中。

每个目标都有四个这样的列表:

\begin{itemize}
\item
COMPATIBLE\_INTERFACE\_BOOL

\item
COMPATIBLE\_INTERFACE\_STRING

\item
COMPATIBLE\_INTERFACE\_NUMBER\_MAX

\item
COMPATIBLE\_INTERFACE\_NUMBER\_MIN
\end{itemize}

将属性添加到它们中的任何一个,都会触发传播和兼容性检查。BOOL列表将检查所有传递到目标目标的属性是否评估为相同的布尔值。类似地,STRING将评估为字符串。NUMBER\_MAX和NUMBER\_MIN略有不同——传递的值不必匹配,但目标目标将只接收最高或最低值。

这个例子将帮助我们了解,如何在实践中对其进行应用:

\filename{ch05/02-propagated/CMakeLists.txt}

\begin{cmake}
cmake_minimum_required(VERSION 3.26)
project(PropagatedProperties CXX)

add_library(source1 empty.cpp)
set_property(TARGET source1 PROPERTY INTERFACE_LIB_VERSION 4)
set_property(TARGET source1 APPEND PROPERTY
             COMPATIBLE_INTERFACE_STRING LIB_VERSION)

add_library(source2 empty.cpp)
set_property(TARGET source2 PROPERTY INTERFACE_LIB_VERSION 4)

add_library(destination empty.cpp)

target_link_libraries(destination source1 source2)
\end{cmake}

这里,创建了三个目标;为了简化,所有目标都使用相同的空源文件。在两个源目标上,指定了带有INTERFACE\_前缀的自定义属性,并将其设置为相同的匹配库版本。两个源目标都链接到相应的目标。最后,我们指定了字符串兼容性作为source1的属性(这里没有添加INTERFACE\_前缀,因为使用target\_link\_libraries链接库文件时,会将类似INTERFACE\_LIB\_VERSION的属性值追加到destination的LIB\_VERSION属性)。

CMake将这个自定义属性传播到相应目标,并检查所有源目标的版本是否完全匹配(兼容性属性只需在目标目标上设置一次)。

已经了解了常规目标是什么,那就来看看其他像目标、闻起来像目标,有时甚至表现得像目标的东西,但最终发现它们并不是真正的目标。

\mySubsubsection{5.2.7.}{识别伪目标}

目标包含的一些概念非常有用,CMake也将这些特性应用在了如外部依赖项、别名等伪目标上。创建目标会在构建系统中生成对应的工件,如可执行文件、库文件等。但是创建的伪目标在构建系统中不会生成具体的工件,它可以用来作为生成目标时的依赖或输入:

\begin{itemize}
\item
导入的目标

\item
别名目标

\item
接口库
\end{itemize}

\mySamllsection{导入的目标}

如果浏览了本书的目录,会了解将讨论CMake如何管理外部依赖项——其他项目、库等。IMPORTED目标是这个过程的产物。CMake可以定义它们,作为find\_package()命令的结果。

可以调整这类目标的属性:编译定义、编译选项、包含目录等——甚至支持目标传递的使用要求。然而,应该将它们视为不可变的目标;不要更改它们的源文件或依赖关系。

IMPORTED目标的定义范围可以是全局的,也可以是定义它们的目录的本地范围(在子目录中可见,但在父目录中不可见)。

\mySamllsection{别名目标}

别名目标的确切作用就是你所期望的——为目标创建另一个不同的名称引用。可以为可执行文件和库创建别名目标:

\begin{shell}
add_executable(<name> ALIAS <target>)
add_library(<name> ALIAS <target>)
\end{shell}

别名目标的属性只读,不能安装或导出别名(在生成的构建系统中不可见)。

为什么还要有别名呢?有时,它们非常有用,比如项目的一部分(如子目录)需要以特定名称引用一个目标,而实际的实现可能因情况而异。例如,希望根据用户的选择构建解决方案中的库或导入它。

\mySamllsection{接口库}

这是一个有趣的构造——一个不编译东西的库,而是作为一个实用目标。整个概念都围绕传播属性(传递使用要求)构建。

接口库主要有两个用途——代表仅包含头文件的库,以及将一堆传播属性打包成一个逻辑单元。

仅包含头文件的库可以用add\_library(INTERFACE)轻松创建:

\begin{cmake}
add_library(Eigen INTERFACE
    src/eigen.h src/vector.h src/matrix.h
)
target_include_directories(Eigen INTERFACE
    $<BUILD_INTERFACE:${CMAKE_CURRENT_SOURCE_DIR}/src>
    $<INSTALL_INTERFACE:include/Eigen>
)
\end{cmake}

在前面的代码片段中,创建了一个包含三个头文件的Eigen接口库。接下来,使用生成器表达式(这些以美元符号和尖括号表示,\$<…>,将在下一章解释),将包含目录设置为当目标导出时为\$\{CMAKE\_CURRENT\_SOURCE\_DIR\}/src,安装时为include/Eigen(这将在本章的末尾解释)。

要使用这样的库,只需链接它即可:

\begin{cmake}
target_link_libraries(executable Eigen)
\end{cmake}

这里不会发生实际的链接,但CMake会将这个命令理解为,向可执行目标传播所有INTERFACE属性的请求。

第二种使用场景利用了相同的机制,但出于不同的目的——创建了一个逻辑目标,可以作为一个传播属性的占位符。然后,可以将这个目标用作其他目标的依赖,并以干净、方便的方式设置属性。这是一个例子:

\begin{cmake}
add_library(warning_properties INTERFACE)
target_compile_options(warning_properties INTERFACE
    -Wall -Wextra -Wpedantic
)
target_link_libraries(executable warning_properties)
\end{cmake}

add\_library(INTERFACE)命令创建了一个逻辑的warning\_properties目标,用于在第二个命令中为可执行目标设置编译选项。我建议使用这些INTERFACE目标,它们可以提高代码的可读性和可重用性,这是将一堆魔法值重构为命名良好的变量的过程。我还建议明确地为接口库添加一个后缀,如\_properties,以便轻松区分接口库和常规库。

\mySubsubsection{5.2.8.}{对象库}

对象库用于将多个源文件,组合成一个单一的逻辑目标,并在构建过程中将它们编译成(.o)对象文件。要创建一个对象库,遵循与创建其他库相同的方法,但使用OBJECT关键字:

\begin{shell}
add_library(<target> OBJECT <sources>)
\end{shell}

构建过程中产生的对象文件可以作为其他目标的编译元素,使用\$<TARGET\_OBJECTS:objlib>生成器表达式:

\begin{shell}
add_library(... $<TARGET_OBJECTS:objlib> ...)
add_executable(... $<TARGET_OBJECTS:objlib> ...)
\end{shell}

或者,可以使用target\_link\_libraries()命令将它们作为依赖项进行添加。

在Calc库的上下文中,对象库将非常有用,以避免为库的静态和共享版本编译库源的冗余。对于共享库,明确地编译具有POSITION\_INDEPENDENT\_CODE启用的对象文件是必要的。

回到项目的目标:calc\_obj将提供编译后的对象文件,然后将用于calc\_static和calc\_shared库。让我们探索这两种类型库之间的实际区别,并理解为什么可能需要创建两者。

伪目标是否穷尽了目标的概念?当然不是!我们仍然需要了解,这些目标是如何用于生成构建系统的。

\mySubsubsection{5.2.9.}{构建目标}

项目的上下文和生成的构建系统中,“目标”一词可以有不同的含义。在生成构建系统的上下文中,CMake将CMake语言编写的列表文件“编译”成所选构建工具的语言,例如为GNU Make创建Makefile。这些生成的Makefile有自己的目标集。其中一些目标是列表文件中定义的目标的直接转换,而其他目标则在构建系统生成过程中隐式创建。

CMake默认为顶级列表文件的所有目标执行构建,如可执行文件和库(自定义目标根据标志选择是否构建),这种目标构建策略为ALL。当运行cmake -{}-build <build tree>时,ALL将构建所有顶级列表文件目标。第一章中,可以通过添加-{}-target <name>参数到cmake构建命令中来选择一个目标。

一些可执行文件或库可能不需要在每次构建中都存在,但我们希望将它们作为项目的一部分,以便在不常需要的场合中使用。为了优化默认构建,可以将它们从ALL目标中排除:

\begin{shell}
add_executable(<name> EXCLUDE_FROM_ALL [<source>...])
add_library(<name> EXCLUDE_FROM_ALL [<source>...])
\end{shell}

自定义目标则相反——默认情况下,它们排除在ALL目标之外,除非明确地使用ALL关键字添加它们,就像我们在BankApp示例中所做的那样。

另一个隐式定义的构建目标是clean,简单地从构建树中删除产生的工件。我们使用它来清除所有旧文件并从头开始构建。重要的是要理解,clean并不会简单地删除构建目录中的所有内容。为了正确工作,需要手动指定自定义目标可能创建的文件作为BYPRODUCTS(请参阅BankApp示例)。

我们对目标和它们不同方面的探索之旅结束了!了解了如何创建它们,配置它们的属性,使用伪目标,并决定它们是否应该默认构建。还有一个有趣的非目标机制,可以创建在所有实际目标中使用的自定义工件——自定义命令(不要与自定义目标混淆)。

