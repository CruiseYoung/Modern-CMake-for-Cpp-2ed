如果你曾经使用过GNU Make,那你已经接触过目标(target)的概念了。本质上,它是一个构建系统遵循的配方,用于将一组文件编译成另一个文件。这可以是一个.cpp实现文件编译成一个.o对象文件,或者是一组.o文件打包成一个.a静态库。在构建系统中,目标和它们的转换有着众多的组合和可能性。

然而,CMake允许你节省时间并跳过定义这些配方的中间步骤;它在更高的抽象层次上工作。它理解大多数语言如何直接从源文件构建可执行文件。所以,你不需要像使用GNU Make那样编写显式的命令来编译你的C++对象文件。所需的一切只是一个add\_executable()命令,后面跟着可执行目标的名字和源文件列表:

\begin{cmake}
add_executable(app1 a.cpp b.cpp c.cpp)
\end{cmake}

我们在前面的章节中使用过这个命令,并且已经知道了在实践中如何使用可执行目标——在生成步骤中,CMake将创建一个构建系统,并用适当的配方填充它,以编译每个源文件并将它们链接成一个单一的二进制可执行文件。

在CMake中,我们可以使用以下三个命令创建一个目标:

\begin{itemize}
\item
add\_executable()

\item
add\_library()

\item
add\_custom\_target()
\end{itemize}

在构建可执行文件或库之前,CMake会进行一次检查,以确定生成的输出是否比源文件旧。这种机制帮助CMake避免重新创建已经是最新的工件。通过比较时间戳,CMake有效地识别出哪些目标需要重新构建,从而减少了不必要的重新编译。

所有定义目标的命令都需要将目标名称作为第一个参数提供,这样就可以在其他命令中引用它,这些命令与目标进行交互,如target\_link\_libraries(),target\_sources()或target\_include\_directories()。我们稍后会学习这些命令,但现在,让我们更仔细地看看我们可以定义哪种目标。

\mySubsubsection{5.2.1.}{定义可执行目标}

定义可执行目标的命令add\_executable()是不言自明的(我们在前面的章节中已经依赖这个事实并使用过它)。其正式结构如下:

\begin{shell}
add_executable(<name> [WIN32] [MACOSX_BUNDLE]
                [EXCLUDE_FROM_ALL]
                [source1] [source2 ...])
\end{shell}

如果我们是为Windows编译,通过添加可选参数WIN32关键字,我们将生成一个不会显示默认控制台窗口(我们通常在这里看到输出流到std::cout)的可执行文件。相反,应用程序将预期生成自己的GUI。

下一个可选参数MACOSX\_BUNDLE在某种程度上类似;它使得为macOS/iOS生成的应用程序可以从Finder作为GUI应用程序启动。

当使用EXCLUDE\_FROM\_ALL关键字时,将防止在常规默认构建中构建可执行目标。这样的目标必须在构建命令中明确提及:

\begin{shell}
cmake --build -t <target>
\end{shell}

最后,我们需要提供将编译成目标的源文件列表。以下扩展名是支持的:

\begin{itemize}
\item
For C: c, m

\item
For C++: C, M, c++, cc, cpp, cxx, m, mm, mpp, CPP, ixx, cppm, ccm, cxxm, c++m
\end{itemize}

请注意,我们没有将任何头文件添加到源文件列表中。这可以通过提供包含这些文件的目录路径和target\_include\_directories()命令隐式完成,或者使用target\_sources()命令的FILE\_SET功能(在CMake 3.23中添加)。这对于可执行文件来说是一个重要的话题,但由于它与目标正交且复杂,我们将在第7章“使用CMake编译C++源文件”中详细讨论。

\mySubsubsection{5.2.2.}{定义库目标}

定义库与定义可执行文件非常相似,但当然,它不需要定义GUI方面的关键字。以下是该命令的签名:

\begin{shell}
add_library(<name> [STATIC | SHARED | MODULE]
            [EXCLUDE_FROM_ALL]
            [<source>...])
\end{shell}

关于名称、排除所有和源的规则与可执行目标完全一致。唯一的区别在于STATIC、SHARED和MODULE关键字。如果你有任何使用库的经验,你会知道这些定义了CMake将生成哪种工件:静态链接库、共享(动态库)或模块。这又是一个相当广泛的主题,我们将在第8章“链接可执行文件和库”中深入讨论。

\mySubsubsection{5.2.3.}{自定义目标}

自定义目标与可执行文件或库略有不同。它们通过执行明确给出的命令行来扩展构建功能,例如,它们可以用于:

\begin{itemize}
\item
计算其他二进制文件的校验和。

\item
运行代码消毒器并收集结果。

\item
将编译报告发送到指标管道。
\end{itemize}

从这张列表中你可以猜到,自定义目标只在相当高级的项目中有用,所以我们只覆盖基础知识,然后继续讨论更重要的主题。

要定义一个自定义目标,请使用以下语法(为了简洁,省略了一些选项):

\begin{shell}
add_custom_target(Name [ALL] [COMMAND command2 [args2...] ...])
\end{shell}

自定义目标有一些需要考虑的缺点。由于它们涉及shell命令,它们可能是系统特定的,可能会限制可移植性。此外,自定义目标可能没有为CMake提供直接的方式来确定生成的具体工件或副产品(如果有的话)。

自定义目标也不像可执行文件和库那样应用陈旧性检查(它们不验证源文件是否比二进制文件更新),因为默认情况下它们没有被添加到依赖关系图中(所以ALL关键字与EXCLUDE\_FROM\_ALL相反)。让我们来看看这个依赖关系图是关于什么的。

\mySubsubsection{5.2.4.}{依赖关系图}

成熟的应用程序通常由许多组件构建而成,特别是内部库。从结构的角度来看,划分项目是有用的。当相关的事物被包装在一个单一的逻辑实体中时,它们可以与其他目标链接:另一个库或一个可执行文件。这对于多个目标使用同一个库的情况尤其方便。请查看图5.1,它描述了一个示例依赖关系图:

\myGraphic{0.5}{content/chapter5/images/1.png}{图5.1:BankApp项目中构建依赖的顺序}

在这个项目中,我们有两个库,两个可执行文件和一个自定义目标。我们的用例是提供一个带有美观GUI的银行应用程序(GuiApp)以及一个作为自动化脚本一部分使用的命令行版本(TerminalApp)。两个可执行文件都依赖于相同的Calculations库,但只有一个需要Drawing库。为了确保我们的应用程序二进制文件是从真实来源下载的,我们还将计算一个校验和,并通过单独的安全渠道分发它。CMake在编写此类解决方案的列表文件时非常灵活:

\mySamllsection{ch05/01-targets/CMakeLists.txt}

\begin{cmake}
cmake_minimum_required(VERSION 3.26)
project(BankApp CXX)

add_executable(terminal_app terminal_app.cpp)
add_executable(gui_app gui_app.cpp)
target_link_libraries(terminal_app calculations)
target_link_libraries(gui_app calculations drawing)

add_library(calculations calculations.cpp)
add_library(drawing drawing.cpp)

add_custom_target(checksum ALL
    COMMAND sh -c "cksum terminal_app>terminal.ck"
    COMMAND sh -c "cksum gui_app>gui.ck"
    BYPRODUCTS terminal.ck gui.ck
    COMMENT "Checking the sums..."
)
\end{cmake}

我们通过使用target\_link\_libraries()命令将库与可执行文件链接。没有它,由于未定义的符号,构建可执行文件将会失败。你注意到我们在声明任何一个库之前就调用了这个命令吗?当CMake配置项目时,它会收集关于目标和它们属性的信息——它们的名称、依赖关系、源文件和其他细节。

在解析所有文件之后,CMake将尝试构建一个依赖关系图。像所有有效的依赖关系图一样,它们是有向无环图(DAGs)。这意味着有一个清晰的哪个目标依赖于哪个目标的方向,并且这样的依赖关系不能形成循环。

当我们在构建模式下执行cmake时,生成的构建系统将检查我们定义的顶级目标,并递归地构建它们的依赖关系。让我们考虑一下图5.1中的例子:

\begin{enumerate}
\item
从顶部开始,构建组1中的两个库。

\item
当Calculations和Drawing库完成后,构建组2 – GuiApp和TerminalApp。

\item
 构建校验和目标;运行指定的命令行以生成校验和(cksum是一个Unix校验和工具,这意味着这个例子不会在其他平台上构建)。
\end{enumerate}

然而,有一个小问题——前面的解决方案不能保证校验和目标会在可执行文件之后构建。CMake不知道校验和依赖于可执行二进制文件的存在,所以它可以自由地首先开始构建它。为了解决这个问题,我们可以在文件末尾放置add\_dependencies()命令:

\begin{cmake}
add_dependencies(checksum terminal_app gui_app)
\end{cmake}

这将确保CMake理解校验和目标与可执行文件之间的关系。

那很好,但target\_link\_libraries()和add\_dependencies()之间有什么区别呢?target\_link\_libraries()旨在与实际库一起使用,并允许你控制属性传播。第二个命令仅用于顶级目标,以设置它们的构建顺序。

随着项目复杂性的增加,依赖树变得难以理解。我们如何简化这个过程呢?

\mySubsubsection{5.2.5.}{可视化依赖关系}

即使是小项目,也可能难以推理并与其他开发者分享。一个整洁的图表将大有帮助。毕竟,一张图片胜过千言万语。我们可以自己动手绘制图表,就像我在图5.1中所做的那样。但这很繁琐,并且每次项目更改时都需要更新。幸运的是,CMake有一个很棒的模块,可以生成dot/graphviz格式的依赖关系图,并且它支持内部和外部依赖!

要使用它,我们可以简单地执行以下命令:

\begin{shell}
cmake --graphviz=test.dot .
\end{shell}

该模块将生成一个文本文件,我们可以将其导入到Graphviz可视化软件中,该软件可以渲染图像或生成PDF或SVG文件,可以作为软件文档的一部分存储。每个人都喜欢优秀的文档,但几乎没有人喜欢创建它——现在,你不需要了!

自定义目标默认情况下是不可见的,我们需要创建一个特殊的配置文件,CMakeGraphVizOptions.cmake,这将允许我们自定义图表。使用set(GRAPHVIZ\_CUSTOM\_TARGETS TRUE)命令在你的图表中启用自定义目标:

\filename{ch05/01-targets/CMakeGraphVizOptions.cmake}

\begin{cmake}
set(GRAPHVIZ_CUSTOM_TARGETS TRUE)
\end{cmake}

其他选项允许添加图表名称、标题、节点前缀,并配置应包含或排除在输出中的目标(按名称或类型)。访问官方CMake文档以获取CMakeGraphVizOptions模块的完整描述。

如果你急于求成,你甚至可以直接在浏览器中运行Graphviz,地址是:\url{https://dreampuf.github.io/GraphvizOnline/}。

你所需要做的就是将test.dot文件的内容复制粘贴到左边的窗口中,你的项目就会被可视化(如图5.2)。非常方便,不是吗?

\myGraphic{0.9}{content/chapter5/images/2.png}{图5.2:在Graphviz中可视化BankApp示例}

使用这种方法,我们可以快速看到所有明确定义的目标。

现在我们理解了目标的概念,我们知道如何定义不同类型的目标,包括可执行文件、库和自定义目标,以及如何创建依赖关系图并打印它。让我们使用这些信息进行更深入的研究,看看如何配置它们。

\mySubsubsection{5.2.6.}{设置目标的属性}

目标具有类似于C++对象字段的属性。其中一些属性是为了被修改而设计的,而有些是只读的。CMake定义了大量“已知属性”(请参阅进一步阅读部分),这些属性取决于目标的类型(可执行文件、库或自定义)。如果你喜欢,你也可以添加自己的属性。使用以下命令来操作目标的属性:

\begin{shell}
get_target_property(<var> <target> <property-name>)
set_target_properties(<target1> <target2> ...
                      PROPERTIES <prop1-name> <value1>
                      <prop2-name> <value2> ...)
\end{shell}

要在屏幕上打印目标属性,我们首先需要将其存储在<var>变量中,然后将其消息传递给用户。读取属性必须逐个进行;设置目标属性允许我们同时为多个目标指定多个属性。

\begin{myNotic}{Note}
属性的概念不仅仅适用于目标;CMake支持为其他范围设置属性:GLOBAL、DIRECTORY、SOURCE、INSTALL、TEST和CACHE。为了操作所有类型的属性,有通用的get\_property()和set\_property()命令。在某些项目中,你会看到这些低级命令被用来精确地完成set\_target\_properties()命令所做的事情,只是需要多做一点工作:

\begin{shell}
set_property(TARGET <target> PROPERTY <name> <value>)
\end{shell}
\end{myNotic}

通常,尽可能多地使用高级命令是更好的选择。在某些情况下,CMake提供了带有额外机制的简写命令。例如,add\_dependencies( )是向MANUALLY\_ADDED\_DEPENDENCIES目标属性追加依赖关系的简写。在这种情况下,我们可以使用get\_target\_property()精确地查询它,就像其他任何属性一样。然而,我们不能使用set\_target\_properties()来更改它(它是只读的),因为CMake坚持使用add\_dependencies()命令来限制操作仅为追加。

在接下来的章节中讨论编译和链接时,我们将介绍更多设置属性的命令。现在,让我们专注于一个目标的属性如何传递给另一个目标。

\mySamllsection{什么是传递使用要求?}

让我们同意,命名是困难的,有时你会得到一个难以理解的标签。“传递使用要求”不幸的是这样一个令人费解的标题,你将在在线CMake文档中遇到。让我们解开这个奇怪的名字,或许提出一个更容易理解的术语。

从中间的术语开始:使用。正如我们之前讨论的,一个目标可能依赖于另一个。CMake文档有时将这种依赖称为使用,即一个目标使用另一个目标。

在某些情况下,被使用的目标会为自己设置特定的属性或依赖项,这些反过来构成了使用它的其他目标的依赖要求:链接某些库、包含目录或要求特定的编译器特性。

我们解谜的最后一部分,“传递”这个词描述了行为是正确的(也许可以更简单一些)。CMake将一些使用目标的属性/要求附加到使用目标的属性中。

可以说,一些属性可以隐式地在目标之间传递(或简单地传播),因此更容易表达依赖关系。

简化这个整个概念,我认为它就像源目标(被使用的目标)和目标目标(使用其他目标的目标)之间的传播属性。

让我们看一个具体的例子来理解它为什么存在以及它是如何工作的:

\begin{shell}
target_compile_definitions(<source> <INTERFACE|PUBLIC|PRIVATE> [items1...])
\end{shell}

这个目标命令将填充一个<source>目标的COMPILE\_DEFINITIONS属性。编译定义就是传递给编译器的-Dname=definition标志,用于配置C++预处理器定义(我们将在第7章“使用CMake编译C++源文件”中讨论这一点)。这里有趣的部分是第二个参数。我们需要指定三个值中的一个,INTERFACE、PUBLIC或PRIVATE,以控制属性应该传递给哪个目标。现在,不要将这些与C++访问修饰符混淆——这是一个独立的全新概念。

传播关键字的工作方式如下:

\begin{itemize}
\item
PRIVATE设置源目标的属性。

\item
INTERFACE设置目标目标的属性。

\item
PUBLIC设置源目标和目标目标的属性。
\end{itemize}

当属性不应传递给任何目标目标时,将其设置为PRIVATE。当需要这样的传递时,选择PUBLIC。如果你处于一个源目标在其实现(.cpp文件)中不使用该属性,而在头文件中使用,并且这些属性传递给消费者目标,则应使用INTERFACE关键字。

这是如何工作的?为了管理这些属性,CMake提供了一些命令,例如我们之前提到的target\_compile\_definitions()。当你指定PRIVATE或PUBLIC关键字时,CMake将在目标的属性中存储提供的值,在这种情况下,COMPILE\_DEFINITIONS。此外,如果关键字是INTERFACE或PUBLIC,它将在具有INTERFACE\_前缀的属性中存储值——INTERFACE\_COMPILE\_DEFINITIONS。在配置阶段,CMake将读取源目标的接口属性并将它们的内容附加到目标目标。就这样——传播属性,或CMake所说的传递使用要求。

使用set\_target\_properties()命令管理的属性可以在\url{https://cmake.org/cmake/help/latest/manual/cmake-properties.7.html}找到,在目标属性部分(并非所有目标属性都是传递的)。这里是最重要的几个:

\begin{itemize}
\item
COMPILE\_DEFINITIONS

\item
COMPILE\_FEATURES

\item
COMPILE\_OPTIONS

\item
INCLUDE\_DIRECTORIES

\item
LINK\_DEPENDS

\item
LINK\_DIRECTORIES

\item
LINK\_LIBRARIES

\item
LINK\_OPTIONS

\item
POSITION\_INDEPENDENT\_CODE

\item
PRECOMPILE\_HEADERS

\item
SOURCES
\end{itemize}

我们将在接下来的页面中讨论大多数这些选项,但请记住,所有这些选项当然都在CMake手册中有详细描述。在以下链接中找到它们的详细描述(将替换为你感兴趣的属性):\url{https://cmake.org/cmake/help/latest/prop_tgt/.html}

接下来,一个自然而然的问题是如何远传播这些属性。属性只设置在第一个目标目标上,还是发送到依赖图的最顶层?你可以决定。

为了在目标之间创建依赖关系,我们使用target\_link\_libraries()命令。这个命令的全名需要一个传播关键字:

\begin{shell}
target_link_libraries(<target>
                    <PRIVATE|PUBLIC|INTERFACE> <item>...
                    [<PRIVATE|PUBLIC|INTERFACE> <item>...]...)
\end{shell}

正如你所看到的,这个签名也指定了传播关键字,它控制源目标的属性如何在目标目标中存储。图5.3展示了生成阶段(配置阶段完成后)传播属性会发生什么情况:

\myGraphic{0.9}{content/chapter5/images/3.png}{图5.3:属性是如何传递到目标目标的}

传播关键字的工作方式如下:

\begin{itemize}
\item
PRIVATE将源值附加到源目标的私有属性。

\item
INTERFACE将源值附加到源目标的接口属性。

\item
PUBLIC将值附加到源目标的两个属性。
\end{itemize}

正如我们之前讨论的,接口属性仅用于将属性进一步传播到链中的下一个目标目标,而源目标在其构建过程中不会使用它们。

我们之前使用的基本target\_link\_libraries(…)命令隐式指定了PUBLIC关键字。

如果你正确地为你的源目标设置了传播关键字,属性将自动放置在目标目标上——除非有冲突…

\mySamllsection{处理冲突的传播属性}

当一个目标依赖于多个其他目标时,可能存在传播属性之间直接冲突的情况。例如,一个被使用的目标将POSITION\_INDEPENDENT\_CODE属性设置为true,而另一个设置为false。CMake将这种冲突理解为错误,并打印出类似于这样的错误信息:

\begin{shell}
CMake Error: The INTERFACE_POSITION_INDEPENDENT_CODE property of "source_ target" does not agree with the value of POSITION_INDEPENDENT_CODE already determined for "destination_target".
\end{shell}

收到这样的消息是有用的,因为我们明确知道我们引入了这种冲突,并且需要解决它。CMake有一些自己的属性,这些属性必须在源目标和目标目标之间“一致”。

在某些情况下,这可能变得重要——例如,如果你在多个目标中使用相同的库,然后将它们链接到一个单一的可执行文件。如果这些源目标使用的是同一库的不同版本,你可能会遇到问题。

为了确保我们只使用特定版本的库,我们可以创建一个自定义接口属性,INTERFACE\_LIB\_VERSION,并在其中存储版本。这还不够解决问题,因为CMake默认不会传播自定义属性(这个机制只适用于内置目标属性)。我们必须明确地将自定义属性添加到“兼容”属性列表中。

每个目标都有四个这样的列表:

\begin{itemize}
\item
COMPATIBLE\_INTERFACE\_BOOL

\item
COMPATIBLE\_INTERFACE\_STRING

\item
COMPATIBLE\_INTERFACE\_NUMBER\_MAX

\item
COMPATIBLE\_INTERFACE\_NUMBER\_MIN
\end{itemize}

将你的属性添加到它们中的任何一个都会触发传播和兼容性检查。BOOL列表将检查所有传递到目标目标的属性是否评估为相同的布尔值。类似地,STRING将评估为字符串。NUMBER\_MAX和NUMBER\_MIN略有不同——传递的值不必匹配,但目标目标将只接收最高或最低值。

这个例子将帮助我们了解如何在实践中应用这一点:

\filename{ch05/02-propagated/CMakeLists.txt}

\begin{cmake}
cmake_minimum_required(VERSION 3.26)
project(PropagatedProperties CXX)

add_library(source1 empty.cpp)
set_property(TARGET source1 PROPERTY INTERFACE_LIB_VERSION 4)
set_property(TARGET source1 APPEND PROPERTY
             COMPATIBLE_INTERFACE_STRING LIB_VERSION)

add_library(source2 empty.cpp)
set_property(TARGET source2 PROPERTY INTERFACE_LIB_VERSION 4)

add_library(destination empty.cpp)

target_link_libraries(destination source1 source2)
\end{cmake}

在这里,我们创建了三个目标;为了简化,所有目标都使用相同的空源文件。在两个源目标上,我们都指定了带有INTERFACE\_前缀的自定义属性,并将它们设置为相同的匹配库版本。两个源目标都被链接到目标目标。最后,我们在源1上指定了字符串兼容性要求作为属性(我们这里没有添加INTERFACE\_前缀)。

CMake将这个自定义属性传播到目标目标,并检查所有源目标的版本是否完全匹配(兼容性属性只需在目标目标上设置一次)。

既然我们已经了解了常规目标是什么,让我们来看看其他看起来像目标、闻起来像目标,有时甚至表现得像目标的东西,但最终发现它们并不是真正的目标。

\mySubsubsection{5.2.7.}{认识伪目标}

目标的概念非常有用,以至于如果能够将其某些行为借用到其他事物上就太好了;这些事物不是构建系统的输出,而是输入——外部依赖项、别名等。这些就是伪目标,或者是不出现在生成的构建系统中的目标:

\begin{itemize}
\item
导入的目标

\item
别名目标

\item
接口库
\end{itemize}

让我们来看看。

\mySamllsection{导入的目标}

如果你浏览了本书的目录,你知道我们将讨论CMake如何管理外部依赖项——其他项目、库等。IMPORTED目标是这个过程的产物。CMake可以定义它们作为find\_package()命令的结果。

你可以调整这类目标的属性:编译定义、编译选项、包含目录等——它们甚至支持传递使用要求。然而,你应该将它们视为不可变的目标;不要更改它们的源文件或依赖关系。

IMPORTED目标的定义范围可以是全局的,也可以是定义它们的目录的本地范围(在子目录中可见,但在父目录中不可见)。

\mySamllsection{别名目标}

别名目标的确切作用就是你所期望的——它们为目标创建另一个不同的名称引用。你可以为可执行文件和库创建别名目标,使用以下命令:

\begin{shell}
add_executable(<name> ALIAS <target>)
add_library(<name> ALIAS <target>)
\end{shell}

别名目标的属性是只读的,你不能安装或导出别名(它们在生成的构建系统中不可见)。

那么,为什么还要有别名呢?在某些情况下,它们非常有用,比如项目的一部分(如子目录)需要以特定名称引用一个目标,而实际的实现可能因情况而异。例如,你可能希望根据用户的选择构建解决方案中的库或导入它。

\mySamllsection{接口库}

这是一个有趣的构造——一个不编译任何东西的库,而是作为一个实用目标。它的整个概念都是围绕传播属性(传递使用要求)构建的。

接口库主要有两个用途——代表仅包含头文件的库,以及将一堆传播属性打包成一个单一的逻辑单元。

仅包含头文件的库可以用add\_library(INTERFACE)轻松创建:

\begin{cmake}
add_library(Eigen INTERFACE
    src/eigen.h src/vector.h src/matrix.h
)
target_include_directories(Eigen INTERFACE
    $<BUILD_INTERFACE:${CMAKE_CURRENT_SOURCE_DIR}/src>
    $<INSTALL_INTERFACE:include/Eigen>
)
\end{cmake}

在前面的代码片段中,我们创建了一个包含三个头的Eigen接口库。接下来,使用生成器表达式(这些以美元符号和尖括号表示,\$<…>,将在下一章解释),我们将它的包含目录设置为当目标导出时为\$\{CMAKE\_CURRENT\_SOURCE\_DIR\}/src,安装时为include/Eigen(这将在本章的末尾解释)。

要使用这样的库,我们只需链接它:

\begin{cmake}
target_link_libraries(executable Eigen)
\end{cmake}

这里不会发生实际的链接,但CMake会将这个命令理解为向可执行目标传播所有INTERFACE属性的请求。

第二种使用场景利用了相同的机制,但出于不同的目的——它创建了一个逻辑目标,可以作为一个传播属性的占位符。然后我们可以将这个目标用作其他目标的依赖,并以干净、方便的方式设置属性。这是一个例子:

\begin{cmake}
add_library(warning_properties INTERFACE)
target_compile_options(warning_properties INTERFACE
    -Wall -Wextra -Wpedantic
)
target_link_libraries(executable warning_properties)
\end{cmake}

add\_library(INTERFACE)命令创建了一个逻辑的warning\_properties目标,用于在第二个命令中为可执行目标设置编译选项。我建议使用这些INTERFACE目标,因为它们可以提高代码的可读性和可重用性。想象一下,这是将一堆魔法值重构为命名良好的变量的过程。我还建议明确地为接口库添加一个后缀,如\_properties,以便轻松区分接口库和常规库。

\mySubsubsection{5.2.8.}{对象库}

对象库用于将多个源文件组合成一个单一的逻辑目标,并在构建过程中将它们编译成(.o)对象文件。要创建一个对象库,我们遵循与创建其他库相同的方法,但使用OBJECT关键字:

\begin{shell}
add_library(<target> OBJECT <sources>)
\end{shell}

构建过程中产生的对象文件可以作为其他目标的编译元素,使用\$<TARGET\_OBJECTS:objlib>生成器表达式:

\begin{shell}
add_library(... $<TARGET_OBJECTS:objlib> ...)
add_executable(... $<TARGET_OBJECTS:objlib> ...)
\end{shell}

或者,你可以使用target\_link\_libraries()命令将它们作为依赖项添加。

在Calc库的上下文中,对象库将非常有用,以避免为库的静态和共享版本编译库源的冗余。对于共享库,明确地编译具有POSITION\_INDEPENDENT\_CODE启用的对象文件是必要的。

回到项目的目标:calc\_obj将提供编译后的对象文件,然后将用于calc\_static和calc\_shared库。让我们探索这两种类型库之间的实际区别,并理解为什么可能需要创建两者。

伪目标是否穷尽了目标的概念?当然不是!这太简单了。我们仍然需要了解这些目标是如何用于生成构建系统的。

\mySubsubsection{5.2.9.}{构建目标}

在项目的上下文和生成的构建系统中,“目标”一词可以有不同的含义。在生成构建系统的上下文中,CMake将CMake语言编写的列表文件“编译”成所选构建工具的语言,例如为GNU Make创建Makefile。这些生成的Makefile有自己的目标集。其中一些目标是列表文件中定义的目标的直接转换,而其他目标则是在构建系统生成过程中隐式创建的。

这样一个构建系统目标就是ALL,CMake默认生成为包含所有顶级列表文件目标的目标,如可执行文件和库(不一定是自定义目标)。当我们运行cmake -{}-build <build tree>时,ALL将构建所有顶级列表文件目标。正如你可能还记得第一章的内容,你可以通过添加-{}-target <name>参数到cmake构建命令中来选择一个目标。

一些可执行文件或库可能不需要在每次构建中都存在,但我们希望将它们作为项目的一部分,以便在那些不常需要的场合中使用。为了优化我们的默认构建,我们可以将它们从ALL目标中排除:

\begin{shell}
add_executable(<name> EXCLUDE_FROM_ALL [<source>...])
add_library(<name> EXCLUDE_FROM_ALL [<source>...])
\end{shell}

自定义目标则相反——默认情况下,它们被排除在ALL目标之外,除非你明确地使用ALL关键字添加它们,就像我们在BankApp示例中所做的那样。

另一个隐式定义的构建目标是clean,它简单地从构建树中删除产生的工件。我们使用它来清除所有旧文件并从头开始构建。重要的是要理解,clean并不会简单地删除构建目录中的所有内容。为了正确工作,你需要手动指定任何自定义目标可能创建的文件作为BYPRODUCTS(请参阅BankApp示例)。

这标志着我们对目标和它们不同方面的探索之旅的结束:我们知道如何创建它们,配置它们的属性,使用伪目标,并决定它们是否应该默认构建。还有一个有趣的非目标机制可以创建可以在所有实际目标中使用的自定义工件——自定义命令(不要与自定义目标混淆)。

