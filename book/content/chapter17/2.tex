

string() 命令用于处理字符串。具有多种模式,可以对字符串执行不同的操作:搜索替换、操作、比较、哈希、生成,以及 JSON 操作(最后一个自 CMake 3.19 开始可用)。

完整的详情可以在在线文档中找到: \url{https://cmake.org/cmake/help/latest/ command/string.html}

\begin{myNotic}{Note}
接受 <input> 参数的 string() 模式将接受多个 <input> 值并在命令执行前将它们连接起来,因此:

\begin{cmake}
string(PREPEND myVariable "a" "b" "c")
\end{cmake}

等同于以下操作:

\begin{cmake}
string(PREPEND myVariable "abc")
\end{cmake}
\end{myNotic}

可用的 string() 模式包括搜索替换、操作、比较、哈希、生成以及 JSON。

\mySubsubsection{A.2.1.}{搜索和替换}

以下是可用的模式:

\begin{itemize}
\item
string(FIND <haystack> <pattern> <out> [REVERSE]):在 <haystack> 字符串中搜索 <pattern> 并将找到的位置作为整数写入 <out> 变量。如果使用了 REVERSE 标志,则从字符串末尾向开头搜索。这仅适用于 ASCII 字符串(不支持多字节字符)。

\item
string(REPLACE <pattern> <replace> <out> <input]):在 <input> 中替换所有 <pattern> 的出现,并将结果存储在 <out> 变量中。

\item
string(REGEX MATCH <pattern> <out> <input>):使用正则表达式匹配 <input> 中 <pattern> 的首次出现,并将结果存储在 <out> 变量中。

\item
string(REGEX MATCHALL <pattern> <out> <input>):使用正则表达式匹配 <input> 中所有 <pattern> 的出现,并将结果作为一个逗号分隔的列表存储在 <out> 变量中。

\item
string(REGEX REPLACE <pattern> <replace> <out> <input>):使用正则表达式替换 <input> 中所有 <pattern> 的出现,并将替换表达式的结果存储在 <out> 变量中。
\end{itemize}

正则表达式操作遵循 C++ 语法,如标准库中的 <regex> 头文件所定义。可以使用捕获组将匹配添加到 <replace> 表达式中,使用数字占位符:\verb|\\|1, \verb|\\|2 ... (需要双反斜杠以正确解析参数)。

\mySubsubsection{A.2.2.}{操作}

以下是可用的模式:

\begin{itemize}
\item
string(APPEND <out> <input>):通过追加 <input> 字符串来修改存储在 <out> 中的字符串。

\item
string(PREPEND <out> <input>):通过前置 <input> 字符串来修改存储在 <out> 中的字符串。

\item
string(CONCAT <out> <input>):连接所有提供的 <input> 字符串并将它们存储在 <out> 变量中。

\item
string(JOIN <glue> <out> <input>):在所有提供的 <input> 字符串之间插入 <glue> 值,并将它们作为连接后的字符串存储在 <out> 变量中(不要将此模式用于列表变量)。

\item
string(TOLOWER <string> <out>):将 <string> 转换为小写并将其存储在 <out> 变量中。

\item
string(TOUPPER <string> <out>):将 <string> 转换为大写并将其存储在 <out> 变量中。

\item
string(LENGTH <string> <out>):计算 <string> 的字节数并将结果存储在 <out> 变量中。

\item
string(SUBSTRING <string> <begin> <length> <out>):提取从 <begin> 字节开始长度为 <length> 的子字符串,并将其存储在 <out> 变量中。将 -1 作为长度被理解为“直到字符串末尾”。

\item
string(STRIP <string> <out>):从 <string> 中移除前后空白并将结果存储在 <out> 变量中。

\item
string(GENEX\_STRIP <string> <out>):从 <string> 中移除所有生成器表达式并将结果存储在 <out> 变量中。

\item
string(REPEAT <string> <count> <out>):生成包含 <count> 次重复的 <string> 的字符串,并将其存储在 <out> 变量中。
\end{itemize}

\mySubsubsection{A.2.3.}{比较}

字符串比较采用以下形式:

\begin{cmake}
string(COMPARE <operation> <stringA> <stringB> <out>)
\end{cmake}

<operation> 参数可为:

\begin{itemize}
\item
LESS

\item
GREATER

\item
EQUAL

\item
NOTEQUAL

\item
LESS\_EQUAL

\item
GREATER\_EQUAL
\end{itemize}

用于比较 <stringA> 和 <stringB>,并将结果(真或假)存储在 <out> 变量中。

\mySubsubsection{A.2.4.}{哈希}

哈希模式具有以下签名:

\begin{cmake}
string(<hashing-algorithm> <out> <string>)
\end{cmake}

使用 <hashing-algorithm> 对 <string> 进行哈希处理,并将结果存储在 <out> 变量中。支持以下算法:

\begin{itemize}
\item
MD5: 消息摘要算法 5,RFC 1321

\item
SHA1: 美国安全哈希算法 1,RFC 3174

\item
SHA224: 美国安全哈希算法,RFC 4634

\item
SHA256: 美国安全哈希算法,RFC 4634

\item
SHA384: 美国安全哈希算法,RFC 4634

\item
SHA512: 美国安全哈希算法,RFC 4634

\item
SHA3\_224: Keccak SHA-3

\item
SHA3\_256: Keccak SHA-3

\item
SHA3\_384: Keccak SHA-3

\item
SHA3\_512: Keccak SHA-3
\end{itemize}

\mySubsubsection{A.2.5.}{生成}

以下是可用的模式:

\begin{itemize}
\item
string(ASCII <number>... <out>):将给定 <number> 的 ASCII 字符存储在 <out> 变量中。

\item
string(HEX <string> <out>):将 <string> 转换为其十六进制表示并将其存储在 <out> 变量中(自 CMake 3.18 开始)。

\item
string(CONFIGURE <string> <out> [@ONLY] [ESCAPE\_QUOTES]):完全像 configure\_file() 一样工作,但针对字符串。结果存储在 <out> 变量中。作为提醒,使用 @ONLY 关键词将限制替换为形如 @VARIABLE@ 的变量。

\item
string(MAKE\_C\_IDENTIFIER <string> <out>):将 <string> 中的非字母数字字符转换为下划线并将结果存储在 <out> 变量中。

\item
string(RANDOM [LENGTH <len>] [ALPHABET <alphabet>] [RANDOM\_SEED <seed>] <out>):使用随机种子 <seed> 和可选的 <alphabet> 生成长度为 <len>(默认为 5)的随机字符串,并将结果存储在 <out> 变量中。

\item
string(TIMESTAMP <out> [<format>] [UTC]):生成代表当前日期和时间的字符串,并将其存储在 <out> 变量中。

\item
string(UUID <out> NAMESPACE <ns> NAME <name> TYPE <type>):生成全局唯一标识符。此模式的应用稍微复杂,需要提供一个命名空间(必须是一个 UUID)、一个名称(例如,域名)以及一个类型(MD5 或 SHA1)。
\end{itemize}

\mySubsubsection{A.2.6.}{JSON}

对 JSON 格式的字符串的操作采用以下签名:

\begin{cmake}
string(JSON <out> [ERROR_VARIABLE <error>] <operation + args>)
\end{cmake}

有几个操作可用。都将结果存储在 <out> 变量中,错误则存储在 <error> 变量中。操作及其参数如下:

\begin{itemize}
\item
GET <json> <member|index>... :使用 <member> 路径或 <index> 从 <json> 字符串中返回一个或多个元素的值。

\item
TYPE <json> <member|index>... :使用 <member> 路径或 <index> 从 <json> 字符串中返回一个或多个元素的类型。

\item
MEMBER <json> <member|index>... :使用 <member> 路径或 <index> 从 <json> 字符串中返回位于 <array-index> 位置的一个或多个数组类型的元素的成员名称。

\item
LENGTH <json> <member|index>...:使用 <member> 路径或 <index> 从 <json> 字符串中返回一个或多个数组类型元素的元素计数。

\item
REMOVE <json> <member|index>...:使用 <member> 路径或 <index> 从 <json> 字符串中移除一个或多个元素。

\item
SET <json> <member|index>... :使用 <member> 路径或 <index> 将 <value> 更新或插入到 <json> 字符串中的一个或多个元素中。

\item
EQUAL <jsonA> <jsonB>:评估 <jsonA> 和 <jsonB> 是否相等。
\end{itemize}




































