

The string() command is used to manipulate strings. It comes with a variety of modes that perform different actions on the string: search and replace, manipulation, comparison, hashing, generation, and JSON operations (the last one available since CMake 3.19).

Full details can be found in the online documentation: \url{https://cmake.org/cmake/help/latest/ command/string.html}.

\begin{myNotic}{Note}
Note that string() modes that accept the <input> argument will accept multiple <input> values and concatenate them before the execution of the command, so:

\begin{cmake}
string(PREPEND myVariable "a" "b" "c")
\end{cmake}

is the equivalent of the following:

\begin{cmake}
string(PREPEND myVariable "abc")
\end{cmake}
\end{myNotic}

Available string() modes are search and replace, manipulation, comparison, hashing, generation, and JSON.

\mySubsubsection{A.2.1.}{Search and replace}

The following modes are available:

\begin{itemize}
\item
string(FIND <haystack> <pattern> <out> [REVERSE]) searches for <pattern> in the <haystack> string and writes the position found as an integer to the <out> variable. If the REVERSE flag was used, it searches from the end of the string to the beginning. This works only for ASCII strings (multibyte support isn’t provided).

\item
string(REPLACE <pattern> <replace> <out> <input>) replaces all occurrences of <pattern> in <input> with <replace> and stores them in the <out> variable.

\item
string(REGEX MATCH <pattern> <out> <input>) regex-matches the first occurrence of <pattern> in <input> with <replace> and stores it in the <out> variable.

\item
string(REGEX MATCHALL <pattern> <out> <input>) regex-matches all occurrences of <pattern> in <input> with <replace> and stores them in the <out> variable as a comma-separated list.

\item
string(REGEX REPLACE <pattern> <replace> <out> <input>) regex-replaces all occurrences of <pattern> in <input> with the <replace> expression and stores them in the <out> variable.
\end{itemize}

Regular expression operations follow C++ syntax, as defined in the standard library in the <regex> header. You can use capturing groups to add matches to the <replace> expression with numeric placeholders: \verb|\\|1, \verb|\\|2... (double backslashes are required so arguments are parsed correctly).

\mySubsubsection{A.2.2.}{Manipulation}

The following modes are available:

\begin{itemize}
\item
string(APPEND <out> <input>)mutates strings stored in <out>by appending the <input> string.

\item
string(PREPEND <out> <input>) mutates strings stored in <out> by prepending the <input> string.

\item
string(CONCAT <out> <input>) concatenates all provided <input> strings and stores them in the <out> variable.

\item
string(JOIN <glue> <out> <input>) interleaves all provided <input> strings with a <glue> value and stores them as a concatenated string in the <out> variable (don’t use this mode for list variables).

\item
string(TOLOWER <string> <out>) converts <string> to lowercase and stores it in the <out> variable.

\item
string(TOUPPER <string> <out>) converts <string> to uppercase and stores it in the <out> variable.

\item
string(LENGTH <string> <out>) counts the bytes of <string> and stores the result in the <out> variable.

\item
string(SUBSTRING <string> <begin> <length> <out>)extracts a substring of <string> of <length> bytes starting at the <begin> byte, and stores it in the <out> variable. Providing -1 as the length is understood as “till the end of the string.”

\item
string(STRIP <string> <out>) removes trailing and leading whitespace from <string> and stores the result in the <out> variable.

\item
string(GENEX\_STRIP <string> <out>) removes all generator expressions used in <string> and stores the result in the <out> variable.

\item
string(REPEAT <string> <count> <out>) generates a string containing <count> repetitions of <string> and stores it in the <out> variable.
\end{itemize}

\mySubsubsection{A.2.3.}{Comparison}

A comparison of strings takes the following form:

\begin{cmake}
string(COMPARE <operation> <stringA> <stringB> <out>)
\end{cmake}

The <operation> argument is one of the following:

\begin{itemize}
\item
LESS

\item
GREATER

\item
EQUAL

\item
NOTEQUAL

\item
LESS\_EQUAL

\item
GREATER\_EQUAL
\end{itemize}

It will be used to compare <stringA> with <stringB> and the result (true or false) will be stored in the <out> variable.

\mySubsubsection{A.2.4.}{Hashing}

The hashing mode has the following signature:

\begin{cmake}
string(<hashing-algorithm> <out> <string>)
\end{cmake}

It hashes <string> with <hashing-algorithm> and stores the result in the <out> variable. The following algorithms are supported:

\begin{itemize}
\item
MD5: Message-Digest Algorithm 5, RFC 1321

\item
SHA1: US Secure Hash Algorithm 1, RFC 3174

\item
SHA224: US Secure Hash Algorithms, RFC 4634

\item
SHA256: US Secure Hash Algorithms, RFC 4634

\item
SHA384: US Secure Hash Algorithms, RFC 4634

\item
SHA512: US Secure Hash Algorithms, RFC 4634

\item
SHA3\_224: Keccak SHA-3

\item
SHA3\_256: Keccak SHA-3

\item
SHA3\_384: Keccak SHA-3

\item
SHA3\_512: Keccak SHA-3
\end{itemize}

\mySubsubsection{A.2.5.}{Generation}

The following modes are available:

\begin{itemize}
\item
string(ASCII <number>... <out>) stores ASCII characters of given <number> in the <out> variable.

\item
string(HEX <string> <out>) converts <string> to its hexadecimal representation and stores it in the <out> variable (since CMake 3.18).

\item
string(CONFIGURE <string> <out> [@ONLY] [ESCAPE\_QUOTES]) works exactly like configure\_file() but for strings. The result is stored in the <out> variable. As a reminder, using the @ONLY keyword restricts replacements to variables in the form of @VARIABLE@.

\item
string(MAKE\_C\_IDENTIFIER <string> <out>) converts non-alphanumeric characters in <string> to underscores and stores the result in the <out> variable.

\item
string(RANDOM [LENGTH <len>] [ALPHABET <alphabet>] [RANDOM\_SEED <seed>] <out>)generates a random string of <len> characters (default 5) using the optional <alphabet> from the random seed, <seed>, and stores the result in the <out> variable.

\item
string(TIMESTAMP <out> [<format>] [UTC]) generates a string representing the current date and time and stores it in the <out> variable.

\item
string(UUID <out> NAMESPACE <ns> NAME <name> TYPE <type>) generates a universally unique identifier. Application of this mode is a bit complex to use; you need to provide a namespace (which has to be a UUID), a name (for example, a domain name), and a type (either MD5 or SHA1).
\end{itemize}

\mySubsubsection{A.2.6.}{JSON}

Operations on JSON-formatted strings use the following signature:

\begin{cmake}
string(JSON <out> [ERROR_VARIABLE <error>] <operation + args>)
\end{cmake}

Several operations are available. They all store their results in the <out> variable, and errors in the <error> variable. Operations and their arguments are as follows:

\begin{itemize}
\item
GET <json> <member|index>... returns the value of one or more elements from a <json> string using the <member> path or <index>.

\item
TYPE <json> <member|index>... returns the type of one or more elements from a <json> string using the <member> path or <index>.

\item
MEMBER <json> <member|index>... <array-index>returns the member name of one or more array-typed elements on the <array-index> position from the <json> string using the <member> path or <index>.

\item
LENGTH <json> <member|index>...returns the element count of one or more array-typed elements from the <json> string using the <member> path or <index>.

\item
REMOVE <json> <member|index>...returns the result of removal of one or more elements from the <json> string using the <member> path or <index>.

\item
SET <json> <member|index>... <value> returns the result of upsertion of <value> to one or more elements from a <json> string using the <member> path or <index>.

\item
EQUAL <jsonA> <jsonB> evaluates whether <jsonA> and <jsonB> are equal.
\end{itemize}




































