
This command provides all kinds of operations related to files: reading, transferring, locking, and archiving. It also provides modes to inspect the filesystem and operations on strings representing paths.

Full details can be found in the online documentation:

\url{https://cmake.org/cmake/help/latest/command/file.html}

The categories for available file() modes are reading, writing, filesystem, path conversion, transfer, locking, and archiving.

\mySubsubsection{A.4.1.}{Reading}

The following modes are available:

\begin{itemize}
\item
file(READ <filename> <out> [OFFSET <o>] [LIMIT <max>] [HEX])reads the file from <filename> to the <out> variable. The read optionally starts at offset <o> and follows the optional limit of <max> bytes. The HEX flag specifies that output should be converted to hexadecimal representation.

\item
file(STRINGS <filename> <out>) reads strings from the file at <filename> to the <out> variable.

\item
file(<hashing-algorithm> <filename> <out>) computes the <hashing-algorithm> hash from the file at <filename> and stores the result in the <out> variable. Available algorithms are the same as for the string() hashing function.

\item
file(TIMESTAMP <filename> <out> [<format>]) generates a string representation of a timestamp of the file at <filename> and stores it in the <out> variable. It optionally accepts a <format> string.

\item
file(GET\_RUNTIME\_DEPENDENCIES [...]) gets runtime dependencies for specified files. This is an advanced command to be used only in install(CODE) or install(SCRIPT) scenarios. Available since CMake 3.21.
\end{itemize}

\mySubsubsection{A.4.2.}{Writing}

The following modes are available:

\begin{itemize}
\item
file(\{WRITE | APPEND\} <filename> <content>...) writes or appends all <content> arguments to the file at <filename>. If the provided system path doesn’t exist, it will be recursively created.

\item
file(\{TOUCH | TOUCH\_NOCREATE\} [<filename>...]) updates the timestamp of the <filename>. If the file doesn’t exist, it will only be created in the TOUCH mode.

\item
file(GENERATE OUTPUT <output-file> [...]) is an advanced mode that generates an output file for each build configuration of the current CMake generator.

\item
file(CONFIGURE OUTPUT <output-file> CONTENT <content> [...]) works like GENERATE\_OUTPUT but also configures the generated files by substituting variable placeholders with values.
\end{itemize}

\mySubsubsection{A.4.3.}{Filesystem}

The following modes are available:

\begin{itemize}
\item
file(\{GLOB | GLOB\_RECURSE\} <out> [...] [<globbing-expression>...]) generates a list of files matching <globbing-expression> and stores it in the <out> variable. GLOB\_RECURSE mode will also scan nested directories.

\item
file(RENAME <oldname> <newname>) moves a file from <oldname> to <newname>.

\item
file(\{REMOVE | REMOVE\_RECURSE \} [<files>...]) deletes <files>. REMOVE\_RECURSE will also remove directories.

\item
file(MAKE\_DIRECTORY [<dir>...]) creates a directory.

\item
file(COPY <file>... DESTINATION <dir> [...]) copies files to the <dir> destination. It offers options for filtering, setting permissions, symlink chain following, and more.

\item
file(COPY\_FILE <file> <destination> [...])copies a single file to the <destination> path. Available since CMake 3.21.

\item
file(SIZE <filename> <out>) reads the size of <filename> in bytes and stores it in the <out> variable.

\item
file(READ\_SYMLINK <linkname> <out>) reads the destination path of the <linkname> symlink and stores it in the <out> variable.

\item
file(CREATE\_LINK <original> <linkname> [...]) creates a symlink to <original> at <linkname>.

\item
file(\{CHMOD|CHMOD\_RECURSE\} <files>... <directories>... PERMISSIONS <permissions>... [...]) sets permissions on files and directories.

\item
file(GET\_RUNTIME\_DEPENDENCIES [...])collects the runtime dependencies for various types of files: executables, libraries, and modules. Use with install(RUNTIME\_DEPENDENCY\_SET).
\end{itemize}

\mySubsubsection{A.4.4.}{Path conversion}

The following modes are available:

\begin{itemize}
\item
file(REAL\_PATH <path> <out> [BASE\_DIRECTORY <dir>]) computes the absolute path from the relative path and stores it in the <out> variable. It optionally accepts the <dir> base directory. Available since CMake 3.19.

\item
file(RELATIVE\_PATH <out> <directory> <file>) computes the <file> path relative to <directory> and stores it in the <out> variable.

\item
file(\{TO\_CMAKE\_PATH | TO\_NATIVE\_PATH\} <path> <out>) converts <path> to a CMake path (directories separated with a forward slash) to the native path of the platform and back. The result is stored in the <out> variable.
\end{itemize}

\mySubsubsection{A.4.5.}{Transfer}

The following modes are available:

\begin{itemize}
\item
file(DOWNLOAD <url> [<path>] [...]) downloads a file from <url> and stores it in <path>.

\item
file(UPLOAD <file> <url> [...]) uploads <file> to a URL.
\end{itemize}

\mySubsubsection{A.4.6.}{Locking}

Locking mode places an advisory lock on the <path> resource:

\begin{shell}
file(LOCK <path> [DIRECTORY] [RELEASE]
    [GUARD <FUNCTION|FILE|PROCESS>]
    [RESULT_VARIABLE <out>] [TIMEOUT <seconds>]
)
\end{shell}

This lock can be optionally scoped to FUNCTION, FILE, or PROCESS and limited with a timeout of <seconds>. To release the lock, provide the RELEASE keyword. The result will be stored in the <out> variable.

\mySubsubsection{A.4.7.}{Archiving}

The creation of archives is provided with the following signature:

\begin{shell}
file(ARCHIVE_CREATE OUTPUT <destination> PATHS <source>...
    [FORMAT <format>]
    [COMPRESSION <type> [COMPRESSION_LEVEL <level>]]
    [MTIME <mtime>] [VERBOSE]
)
\end{shell}

It creates an archive at the <destination> path comprising <source> files in one of the supported formats: 7zip, gnutar, pax, paxr, raw, or zip (paxr is the default). If the chosen format supports the compression level, it can be provided as a single-digit integer 0-9, with 0 being the default.

The extraction mode has the following signature:

\begin{shell}
file(ARCHIVE_EXTRACT INPUT <archive> [DESTINATION <dir>]
    [PATTERNS <patterns>...] [LIST_ONLY] [VERBOSE]
)
\end{shell}

It extracts files matching optional <patterns> values from <archive> to the destination <dir>. If the LIST\_ONLY keyword is provided, files won’t be extracted but will only be listed instead.



