
此命令提供了与文件相关的各种操作:读取、传输、锁定及归档。它还提供了用于检查文件系统和对路径字符串进行操作的模式。

完整详情可以在在线文档中找到:

\url{https://cmake.org/cmake/help/latest/command/file.html}

可用的 file() 模式的类别包括读取、写入、文件系统、路径转换、传输、锁定及归档。

\mySubsubsection{A.4.1.}{读取}

以下是可用的模式:

\begin{itemize}
\item
file(READ <filename> <out> [OFFSET <o>] [LIMIT <max>] [HEX]):从 <filename> 读取文件到 <out> 变量。读取可选地从偏移量 <o> 开始,并遵循最大字节数 <max> 的限制。HEX 标志指明输出应转换为十六进制表示。

\item
file(STRINGS <filename> <out>):从 <filename> 中读取字符串到 <out> 变量。

\item
file(<hashing-algorithm> <filename> <out>):计算 <filename> 文件的 <hashing-algorithm> 哈希值,并将结果存储在 <out> 变量中。可用的算法与 string() 哈希函数相同。

\item
file(TIMESTAMP <filename> <out> [<format>]):生成 <filename> 文件的时间戳字符串表示,并将其存储在 <out> 变量中。可选地接受一个 <format> 字符串。

\item
file(GET\_RUNTIME\_DEPENDENCIES [...]):获取指定文件的运行时依赖项。这是一个高级命令,只应在 install(CODE) 或 install(SCRIPT) 场景中使用。自 CMake 3.21 开始可用。
\end{itemize}

\mySubsubsection{A.4.2.}{写入}

以下是可用的模式:

\begin{itemize}
\item
file(\{WRITE | APPEND\} <filename> <content>...):将所有 <content> 参数写入或追加到 <filename> 文件中。如果提供的系统路径不存在,它将会递归创建。

\item
file(\{TOUCH | TOUCH\_NOCREATE\} [<filename>...]):更新 <filename> 的时间戳。如果文件不存在,则仅在 TOUCH 模式下创建。

\item
file(GENERATE OUTPUT <output-file> [...]):这是一种高级模式,为当前 CMake 生成器的每个构建配置生成输出文件。

\item
file(CONFIGURE OUTPUT <output-file> CONTENT <content> [...]) :其作用类似于 GENERATE\_OUTPUT,但同时通过替换变量占位符来配置生成的文件。
\end{itemize}

\mySubsubsection{A.4.3.}{文件系统}

以下是可用的模式:

\begin{itemize}
\item
file(\{GLOB | GLOB\_RECURSE\} <out> [...] [<globbing-expression>...]):生成匹配 <globbing-expression> 的文件列表,并将其存储在 <out> 变量中。GLOB\_RECURSE 模式也会扫描嵌套目录。

\item
file(RENAME <oldname> <newname>):将文件从 <oldname> 移动到 <newname>。

\item
file(\{REMOVE | REMOVE\_RECURSE \} [<files>...]):删除 <files>。REMOVE\_RECURSE 也会删除目录。

\item
file(MAKE\_DIRECTORY [<dir>...]):创建目录。

\item
file(COPY <file>... DESTINATION <dir> [...]):将文件复制到 <dir> 目的地。它提供了过滤、设置权限、跟随符号链接链等选项。

\item
file(COPY\_FILE <file> <destination> [...]):将单个文件复制到 <destination> 路径。自 CMake 3.21 开始可用。

\item
file(SIZE <filename> <out>):读取 <filename> 的大小(以字节为单位),并将其存储在 <out> 变量中。

\item
file(READ\_SYMLINK <linkname> <out>):读取 <linkname> 符号链接的目标路径,并将其存储在 <out> 变量中。

\item
file(CREATE\_LINK <original> <linkname> [...]):在 <linkname> 创建指向 <original> 的符号链接。

\item
file(\{CHMOD|CHMOD\_RECURSE\} <files>... <directories>... PERMISSIONS <permissions>... [...]):设置文件和目录的权限。

\item
file(GET\_RUNTIME\_DEPENDENCIES [...]):收集各种类型文件的运行时依赖项:可执行文件、库和模块。与 install(RUNTIME\_DEPENDENCY\_SET) 结合使用。
\end{itemize}

\mySubsubsection{A.4.4.}{路径转换}

以下是可用的模式:

\begin{itemize}
\item
file(REAL\_PATH <path> <out> [BASE\_DIRECTORY <dir>]):从相对路径计算绝对路径,并将其存储在 <out> 变量中。可选地接受 <dir> 基础目录。自 CMake 3.19 开始可用。

\item
file(RELATIVE\_PATH <out> <directory> <file>):计算 <file> 相对于 <directory> 的路径,并将其存储在 <out> 变量中。

\item
file(\{TO\_CMAKE\_PATH | TO\_NATIVE\_PATH\} <path> <out>):将 <path> 转换为 CMake 路径(目录以正斜杠分隔)到平台原生路径及反之。结果存储在 <out> 变量中。
\end{itemize}

\mySubsubsection{A.4.5.}{传输}

以下是可用的模式:

\begin{itemize}
\item
file(DOWNLOAD <url> [<path>] [...]):从 <url> 下载文件,并将其存储在 <path> 中。

\item
file(UPLOAD <file> <url> [...]):将 <file> 上传到 URL。
\end{itemize}

\mySubsubsection{A.4.6.}{锁定}

锁定模式在 <path> 资源上放置了一个建议性锁

\begin{shell}
file(LOCK <path> [DIRECTORY] [RELEASE]
    [GUARD <FUNCTION|FILE|PROCESS>]
    [RESULT_VARIABLE <out>] [TIMEOUT <seconds>]
)
\end{shell}

此锁可选地限定在 FUNCTION、FILE 或 PROCESS,并且可以用 <seconds> 的超时时间限制。为了释放锁,提供 RELEASE 关键词。结果将存储在 <out> 变量中。

\mySubsubsection{A.4.7.}{归档}

归档创建提供以下签名:

\begin{shell}
file(ARCHIVE_CREATE OUTPUT <destination> PATHS <source>...
    [FORMAT <format>]
    [COMPRESSION <type> [COMPRESSION_LEVEL <level>]]
    [MTIME <mtime>] [VERBOSE]
)
\end{shell}

它在 <destination> 路径创建一个归档文件,其中包含 <source> 文件之一种支持的格式:7zip、gnutar、pax、paxr、raw 或 zip(默认为 paxr)。如果选定的格式支持压缩级别,则可以提供一个单数字整数 0-9,其中 0 为默认值。

提取模式具有以下签名:

\begin{shell}
file(ARCHIVE_EXTRACT INPUT <archive> [DESTINATION <dir>]
    [PATTERNS <patterns>...] [LIST_ONLY] [VERBOSE]
)
\end{shell}

它从 <archive> 中提取匹配可选 <patterns> 值的文件到目的地 <dir>。如果提供了 LIST\_ONLY 关键词,则文件不会被提取,而只会列出。



