

此命令提供了基本的列表操作:读取、查找、修改及排序。某些模式会改变列表(即修改原始值)。如果你稍后还需要原始值,请确保先复制一份。

完整详情可以在在线文档中找到:

\url{https://cmake.org/cmake/help/latest/command/list.html}

可用的 list() 模式的类别包括读取、查找、修改及排序。

\mySubsubsection{A.3.1.}{读取}

以下是可用的模式:

\begin{itemize}
\item
list(LENGTH <list> <out>):计算 <list> 变量中的元素数量,并将结果存储在 <out> 变量中。

\item
list(GET <list> <index>... <out>):将由 <index> 索引列表指定的 <list> 元素复制到 <out> 变量中。

\item
list(JOIN <list> <glue> <out>):使用 <glue> 分隔符交错 <list> 元素,并将结果字符串存储在 <out> 变量中。

\item
list(SUBLIST <list> <begin> <length> <out>):其作用类似于 GET 模式,但不是使用显式索引而是使用范围。如果 <length> 是 -1,则从 <begin> 索引到 <list> 变量中的列表末尾的所有元素都会被返回。
\end{itemize}

\mySubsubsection{A.3.2.}{查找}

此模式仅仅是在 <list> 变量中查找 <needle> 元素的索引,并将结果存储在 <out> 变量中(如果没有找到该元素,则存储 -1):

\begin{cmake}
list(FIND <list> <needle> <out>)
\end{cmake}

\mySubsubsection{A.3.3.}{修改}

以下是可用的模式:

\begin{itemize}
\item
list(APPEND <list> <element>...) :将一个或多个 <element> 值添加到 <list> 变量的末尾。
\item
list(PREPEND <list> [<element>...]):其作用类似于 APPEND,但将元素添加到 <list> 变量的开头。

\item
list(FILTER <list> \{INCLUDE | EXCLUDE\} REGEX <pattern>):根据 <pattern> 值过滤 <list> 变量,以 INCLUDE 或 EXCLUDE 匹配的元素。

\item
list(INSERT <list> <index> [<element>...]):在给定的 <index> 位置将一个或多个 <element> 值添加到 <list> 变量中。

\item
list(POP\_BACK <list> [<out>...]):从 <list> 变量的末尾移除一个元素,并将它存储在可选的 <out> 变量中。如果提供了多个 <out> 变量,则会移除更多元素来填充它们。

\item
list(POP\_FRONT <list> [<out>...]):其作用类似于 POP\_BACK,但从 <list> 变量的开头移除元素。

\item
list(REMOVE\_ITEM <list> <value>...):是 FILTER EXCLUDE 的简写,但不支持正则表达式。

\item
list(REMOVE\_AT <list> <index>...):从 <list> 中移除特定 <index> 的元素。

\item
list(REMOVE\_DUPLICATES <list>):从 <list> 中移除重复项。

\item
list(TRANSFORM <list> <action> [<selector>] [OUTPUT\_VARIABLE <out>]) :对 <list> 元素应用特定的转换。默认情况下,操作应用于所有元素,但我们可以通过添加 <selector> 来限制其效果。除非提供 OUTPUT\_VARIABLE 关键词,在这种情况下结果将存储在 <out> 变量中,否则提供的列表将被修改(就地更改)。
\end{itemize}

以下选择器可用:AT <index>,FOR <start> <stop> [<step>],以及 REGEX <pattern>。

操作包括 APPEND <string>,PREPEND <string>,TOLOWER,TOUPPER,STRIP,GENEX\_STRIP,以及 REPLACE <pattern> <expression>。它们的工作方式与具有相同名称的 string() 模式完全相同。

\mySubsubsection{A.3.4.}{排序}

以下是可用的模式:

\begin{itemize}
\item
list(REVERSE <list>):简单地反转 <list> 的顺序。

\item
list(SORT <list>):按字母顺序对列表进行排序。
\end{itemize}

有关更高级选项的参考,请参阅在线手册。











