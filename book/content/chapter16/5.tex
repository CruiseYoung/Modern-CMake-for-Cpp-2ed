工作流预设是我们项目的终极自动化解决方案。它们允许我们按照预定顺序自动执行多个特定阶段的预设。这样,我们实际上可以用一步完成端到端的构建。

要发现项目的可用工作流,我们可以执行以下命令:

\begin{shell}
cmake --workflow --list-presets
\end{shell}

要选择并应用一个预设,使用以下命令:

\begin{shell}
cmake –workflow --preset <preset-name>
\end{shell}

此外,使用 -{}-fresh 标志,我们可以清除构建树和缓存。

定义工作流预设非常简单;我们需要定义一个名称,并且可以像为特定阶段的预设那样可选地提供 displayName 和 description。之后,我们必须枚举出工作流应该执行的所有特定阶段的预设。这是通过提供一个 steps 数组来完成的,数组中的对象包含 type 和 name 属性,如下面所示:

\filename{ch16/01-presets/CMakePresets.json (续)}

\begin{json}
...
"workflowPresets": [
{
    "name": "myWorkflow",
    "steps": [
        {
            "type": "configure",
            "name": "myConfigure"
        },
        {
            "type": "build",
            "name": "myBuild"
        },
        {
            "type": "test",
            "name": "myTest"
        },
        {
            "type": "package",
            "name": "myPackage"
        },
        {
            "type": "build",
            "name": "myInstall"
        }
    ]
...
\end{json}

steps 数组中的每个对象引用了我们在本章中早些时候定义的预设,指示其类型(configure, build, test, 或 package)和名称。这些预设一起执行所有必要的步骤,只需一条命令即可从头开始完全构建和安装项目:

\begin{shell}
cmake --workflow --preset myWorkflow
\end{shell}

工作流预设是用于自动化 C++ 构建、测试、打包和安装的终极解决方案。接下来,让我们探讨如何通过条件和宏来管理一些特殊情况。




