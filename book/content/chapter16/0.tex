CMake 3.19 版本中引入了预设(Presets),目的是简化项目设置的管理。预设出现之前,用户需要记住冗长的命令行配置,或是直接在项目文件中设置覆盖选项,这往往会变得复杂且容易出错。预设让用户能够以更简单的方式处理,诸如用于配置项目的生成器、并发构建任务的数量,以及要构建或测试的项目组件等设置。通过使用预设,CMake 变得更加易于使用。用户可以一次性设置预设,并在需要时使用它们,从而使每次 CMake 执行更加一致且易于理解。还有助于跨不同用户和计算机标准化设置,从而简化协作项目的工作流程。

预设兼容 CMake 的四种主要模式:配置构建系统、构建、运行测试和打包。允许用户将这些部分链接在一起形成工作流,使整个过程更加自动化和有序。此外,预设提供了如条件和宏表达式(或简称宏)等功能,给予用户更大的控制力。

本章中,将包含以下内容:

\begin{itemize}
\item
使用项目中定义的预设

\item
编写预设文件

\item
定义特定阶段的预设

\item
定义工作流预设

\item
添加条件和宏
\end{itemize}


