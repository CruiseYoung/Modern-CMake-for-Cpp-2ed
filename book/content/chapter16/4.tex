

特定阶段的预设仅仅是配置各个 CMake 阶段的预设:配置、构建、测试、打包和安装,允许以精细和结构化的方式来定义构建配置。以下是所有预设阶段共有的特性概览,随后将介绍如何为各个阶段定义预设。

\mySubsubsection{16.4.1.}{各预设阶段的共同特性}

无论 CMake 阶段如何,有三个特性用于配置预设,这些特性包括独特的名称字段、可选字段,以及与配置阶段预设的关联。

\mySamllsection{独特的名称字段}

每个预设在其所属阶段内必须有一个独特的名称字段。如果 CMakeUserPresets.json 存在,它会隐式地包含 CMakePresets.json(如果该文件也存在),所以两个文件共享命名空间,避免了名称重复。例如,不能在两个文件中都有名为 myPreset 的打包阶段预设。

一个最小化的预设文件可能如下所示:

\begin{json}
{
    "version": 6,
    "configurePresets": [
    {
        "name": "myPreset"
    },
    {
        "name": "myPreset2"
    }
    ]
}
\end{json}

\mySamllsection{可选字段}

每个特定阶段的预设都使用相同的可选字段:

\begin{itemize}
\item
displayName: 一个字符串,为预设提供一个用户友好的名称。

\item
description: 一个字符串,说明预设的功能。

\item
inherits: 一个字符串或字符串数组,有效地复制了在此字段中命名的预设的配置作为基础,进一步扩展或修改。

\item
hidden: 一个布尔值,隐藏预设使其不显示在列表中;这样的隐藏预设只能通过继承使用。

\item
environment: 一个对象,为这个阶段覆盖环境变量;每个键标识一个单独的变量,值可以是字符串或 null;它支持宏。

\item
condition: 一个对象,启用或禁用此预设(稍后会有更多介绍)。

\item
vendor: 一个自定义对象,包含供应商特定的值,并遵循与根级 vendor 字段相同的约定。
\end{itemize}

预设可以形成类似图的继承结构,前提是不存在循环依赖。CMakeUserPresets.json 可以从项目级别的预设继承,但反之则不行。

\mySamllsection{与配置阶段预设的关联}

所有特定阶段的预设都必须与配置预设相关联,它们必须知道构建树的位置。虽然配置预设本质上与自身关联,但构建、测试和打包预设,需要明确地通过 configurePreset 字段定义这种关联。

这种关联并不意味着当决定运行后续预设时,CMake 将自动执行配置预设。仍然需要手动执行每个预设,或者使用工作流预设(稍后会介绍这一点)。

有了这些基础概念,可以继续深入探讨各个阶段的预设细节,从配置阶段开始。随着进展,将探索这些预设如何相互作用,以及如何用来简化 CMake 中的项目配置和构建过程。

\mySubsubsection{16.4.2.}{定义配置阶段预设}

配置预设位于 configurePresets 数组中,可以通过向命令行添加 -{}-list-presets 参数来列出,具体针对配置阶段:

\begin{shell}
cmake --list-presets
\end{shell}

要用选定的预设配置项目,请在 -{}-preset 参数后指定其名称:

\begin{shell}
cmake --preset myConfigurationPreset
\end{shell}

配置预设有像名称和描述这样的通用字段,但它也有自己的一套可选字段。以下是最重要的几个字段的简化描述:

\begin{itemize}
\item
generator: 一个字符串,指定了预设使用的生成器;对于小于第 3 版的架构是必需的。

\item
architecture 和 toolset: 一个字符串,配置支持这些选项的生成器。

\item
binaryDir: 一个字符串,提供了构建树的相对或绝对路径;对于小于第 3 版的架构是必需的;支持宏。

\item
installDir: 一个字符串,提供了安装目录的相对或绝对路径;对于小于第 3 版的架构是必需的;支持宏。

\item
cacheVariables: 一个映射/map,定义了缓存变量;值支持宏。
\end{itemize}

定义 cacheVariables 映射时,要记住项目中变量解析的顺序。如图 16.1 所示,通过命令行定义的任何缓存变量都会覆盖预设变量。缓存或环境预设变量都会覆盖来自缓存文件或主机环境的变量。

\myGraphic{0.4}{content/chapter16/images/1.png}{图16.1: 预设如何覆盖CMakeCache.txt和系统环境变量}

声明一个简单的 myConfigure 配置预设,其指定了生成器、构建树和安装路径:

\filename{ch16/01-presets/CMakePresets.json (continued)}

\begin{json}
...
    "configurePresets": [
        {
            "name": "myConfigure",
            "displayName": "Configure Preset",
            "description": "Ninja generator",
            "generator": "Ninja",
            "binaryDir": "${sourceDir}/build",
            "installDir": "${sourceDir}/build/install"
        }
    ],
...
\end{json}

对配置预设的介绍已经完成,接下来是构建阶段预设。

\mySubsubsection{16.4.3.}{定义构建阶段预设}

不会惊讶于发现构建预设位于 buildPresets 数组中。可以通过向命令行添加 -{}-list-presets 参数来列出它们,具体针对构建阶段:

\begin{shell}
cmake --build --list-presets
\end{shell}

要用选定的预设构建项目,请在 -{}-preset 参数后指定其名称:

\begin{shell}
cmake --build --preset myBuildingPreset
\end{shell}

构建预设有像名称和描述这样的通用字段,其也有一套自己的可选字段。以下是最重要的几个字段的简化描述:

\begin{itemize}
\item
jobs: 一个整数,设置了用于构建项目的并行作业的数量。

\item
targets: 一个字符串或字符串数组,设置了要构建的目标,并支持宏。

\item
configuration: 一个字符串,确定了多配置生成器的构建类型(Debug, Release等)。

\item
cleanFirst: 一个布尔值,确保在构建前总是清理项目。
\end{itemize}

现在,可以这样写一个构建预设:

\filename{ch16/01-presets/CMakePresets.json (续)}

\begin{json}
...
    "buildPresets": [
        {
            "name": "myBuild",
            "displayName": "Build Preset",
            "description": "Four jobs",
            "configurePreset": "myConfigure",
            "jobs": 4
        }
    ],
...
\end{json}

必需的 configurePreset 字段设置为指向我们在上一节中定义的 myConfigure 预设。现在,可以继续介绍测试预设。

\mySubsubsection{16.4.4.}{定义测试阶段预设}

测试预设位于 testPresets 数组中。可以通过向命令行添加 -{}-list-presets 参数来列出它们,具体针对测试阶段:

\begin{shell}
ctest --list-presets
\end{shell}

要用预设测试项目,请在 -{}-preset 参数后指定其名称:

\begin{shell}
ctest --preset myTestPreset
\end{shell}

测试预设有它自己的一套可选字段。以下是最重要的几个字段的简化描述:

\begin{itemize}
\item
configuration: 一个字符串,确定了多配置生成器的构建类型 (Debug, Release等)

\item
output: 一个对象,配置输出。

\item
filter: 一个对象,指定要运行哪些测试。

\item
execution: 一个对象,配置测试的执行。
\end{itemize}

每个对象将相应的命令行选项映射到配置值,但这并不是一个详尽的列表。请参阅“扩展阅读”部分获取完整的参考。

output 对象的可选条目包括:

\begin{itemize}
\item
shortProgress: 布尔值;进度将在一行内报告。

\item
verbosity:  一个字符串,将输出的详细程度设置为以下级别之一:默认、详细或其他信息。

\item
outputOnFailure: 布尔值;在测试失败时打印程序输出。

\item
quiet: 布尔值;抑制所有输出。
\end{itemize}

对于 exclude,接受的一些条目是:

\begin{itemize}
\item
name: 一个字符串,排除匹配正则表达式模式的测试名称,并支持宏。

\item
label:  一个字符串,排除匹配正则表达式模式的测试标签,并支持宏。

\item
fixtures: 一个对象,确定要排除哪些测试夹具(参见官方文档了解更多详情)。
\end{itemize}

最后,execution 对象接受以下可选条目:

\begin{itemize}
\item
outputLogFile: 一个字符串,指定了输出日志文件的路径,并支持宏。
\end{itemize}

filter 对象接受 include 和 exclude 键来配置测试案例的筛选;

\begin{json}
    "testPresets": [
        {
            "name": "myTest",
            "configurePreset": "myConfigure",
            "filter": {
                "include": {
                    ... name, label, index, useUnion ...
                },
                "exclude": {
                    ... name, label, fixtures ...
                }
            }
        }
    ],
...
\end{json}

每个键定义了自己的选项对象:

对于 include:

\begin{itemize}
\item
name:  一个字符串,包含匹配正则表达式模式的测试名称,并支持宏。

\item
label: 一个字符串,包含匹配正则表达式模式的测试标签,并支持宏。

\item
index: 一个对象,通过接受起始、结束和步长整数以及,一个特定测试的整数数组来选择要运行的测试,支持宏。

\item
useUnion: 一个布尔值,启用使用由 index 和 name 确定的测试的并集,而不是交集。
\end{itemize}

对于 exclude,条目包括:

\begin{itemize}
\item
name: 一个字符串,排除匹配正则表达式模式的测试名称,并支持宏。

\item
label: 一个字符串,排除匹配正则表达式模式的测试标签,并支持宏。

\item
fixtures: 一个对象,确定要从测试中排除哪些夹具(参见官方文档了解更多信息)。
\end{itemize}

最后,execution 对象可以在这里添加:

\begin{json}
    "testPresets": [
        {
            "name": "myTest",
            "configurePreset": "myConfigure",
            "execution": {
                ... stopOnFailure, enableFailover, ...
                ... jobs, repeat, scheduleRandom, ...
                ... timeout, noTestsAction ...
            }
        }
    ],
...
\end{json}

接受以下可选条目:

\begin{itemize}
\item
stopOnFailure: 一个布尔值,任何测试失败则停止测试。

\item
enableFailover: 一个布尔值,恢复之前中断的测试。

\item
jobs: 一个整数,以并行方式运行多个测试。

\item
repeat: 一个对象,确定如何重复测试;该对象必须具有以下字段:
\begin{itemize}
\item
mode – 一个字符串,其值可以是:until-fail, until-pass, aftertimeout。

\item
count – 一个整数,确定重复次数。
\end{itemize}

\item
scheduleRandom: 一个布尔值,启用随机顺序执行测试。

\item
timeout: 一个整数,设置所有测试总执行时间的限制(秒)。

\item
noTestsAction: 一个字符串,定义如果没有找到测试时的动作,选项包括:default, error, 和 ignore。
\end{itemize}

尽管有许多配置选项,简单的预设也可行:

\filename{ch16/01-presets/CMakePresets.json (续)}

\begin{json}
...
"testPresets": [
    {
        "name": "myTest",
        "displayName": "Test Preset",
        "description": "Output short progress",
        "configurePreset": "myConfigure",
        "output": {
            "shortProgress": true
        }
    }
],
...
\end{json}

与构建预设一样,我们也为新的测试预设设置了必需的 configurePreset 字段,以便整洁地将它们联系起来。来看一下最后一个特定阶段的预设类型:打包预设。

\mySubsubsection{16.4.5.}{定义打包阶段预设}

打包预设是在第 6 版架构中引入的,所以需要至少使用 CMake 3.25 才能使用它们。这些预设应该包含在 packagePresets 数组中,也可以通过向命令行追加 -{}-list-presets 参数来显示,具体针对打包阶段:

\begin{shell}
cpack --list-presets
\end{shell}

要用预设创建项目包,请在 -{}-preset 参数后指定其名称:

\begin{shell}
cpack --preset myTestPreset
\end{shell}

打包预设利用与其他预设相同的共享字段,并引入了一些特定于自身的可选字段:

\begin{itemize}
\item
generators: 一个字符串数组,设置了要使用的包生成器(ZIP, 7Z, DEB 等)。

\item
configuration: 一个字符串数组,确定了 CPack 要打包的构建类型列表。 (Debug, Release等)。

\item
filter:  一个对象,指定了要运行的测试。

\item
packageName, packageVersion, packageDirectory 和 vendorName: 字符串,指定了创建的包的元数据。
\end{itemize}

在预设文件中添加一个简洁的打包预设:

\filename{ch16/01-presets/CMakePresets.json (续)}

\begin{json}
...
    "packagePresets": [
        {
            "name": "myPackage",
            "displayName": "Package Preset",
            "description": "ZIP generator",
            "configurePreset": "myConfigure",
            "generators": [
            "ZIP"
            ]
        }
    ],
...
\end{json}

这样的配置将简化项目包的创建,但还缺少一个关键成分:项目安装。

\mySubsubsection{16.4.6.}{添加安装预设}

可能已经注意到 CMakePresets.json 对象不支持定义 "installPresets"。没有明确的方式通过预设安装项目,这似乎有些奇怪,因为配置预设提供了 installDir 字段!那么是否需要诉诸于手动安装命令?

幸运的是,不需要。有一个变通方法可以让我们使用构建预设实现目标。

来看一下:

\filename{ch16/01-presets/CMakePresets.json (续)}

\begin{json}
...
    "buildPresets": [
        {
            "name": "myBuild",
            ...
        },
        {
            "name": "myInstall",
            "displayName": "Installation",
            "targets" : "install",
            "configurePreset": "myConfigure"
        }
    ],
...
\end{json}

可以创建一个构建预设,其中 targets 字段设置为 install。当正确配置安装时,install 目标由项目隐式定义。使用这个预设构建将执行必要的步骤,来安装项目到在关联的配置预设中指定的 installDir(如果 installDir 字段为空,则使用默认位置):

\begin{shell}
$ cmake --build --preset myInstall
[0/1] Install the project...
-- Install configuration: ""
-- Installing: .../install/include/calc/basic.h
-- Installing: .../install/lib/libcalc_shared.so
-- Installing: .../install/lib/libcalc_static.a
-- Installing: .../install/lib/calc/cmake/CalcLibrary.cmake
-- Installing: .../install/lib/calc/cmake/CalcLibrary-noconfig.cmake
-- Installing: .../install/lib/calc/cmake/CalcConfig.cmake
-- Installing: .../install/bin/calc_console
-- Set non-toolchain portion of runtime path of ".../install/bin/calc_
console" to ""
\end{shell}

这个巧妙的方法可以帮助我们节省一些步骤。如果可以为最终用户提供一个单一命令,从配置到安装全部处理好,那就更好了。确实可以做到,通过工作流预设。

