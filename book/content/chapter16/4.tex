

特定阶段的预设仅仅是配置各个 CMake 阶段的预设:配置、构建、测试、打包和安装。它们允许以精细和结构化的方式来定义构建配置。以下是所有预设阶段共有的特性概览,随后我们将介绍如何为各个阶段定义预设。

\mySubsubsection{16.4.1.}{各预设阶段的共同特性}

无论 CMake 阶段如何,有三个特性用于配置预设。具体来说,这些特性包括独特的名称字段、可选字段以及与配置阶段预设的关联。接下来的各节将分别介绍这些特性


\mySamllsection{独特的名称字段}

每个预设在其所属阶段内必须有一个独特的名称字段。考虑到如果 CMakeUserPresets.json 存在,它会隐式地包含 CMakePresets.json(如果该文件也存在),这意味着两个文件共享命名空间,防止了名称重复。例如,你不能在两个文件中都有名为 myPreset 的打包阶段预设。

一个最小化的预设文件可能如下所示:

\begin{json}
{
    "version": 6,
    "configurePresets": [
    {
        "name": "myPreset"
    },
    {
        "name": "myPreset2"
    }
    ]
}
\end{json}


\mySamllsection{可选字段}

每个特定阶段的预设都可以使用相同的可选字段:

\begin{itemize}
\item
displayName: 一个字符串,为预设提供一个用户友好的名称。

\item
description: 一个字符串,说明预设的功能。

\item
inherits: 一个字符串或字符串数组,有效地复制了在此字段中命名的预设的配置作为基础,进一步扩展或修改。

\item
hidden: 一个布尔值,隐藏预设使其不显示在列表中;这样的隐藏预设只能通过继承使用。

\item
environment: 一个对象,为这个阶段覆盖环境变量;每个键标识一个单独的变量,值可以是字符串或 null;它支持宏。

\item
condition: 一个对象,启用或禁用此预设(稍后会有更多介绍)。

\item
vendor: 一个自定义对象,包含供应商特定的值,并遵循与根级 vendor 字段相同的约定。
\end{itemize}

预设可以形成类似图的继承结构,前提是不存在循环依赖。CMakeUserPresets.json 可以从项目级别的预设继承,但反之则不行。

\mySamllsection{与配置阶段预设的关联}

所有特定阶段的预设都必须与一个配置预设相关联,因为它们必须知道构建树的位置。虽然配置预设本质上与自身关联,但构建、测试和打包预设需要明确地通过 configurePreset 字段定义这种关联。

与你可能认为的不同,这种关联并不意味着当你决定运行任何后续预设时 CMake 将自动执行配置预设。你仍然需要手动执行每个预设,或者使用工作流预设(稍后我们会介绍这一点)。

有了这些基础概念,我们可以继续深入探讨各个阶段的预设细节,从配置阶段开始。随着进展,我们将探索这些预设是如何相互作用的,以及它们如何被用来简化 CMake 中的项目配置和构建过程。

\mySubsubsection{16.4.2.}{定义配置阶段预设}

正如前面所述,配置预设位于 configurePresets 数组中。可以通过向命令行添加 -{}-list-presets 参数来列出它们,具体针对配置阶段:

\begin{shell}
cmake --list-presets
\end{shell}

要用选定的预设配置项目,请在 -{}-preset 参数后指定其名称:

\begin{shell}
cmake --preset myConfigurationPreset
\end{shell}

配置预设有像名称和描述这样的通用字段,但它也有自己的一套可选字段。以下是最重要的几个字段的简化描述:

\begin{itemize}
\item
generator: 一个字符串,指定了预设使用的生成器;对于小于第 3 版的架构是必需的。

\item
architecture 和 toolset: 一个字符串,配置支持这些选项的生成器。

\item
binaryDir: 一个字符串,提供了构建树的相对或绝对路径;对于小于第 3 版的架构是必需的;支持宏。

\item
installDir: 一个字符串,提供了安装目录的相对或绝对路径;对于小于第 3 版的架构是必需的;支持宏。

\item
cacheVariables: 一个映射,定义了缓存变量;值支持宏。
\end{itemize}

在定义 cacheVariables 映射时,要记住项目中变量解析的顺序。如图 16.1 所示,通过命令行定义的任何缓存变量都会覆盖预设变量。任何缓存或环境预设变量都会覆盖来自缓存文件或主机环境的变量。

\myGraphic{0.4}{content/chapter16/images/1.png}{图16.1: 预设如何覆盖CMakeCache.txt和系统环境变量}

让我们声明一个简单的 myConfigure 配置预设,它指定了生成器、构建树和安装路径:

\filename{ch16/01-presets/CMakePresets.json (continued)}

\begin{json}
...
    "configurePresets": [
        {
            "name": "myConfigure",
            "displayName": "Configure Preset",
            "description": "Ninja generator",
            "generator": "Ninja",
            "binaryDir": "${sourceDir}/build",
            "installDir": "${sourceDir}/build/install"
        }
    ],
...
\end{json}

我们对配置预设的介绍已经完成,接下来是构建阶段预设。

\mySubsubsection{16.4.3.}{定义构建阶段预设}

你不会惊讶地发现构建预设位于 buildPresets 数组中。可以通过向命令行添加 -{}-list-presets 参数来列出它们,具体针对构建阶段:

\begin{shell}
cmake --build --list-presets
\end{shell}

要用选定的预设构建项目,请在 -{}-preset 参数后指定其名称:

\begin{shell}
cmake --build --preset myBuildingPreset
\end{shell}

构建预设有像名称和描述这样的通用字段,它也有一套自己的可选字段。以下是最重要的几个字段的简化描述:

\begin{itemize}
\item
jobs: 一个整数,设置了用于构建项目的并行作业的数量。

\item
targets: 一个字符串或字符串数组,设置了要构建的目标,并支持宏。

\item
configuration: 一个字符串,确定了多配置生成器的构建类型(Debug, Release等)。

\item
cleanFirst: 一个布尔值,确保在构建前总是清理项目。
\end{itemize}

现在,我们可以这样写一个构建预设:

\filename{ch16/01-presets/CMakePresets.json (续)}

\begin{json}
...
    "buildPresets": [
        {
            "name": "myBuild",
            "displayName": "Build Preset",
            "description": "Four jobs",
            "configurePreset": "myConfigure",
            "jobs": 4
        }
    ],
...
\end{json}

你会注意到必需的 configurePreset 字段设置为指向我们在上一节中定义的 myConfigure 预设。现在,我们可以继续介绍测试预设。

\mySubsubsection{16.4.4.}{定义测试阶段预设}

测试预设位于 testPresets 数组中。可以通过向命令行添加 -{}-list-presets 参数来列出它们,具体针对测试阶段:

\begin{shell}
ctest --list-presets
\end{shell}

要用预设测试项目,请在 -{}-preset 参数后指定其名称:

\begin{shell}
ctest --preset myTestPreset
\end{shell}

测试预设有它自己的一套可选字段。以下是最重要的几个字段的简化描述:

\begin{itemize}
\item
configuration: 一个字符串,确定了多配置生成器的构建类型 (Debug, Release等)

\item
output: 一个对象,配置输出。

\item
filter: 一个对象,指定要运行哪些测试。

\item
execution: 一个对象,配置测试的执行。
\end{itemize}

每个对象将相应的命令行选项映射到配置值。我们将突出一些基本选项,但这并不是一个详尽的列表。请参阅“进一步阅读”部分获取完整的参考。

output 对象的可选条目包括:

\begin{itemize}
\item
shortProgress: 布尔值;进度将在一行内报告。

\item
verbosity:  一个字符串,将输出的详细程度设置为以下级别之一:默认、详细或额外详细。

\item
outputOnFailure: 布尔值;在测试失败时打印程序输出。

\item
quiet: 布尔值;抑制所有输出。
\end{itemize}

对于 exclude,接受的一些条目是:

\begin{itemize}
\item
name: 一个字符串,排除匹配正则表达式模式的测试名称,并支持宏。

\item
label:  一个字符串,排除匹配正则表达式模式的测试标签,并支持宏。

\item
fixtures: 一个对象,确定要排除哪些测试夹具(参见官方文档了解更多详情)。
\end{itemize}

最后,execution 对象接受以下可选条目:

\begin{itemize}
\item
outputLogFile: 一个字符串,指定了输出日志文件的路径,并支持宏。
\end{itemize}

filter 对象接受 include 和 exclude 键来配置测试案例的筛选;这里是一个部分填充的结构以说明这一点

\begin{json}
    "testPresets": [
        {
            "name": "myTest",
            "configurePreset": "myConfigure",
            "filter": {
                "include": {
                    ... name, label, index, useUnion ...
                },
                "exclude": {
                    ... name, label, fixtures ...
                }
            }
        }
    ],
...
\end{json}

每个键定义了自己的选项对象:

对于 include,条目包括:

\begin{itemize}
\item
name:  一个字符串,包含匹配正则表达式模式的测试名称,并支持宏。

\item
label: 一个字符串,包含匹配正则表达式模式的测试标签,并支持宏。

\item
index: 一个对象,通过接受起始、结束和步长整数以及一个特定测试的整数数组来选择要运行的测试;它支持宏。

\item
useUnion: 一个布尔值,启用使用由 index 和 name 确定的测试的并集,而不是交集。
\end{itemize}

对于 exclude,条目包括:

\begin{itemize}
\item
name: 一个字符串,排除匹配正则表达式模式的测试名称,并支持宏。

\item
label: 一个字符串,排除匹配正则表达式模式的测试标签,并支持宏。

\item
fixtures: 一个对象,确定要从测试中排除哪些夹具(参见官方文档了解更多信息)。
\end{itemize}

最后,execution 对象可以在这里添加:

\begin{json}
    "testPresets": [
        {
            "name": "myTest",
            "configurePreset": "myConfigure",
            "execution": {
                ... stopOnFailure, enableFailover, ...
                ... jobs, repeat, scheduleRandom, ...
                ... timeout, noTestsAction ...
            }
        }
    ],
...
\end{json}

它接受以下可选条目:

\begin{itemize}
\item
stopOnFailure: 一个布尔值,如果任何测试失败则停止测试。

\item
enableFailover: 一个布尔值,恢复之前中断的测试。

\item
jobs: 一个整数,以并行方式运行多个测试。

\item
repeat: 一个对象,确定如何重复测试;该对象必须具有以下字段:
\begin{itemize}
\item
mode – 一个字符串,其值可以是:until-fail, until-pass, aftertimeout。

\item
count – 一个整数,确定重复次数。
\end{itemize}

\item
scheduleRandom: 一个布尔值,启用随机顺序执行测试。

\item
timeout: 一个整数,设置所有测试总执行时间的限制(秒)。

\item
noTestsAction: 一个字符串,定义如果没有找到测试时的动作,选项包括:default, error, 和 ignore。
\end{itemize}

尽管有许多配置选项,简单的预设也是可行的:

\filename{ch16/01-presets/CMakePresets.json (续)}

\begin{json}
...
"testPresets": [
    {
        "name": "myTest",
        "displayName": "Test Preset",
        "description": "Output short progress",
        "configurePreset": "myConfigure",
        "output": {
            "shortProgress": true
        }
    }
],
...
\end{json}

与构建预设一样,我们也为新的测试预设设置了必需的 configurePreset 字段,以便整洁地将它们联系起来。让我们看一下最后一个特定阶段的预设类型:打包预设。

\mySubsubsection{16.4.5.}{定义打包阶段预设}

打包预设是在第 6 版架构中引入的,这意味着你需要至少使用 CMake 3.25 才能利用它们。这些预设应该包含在 packagePresets 数组中。也可以通过向命令行追加 -{}-list-presets 参数来显示它们,具体针对打包阶段:

\begin{shell}
cpack --list-presets
\end{shell}

要用预设创建项目包,请在 -{}-preset 参数后指定其名称:

\begin{shell}
cpack --preset myTestPreset
\end{shell}

打包预设利用与其他预设相同的共享字段,并引入了一些特定于自身的可选字段:

\begin{itemize}
\item
generators: 一个字符串数组,设置了要使用的包生成器(ZIP, 7Z, DEB 等)。

\item
configuration: 一个字符串数组,确定了 CPack 要打包的构建类型列表。 (Debug, Release等)。

\item
filter:  一个对象,指定了要运行的测试。

\item
packageName, packageVersion, packageDirectory 和 vendorName: 字符串,指定了创建的包的元数据。
\end{itemize}

让我们在预设文件中添加一个简洁的打包预设:

\filename{ch16/01-presets/CMakePresets.json (续)}

\begin{json}
...
    "packagePresets": [
        {
            "name": "myPackage",
            "displayName": "Package Preset",
            "description": "ZIP generator",
            "configurePreset": "myConfigure",
            "generators": [
            "ZIP"
            ]
        }
    ],
...
\end{json}

这样的配置将使我们能够简化项目包的创建,但我们还缺少一个关键成分:项目安装。让我们看看如何让它工作。

\mySubsubsection{16.4.6.}{添加安装预设}

你可能已经注意到 CMakePresets.json 对象不支持定义 "installPresets"。没有明确的方式通过预设安装你的项目,这似乎有些奇怪,因为配置预设提供了 installDir 字段!那么我们是否需要诉诸于手动安装命令?

幸运的是,不需要。有一个变通方法可以让我们使用构建预设实现目标。

来看一下:

\filename{ch16/01-presets/CMakePresets.json (continued)}

\begin{json}
...
    "buildPresets": [
        {
            "name": "myBuild",
            ...
        },
        {
            "name": "myInstall",
            "displayName": "Installation",
            "targets" : "install",
            "configurePreset": "myConfigure"
        }
    ],
...
\end{json}

我们可以创建一个构建预设,其中 targets 字段设置为 install。当正确配置安装时,install 目标由项目隐式定义。使用这个预设构建将执行必要的步骤来安装项目到在关联的配置预设中指定的 installDir(如果 installDir 字段为空,则使用默认位置):

\begin{shell}
$ cmake --build --preset myInstall
[0/1] Install the project...
-- Install configuration: ""
-- Installing: .../install/include/calc/basic.h
-- Installing: .../install/lib/libcalc_shared.so
-- Installing: .../install/lib/libcalc_static.a
-- Installing: .../install/lib/calc/cmake/CalcLibrary.cmake
-- Installing: .../install/lib/calc/cmake/CalcLibrary-noconfig.cmake
-- Installing: .../install/lib/calc/cmake/CalcConfig.cmake
-- Installing: .../install/bin/calc_console
-- Set non-toolchain portion of runtime path of ".../install/bin/calc_
console" to ""
\end{shell}

这个巧妙的方法可以帮助我们节省一些步骤。如果我们可以为最终用户提供一个单一命令,从配置到安装全部处理好,那就更好了。确实可以做到,通过工作流预设。让我们来看看。

