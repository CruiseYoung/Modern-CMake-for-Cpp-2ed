CMake 在项目的顶层目录中搜索 CMakePresets.json 和 CMakeUserPresets.json 文件。这两个文件使用相同的 JSON 结构来定义预设,因此它们之间的区别不大。其格式是一个 JSON 对象,包含以下键:

\begin{itemize}
\item
version:  这是一个必需的整数,指定了预设 JSON 架构的版本。

\item
cmakeMinimumRequired: 这是一个对象,指定了所需的 CMake 版本。

\item
include:这是一个字符串数组,它从数组中提供的文件路径包含外部预设(自第 4 版架构开始)。

\item
configurePresets: 这是一个对象数组,定义了配置阶段的预设。

\item
buildPresets: 这是一个对象数组,定义了构建阶段的预设。

\item
testPresets: 这是一个对象数组,专门针对测试阶段的预设。

\item
packagePresets: 这是一个对象数组,专门针对打包阶段的预设。

\item
workflowPresets: 这是一个对象数组,专门针对工作流模式的预设。

\item
vendor: 这是一个对象,包含由 IDE 和其他供应商定义的自定义设置;CMake 不处理这个字段。
\end{itemize}

在编写预设时,CMake 要求存在 version 入口;其他值则是可选的。下面是一个预设文件的例子(实际的预设将在后续部分添加):

\filename{ch16/01-presets/CMakePresets.json}

\begin{json}
{
    "version": 6,
    "cmakeMinimumRequired": {
        "major": 3,
        "minor": 26,
        "patch": 0
    },
    "include": [],
    "configurePresets": [],
    "buildPresets": [],
    "testPresets": [],
    "packagePresets": [],
    "workflowPresets": [],
    "vendor": {
        "data": "IDE-specific information"
    }
}
\end{json}

在上述例子中,并不需要添加空数组;除了 version 以外的条目都是可选的。顺便说一下,对于 CMake 3.26 的适当架构版本是 6。

既然我们已经了解了预设文件的结构,接下来我们就学习如何实际定义这些预设。
