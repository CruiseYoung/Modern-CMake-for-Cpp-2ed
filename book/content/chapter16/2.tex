项目的配置可能成为一个复杂的任务,特别是需要具体指定缓存变量、选定的生成器等元素时——尤其是当项目有多种构建方式的时候,这时预设就变得非常有用。不必记住命令行参数或是编写 shell 脚本,来以不同的参数运行 cmake,而是可以创建一个预设文件,并将所需的配置存储在项目中。

CMake 使用两个可选文件来存储项目预设:

\begin{itemize}
\item
CMakePresets.json: 由项目作者提供的官方预设。

\item
CMakeUserPresets.json: 专为希望向项目添加自定义预设的用户设计。项目应当将此文件添加到版本控制系统忽略列表中,以确保自定义设置不会共享到仓库中。
\end{itemize}

预设文件必须放置在项目的顶层目录中以便 CMake 能够识别。每个预设文件可以为每个阶段定义多个预设:配置 (configure)、构建 (build)、测试 (test)、打包 (package),以及涵盖多个阶段的工作流 (workflow) 预设。然后用户可以通过集成开发环境 (IDE)、图形用户界面 (GUI) 或者命令行选择并执行一个预设。

预设可以通过在命令行中添加 -{}-list-presets 参数来列出,具体取决于我们要列出的阶段。例如,列出构建预设可以用下面的命令:

\begin{shell}
cmake --build --list-presets
\end{shell}

列出测试预设可以用下面的命令:

\begin{shell}
ctest --list-presets
\end{shell}

要使用一个预设,需要遵循相同的模式,并在 -{}-preset 参数之后提供预设名称。

另外,不能使用 cmake 命令来列出打包预设,需要使用 cpack。这里是一个用于打包预设的命令行示例:

\begin{shell}
cpack --preset <preset-name>
\end{shell}

选择好预设后,还可以添加特定于阶段的命令行参数,例如指定构建树或者安装路径。添加的参数会覆盖预设中设置的内容。

工作流预设有特殊情况,运行 cmake 命令时如果存在 -{}-workflow 参数,则可以列出并应用工作流预设:

\begin{shell}
$ cmake --workflow --list-presets
Available workflow presets:
    "myWorkflow"
$ cmake --workflow --preset myWorkflow
Executing workflow step 1 of 4: configure preset "myConfigure"
...
\end{shell}

这就是如何在项目中应用和查看可用预设的方法。现在,让我们探索一下如何构建预设文件。

























