简单的编译场景通常由工具链的默认配置处理,或者由集成开发环境(IDE)直接提供。然而,在专业环境中,商业需求往往需要更高级的功能。可能是对更高性能、更小的二进制文件、更好的可移植性、自动化测试或广泛的调试功能的需求——不胜枚举。要以连贯且面向未来的方式管理所有这些需求,很快就会变得复杂而混乱(特别是在需要支持多个平台时)。

编译过程在C++书籍中通常没有解释得足够清楚(深入主题,如虚基类,似乎更有趣)。在本章中,我们将通过探讨编译的不同方面来纠正这一点:我们将了解编译是如何工作的,它的内部阶段是什么,以及它们如何影响二进制输出。

之后,我们将关注先决条件——我们将讨论可以使用哪些命令来微调编译过程,如何要求编译器具有特定功能,以及如何正确指导编译器处理哪些输入文件。

然后,我们将关注编译的第一阶段——预处理器。我们将提供包含头文件的路径,并且我们将学习如何通过预处理器定义将CMake和构建环境中的变量插入。我们将涵盖最有趣的使用案例,并学习如何暴露CMake变量,以便它们可以从C++代码中访问。

紧接着,我们将讨论优化器以及不同的标志如何影响性能。我们还将讨论优化的成本,特别是它如何影响产生的二进制的可调试性,以及如果不希望这样该怎么办。

最后,我们将解释如何通过使用预编译头文件和统一构建来管理编译过程,以减少编译时间。我们将学习如何调试构建过程并找出我们可能犯的任何错误。

在本章中,我们将包含以下主要内容:

\begin{itemize}
\item
编译的基础知识

\item
配置预处理器

\item
配置优化器

\item
管理编译过程
\end{itemize}






