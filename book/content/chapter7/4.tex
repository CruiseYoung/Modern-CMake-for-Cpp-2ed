
优化器将分析前几个阶段的结果,并使用多种策略,这些策略开发者通常不会直接使用,它们不符合干净代码的原则。但这没关系——优化器的基本作用是提高代码性能,力求降低CPU使用率、最小化寄存器使用和减少内存占用。当优化器遍历源代码时,针对目标CPU,会将代码大量变形为无法识别的形式。

优化器不仅会决定哪些函数可以删除或压缩;还会移动代码,甚至复制代码!如果明确确定某些代码多余,甚至会从重要函数的中间删除这些代码。通过回收内存,使得多个变量可以在不同时间占据同一个槽位。如果这样做可以减少几个周期的耗时,甚至可以将控制结构重塑为完全不同的形式。

如果开发者手动将这些技术应用于源码,代码将变成一个糟糕的、难以阅读的混乱,难以编写和理解。然而,因为编译器严格遵循提供的指令,所以当编译器应用这些技术时是有利的。优化器是一个不知疲倦的怪兽,只有一个目的:加速执行速度,不管输出变得多么扭曲。

每个编译器都有自己的独特技巧,这与它支持的平台和遵循的思想一致。我们将了解GNU GCC和LLVM Clang中最常见的一些,以清楚什么是实用的和可实现的。

问题是——许多编译器默认情况下不会启用优化(包括GCC)。当可以快时,为什么要慢下来?为了纠正这一点,可以使用target\_compile\_options()命令,明确声明我们对编译器的期望。

这个命令的语法与本章中的其他命令类似:

\begin{shell}
target_compile_options(<target> [BEFORE]
                       <INTERFACE|PUBLIC|PRIVATE> [items1...]
                      [<INTERFACE|PUBLIC|PRIVATE> [items2...]
...])
\end{shell}

我们提供在构建目标时使用的命令行选项,还指定传播关键字。执行时,CMake会将给定的选项附加到目标的适当COMPILE\_OPTIONS变量中。如果要前置,可以使用可选的BEFORE关键字。某些情况下,顺序很重要。

注意,target\_compile\_options()是一个通用命令,还可以用来提供其他类似编译器的参数,例如-D定义,CMake为此提供了target\_compile\_definition()命令。始终建议尽可能使用最专业的CMake命令,以保证在所有支持的编译器上以相同的方式工作。

\mySubsubsection{7.4.1.}{通用级别}

优化器的所有不同行为,都可以通过特定的标志配置,可以作为编译选项传递。了解所有这些标志需要花费大量时间,并且需要对编译器、处理器和内存的内部有深入的了解。如果只想得到在大多数情况下都很好的最佳方案,能做什么呢?可以寻求一个通用解决方案——优化级别指定器。

大多数编译器提供从0到3的四个基本优化级别,使用-O<级别>选项指定。-O0表示没有优化,通常它是编译器的默认级别。另一方面,-O2认为是完全优化,会生成高度优化的代码,但代价是编译时间最慢。

还有一个介于两者之间的-O1级别,可以是一个很好的折衷方案——启用了一定数量的优化机制,而不会过度减慢编译速度。

最后,可以使用-O3,是与-O2相同的完全优化,但对子程序内联和循环向量化采取了更激进的方法。

还有一些优化变体,针对生成文件的大小(不一定是速度)进行优化——-Os。有一种超级积极的优化,-Ofast,是一个不严格遵循C++标准的-O3优化。最明显的区别是使用-ffast-math和-ffinite-math标志,如果程序是关于精确计算(大多数是),最好别使用。

CMake知道并非所有编译器都相同,为了给开发者提供标准化的体验,其为编译器提供了一些默认标志。它们存储在系统范围(非目标特定)变量中,根据使用的语言(C++为CXX)和构建配置(DEBUG或RELEASE),变量名为:

\begin{itemize}
\item
CMAKE\_CXX\_FLAGS\_DEBUG 等于 -g

\item
CMAKE\_CXX\_FLAGS\_RELEASE 等于 -O3 -DNDEBUG
\end{itemize}

调试配置不启用优化,而发布配置直接使用-O3。可以使用set()命令直接更改它们,或者只添加一个目标编译选项,这将覆盖此默认行为。其他两个标志(-g,-DNDEBUG)与调试有关。

诸如CMAKE\_<LANG>\_FLAGS\_<CONFIG>这样的变量是全局的——适用于所有目标。建议通过属性和命令(如target\_compile\_options())配置目标,而不是依赖全局变量。这样,可以更细致地控制目标。

通过选择带有-O<级别的>优化级别,间接的设置了一系列标志,每个标志控制特定的优化行为。然后,可以通过更多标志来微调优化:

\begin{itemize}
\item
使用-f选项启用 -finline-functions.

\item
使用-fno选项禁用 -fno-inline-functions.
\end{itemize}

这些标志中的一些值得更好地理解,因为会影响到程序的工作方式,以及如何调试。

\mySubsubsection{7.4.2.}{函数内联}

可以通过在类声明块内定义函数或显式使用inline关键字,鼓励编译器内联某些函数:

\begin{cpp}
struct X {
    void im_inlined(){ cout << "hi\n"; };
    void me_too();
};
inline void X::me_too() { cout << "bye\n"; };
\end{cpp}

是否内联一个函数最终由编译器决定。如果启用了内联,并且函数在单个地方使用(或者相对较小的函数使用),则很可能会发生内联。

函数内联是一种有趣的优化技术,通过从目标函数中提取代码,并将其嵌入到函数调用的所有位置来操作。这个过程替换了原始调用,并节省了宝贵的CPU周期。

使用我刚刚定义的类:

\begin{cpp}
int main() {
    X x;
    x.im_inlined();
    x.me_too();
    return 0;
}
\end{cpp}

如果没有内联,代码将在main()中执行,直到方法调用。然后,将为im\_inlined()创建一个新的帧,在单独的作用域中执行,然后返回到main()。对于me\_too()方法也会发生同样的情况。

然而,当发生内联时,编译器将替换调用,如下所示:

\begin{cpp}
int main() {
    X x;
    cout << "hi\n";
    cout << "bye\n";
    return 0;
}
\end{cpp}

这并不是一个精确的表现,因为内联发生在汇编或机器代码级别(而不是源代码级别),但它确实提供了一个大致的表现。

编译器使用内联来节省时间,绕过了创建和拆除新调用的需要,以及查找下一个要执行的指令的地址(并返回)的需求,并由于在接近的位置,从而增强了指令缓存。

然而,内联确实带来了一些显著的副作用。如果一个函数多次使用,必须复制到所有位置,导致文件大小增大和内存使用增加,特别是在为内存有限的低端设备开发软件时,需要注意。

此外,内联对调试产生了严重影响。内联代码不再位于原始行号,使得跟踪变得更加困难,有时甚至不可能。这就是为什么在内联的函数中放置的调试器断点,永远不会命中(尽管代码仍然以某种方式执行)。为了绕过这个问题,需要为调试构建禁用内联(以不测试确切发布构建版本为代价)。

可以通过为目标指定-O0(零级别),或直接处理负责内联的标志来实现这一点:

\begin{itemize}
\item
-finline-functions-called-once: 适用于GCC。

\item
-finline-functions: 适用于Clang和GCC。

\item
-finline-hint-functions: 适用于Clang。
\end{itemize}

可以使用-fno-inline-…明确禁用内联,为了详细信息,建议参考特定编译器的文档版本。

\mySubsubsection{7.4.3.}{循环展开}

循环展开是一种优化技术,这种策略旨在将循环转换为一系列完成相同结果的语句。这种方法以程序的小尺寸换取执行速度,消除了循环控制指令、指针算术和循环结束检查。

看一下示例:

\begin{cpp}
void func() {
    for(int i = 0; i < 3; i++)
    cout << "hello\n";
}
\end{cpp}

之前的代码将变成如下形式:

\begin{cpp}
void func() {
    cout << "hello\n";
    cout << "hello\n";
    cout << "hello\n";
}
\end{cpp}

结果保持不变,但不再需要分配i变量,也不需要对其进行递增或与值3进行比较三次。如果在程序的生命周期中足够多次调用func(),即使是这样一个短小和简单的函数,其展开也会产生显著的差异。

重要的是要理解两个限制因素。首先,循环展开只有在编译器知道或能够准确估计迭代次数时才有效。其次,循环展开可能会在现代CPU上产生不希望的结果,因为代码大小的增加可能会阻碍缓存命中。

每个编译器提供了这个标志的略有不同的版本:

\begin{itemize}
\item
-floop-unroll: GCC版本。

\item
-funroll-loops: Clang版本。
\end{itemize}

如果不确定,请广泛测试这个标志是否影响特定程序,并明确地启用或禁用。GCC上,与隐式启用的-O3一起,作为-floop-unroll-and-jam标志的一部分启用。

\mySubsubsection{7.4.4.}{循环向量化}

称为单指令多数据(SIMD)的机制在20世纪60年代初为实现并行而开发。正如其名称,旨在同时对多个数据执行相同的操作。通过以下示例来了解一下:

\begin{cpp}
int a[128];
int b[128];
// initialize b
for (i = 0; i<128; i++)
    a[i] = b[i] + 5;
\end{cpp}

这样的代码会循环128次,但如果CPU能力强大,代码的执行可以通过同时计算两个或更多的数组元素而显著加速。这是由于连续元素之间不存在依赖关系,以及数组之间的数据重叠。聪明的编译器可以将前一个循环转换为类似以下形式(这发生在汇编级别):

\begin{cpp}
for (i = 0; i<32; i+=4) {
    a[ i ] = b[ i ] + 5;
    a[i+1] = b[i+1] + 5;
    a[i+2] = b[i+2] + 5;
    a[i+3] = b[i+3] + 5;
}
\end{cpp}

GCC将在-O3级别启用这样的循环向量化,Clang默认启用。两个编译器都提供不同的标志来启用/禁用特定情况下的向量化:

\begin{itemize}
\item
-ftree-vectorize -ftree-slp-vectorize: GCC中启用向量化

\item
-fno-vectorize -fno-slp-vectorize:Clang中禁用向量化
\end{itemize}

向量化的效率来自于CPU制造商提供的特殊指令的利用,而不是仅仅将循环的原始形式替换为展开版本。

优化器在提高程序的运行时性能方面起着关键作用。通过有效地运用其策略,可以得到更多的价值。效率不仅在于编码完成之后,而且在软件开发过程中也至关重要。如果编译时间过长,可以通过更好地管理过程来改善。














