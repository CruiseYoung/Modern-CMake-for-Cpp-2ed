
作为程序员和构建工程师,还必须考虑编译的其他方面,比如完成编译所需的时间,以及解决方案构建过程中识别和纠正错误的速度。

\mySubsubsection{7.5.1.}{减少编译时间}

需要频繁重新编译的项目(可能每小时多次)中,确保编译过程尽可能快是至关重要的。这不仅影响代码-编译-测试循环的效率,还会影响专注度和工作流程。

由于单独的翻译单元,C++在管理编译时间方面已经做得很好了。CMake将仅重新编译受最近更改影响的源文件。如果需要进一步改进,可以使用一些技术:预编译头文件和统一构建。

\mySamllsection{预编译头文件}

头文件(.h)在编译开始前由预处理器包含在翻译单元中,所以每次.cpp实现文件更改时,都必须重新编译。此外,如果多个翻译文件使用相同的共享头文件,每次包含时都必须编译。这种方式效率低下,但长期以来一直是标准做法。

自从3.16版本以来,CMake提供了一个命令来启用头文件预编译。这允许编译器单独处理头文件,从而加快编译过程。以下是该命令的语法:

\begin{shell}
target_precompile_headers(<target>
                          <INTERFACE|PUBLIC|PRIVATE> [header1...]
                         [<INTERFACE|PUBLIC|PRIVATE> [header2...]
...])
\end{shell}

添加的头文件列表存储在PRECOMPILE\_HEADERS目标属性中。我们可以通过选择PUBLIC或INTERFACE关键字使用传播属性将头文件与任何依赖目标共享,但不应该对使用install()命令导出的目标执行此操作。

\begin{myNotic}{Note}
使用第6章中的\$<BUILD\_INTERFACE:…>生成器表达式,避免预编译头文件出现在安装时。然而,仍然可能通过export()命令从构建树导出的目标添加。如果现在这似乎令人困惑 - 不要担心,这将在第14章中进行介绍。
\end{myNotic}

CMake将所有头文件的名称放入cmake\_pch.h或cmake\_pch.hxx文件中,然后将其预编译为具有.pch、.gch或.pchi扩展名的编译器特定二进制文件。

可以在列表文件中这样使用:

\filename{ch07/06-precompile/CMakeLists.txt}

\begin{cmake}
add_executable(precompiled hello.cpp)
target_precompile_headers(precompiled PRIVATE <iostream>)
\end{cmake}

也可以在相应的源文件中使用:

\filename{ch07/06-precompile/hello.cpp}

\begin{cmake}
int main() {
    std::cout << "hello world" << std::endl;
}
\end{cmake}

main.cpp文件中,不需要包含cmake\_pch.h或其他头文件 - CMake会使用编译器特定的命令行选项包含。

前面的示例中,使用了一个内置头文件;然而,可以轻松地添加自己的带有类或函数定义的头文件。使用以下两种形式之一引用头文件:

\begin{itemize}
\item
header.h (直接路径)将解析为相对于当前源目录的路径,并将使用绝对路径包含。

\item
使用[["header.h"]](双括号和引号)时,路径将根据目标的INCLUDE\_DIRECTORIES属性进行扫描,该属性可以使用target\_include\_directiories()进行配置。
\end{itemize}

一些在线参考资料可能不鼓励预编译不是标准库(如)一部分的头文件,或者根本不使用预编译头文件。这是因为更改列表或编辑自定义头文件,将导致目标中的所有翻译单元重新编译。使用CMake,这个问题并不严重,特别是已经正确地结构化项目(具有相对较小的目标,专注于狭窄的领域)。每个目标都有一个单独的预编译头文件,这限制了头文件更改的影响。

如果头文件相对稳定,可能会决定在目标中重用预编译头文件。为此,CMake提供了一个方便的命令:

\begin{shell}
target_precompile_headers(<target> REUSE_FROM <other_target>)
\end{shell}

这将设置重用头文件的目标的PRECOMPILE\_HEADERS\_REUSE\_FROM属性,并在这些目标之间创建依赖关系。使用此方法,使用目标不能再指定自己的预编译头文件。此外,所有编译选项、编译标志和编译定义必须在目标之间匹配。

注意,特别是头文件使用双括号格式([[“header.h”]])。两个目标都需要适当地设置它们的包含路径,以确保编译器能找到这些头文件。

\mySamllsection{统一构建}

CMake 3.16引入了另一个编译时间优化功能——统一构建,也称为统一构建或超级构建。统一构建通过利用\#include指令将多个实现源文件组合在一起。这带来了一些有趣的后果,其中一些有益,而其他一些可能有害。

最明显的优势是在CMake创建统一构建文件时,会避免在不同翻译单元中重新编译头文件:

\begin{cpp}
#include "source_a.cpp"
#include "source_b.cpp"
\end{cpp}

当两个源文件都包含一个\#include "header.h"行时,引用的文件将只解析一次。虽然这不如预编译头文件那样精细,但它是一种替代方案。

这种构建类型的第二个好处是,优化器现在可以在更大的范围内进行操作,并对捆绑的所有源文件中的跨过程调用进行优化。这与第4章中讨论的链接时间优化类似。

然而,这些好处是有代价的。随着减少了对象文件和处理步骤的数量,也增加了处理更大文件所需的内存量。此外,减少了可并行工作的量。编译器并不是特别擅长多线程编译,通常不需要——构建系统通常会启动许多编译任务,以便在不同的线程上同时执行所有文件。将所有文件组合在一起会复杂化这个问题,因为CMake现在可以并行编译的文件更少了。

统一构建中,还需要考虑一些C++语义上的影响,这些影响可能并不那么明显——匿名命名空间隐藏跨文件的符号现在会限定在统一文件中,而不是单个翻译单元。同样,静态全局变量、函数和宏定义也会发生这种情况。这可能导致名称冲突,或者执行错误的函数重载。

重新编译时,统一构建的性能并不理想,因为会编译许多不必要的文件。代码需要尽可能快地编译所有文件时表现最佳。在Qt Creator(一个流行的GUI库)上进行的测试显示,可以期待性能提升在20\%到50\%之间(取决于使用的编译器)。

要启用统一构建,有两种选择:

\begin{itemize}
\item
将CMAKE\_UNITY\_BUILD变量设置为true——将在定义的所有目标上初始化UNITY\_BUILD属性。

\item
手动将UNITY\_BUILD目标属性设置为true,适用于应该使用统一构建的所有目标。
\end{itemize}

第二个选项通过以下方式实现:

\begin{shell}
set_target_properties(<target1> <target2> ...
                      PROPERTIES UNITY_BUILD true)
\end{shell}

当然,手动设置许多目标的这些属性会增加工作量和维护成本,但可能需要这样做以在更精细的层面上控制这个设置。

默认情况下,CMake将创建包含八个源文件的构建,这是通过目标的目标属性UNITY\_BUILD\_BATCH\_SIZE指定的(创建目标时从CMAKE\_UNITY\_BUILD\_BATCH\_SIZE变量复制),可以更改目标属性或默认变量。

从版本3.18开始,可以显式定义文件应该如何分组,并为它们命名。为此,请将目标的UNITY\_BUILD\_MODE属性更改为GROUP(默认是BATCH)。然后,通过设置它们的UNITY\_GROUP属性为选择的名称分组源文件:

\begin{shell}
set_property(SOURCE <src1> <src2> PROPERTY UNITY_GROUP "GroupA")
\end{shell}

CMake将忽略UNITY\_BUILD\_BATCH\_SIZE,并将所有来自该组的文件添加到一个统一的构建中。

Make的文档建议不要为公共项目默认启用统一构建。建议应用程序的最终用户通过提供-DCMAKE\_UNITY\_BUILD命令行参数来决定否想要统一构建。如果代码编写方式导致统一构建出现问题,应该明确地将目标的属性设置为false。然而,对于内部使用的代码(如在公司内部或私人项目中),可以启用这个特性。

这些是为了CMake减少编译时间。编程的其他方面往往也会花费我们很多时间——其中耗时最多的就是调试。

\mySubsubsection{7.5.2.}{发现错误}

作为开发者,我们花大量时间寻找bug。识别错误并纠正它们的过程往往令人沮丧,尤其是当它需要长时间时。当我们没有工具来应对这些挑战性的情况时,难度会进一步增加。因此,必须要非常关注环境设置,以便简化这个过程,使其尽可能容易和可承受。一种实现这一目标的方法是通过配置编译器使用target\_compile\_options()。那么,哪些编译选项可以帮助我们呢?

\mySamllsection{配置错误和警告}

软件开发中有许多令人压力大的事情——半夜修复关键bug,在大系统中处理高可见性、成本高昂的失败,以及处理恼人的编译错误。有些错误很难理解,而另一些则需要艰苦努力来修复。简化工作并减少失败的机会的探索中,会发现许多关于如何配置编译器警告的建议。

一个很好的建议是默认为所有构建启用-Werror标志。这个标志的功能表面上很简单——将所有的警告视为错误,阻止代码编译,直到解决。虽然这可能看起来是一个有益的方法,但很少是这样。

警告没有分类为错误是有原因的:是为了警告你。决定如何处理这些警告取决于你。当在实验或原型解决方案时,忽略一个警告的权限往往无价。

另一方面,如果代码是完美的、没有警告,那么允许未来的修改破坏这个完美的状态似乎是有遗憾的。开启它并让它保持在那里有什么害处?至少在编译器升级之前,似乎没有什么害处。新版本的编译器往往对过时的特性更严格,或者更擅长提供改进建议。虽然当警告保持为警告时是有益的,但当需要快速修复与新警告无关的问题时,可能导致未更改的代码出现意外的构建失败。

那么,什么时候允许启用所有可能的警告呢?答案是当你创建一个公共库时。在这种情况下,预先阻止问题,指出代码在比开发环境更严格的环境中可能会行为不当。如果选择启用这个设置,确保更新对新编译器版本引入的警告。此外,明确管理这个更新过程也很重要,这可以与任何代码更改分开处理。

否则,让警告保持原样,专注于错误。如果觉得有必要追求完美,可以使用-Wpedantic标志。这个特定的标志启用了由严格的ISO C和ISO C++标准要求的警告。然而,这个标志并不确认符合标准,只是标识需要诊断消息的非ISO实践。

更宽松、更接地气的程序员会满足于-Wall,可以选择与-Wextra结合,增加一点复杂性,这应该就足够了。这些警告认为是真正有用的,当时间允许时,应该在代码中处理它们。

根据项目类型,可能还有其他一些警告标志有帮助。我建议阅读编译器的手册,看看有哪些选项可用。

\mySamllsection{调试构建}

偶尔,编译会失败。这通常发生在我们尝试重构大量代码或清理构建系统时。有时,问题可以轻松解决;然而,有些更复杂的问题需要深入调查配置步骤。我们已经在第1章中讨论了如何打印更详细的CMake输出,但如何分析在每个阶段实际发生了什么?

\mySamllsubsection{调试各个阶段}

-save-temps,可以传递给GCC和Clang编译器,允许调试编译的各个阶段。这个标志将指示编译器将某些编译阶段的输出存储在文件中,而不是内存中。

\filename{ch07/07-debug/CMakeLists.txt}

\begin{cmake}
add_executable(debug hello.cpp)
target_compile_options(debug PRIVATE -save-temps=obj)
\end{cmake}

启用这个选项将产生两个文件(.ii和.s)每个翻译单元。

第一个,<build-tree>/CMakeFiles/<target>.dir/<source>.ii,存储了预处理阶段的输出,其中包含注释,解释了源代码的每个部分来自哪里:

\begin{shell}
# 1 "/root/examples/ch07/06-debug/hello.cpp"
# 1 "<built-in>"
# 1 "<command-line>"
# 1 "/usr/include/stdc-predef.h" 1 3 4
# / / / ... removed for brevity ... / / /
# 252 "/usr/include/x86_64-linux
  gnu/c++/9/bits/c++config.h" 3
namespace std
{
    typedef long unsigned int size_t;
    typedef long int ptrdiff_t;
    typedef decltype(nullptr) nullptr_t;
}
...
\end{shell}

第二个,<build-tree>/CMakeFiles/<target>.dir/<source>.s, 包含了语言分析阶段的输出,准备进入汇编阶段:

\begin{shell}
        .file "hello.cpp"
        .text
        .section .rodata
        .type _ZStL19piecewise_construct, @object
        .size _ZStL19piecewise_construct, 1
_ZStL19piecewise_construct:
        .zero 1
        .local _ZStL8__ioinit
        .comm _ZStL8__ioinit,1,1
.LC0:
        .string "hello world"
        .text
        .globl main
        .type main, @function
main:
( ... )
\end{shell}

根据问题的类型,通常可以揭示实际的问题。例如,预处理器的输出可以识别bug,如错误的include路径(可能提供错误的库版本),或者定义错误导致的错误的\#ifdef评估。

同时,语言分析的输出对于针对特定处理器和解决关键优化问题特别有益。

\mySamllsubsection{调试头文件包含问题}

调试错误包含的文件可能是一项具有挑战性的任务。我应该知道——在第一家公司工作时,我需要将整个代码库从一个构建系统移植到另一个构建系统。如果发现自己处于需要精确理解请求头文件包含路径的情况,可以考虑使用-H编译选项:

\filename{ch07/07-debug/CMakeLists.txt}

\begin{cmake}
add_executable(debug hello.cpp)
target_compile_options(debug PRIVATE -H)
\end{cmake}

产生的输出将类似于以下内容:

\begin{shell}
[ 25%] Building CXX object
  CMakeFiles/inclusion.dir/hello.cpp.o
. /usr/include/c++/9/iostream
.. /usr/include/x86_64-linux-gnu/c++/9/bits/c++config.h
... /usr/include/x86_64-linux-gnu/c++/9/bits/os_defines.h
.... /usr/include/features.h
-- removed for brevity --
.. /usr/include/c++/9/ostream
\end{shell}

对象文件名称之后,输出中的每一行都包含一个头文件的路径。这个例子中,行首的一个点表示顶级包含(包含指令在hello.cpp中)。两个点表示该文件被后续文件(iostream)包含。每个额外的点表示嵌套的另一个级别。

输出的末尾,可能会找到一些代码改进的建议:

\begin{shell}
Multiple include guards may be useful for:
/usr/include/c++/9/clocale
/usr/include/c++/9/cstdio
/usr/include/c++/9/cstdlib
\end{shell}

虽然不需要处理标准库中的问题,但可能会看到一些自己的头文件列出。在这种情况下,再考虑进行修改。

\mySamllsection{为调试器提供信息}

机器代码是一系列指令和数据的神秘列表,这些指令和数据以二进制格式编码,不传达任何更大的意义或目标。这是因为CPU不在乎程序的目标是什么,或者所有指令的含义是什么。唯一的要求是代码的正确性。编译器会将所有这些内容转换为CPU指令的数字标识符,并在需要时存储数据以初始化内存,并提供成千上万的内存地址。换句话说,最终的二进制文件不需要包含实际的源代码、变量名、函数签名或其他程序员关心的细节。这是编译器的默认输出——原始而简单。

这样做是为了节省空间,并执行时避免过多的开销。巧合的是,我们也以某种方式保护了应用程序免受逆向工程的影响。是的,可以理解每个CPU指令做什么,而无需源代码(将这个值复制到那个寄存器)。

如果你是一个特别有动力的人,可以使用一个叫做反汇编器的工具,并且凭借大量的知识(以及一些运气),将能够解码可能发生的事情。但这种方法并不实用,因为反汇编代码没有原始符号,这使得很难和慢地弄清楚它们去了哪里。

相反,可以要求编译器将源代码存储在生成的二进制文件中,并与编译后的代码和原始代码之间的引用映射一起。然后,可以将调试器附加到一个正在运行的程序上,并查看在任何给定的时刻执行的是哪一行源代码。当编写新功能或更正错误时,这必不可少。

这两个用例是存在两种构建配置的原因:Debug和Release。CMake会默认向编译器提供一些标志来管理这个过程,首先存储在全局变量中:

\begin{itemize}
\item
CMAKE\_CXX\_FLAGS\_DEBUG 包含 -g

\item
CMAKE\_CXX\_FLAGS\_RELEASE 包含 -DNDEBUG
\end{itemize}

-g标志的意思是“添加调试信息”,以操作系统的原生格式提供:stabs、COFF、XCOFF或DWARF。这些格式随后可以被像gdb(GNU调试器)这样的调试器访问。通常,这对像CLion这样的IDE来说已经足够了(后台使用gdb)。其他情况下,请参考提供的调试器手册,并了解选择的编译器的适当标志。

对于Release配置,CMake将添加-DNDEBUG标志。这是一个预处理器定义——“非调试构建”,这个选项会禁用一些面向调试的宏。其中之一是assert,在<assert.h>头文件中可用。如果在生产代码中使用断言,其将不起作用:

\begin{cpp}
int main(void)
{
    assert(false); // blod
    std::cout << "This shouldn't run. \n";
    return 0;
}
\end{cpp}

在Release配置中,assert(false)调用不会有任何效果,但在Debug中它会停止执行。如果在实践断言性编程,仍需要在Release构建中使用assert(),可以改变CMake提供的默认设置(从CMAKE\_CXX\_FLAGS\_RELEASE中移除NDEBUG),或者通过在包含头文件之前取消定义宏来实现硬编码覆盖:

\begin{cpp}
#undef NDEBUG
#include <assert.h>
\end{cpp}

更多信息请参阅assert参考:\url{https://en.cppreference.com/w/c/error/assert}。

如果断言可以在编译时完成,可以考虑用C++11引入的static\_assert()替换assert(),因为这个函数不像assert()那样受到\#ifndef(NDEBUG)预处理器指令的保护。

通过这些,我们已经了解了如何管理编译过程。











