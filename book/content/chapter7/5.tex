
作为程序员和构建工程师,我们还必须考虑编译的其他方面,比如完成编译所需的时间以及我们在解决方案构建过程中识别和纠正错误的速度。

\mySubsubsection{7.5.1.}{减少编译时间}

在需要频繁重新编译的项目(可能每小时多次)中,确保编译过程尽可能快是至关重要的。这不仅影响你的代码-编译-测试循环的效率,还会影响你的专注度和工作流程。

幸运的是,由于单独的翻译单元,C++在管理编译时间方面已经做得很好了。CMake将仅重新编译受最近更改影响的源文件。然而,如果我们需要进一步改进,可以使用一些技术:头文件预编译和统一构建。

\mySamllsection{头文件预编译}

头文件(.h)在编译开始前由预处理器包含在翻译单元中。这意味着每次.cpp实现文件更改时,它们都必须重新编译。此外,如果多个翻译文件使用相同的共享头文件,每次包含时都必须编译它。这种方式效率低下,但长期以来一直是标准做法。

幸运的是,自从3.16版本以来,CMake提供了一个命令来启用头文件预编译。这允许编译器单独处理头文件,从而加快编译过程。以下是该命令的语法:

\begin{shell}
target_precompile_headers(<target>
                          <INTERFACE|PUBLIC|PRIVATE> [header1...]
                         [<INTERFACE|PUBLIC|PRIVATE> [header2...]
...])
\end{shell}

添加的头文件列表存储在PRECOMPILE\_HEADERS目标属性中。正如我们在第5章“使用目标”的“What are transitive usage requirements?”部分讨论的那样,我们可以通过选择PUBLIC或INTERFACE关键字使用传播属性将头文件与任何依赖目标共享;但是,不应该对使用install()命令导出的目标执行此操作。其他项目不应该被迫使用我们的预编译头文件,因为这不是常规做法。

\begin{myNotic}{Note}
使用第6章“使用生成器表达式”中描述的\$<BUILD\_INTERFACE:…>生成器表达式,防止预编译头文件在安装时出现在目标的用法要求中。然而,它们仍然会被通过export()命令从构建树导出的目标添加。如果现在这似乎令人困惑 - 不要担心,这将在第14章“安装和打包”中完全解释。
\end{myNotic}

CMake将所有头文件的名称放入cmake\_pch.h或cmake\_pch.hxx文件中,然后将其预编译为具有.pch、.gch或.pchi扩展名的编译器特定二进制文件。

我们可以在列表文件中这样使用它:

\filename{ch07/06-precompile/CMakeLists.txt}

\begin{cmake}
add_executable(precompiled hello.cpp)
target_precompile_headers(precompiled PRIVATE <iostream>)
\end{cmake}

我们也可以在相应的源文件中使用它:

\filename{ch07/06-precompile/hello.cpp}

\begin{cmake}
int main() {
    std::cout << "hello world" << std::endl;
}
\end{cmake}

请注意,在我们的main.cpp文件中,我们不需要包含cmake\_pch.h或任何其他头文件 - CMake会使用编译器特定的命令行选项包含它。

在前面的示例中,我使用了一个内置头文件;然而,你可以轻松地添加你自己的带有类或函数定义的头文件。使用以下两种形式之一引用头文件:

\begin{itemize}
\item
header.h 直接路径)将被解释为相对于当前源目录的路径,并将使用绝对路径包含。

\item
[["header.h"]](双括号和引号)路径将根据目标的INCLUDE\_DIRECTORIES属性进行扫描,该属性可以使用target\_include\_directiories()进行配置。
\end{itemize}

一些在线参考资料可能不鼓励预编译不是标准库(如)一部分的头文件,或者根本不使用预编译头文件。这是因为更改列表或编辑自定义头文件将导致目标中的所有翻译单元重新编译。使用CMake,这个问题并不那么严重,特别是如果你正确地结构化你的项目(具有相对较小的目标,专注于狭窄的领域)。每个目标都有一个单独的预编译头文件,这限制了头文件更改的影响。

如果你的头文件相对稳定,你可能会决定在你的目标中重用预编译头文件。为此,CMake提供了一个方便的命令:

\begin{shell}
target_precompile_headers(<target> REUSE_FROM <other_target>)
\end{shell}

这将设置重用头文件的目标的PRECOMPILE\_HEADERS\_REUSE\_FROM属性,并在这些目标之间创建依赖关系。使用此方法,消费目标不能再指定自己的预编译头文件。此外,所有编译选项、编译标志和编译定义必须在目标之间匹配。

注意要求,特别是如果你的头文件使用双括号格式([[“header.h”]])。两个目标都需要适当地设置它们的包含路径,以确保这些头文件被编译器找到。

\mySamllsection{统一构建}

CMake 3.16引入了另一个编译时间优化功能——统一构建,也称为统一构建或超大构建。统一构建通过利用\#include指令将多个实现源文件组合在一起。这带来了一些有趣的含义,其中一些是有益的,而其他一些可能是有害的。

最明显的优势是在CMake创建统一构建文件时避免在不同翻译单元中重新编译头文件:

\begin{cpp}
#include "source_a.cpp"
#include "source_b.cpp"
\end{cpp}

当两个源文件都包含一个\#include "header.h"行时,引用的文件将只被解析一次,感谢include guards(假设它们已经正确添加)。虽然这不如预编译头文件那样精细,但它是一种替代方案。

这种构建类型的第二个好处是,优化器现在可以在更大的范围内进行操作,并对捆绑的所有源文件中的跨过程调用进行优化。这与我们在第4章“设置你的第一个CMake项目”中讨论的链接时间优化类似。

然而,这些好处是有代价的。随着我们减少了对象文件和处理步骤的数量,我们也增加了处理更大文件所需的内存量。此外,我们减少了可并行工作的量。编译器并不是特别擅长多线程编译,因为它们通常不需要——构建系统通常会启动许多编译任务,以便在不同的线程上同时执行所有文件。将所有文件组合在一起会复杂化这个问题,因为CMake现在可以并行编译的文件更少了。

在统一构建中,你还需要考虑一些C++语义上的影响,这些影响可能并不那么明显——匿名命名空间隐藏跨文件的符号现在被限定在统一文件中,而不是单个翻译单元。同样,静态全局变量、函数和宏定义也会发生这种情况。这可能导致名称冲突,或者执行错误的函数重载。

在重新编译时,超大构建的性能并不理想,因为它们会编译许多不必要的文件。它们在代码需要尽可能快地编译所有文件时表现最佳。在Qt Creator(一个流行的GUI库)上进行的测试显示,你可以期待性能提升在20\%到50\%之间(取决于使用的编译器)。

要启用统一构建,我们有两种选择:

\begin{itemize}
\item
将CMAKE\_UNITY\_BUILD变量设置为true——它将在定义的所有目标上初始化UNITY\_BUILD属性。

\item
手动将UNITY\_BUILD目标属性设置为true,适用于应该使用统一构建的所有目标。
\end{itemize}

第二个选项是通过以下方式实现的:

\begin{shell}
set_target_properties(<target1> <target2> ...
                      PROPERTIES UNITY_BUILD true)
\end{shell}

当然,手动设置许多目标的这些属性会增加工作量和维护成本,但你可能需要这样做以在更精细的层面上控制这个设置。

默认情况下,CMake将创建包含八个源文件的构建,这是通过目标的目标属性UNITY\_BUILD\_BATCH\_SIZE指定的(在创建目标时从CMAKE\_UNITY\_BUILD\_BATCH\_SIZE变量复制)。你可以更改目标属性或默认变量。

从版本3.18开始,你可以显式定义文件应该如何分组,并为它们命名。为此,请将目标的UNITY\_BUILD\_MODE属性更改为GROUP(默认是BATCH)。然后,通过设置它们的UNITY\_GROUP属性为你选择的名称来将源文件分配给组:

\begin{shell}
set_property(SOURCE <src1> <src2> PROPERTY UNITY_GROUP "GroupA")
\end{shell}

CMake将忽略UNITY\_BUILD\_BATCH\_SIZE,并将所有来自该组的文件添加到一个统一的构建中。

Make的文档建议不要为公共项目默认启用统一构建。建议应用程序的最终用户能够通过提供-DCMAKE\_UNITY\_BUILD命令行参数来决定他们是否想要超大构建。如果你的代码编写方式导致统一构建出现问题,你应该明确地将目标的属性设置为false。然而,对于内部使用的代码(如在公司内部或你的私人项目中),你可以自由启用这个特性。

这些是使用CMake减少编译时间的最重要的方面。编程的其他方面往往也会花费我们很多时间——其中最臭名昭著的是调试。 让我们看看我们如何改进这一点。

\mySubsubsection{7.5.2.}{发现错误}

作为程序员,我们花大量时间寻找bug。不幸的是,这是我们的职业事实。识别错误并纠正它们的过程往往令人沮丧,尤其是当它需要长时间时。当我们没有必要的工具来帮助我们导航这些挑战性的情况时,难度会进一步增加。因此,我们必须要非常关注我们的环境设置,以便简化这个过程,使其尽可能容易和可承受。一种实现这一目标的方法是通过配置编译器使用target\_compile\_options()。那么,哪些编译选项可以帮助我们呢?

\mySamllsection{配置错误和警告}

软件开发中有许多令人压力大的事情——在半夜修复关键bug,在大系统中处理高可见性、成本高昂的失败,以及处理恼人的编译错误。有些错误很难理解,而另一些则需要艰苦努力来修复。在简化你的工作并减少失败的机会的探索中,你会发现许多关于如何配置编译器警告的建议。

一个很好的建议是默认为所有构建启用-Werror标志。这个标志的功能表面上很简单——它将所有的警告视为错误,阻止代码编译,直到你解决每一个。虽然这可能看起来是一个有益的方法,但很少是这样。

你看,警告没有被分类为错误是有原因的:它们是为了警告你。决定如何处理这些警告取决于你。当你在实验或原型解决方案时,忽略一个警告的权限往往是无价的。

另一方面,如果你的代码是完美的、没有警告、闪闪发光的,那么允许未来的修改破坏这个完美的状态似乎是遗憾的。开启它并让它保持在那里有什么害处?至少在编译器升级之前,似乎没有什么害处。新版本的编译器往往对过时的特性更严格,或者更擅长提供改进建议。虽然当警告保持为警告时这是有益的,但它可能导致未更改的代码出现意外的构建失败,或者更令人沮丧的是,当你需要快速修复与新警告无关的问题时。

那么,什么时候允许启用所有可能的警告呢?答案是当你创建一个公共库时。在这种情况下,你希望预先阻止任何问题,指出你的代码在比你的环境更严格的环境中行为不当。如果你选择启用这个设置,确保你保持对新编译器版本及其引入的警告的更新。此外,明确管理这个更新过程也很重要,这与进行任何代码更改分开。

否则,让警告保持原样,专注于错误。如果你觉得有必要追求完美,可以使用-Wpedantic标志。这个特定的标志启用了由严格的ISO C和ISO C++标准要求的警告。然而,请注意,这个标志并不确认符合标准;它只是标识需要诊断消息的非ISO实践。

更宽松、更接地气的程序员会满足于-Wall,可以选择与-Wextra结合,增加一点复杂性,这应该就足够了。这些警告被认为是真正有用的,当时间允许时,你应该在代码中处理它们。

根据你的项目类型,可能还有其他一些警告标志对你有帮助。我建议你阅读你选择的编译器的手册,看看有哪些选项可用。

\mySamllsection{调试构建}

偶尔,编译会失败。这通常发生在我们尝试重构大量代码或清理我们的构建系统时。有时,问题可以轻松解决;然而,有些更复杂的问题需要深入调查配置步骤。我们已经在第1章“CMake入门”中讨论了如何打印更详细的CMake输出,但如何分析在每个阶段实际发生了什么?

\mySamllsubsection{调试各个阶段}

-save-temps,它可以传递给GCC和Clang编译器,允许我们调试编译的各个阶段。这个标志将指示编译器将某些编译阶段的输出存储在文件中,而不是内存中。

\filename{ch07/07-debug/CMakeLists.txt}

\begin{cmake}
add_executable(debug hello.cpp)
target_compile_options(debug PRIVATE -save-temps=obj)
\end{cmake}

启用这个选项将产生两个额外的文件(.ii和.s)每个翻译单元。


第一个,<build-tree>/CMakeFiles/<target>.dir/<source>.ii,存储了预处理阶段的输出,其中包含注释,解释了源代码的每个部分来自哪里:

\begin{shell}
# 1 "/root/examples/ch07/06-debug/hello.cpp"
# 1 "<built-in>"
# 1 "<command-line>"
# 1 "/usr/include/stdc-predef.h" 1 3 4
# / / / ... removed for brevity ... / / /
# 252 "/usr/include/x86_64-linux
  gnu/c++/9/bits/c++config.h" 3
namespace std
{
    typedef long unsigned int size_t;
    typedef long int ptrdiff_t;
    typedef decltype(nullptr) nullptr_t;
}
...
\end{shell}

第二个,<build-tree>/CMakeFiles/<target>.dir/<source>.s, 包含了语言分析阶段的输出,准备进入汇编阶段:

\begin{shell}
        .file "hello.cpp"
        .text
        .section .rodata
        .type _ZStL19piecewise_construct, @object
        .size _ZStL19piecewise_construct, 1
_ZStL19piecewise_construct:
        .zero 1
        .local _ZStL8__ioinit
        .comm _ZStL8__ioinit,1,1
.LC0:
        .string "hello world"
        .text
        .globl main
        .type main, @function
main:
( ... )
\end{shell}

根据问题的类型,我们通常可以揭示实际的问题。例如,预处理器的输出可以帮助我们识别bug,如错误的include路径(可能提供错误的库版本),或者定义错误导致的错误的\#ifdef评估。

同时,语言分析的输出对于针对特定处理器和解决关键优化问题特别有益。

\mySamllsubsection{调试头文件包含问题}

调试错误包含的文件可能是一项具有挑战性的任务。我应该知道——在我第一份公司工作时,我不得不将整个代码库从一个构建系统移植到另一个构建系统。如果你发现自己处于需要精确理解请求头文件包含路径的情况,可以考虑使用-H编译选项:

\filename{ch07/07-debug/CMakeLists.txt}

\begin{cmake}
add_executable(debug hello.cpp)
target_compile_options(debug PRIVATE -H)
\end{cmake}

产生的输出将类似于以下内容:

\begin{shell}
[ 25%] Building CXX object
  CMakeFiles/inclusion.dir/hello.cpp.o
. /usr/include/c++/9/iostream
.. /usr/include/x86_64-linux-gnu/c++/9/bits/c++config.h
... /usr/include/x86_64-linux-gnu/c++/9/bits/os_defines.h
.... /usr/include/features.h
-- removed for brevity --
.. /usr/include/c++/9/ostream
\end{shell}

在对象文件名称之后,输出中的每一行都包含一个头文件的路径。在这个例子中,行首的一个点表示顶级包含(包含指令在hello.cpp中)。两个点表示该文件被后续文件(iostream)包含。每个额外的点表示嵌套的另一个级别。

在这个输出的末尾,你可能会找到一些代码改进的建议:

\begin{shell}
Multiple include guards may be useful for:
/usr/include/c++/9/clocale
/usr/include/c++/9/cstdio
/usr/include/c++/9/cstdlib
\end{shell}

虽然你不需要处理标准库中的问题,但你可能会看到一些自己的头文件被列出。在这种情况下,你可能会考虑进行修改。

\mySamllsection{为调试器提供信息}

机器代码是一系列指令和数据的神秘列表,这些指令和数据以二进制格式编码。它不传达任何更大的意义或目标。这是因为CPU不在乎程序的目标是什么,或者所有指令的含义是什么。唯一的要求是代码的正确性。编译器会将所有这些内容转换为CPU指令的数字标识符,并在需要时存储数据以初始化内存,并提供成千上万的内存地址。换句话说,最终的二进制文件不需要包含实际的源代码、变量名、函数签名或其他程序员关心的细节。这是编译器的默认输出——原始而简单。

这样做主要是为了节省空间并执行时避免过多的开销。巧合的是,我们也以某种方式保护了我们的应用程序免受逆向工程的影响。是的,你可以理解每个CPU指令做什么,而无需源代码(例如,将这个值复制到那个寄存器)。但是,即使是基本的程序也包含太多的这些指令,以至于难以理解它们。

如果你是一个特别有动力的人,你可以使用一个叫做反汇编器的工具,并且凭借大量的知识(以及一些运气),你将能够解码可能发生的事情。然而,这种方法并不实用,因为反汇编代码没有原始符号,这使得很难和慢地弄清楚它们去了哪里。

相反,我们可以要求编译器将源代码存储在生成的二进制文件中,并与编译后的代码和原始代码之间的引用映射一起。然后,我们可以将调试器附加到一个正在运行的程序上,并查看在任何给定的时刻执行的是哪一行源代码。当我们编写新功能或更正错误时,这是必不可少的。

这两个用例是存在两种构建配置的原因:Debug和Release。如我们之前所见,CMake会默认向编译器提供一些标志来管理这个过程,首先存储在全局变量中:

\begin{itemize}
\item
CMAKE\_CXX\_FLAGS\_DEBUG 包含 -g

\item
CMAKE\_CXX\_FLAGS\_RELEASE 包含 -DNDEBUG
\end{itemize}

-g标志的意思是“添加调试信息”。它以操作系统的原生格式提供:stabs、COFF、XCOFF或DWARF。这些格式随后可以被像gdb(GNU调试器)这样的调试器访问。通常,这对像CLion这样的IDE来说已经足够了(因为它们在幕后使用gdb)。在其他情况下,请参考提供的调试器手册,并检查您选择的编译器的适当标志。

对于Release配置,CMake将添加-DNDEBUG标志。这是一个预处理器定义,意味着“不是调试构建”。这个选项会故意禁用一些面向调试的宏。其中之一是assert,它在<assert.h>头文件中可用。如果你决定在你的生产代码中使用断言,它们将不起作用:

\begin{cpp}
int main(void)
{
    assert(false); // blod
    std::cout << "This shouldn't run. \n";
    return 0;
}
\end{cpp}

在Release配置中,assert(false)调用不会有任何效果,但在Debug中它会停止执行。如果你在实践断言性编程,但仍需要在Release构建中使用assert(),你可以改变CMake提供的默认设置(从CMAKE\_CXX\_FLAGS\_RELEASE中移除NDEBUG),或者通过在包含头文件之前取消定义宏来实现硬编码覆盖:

\begin{cpp}
#undef NDEBUG
#include <assert.h>
\end{cpp}

更多信息请参阅assert参考:\url{https://en.cppreference.com/w/c/error/assert}。

你可以考虑用C++11引入的static\_assert()替换assert(),如果你的断言可以在编译时完成,因为这个函数不像assert()那样受到\#ifndef(NDEBUG)预处理器指令的保护。

通过这些,我们已经学会了如何管理编译过程。











