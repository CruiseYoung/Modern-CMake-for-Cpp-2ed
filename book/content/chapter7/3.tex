
预处理器在构建过程中扮演着重要角色。考虑到其功能看起来相当直接和有限,在以下各节中,我们将介绍提供包含文件的路径以及使用预处理器定义。还将解释如何使用CMake来配置包含的头文件。

\mySubsubsection{7.3.1.}{提供包含文件的路径}

预处理器最基本的特性是能够使用\#include指令包含.h和.hpp头文件,它有两种形式:

\begin{itemize}
\item
尖括号形式:\#include <path-spec>

\item
引号形式: \#include "path-spec"
\end{itemize}

预处理器将用路径规范中指定的文件内容替换这些指令。找到这些文件可能是一个挑战。应该搜索哪些目录,以及按照什么顺序?遗憾的是,C++标准并没有确切规定这一点,必须查阅正在使用的编译器的手册。

通常,尖括号形式将检查标准包含目录,这些目录包括存储在系统中的标准C++库和标准C库头文件的目录。

引号形式首先在当前文件的目录中搜索包含的文件,然后检查尖括号形式的目录。
CMake提供了一个命令来操作搜索包含文件的路径

\begin{shell}
target_include_directories(<target> [SYSTEM] [AFTER|BEFORE]
                           <INTERFACE|PUBLIC|PRIVATE> [item1...]
                          [<INTERFACE|PUBLIC|PRIVATE> [item2...]
...])
\end{shell}

这可以添加自定义路径,以便编译器扫描。CMake将其添加到生成的构建系统中的编译器中,将使用特定的标志以提供(通常是-I)给编译器。

target\_include\_directories()命令通过追加或预置目录,来修改目标的INCLUDE\_DIRECTORIES属性,具体取决于是否使用了AFTER或BEFORE关键字。然而,由编译器决定,是否在默认目录之前或之后检查这里提供的目录(通常是之前)。

SYSTEM关键字告诉编译器给定的目录应该视为标准系统目录(与尖括号形式一起使用)。对于许多编译器来说,这些目录是通过-isystem标志传递的。

\mySubsubsection{7.3.2.}{预处理器定义}

回想一下前面在讨论编译阶段时,提到预处理器的\#define和\#if、\#elif,以及\#endif指令。来检查一下以下示例:

\filename{ch07/02-definitions/definitions.cpp}

\begin{cpp}
#include <iostream>
int main() {
#if defined(ABC)
    std::cout << "ABC is defined!" << std::endl;
#endif
#if (DEF > 2*4-3)
    std::cout << "DEF is greater than 5!" << std::endl;
#endif
}
\end{cpp}

目前,这个示例什么也没完成,因为ABC和DEF都没有定义(DEF将默认为0)。我们可以通过在此代码顶部添加两行来轻松更改:

\begin{cpp}
#define ABC
#define DEF 8
\end{cpp}

编译并执行此代码后,可以在控制台看到两条消息:

\begin{shell}
ABC is defined!
DEF is greater than 5!
\end{shell}

这可能看起来足够简单,但如果想基于外部因素(如操作系统、架构或其他因素)来条件这些部分呢?可以将值从CMake传递给C++编译器。

target\_compile\_definitions()命令就足够了:

\filename{ch07/02-definitions/CMakeLists.txt}

\begin{cmake}
set(VAR 8)
add_executable(defined definitions.cpp)
target_compile_definitions(defined PRIVATE ABC "DEF=${VAR}")
\end{cmake}

前面的代码将表现得与两个\#define语句完全一样,但有使用CMake的变量和生成器表达式的灵活性,并且可以将命令放在条件块中。

这些定义是通过-D标志(例如,-DFOO=1)传递给编译器的,一些开发者会继续在这个命令中使用这个标志:

\begin{cmake}
target_compile_definitions(hello PRIVATE -DFOO)
\end{cmake}

CMake识别这一点,并将自动移除前导的-D标志。它还会忽略空字符串,所以以下命令也是有效的:

\begin{cmake}
target_compile_definitions(hello PRIVATE -D FOO)
\end{cmake}

这种情况下,-D是一个单独的参数,移除后变成一个空字符串,然后忽略,从而确保正确的行为。

\mySamllsection{避免在单元测试中访问私有类字段}

一些在线资源建议使用特定的-D定义与\#ifdef/ifndef指令结合用于单元测试的目的。这种方法最直接的应用是将公共访问说明符包含在条件包含中,当定义了UNIT\_TEST时,有效地使所有字段变为公共的(默认情况下,类字段私有):

\begin{cpp}
class X {
#ifdef UNIT_TEST
    public:
#endif
    int x_;
}
\end{cpp}

虽然这种方法提供了便利(允许测试直接访问私有成员),但并没有产生干净的代码。单元测试应该专注于验证公共接口内的方法的功能性,将底层实现视为一个黑盒,我建议只在万不得已的情况下使用这种方法。

\mySamllsection{使用git提交跟踪编译版本}

思考一下哪些用例需要了解关于环境或文件系统的详细信息。专业环境中,一个典型的例子可能涉及,传递用于构建二进制文件的修订版本或提交SHA。可以这样实现:

\filename{ch07/03-git/CMakeLists.txt}

\begin{cmake}
add_executable(print_commit print_commit.cpp)
execute_process(COMMAND git log -1 --pretty=format:%h
                OUTPUT_VARIABLE SHA)
target_compile_definitions(print_commit
                           PRIVATE "SHA=${SHA}")
\end{cmake}

然后可以在应用程序中使用SHA:

\filename{ch07/03-git/print\_commit.cpp}

\begin{cpp}
#include <iostream>
// special macros to convert definitions into c-strings:
#define str(s) #s
#define xstr(s) str(s)
int main()
{
#if defined(SHA)
    std::cout << "GIT commit: " << xstr(SHA) << std::endl;
#endif
}
\end{cpp}

前面的代码要求用户安装了Git,并将其添加到PATH中。当运行在生产服务器上的程序是持续集成/部署管道的结果时,这个功能尤其有用。如果软件出现问题,可以快速检查使用了哪个确切的Git提交来构建了有故障的产品。

跟踪确切的提交对于调试目的非常有用。将单个变量传递给C++代码很简单,但是如果需要将数十个变量传递给的头文件,该如何处理呢?

\mySubsubsection{7.3.3.}{配置头文件}

通过target\_compile\_definitions()传递定义在变量众多时会变得繁琐。提供带有引用这些变量的占位符的头文件,并允许CMake填充它们,这样不是更容易吗?当然可以!

CMake的configure\_file(<input> <output>)命令使你能够从模板生成新文件,如下例所示:

\filename{ch07/04-configure/configure.h.in}

\begin{cmake}
#cmakedefine FOO_ENABLE
#cmakedefine FOO_STRING1 "@FOO_STRING1@"
#cmakedefine FOO_STRING2 "${FOO_STRING2}"
#cmakedefine FOO_UNDEFINED "@FOO_UNDEFINED@"
\end{cmake}

可以按以下方式使用这个命令:

\filename{ch07/04-configure/CMakeLists.txt}

\begin{cmake}
add_executable(configure configure.cpp)
set(FOO_ENABLE ON)
set(FOO_STRING1 "abc")
set(FOO_STRING2 "def")
configure_file(configure.h.in configured/configure.h)
target_include_directories(configure PRIVATE
                           ${CMAKE_CURRENT_BINARY_DIR})
\end{cmake}

然后CMake生成如下输出文件:

\filename{ch07/04-configure/<build\_tree>/configured/configure.h}

\begin{cpp}
#define FOO_ENABLE
#define FOO_STRING1 "abc"
#define FOO_STRING2 "def"
/* #undef FOO_UNDEFINED */
\end{cpp}

,@VAR@和\$\{VAR\}变量占位符使用CMake列表文件中的值替换了。此外,对于已定义的变量,\#cmakedefine替换为\#define,对于未定义的变量,则替换为/* \#undef VAR */。如果要为\#if块显式地\#define 1或\#define 0,请使用\#cmakedefine01。

可以通过在实现文件中包含这个配置好的头文件,将其合并到应用程序中:

\filename{ch07/04-configure/configure.cpp}

\begin{cpp}
#include <iostream>
#include "configured/configure.h"

// special macros to convert definitions into c-strings:
#define str(s) #s
#define xstr(s) str(s)

using namespace std;
int main()
{
#ifdef FOO_ENABLE
    cout << "FOO_ENABLE: ON" << endl;
#endif
    cout << "FOO_STRING1: " << xstr(FOO_STRING1) << endl;
    cout << "FOO_STRING2: " << xstr(FOO_STRING2) << endl;
    cout << "FOO_UNDEFINED: " << xstr(FOO_UNDEFINED) << endl;
}
\end{cpp}

使用target\_include\_directories()命令将二进制树添加到包含搜索路径中,可以编译示例并从CMake接收填充的输出:

\begin{shell}
FOO_ENABLE: ON
FOO_STRING1: "abc"
FOO_STRING2: "def"
FOO_UNDEFINED: FOO_UNDEFINED
\end{shell}

configure\_file()命令还包括一系列的格式化和文件权限选项,但由于篇幅限制,我们在这里不深入讨论。如果感兴趣,可以参考在线文档以获取更多详细信息(请参阅本章的“扩展阅读”部分)。

准备好头文件和源文件的完整编译之后,让我们讨论在后续步骤中输出代码如何形成。虽然无法直接控制语言分析或组装(这些步骤遵循严格的标准),但可以操纵优化器的配置。






