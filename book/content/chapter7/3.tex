
预处理器在构建过程中扮演着重要角色。考虑到其功能看起来相当直接和有限,这可能有点令人惊讶。在以下各节中,我们将介绍提供包含文件的路径以及使用预处理器定义。我们还将解释如何使用CMake来配置包含的头文件。

\mySubsubsection{7.3.1.}{提供包含文件的路径}

预处理器最基本的特性是能够使用\#include指令包含.h和.hpp头文件,它有两种形式:

\begin{itemize}
\item
尖括号形式: \#include <path-spec>

\item
引号形式: \#include "path-spec"
\end{itemize}

我们知道,预处理器将用路径规范中指定的文件内容替换这些指令。找到这些文件可能是一个挑战。应该搜索哪些目录,以及按照什么顺序?遗憾的是,C++标准并没有确切规定这一点。我们必须查阅正在使用的编译器的手册。

通常,尖括号形式将检查标准包含目录,这些目录包括存储在系统中的标准C++库和标准C库头文件的目录。

引号形式首先在当前文件的目录中搜索包含的文件,然后检查尖括号形式的目录。
CMake提供了一个命令来操作搜索包含文件的路径

\begin{shell}
target_include_directories(<target> [SYSTEM] [AFTER|BEFORE]
                           <INTERFACE|PUBLIC|PRIVATE> [item1...]
                          [<INTERFACE|PUBLIC|PRIVATE> [item2...]
...])
\end{shell}

这允许我们添加自定义路径,以便编译器扫描。CMake将它们添加到生成的构建系统中的编译器调用中。它们将使用特定编译器的适当标志提供(通常是-I)。

target\_include\_directories()命令通过追加或预置目录来修改目标的INCLUDE\_DIRECTORIES属性,具体取决于是否使用了AFTER或BEFORE关键字。然而,由编译器决定是否在默认目录之前或之后检查这里提供的目录(通常是之前)。

SYSTEM关键字告诉编译器给定的目录应该被视为标准系统目录(与尖括号形式一起使用)。对于许多编译器来说,这些目录是通过-isystem标志传递的。

\mySubsubsection{7.3.2.}{预处理器定义}

回想一下前面在讨论编译阶段时提到的预处理器的\#define和\#if、\#elif,以及\#endif指令。让我们检查以下示例:

\filename{ch07/02-definitions/definitions.cpp}

\begin{cpp}
#include <iostream>
int main() {
#if defined(ABC)
    std::cout << "ABC is defined!" << std::endl;
#endif
#if (DEF > 2*4-3)
    std::cout << "DEF is greater than 5!" << std::endl;
#endif
}
\end{cpp}

目前,这个示例什么也没完成,因为ABC和DEF都没有定义(在这个例子中,DEF将默认为0)。我们可以通过在此代码顶部添加两行来轻松更改:

\begin{cpp}
#define ABC
#define DEF 8
\end{cpp}

编译并执行此代码后,我们可以在控制台看到两条消息:

\begin{shell}
ABC is defined!
DEF is greater than 5!
\end{shell}

这可能看起来足够简单,但是如果我们想基于外部因素(如操作系统、架构或其他因素)来条件这些部分呢?好消息是,你可以将值从CMake传递给C++编译器,而且这并不复杂。

target\_compile\_definitions()命令就足够了:

\filename{ch07/02-definitions/CMakeLists.txt}

\begin{cmake}
set(VAR 8)
add_executable(defined definitions.cpp)
target_compile_definitions(defined PRIVATE ABC "DEF=${VAR}")
\end{cmake}

前面的代码将表现得与两个\#define语句完全一样,但我们有使用CMake的变量和生成器表达式的灵活性,并且可以将命令放在条件块中。

传统上,这些定义是通过-D标志(例如,-DFOO=1)传递给编译器的,一些程序员继续在这个命令中使用这个标志:

\begin{cmake}
target_compile_definitions(hello PRIVATE -DFOO)
\end{cmake}

CMake识别这一点,并将自动移除任何前导的-D标志。它还会忽略空字符串,所以以下命令是完全有效的:

\begin{cmake}
target_compile_definitions(hello PRIVATE -D FOO)
\end{cmake}

在这种情况下,-D是一个单独的参数,移除后变成一个空字符串,然后被忽略,从而确保正确的行为。

\mySamllsection{避免在单元测试中访问私有类字段}

一些在线资源建议使用特定的-D定义与\#ifdef/ifndef指令结合用于单元测试的目的。这种方法最直接的应用是将公共访问说明符包含在条件包含中,当定义了UNIT\_TEST时,有效地使所有字段变为公共的(默认情况下,类字段是私有的):

\begin{cpp}
class X {
#ifdef UNIT_TEST
    public:
#endif
    int x_;
}
\end{cpp}

虽然这种方法提供了便利(允许测试直接访问私有成员),但它并没有产生干净的代码。理想情况下,单元测试应该专注于验证公共接口内的方法的功能性,将底层实现视为一个黑盒。因此,我建议只在万不得已的情况下使用这种方法。

\mySamllsection{使用git提交跟踪编译版本}

让我们思考一下哪些用例需要了解关于环境或文件系统的详细信息。在专业环境中,一个典型的例子可能涉及传递用于构建二进制文件的修订版本或提交SHA。可以这样实现:

\filename{ch07/03-git/CMakeLists.txt}

\begin{cmake}
add_executable(print_commit print_commit.cpp)
execute_process(COMMAND git log -1 --pretty=format:%h
                OUTPUT_VARIABLE SHA)
target_compile_definitions(print_commit
                           PRIVATE "SHA=${SHA}")
\end{cmake}

然后我们可以在应用程序中使用SHA,如下所示:

\filename{ch07/03-git/print\_commit.cpp}

\begin{cpp}
#include <iostream>
// special macros to convert definitions into c-strings:
#define str(s) #s
#define xstr(s) str(s)
int main()
{
#if defined(SHA)
    std::cout << "GIT commit: " << xstr(SHA) << std::endl;
#endif
}
\end{cpp}

当然,前面的代码要求用户安装了Git并将其添加到PATH中。当运行在生产服务器上的程序是持续集成/部署管道的结果时,这个功能尤其有用。如果我们的软件出现问题,我们可以快速检查使用了哪个确切的Git提交来构建有故障的产品。

跟踪确切的提交对于调试目的非常有用。将单个变量传递给C++代码很简单,但是如果我们需要将数十个变量传递给我们的头文件,我们该如何处理呢?

\mySubsubsection{7.3.3.}{配置头文件}

通过target\_compile\_definitions()传递定义在变量众多时会变得繁琐。提供带有引用这些变量的占位符的头文件,并允许CMake填充它们,这样不是更容易吗?当然可以!

CMake的configure\_file(<input> <output>)命令使你能够从模板生成新文件,如下例所示:

\filename{ch07/04-configure/configure.h.in}

\begin{cmake}
#cmakedefine FOO_ENABLE
#cmakedefine FOO_STRING1 "@FOO_STRING1@"
#cmakedefine FOO_STRING2 "${FOO_STRING2}"
#cmakedefine FOO_UNDEFINED "@FOO_UNDEFINED@"
\end{cmake}

你可以按以下方式使用这个命令:

\filename{ch07/04-configure/CMakeLists.txt}

\begin{cmake}
add_executable(configure configure.cpp)
set(FOO_ENABLE ON)
set(FOO_STRING1 "abc")
set(FOO_STRING2 "def")
configure_file(configure.h.in configured/configure.h)
target_include_directories(configure PRIVATE
                           ${CMAKE_CURRENT_BINARY_DIR})
\end{cmake}

然后CMake生成如下输出文件:

\filename{ch07/04-configure/<build\_tree>/configured/configure.h}

\begin{cpp}
#define FOO_ENABLE
#define FOO_STRING1 "abc"
#define FOO_STRING2 "def"
/* #undef FOO_UNDEFINED */
\end{cpp}

如你所见,@VAR@和\$\{VAR\}变量占位符被CMake列表文件中的值替换了。此外,对于已定义的变量,\#cmakedefine被替换为\#define,对于未定义的变量,则替换为/* \#undef VAR */。如果你需要为\#if块显式地\#define 1或\#define 0,请使用\#cmakedefine01。

你可以通过简单地在实现文件中包含这个配置好的头文件,将其合并到你的应用程序中:

\filename{ch07/04-configure/configure.cpp}

\begin{cpp}
#include <iostream>
#include "configured/configure.h"

// special macros to convert definitions into c-strings:
#define str(s) #s
#define xstr(s) str(s)

using namespace std;
int main()
{
#ifdef FOO_ENABLE
    cout << "FOO_ENABLE: ON" << endl;
#endif
    cout << "FOO_STRING1: " << xstr(FOO_STRING1) << endl;
    cout << "FOO_STRING2: " << xstr(FOO_STRING2) << endl;
    cout << "FOO_UNDEFINED: " << xstr(FOO_UNDEFINED) << endl;
}
\end{cpp}

通过使用target\_include\_directories()命令将二进制树添加到我们的包含路径中,我们可以编译示例并从CMake接收填充的输出:

\begin{shell}
FOO_ENABLE: ON
FOO_STRING1: "abc"
FOO_STRING2: "def"
FOO_UNDEFINED: FOO_UNDEFINED
\end{shell}

configure\_file()命令还包括一系列的格式化和文件权限选项,但由于篇幅限制,我们在这里不深入讨论。如果你感兴趣,可以参考在线文档以获取更多详细信息(请参阅本章的“进一步阅读”部分)。

在准备好我们的头文件和源文件的完整编译之后,让我们讨论在后续步骤中输出代码是如何形成的。虽然我们无法直接控制语言分析或组装(因为这些步骤遵循严格的标准),但我们可以操纵优化器的配置。让我们探讨这如何影响最终结果。






