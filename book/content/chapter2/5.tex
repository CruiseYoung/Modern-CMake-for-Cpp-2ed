CMake语言要是没有控制结构就不完整了!其以命令的形式提供,并且分为三类:条件块、循环和命令定义。控制结构在脚本执行和项目构建系统生成期间执行。

\mySubsubsection{2.5.1}{条件块}

CMake支持的条件块是if()命令。所有条件块都必须用endif()命令关闭,并且可以按照以下顺序有任意数量的elseif()命令和一个可选的else()命令:

\begin{shell}
if(<condition>)
    <commands>
elseif(<condition>) # optional block, can be repeated
    <commands>
else() # optional block
    <commands>
endif()
\end{shell}

与许多其他命令式语言一样,if()-endif()块控制执行的命令集:

\begin{itemize}
\item
如果if()中指定的<condition>表达式成立,将执行第一个部分。

\item
否则,CMake将执行属于此块中,第一个满足其条件的elseif()中的命令。

\item
如果没有这样的命令,CMake将检查是否提供了else()命令,并执行代码该部分中的命令。

\item
如果前面的条件都不满足,执行将在endif()命令之后继续。
\end{itemize}

注意,条件块中不存在局部作用域。

提供的<condition>表达式,会根据非常简单的语法进行计算——让我们了解更多关于它的信息。

\mySamllsection{语法}

if()、elseif()和while()命令的语法基本相同。

\mySamllsubsection{逻辑运算符}

if()条件支持NOT、AND和OR逻辑运算符:

\begin{itemize}
\item
NOT <condition>

\item
<condition> AND <condition>

\item
<condition> OR <condition>
\end{itemize}

此外,嵌套条件可以,需要匹配的圆括号对(())。与所有体面的语言一样,CMake语言遵循求值顺序,并从最内层的括号开始:

\begin{shell}
(<condition>) AND (<condition> OR (<condition>))
\end{shell}

\mySamllsubsection{字符串和变量的求值}

由于历史原因(变量引用(\$\{\})语法并不总是存在),CMake尝试计算未引用的参数,就好像是变量引用一样。换句话说,在条件中使用一个普通的变量名(例如,QUX)等于\$\{QUX\}。这里有一个例子,也是一个陷阱:

\begin{cmake}
set(BAZ FALSE)
set(QUX "BAZ")
if(${QUX})
\end{cmake}

这里的if()条件工作方式有些复杂——首先,将\$\{QUX\}求值为BAZ,这是一个已识别的变量,进而解析为一个包含五个字符的字符串,拼写为FALSE。字符串只有在等于ON、Y、YES、TRUE或非零数字,才可视为真(这些比较不区分大小写)。

所以,前面示例中的条件为假。

这里还有一个陷阱——如果一个条件的未加引号参数名,一个包含BAR这样的值的变量,那结果会是什么?看看以下代码示例:

\begin{cmake}
set(FOO BAR)
if(FOO)
\end{cmake}

根据目前所知,条件将解析为假,因为BAR字符串不符合求值为真的条件。但情况并非如此,因为CMake在处理未加引号的变量引用时有一个例外。与加引号的参数不同,FOO不会求值为BAR,以生成if(“BAR”)语句(这将是不正确的)。相反,CMake只有在以下情况下,才会将if(FOO)计算为假:

\begin{itemize}
\item
OFF, NO, FALSE, N, IGNORE或 NOTFOUND

\item
以-NOTFOUND结尾的字符串

\item
空字符串

\item
零
\end{itemize}

简单地判断一个未定义的变量,将为假:

\begin{cmake}
if (CORGE)
\end{cmake}

当变量事先定义后,情况就改变了,条件计算为真:

\begin{cmake}
set(CORGE "A VALUE")
if (CORGE)
\end{cmake}

\begin{myNotic}{Note}
如果认为未加引号的if()参数的递归求值令人困惑,可以将变量引用用引号括起来:if(“\$\{CORGE\}”)。这将导致在提供的参数传递到if()命令之前对其进行求值,其行为与字符串一致。
\end{myNotic}

换句话说,CMake假定传递变量名给if()命令的用户,是在询问该变量是否定义了不等于“假”的值。为了明确检查变量是否已定义(忽略其值),可以使用以下方法:

\begin{cmake}
if(DEFINED <name>)
if(DEFINED CACHE{<name>})
if(DEFINED ENV{<name>})
\end{cmake}

\mySamllsubsection{比较值}

支持以下操作符进行比较操作:

EQUAL, LESS, LESS\_EQUAL, GREATER 和 GREATER\_EQUAL

其他语言中常见的比较操作符,在 CMake 中不起作用。

它们可以用来比较数值:

\begin{cmake}
if (1 LESS 2)
\end{cmake}

可以通过在操作符前加上 VERSION\_ 前缀来按照 [major[.minor[.patch[.tweak]]]] 的格式比较软件版本:

\begin{cmake}
if (1.3.4 VERSION_LESS_EQUAL 1.4)
\end{cmake}

省略的组件视为零,非整数的版本组件将在此点截断比较字符串。

对于字典序字符串比较,需要在操作符前加上 STR 前缀(没有下划线):

\begin{cmake}
if ("A" STREQUAL "${B}")
\end{cmake}

我们经常需要比简单相等更高级的比较机制,CMake 支持 POSIX 正则表达式匹配(CMake 文档暗示支持扩展正则表达式(ERE)风格,但未提到对特定正则字符类的支持)。可以按以下方式使用 MATCHES 操作符:

\begin{shell}
<VARIABLE|STRING> MATCHES <regex>
\end{shell}

匹配组会被捕获到CMAKE\_MATCH\_<n>变量中。

\mySamllsubsection{简单检查}

之前提到了一个简单检查,DEFINED。也还有其他的检查,如果满足条件则返回真。

还可以检查以下内容:

\begin{itemize}
\item
值是否在列表中:<VARIABLE|STRING> IN\_LIST <VARIABLE>

\item
此版本的CMake中是否可以调用某个命令: COMMAND <command-name>

\item
是否存在一个CMake策略: POLICY <policy-id>

\item
是否使用add\_test()添加了CTest测试:TEST <test-name>

\item
是否定义了一个构建目标: TARGET <target-name>
\end{itemize}

我们将在第5章,使用目标中探讨构建目标。现在,让我们简单的认为,目标是通过add\_executable(),add\_library()或add\_custom\_target()命令创建的基本项目单元(构建过程的)。

\mySamllsubsection{检查文件系统}

CMake提供了许多处理文件的方法。我们很少需要直接操作,通常会使用高级方法。作为参考,本书将在附录中提供与文件相关的命令的简短列表。以下操作符会经常使用(仅对绝对路径有明确的行为定义):

\begin{itemize}
\item
EXISTS <path-to-file-or-directory>: 检查文件或目录是否存在。

也会符号链接进行解析(如果符号链接的目标存在,则返回真)。

\item
<file1> IS\_NEWER\_THAN <file2>: 检查哪个文件修改时间更晚。

如果文件1比文件2修改时间更晚(或等于)或者两个文件中有一个不存在,则返回真。

\item
IS\_DIRECTORY <path-to-directory>: 检查路径是否为目录。

\item
IS\_SYMLINK <file-name>: 检查路径是否为符号链接。

\item
IS\_ABSOLUTE <path>: 检查路径是否为绝对路径。
\end{itemize}

3.24版本开始,CMake支持简单的路径比较检查,这将折叠多个路径分隔符,但不会进行其他标准化:

\begin{cmake}
if ("/a////b/c" PATH_EQUAL "/a/b/c") # returns true
\end{cmake}

对于更高级的路径操作,请参阅cmake\_path()命令的文档。

完成了条件命令的语法介绍,接下来我们将讨论的控制结构是——循环。

\mySubsubsection{2.5.2}{循环}

CMake中,循环很简单——可以使用while()循环或foreach()循环,来重复执行同一组命令。这两个命令都支持循环控制机制:

\begin{itemize}
\item
break()循环将停止执行剩余的块,并跳出包围的循环。

\item
continue()循环将停止当前迭代的执行,并从下一个迭代的开头开始。
\end{itemize}

注意,循环块中不会创建局部作用域。

\mySamllsection{while()}

循环块使用while()命令开启,并使用endwhile()命令关闭。只要while()中提供的<condition>表达式为真,封闭的命令都将会执行。条件部分的语法与if()命令相同:

\begin{shell}
while(<condition>)
    <commands>
endwhile()
\end{shell}

通过一些变量——while循环可以替代for循环。实际上,使用foreach()循环会更容易。

\mySamllsection{foreach()}

foreach()块有几种变体,为给定列表中的每个值执行封闭的命令。像其他块一样,有打开和关闭命令:foreach()和endforeach()。

foreach()的最简单形式类似C++的for循环:

\begin{shell}
foreach(<loop_var> RANGE <max>)
    <commands>
endforeach()
\end{shell}

CMake将从0迭代到<max>(包括)。如果需要更多控制,可以使用第二个变体,提供<min>,<max>,以及可选的<step>。所有参数都必须是非负整数,且<min>必须小于<max>:

\begin{shell}
foreach(<loop_var> RANGE <min> <max> [<step>])
\end{shell}

当foreach()处理列表时,才是彰显特色的时候:

\begin{shell}
foreach(<loop_variable> IN [LISTS <lists>] [ITEMS <items>])
\end{shell}

CMake将从一个或多个指定的<lists>列表变量中检索元素,以及一行定义的<items>值列表,将其放入<loop variable>中。然后,针对列表中的每个项执行所有命令,可以选择只提供列表或值(或者两者都提供):

\filename{ch02/06-loops/foreach.cmake}

\begin{cmake}
set(MyList 1 2 3)
foreach(VAR IN LISTS MyList ITEMS e f)
    message(${VAR})
endforeach()
\end{cmake}

将输出以下内容:

\begin{shell}
1
2
3
e
f
\end{shell}

或者,可以使用简短版本(跳过IN关键字)得到相同的结果:

\begin{cmake}
foreach(VAR 1 2 3 e f)
\end{cmake}

自从3.17版本以来,foreach()支持压缩列表(ZIP\_LISTS):

\begin{shell}
foreach(<loop_var>... IN ZIP_LISTS <lists>)
\end{shell}

压缩列表的过程,涉及迭代多个列表,并对具有相同索引的对应项进行操作:

\filename{ch02/06-loops/foreach.cmake}

\begin{cmake}
set(L1 "one;two;three;four")
set(L2 "1;2;3;4;5")
foreach(num IN ZIP_LISTS L1 L2)
    message("word=${num_0}, num=${num_1}")
endforeach()
\end{cmake}

CMake将为提供的每个列表创建一个num\_<N>变量,将用每个列表中的项填充这些变量。

可以传递多个变量名(每个列表一个),每个列表将使用单独的变量来存储其项:

\begin{cmake}
foreach(word num IN ZIP_LISTS L1 L2)
    message("word=${word}, num=${num}")
\end{cmake}

ZIP\_LISTS的两个例子都将产生相同的输出:

\begin{shell}
word=one, num=1
word=two, num=2
word=three, num=3
word=four, num=4
\end{shell}

如果列表之间的项数不同,较短的列表的变量将为空。

自从3.21版本以来,foreach()中的循环变量会限制在循环的局部作用域内使用。

好了,到此为止,关于循环的讨论也就差不多了结束了。

\mySubsubsection{2.5.3}{自定义命令}

有两种方法可以自定义命令:可以使用macro()命令或function()命令。这两个命令相当于C语言中的预处理器宏和函数:

macro()命令更像是查找替换指令,而不是function()那样的函数调用,所以宏不会在调用堆栈上保留信息。 所以,宏中调用return()将返回到比使用点更高一层的调用语句(如果已经在顶层作用域中,可能会终止执行)。

function()命令为变量创建局部作用域,这与macro()命令不同。我们会在下一节中,讨论这些细节。

这两种定义方法都允许,在定义命令的局部作用域中,引用的命名参数。此外,CMake提供了以下变量来访问与调用相关的参数值:

\begin{itemize}
\item
\$\{ARGC\}: 参数计数

\item
\$\{ARGV\}: 参数列表(所有参数)

\item
\$\{ARGV<index>\}: 特定索引(从0开始)处的参数值

\item
\$\{ARGN\}: 调用者在最后一个参数后传递的匿名参数列表
\end{itemize}

使用超出ARGC范围的索引访问数字参数,会导致未定义行为。要处理高级场景(通常参数数量未知),可以去阅读官方文档中,关于cmake\_parse\_arguments()的内容。如果决定使用带有命名参数的命令,每次调用都必须传递这些参数,否则调用将无效。

\mySamllsection{宏}

宏与其他块定义类似:

\begin{shell}
macro(<name> [<argument>…])
    <commands>
endmacro()
\end{shell}

此声明之后,可以通过调用其名称来执行宏(函数调用不区分大小写)。

宏不会在修改调用堆栈信息,或为变量添加新的作用域。以下示例展示了宏的这一行为:

\filename{ch02/08-definitions/macro.cmake}

\begin{cmake}
macro(MyMacro myVar)
    set(myVar "new value")
    message("argument: ${myVar}")
endmacro()
set(myVar "first value")
message("myVar is now: ${myVar}")
MyMacro("called value")
message("myVar is now: ${myVar}")
\end{cmake}

以下是脚本的输出

\begin{shell}
$ cmake -P ch02/08-definitions/macro.cmake
myVar is now: first value
argument: called value
myVar is now: new value
\end{shell}

发生了什么?尽管将myVar设置为“新值”,但它并没有影响到message(“argument:\$\{myVar\}”)的输出!因为传递给宏的参,数不是作为真正的变量处理,而是作为常量的查找替换指令。

另一方面,全局作用域中的myVar变量从“初始值”修改为“新值”。这种行为有一定的副作用,被认为是一种不良实践。因为如果不阅读宏,就无法知道宏会修改哪些全局变量。建议尽可能使用函数,可能避免这种问题。

\mySamllsection{函数}

要将命令声明为函数,请遵循以下语法:

\begin{shell}
function(<name> [<argument>...])
    <commands>
endfunction()
\end{shell}

函数需要一个名称,可以选择接受一系列参数名称。如前所述,函数会创建变量作用域。可以使用set(),提供函数的一个命名参数,并且任何更改都将局限于函数内部(除非指定了PARENT\_SCOPE)。

函数遵循调用堆栈的规则,允许使用return()命令返回到调用作用域。从CMake 3.25开始,return()命令允许一个可选的PROPAGATE关键字,后跟一系列变量名称。其目的与block()命令类似——指定变量的值从局部作用域传递到调用作用域。

CMake为每个函数设置了以下变量(这些变量自3.17版本起可用):

\begin{itemize}
\item
CMAKE\_CURRENT\_FUNCTION

\item
CMAKE\_CURRENT\_FUNCTION\_LIST\_DIR

\item
CMAKE\_CURRENT\_FUNCTION\_LIST\_FILE

\item
CMAKE\_CURRENT\_FUNCTION\_LIST\_LINE
\end{itemize}

来看看这些函数变量在实际中的使用:

\filename{ch02/08-definitions/function.cmake}

\begin{cmake}
function(MyFunction FirstArg)
    message("Function: ${CMAKE_CURRENT_FUNCTION}")
    message("File: ${CMAKE_CURRENT_FUNCTION_LIST_FILE}")
    message("FirstArg: ${FirstArg}")
    set(FirstArg "new value")
    message("FirstArg again: ${FirstArg}")
    message("ARGV0: ${ARGV0} ARGV1: ${ARGV1} ARGC: ${ARGC}")
endfunction()
set(FirstArg "first value")
MyFunction("Value1" "Value2")
message("FirstArg in global scope: ${FirstArg}")
\end{cmake}

使用cmake -P function.cmake运行这个脚本将输出以下输出:

\begin{shell}
Function: MyFunction
File: /root/examples/ch02/08-definitions/function.cmake
FirstArg: Value1
FirstArg again: new value
ARGV0: Value1 ARGV1: Value2 ARGC: 2
FirstArg in global scope: first value
\end{shell}

函数的总体语法和概念与宏非常相似,但不易受到隐式错误的影响。

\mySamllsection{CMake中的过程范式}

编写CMake代码,需要创建一个CMakeLists.txt列表文件,该文件将调用三个定义的命令,这些命令可能还会调用其自定义命令。图2.3对这种可能性,进行了展示:

\myGraphic{0.8}{content/chapter2/images/3.png}{图2.3:过程调用图}

在CMake中,以过程风格编写代码可能会出现问题,必须在调用之前提供定义。CMake解析器不接受其他方式,所以代码可能会写成这样:

\begin{cmake}
cmake_minimum_required(...)
project(Procedural)

# Definitions
function(pull_shared_protobuf)
function(setup_first_target)
function(calculate_version)
function(setup_second_target)
function(setup_tests)

# Calls
setup_first_target()
setup_second_target()
setup_tests()
\end{cmake}

真是噩梦!一切都颠倒了!由于最低抽象级别的代码位于文件开头,这将使整段代码难以理解。结构正确的代码,应该在第一个子程序中列出最一般的步骤,然后提供详细一点的子程序,并将最详细的步骤放在文件的最后。

这个问题有解决方案,比如将命令定义移动到其他文件中,并在目录之间划分作用域。但还有一个简单而优雅的方法——在文件顶部声明一个入口点宏,并在文件末尾调用:

\begin{cmake}
macro(main)
    first_step()
    second_step()
    third_step()
endmacro()

function(first_step)
function(second_step)
function(third_step)

main()
\end{cmake}

这种方法会逐渐缩小作用域,并且在文件末尾才实际调用main()宏,所以CMake不会因为执行未定义的命令而抱怨。

那为什么使用宏而不是函数?能够不受限制地访问全局变量是很好的,而且由于无需向main()传递任何参数,所以不需要有任何的担心。

可以在本书的GitHub库的ch02/09-procedural/CMakeLists.txt列表文件中,找到这个简单的示例。

\mySamllsection{关于命名约定}

软件开发中,命名是一项难题,并且保持一个易于阅读和理解的解决方案仍然非常重要。对于CMake脚本和项目,我们应该遵循干净代码的规则:

\begin{itemize}
\item
遵循一致的命名风格(在CMake社区中,snake\_case是可接受的标准)。

\item
使用简短但有意义的名称(避免使用func(),f()等)。

\item
避免在命名中使用双关语和只有编写者知道的东西。

\item
使用可发音、可搜索的名称,不需要进行心智映射。
\end{itemize}

这样,我们结束了“如何使用正确的语法调用命令”。

接下来,来探讨一下哪些命令比较适合于入门。



















