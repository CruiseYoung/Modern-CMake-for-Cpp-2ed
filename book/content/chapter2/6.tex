
CMake 提供了许多脚本命令,允许处理变量和环境。其中一些在附录中已经进行了详细介绍,例如 :list()、string() 和 file()。其他,如: find\_file()、find\_package() 和 find\_path(),更适合放在属于它们主题的章节中介绍。

本节中,我们将简述一些常用命令:

\begin{itemize}
\item
message()

\item
include()

\item
include\_guard()

\item
file()

\item
execute\_process()
\end{itemize}

\mySubsubsection{2.6.1}{message()}

message() 命令,将文本输出到标准输出,但其功能远不止于此。其有一个 MODE 参数,可以定义命令的行为,如下所示:message(<MODE> "text to print")。

可用的MODE如下所示:

\begin{itemize}
\item
FATAL\_ERROR: 停止处理和生成。

\item
SEND\_ERROR: 继续处理,但跳过生成。

\item
WARNING: 继续处理。

\item
AUTHOR\_WARNING: 输出警告,但继续处理。

\item
DEPRECATION: 如果启用了 CMAKE\_ERROR\_DEPRECATED 或 CMAKE\_WARN\_DEPRECATED,则输出相应地信息。

\item
NOTICE 或省略模式(默认): 输出消息到 stderr,以吸引使用者的注意。

\item
STATUS: 继续处理,推荐用于向用户显示的主要消息。

\item
VERBOSE: 继续处理,应用于更详细的信息,通常不是非常必要。

\item
DEBUG: 继续处理,应包含项目出现问题时,对处理问题有帮助的详细信息。

\item
TRACE: 继续处理,建议在项目开发期间输出消息。通常,这类消息会在发布项目之前移除。
\end{itemize}

选择正确的模式需要花点心思,可以通过基于严重性(自 3.21 版本起)给输出文本着色来节省调试时间,甚至在声明不可恢复的错误后停止执行:

\filename{ch02/10-useful/message\_error.cmake}

\begin{cmake}
message(FATAL_ERROR "Stop processing")
message("This won't be printed.")
\end{cmake}

消息将根据当前的日志级别(默认为 STATUS)打印,在上一章的调试和跟踪选项部分讨论了如何更改此设置。

在第 1 章中,我提到了使用 CMAKE\_MESSAGE\_CONTEXT 进行调试,现在是深入研究它的时候了。在此期间,我们来了解了这个主题三个关键部分:列表、作用域和函数。

在复杂的调试场景中,指出消息发生的上下文会非常有用。考虑以下输出,其中在 foo 函数中打印的消息有适当的前缀:

\begin{shell}
$ cmake -P message_context.cmake --log-context
[top] Before `foo`
[top.foo] foo message
[top] After `foo`
\end{shell}

具体代码:

\filename{ch02/10-useful/message\_context.cmake}

\begin{cmake}
function(foo)
    list(APPEND CMAKE_MESSAGE_CONTEXT "foo")
    message("foo message")
endfunction()

list(APPEND CMAKE_MESSAGE_CONTEXT "top")
message("Before `foo`")
foo()
message("After `foo`")
\end{cmake}

来分解一下:

\begin{enumerate}
\item
首先,将 top 追加到上下文跟踪变量 CMAKE\_MESSAGE\_CONTEXT,然后打印最初的“Before 'foo' ”,匹配的前缀 [top] 将添加到输出中。

\item
接下来,进入 foo() 函数时,我们在属于该函数的列表后,追加一个名为 foo 的新上下文,并输出另一个消息,该消息在输出中显示扩展的 [top.foo] 前缀。

\item
最后,在函数执行完成后,我们打印“After 'foo'”。消息以原始的 [foo] 作用域打印。为什么?因为变量作用域规则:更改的 CMAKE\_MESSAGE\_CONTEXT 变量只在函数作用域结束前有效,然后恢复为原始未更改的版本
\end{enumerate}

message() 的另一个技巧是向 CMAKE\_MESSAGE\_INDENT 列表添加缩进(与 CMAKE\_MESSAGE\_CONTEXT 完全相同的方式):

\begin{cmake}
list(APPEND CMAKE_MESSAGE_INDENT " ")
message("Before `foo`")
foo()
message("After `foo`")
\end{cmake}

然后,脚本输出看起来更简单:

\begin{shell}
Before `foo`
    foo message
After `foo`
\end{shell}

由于CMake没有提供断点或其他调试工具,当事情没有按计划进行时,清晰的日志信息就显得尤为重要。

\mySubsubsection{2.6.2}{include()}

将代码分割成不同的文件,以保持有序和分离。然后,可以通过 include() 从父列表文件引用它们:

\begin{shell}
include(<file|module> [OPTIONAL] [RESULT_VARIABLE <var>])
\end{shell}

如果提供一个文件名(带有 .cmake 扩展名的路径),CMake 将尝试打开并执行。

注意,这也不会创建独立的变量作用域,因此该文件中对变量的修改都会对调用作用域产生影响。

如果文件不存在,CMake 将报错,除非使用 OPTIONAL 关键字指定相应文件为“可选的”。当需要知道 include() 是否成功时,可以提供 RESULT\_VARIABLE 关键字以及变量的名称。在成功时,它的内容为文件的完整路径;在失败时,则为 NOTFOUND。

使用脚本模式下运行时,相对路径都将以当前工作目录解析相对路径。要强制相对于脚本本身进行搜索,请提供绝对路径:

\begin{cmake}
include("${CMAKE_CURRENT_LIST_DIR}/<filename>.cmake")
\end{cmake}

如果不提供路径,但提供了模块的名称(不带 .cmake 或其他),CMake 将尝试找到一个模块并包含它。CMake 将在 CMAKE\_MODULE\_PATH 中搜索名为 <模块>.cmake 的文件,然后是在 CMake 模块目录中搜索。

当 CMake 遍历源树,并包含了不同的列表文件时,将设置以下变量:

\begin{itemize}
\item
CMAKE\_CURRENT\_LIST\_DIR

\item
CMAKE\_CURRENT\_LIST\_FILE

\item
CMAKE\_PARENT\_LIST\_FILE

\item
CMAKE\_CURRENT\_LIST\_LINE
\end{itemize}

\mySubsubsection{2.6.3}{include\_guard()}

对于有些文件,我们只希望包含一次,这时 include\_guard([DIRECTORY|GLOBAL]) 就可以使用了。

将 include\_guard() 放在包含文件的顶部。当 CMake 第一次遇到它时,将在当前作用域中进行记录。如果文件再次包含,CMake就不会再对该文件进行处理了。

避免在隔离相关的作用域,应该提供 DIRECTORY 或 GLOBAL 参数。顾名思义,DIRECTORY 关键字将在当前目录及其子目录应用保护,而 GLOBAL 关键字将把保护应用于整个构建。


\mySubsubsection{2.6.4}{file()}

为了了解可以在 CMake 脚本中执行的操作,来了解一下文件操作命令:

\begin{shell}
file(READ <filename> <out-var> [...])
file({WRITE | APPEND} <filename> <content>...)
file(DOWNLOAD <url> [<file>] [...])
\end{shell}

简而言之,file() 命令可以以系统无关的方式读取、写入和传输文件,以及与文件系统、文件锁、路径和存档等交互。有关更多详细信息,请参阅附录。

\mySubsubsection{2.6.5}{execute\_process()}

有时候,需要使用系统中的工具(毕竟,CMake 主要是一个构建系统生成器)。CMake 提供了一个命令以实现此目的:可以使用 execute\_process() 来运行其他进程,并收集输出。这个命令非常适合脚本使用,也可以在项目中使用,但它仅在配置阶段有效。

以下是该命令的一般形式:

\begin{shell}
execute_process(COMMAND <cmd1> [<arguments>]... [OPTIONS])
\end{shell}

CMake 将使用操作系统的 API 来创建一个子进程(因此,像 \&\&, || 和 > 这样的 shell 操作符将不起作用),可以通过多次提供 COMMAND 参数来链接命令,并将一个命令的输出传递给另一个。

可选地 TIMEOUT 参数,用来在进程未在所需限制内完成任务时终止该进程,并且可以根据需要设置 WORKING\_DIRECTORY 。

所有任务的退出代码可以通过提供 RESULTS\_VARIABLE 参数来收集到一个列表中。如果只对最后执行的命令的结果感兴趣,请使用单数形式:RESULT\_VARIABLE 。

为了收集输出,CMake 提供了两个参数:OUTPUT\_VARIABLE 和 ERROR\_VARIABLE(用法类似)。如果想合并 stdout 和 stderr,请为这两个参数使用相同的变量。

当为其他用户编写项目时,应该确保计划使用的命令,在声明支持的平台上可用。








