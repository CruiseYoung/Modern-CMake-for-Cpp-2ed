本章打开了使用 CMake 编程的大门 —— 现在能够编写出色、富有信息量的注释,并使用内置命令,了解了如何正确地提供各种类型的参数。仅凭这些知识,就能理解 CMake 列表文件的不寻常语法。我们讨论了 CMake 中的变量 —— 特别是如何引用、设置和取消设置普通变量、缓存变量和环境变量。还深入探讨了文件和目录变量作用域的工作原理,如何创建它们,以及可能遇到的问题和问题的解决办法。我们还介绍了列表和控制结构,检查了条件的语法、逻辑操作、未引用参数的计算,以及字符串和变量。还学习了如何比较值、进行简单的检查,以及检查系统中文件的状态。这样,就能够编写条件块和 while 循环;在讨论循环时,我们还介绍了 foreach 循环的语法。

理解如何使用宏和函数语句,自定义命令无疑将促进编写更清晰、更有条理的代码。我们还讨论了改进代码结构和创建更具可读性名称的策略。

最后,介绍了 message() 命令及其多个日志级别。还研究了如何划分和包含列表文件,并且发现了一些有趣的命令。

全部的这些,为迎接下一章做好了充分的准备。
