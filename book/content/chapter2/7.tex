本章为我们打开了实际使用 CMake 编程的大门 —— 您现在能够编写出色、富有信息量的注释并利用内置命令,并且您了解了如何正确地向它们提供各种类型的参数。仅凭这些知识,就能帮助您理解在其他人的项目中可能看到的 CMake 列表文件的不寻常语法。我们讨论了 CMake 中的变量 —— 特别是如何引用、设置和取消设置普通变量、缓存变量和环境变量。我们深入探讨了文件和目录变量作用域的工作原理,如何创建它们,以及我们可能遇到的问题以及如何解决这些问题。我们还介绍了列表和控制结构。我们检查了条件的语法、逻辑操作、未引用参数的评估,以及字符串和变量。我们学习了如何比较值、进行简单的检查,以及检查系统中文件的状态。这使我们能够编写条件块和 while 循环;在讨论循环时,我们还掌握了 foreach 循环的语法。

理解如何使用宏和函数语句定义自定义命令无疑将促进编写更清晰、更有条理的代码。我们还讨论了改进代码结构和创建更具可读性的名称的策略。

最后,我们正式介绍了 message() 命令及其多个日志级别。我们还研究了如何划分和包含列表文件,并且我们发现了一些其他有用的命令。有了这些信息,我们已经为迎接下一章,第 3 章,使用 CMake 在流行 IDE 中做好了充分的准备。
