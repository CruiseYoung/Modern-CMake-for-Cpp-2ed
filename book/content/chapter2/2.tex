编写 CMake 代码类似于其他命令式语言:从上到下、从左到右逐行执行,偶尔会进入包含的文件或调用的函数。执行的起点取决于模式,可以是从源码树的根文件 (CMakeLists.txt) 开始,也可以是从作为 cmake 命令参数提供的 .cmake 脚本文件开始。

由于 CMake 脚本提供了对 CMake 语言的广泛支持(除了与项目相关的特性),我们将利用它们来练习本章中的 CMake 语法。当我们掌握了编写简单的列表文件,就可以进一步创建实际的项目文件。

脚本可以通过以下命令运行:cmake -P script.cmake

\begin{myNotic}{Note}
CMake支持7位ASCII文本文件,以便在所有平台之间移植,可以使用 \verb|\|n 或 \verb|\|r\verb|\|n 行结束符。版本高于3.2的CMake,支持带有可选字节序标记(BOM)的UTF-8和UTF-16。

\end{myNotic}

CMake列表文件中的所有内容要么是注释,要么是命令。

\mySubsubsection{2.1.1}{注释}

与C++一样,有两种类型的注释:单行注释和括号(多行)注释。但与C++不同的是,括号注释可以嵌套。单行注释以井号符号开始,\#:

\begin{cmake}
# they can be placed on an empty line
message("Hi"); # or after a command like here.
\end{cmake}

多行括号注释的名字来源于它们的符号——以 \# 开始,后跟开方括号 [, 任意数量的等号 =(也可以是0个),以及另一个方括号 [. 要关闭括号注释,请使用相同数量的等号并反转方括号 ]:

\begin{cmake}
#[=[
bracket comment
  #[[
    nested bracket comment
  #]]
#]=]
\end{cmake}

可以通过在括号注释的初始行,添加另一个 \# 来快速禁用多行注释:

\begin{cmake}
##[=[ this is a single-line comment now
no longer commented
  #[[
    still, a nested comment
  #]]
#]=] this is a single-line comment now
\end{cmake}

了解如何使用注释肯定有用,但这又引出了另一个问题:应该在什么时候使用?由于编写列表文件本质上就是编程,将我们最好的编码实践带到其中。

遵循这些实践的代码通常称为\textbf{整洁代码}——这个术语多年来被软件开发大师如Robert C. Martin、Martin Fowler,以及许多其他作者使用。

围绕哪些实践是有益或有害,常常存在很多争议,注释也没有免于这些辩论。一切应该根据具体情况来判断,但普遍认同的指导原则认为,好的注释至少提供以下一项:

\begin{itemize}
\item
信息:可以解开复杂性的问题,如正则表达式模式或格式化字符串。

\item
意图:当代码的意图从实现或接口中不明显时,可以解释代码意图。

\item
澄清:可以解释难以重构或更改的概念。

\item
警告后果:可以提供警告,尤其是在可能破坏其他事物时。

\item
强调:可以强调难以在代码中表达的重要想法。

\item
法律条款:有必要的时候需要添加,但这通常不是开发者擅长的领域。
\end{itemize}

最好通过应用更好的命名、重构或纠正代码来避免注释。省略以下注释:

\begin{itemize}
\item
强制性:这些是为了完整性而添加的,但并不是很重要。

\item
冗余:这些重复了代码中已经清晰写明的内容。

\item
误导性:如果不跟随代码更改,这些可能是过时或不正确的。

\item
日志:这些记录了更改了什么,以及何时更改(应使用版本控制系统(VCS)代替)。

\item
分隔符:只用于标记。
\end{itemize}

避免添加注释,采用更好的命名实践,并重构或纠正代码。编写优雅的代码是一项具有挑战性的任务,但它增强了读者的体验。由于阅读代码的时间比编写代码的时间多,应始终努力编写易于阅读的代码,而不仅仅是试图快速完成它。我建议查看本章末尾的“扩展阅读”,那里有一些关于整洁代码的好参考资料。如果对注释感兴趣,可在链接到我的YouTube视频“Which comments in your code ARE GOOD?”,其中深入探讨这一内容。

\mySubsubsection{2.1.2}{命令调用}

命令调用是 CMake 列表文件的核心。为了运行一个命令,必须指定它的名字,然后是括号,在括号内可以包含由空格分隔的命令参数列表。

\myGraphic{0.3}{content/chapter2/images/1.png}{图 2.1:命令示例}

命令名称不区分大小写,但在 CMake 社区中有一个约定,使用 snake\_case(即用下划线连接的小写单词)。还可以定义自己的命令。

CMake 和 C++ 之间的一个重要区别是,CMake 中的命令调用不是表达式。所以不能将另一个命令作为参数传递给调用的命令,因为括号内的所有内容都视为特定命令的参数。

CMake 命令后也不需要加分号。这是因为源代码的每一行只能包含一个命令。

命令后面可以(可选地)跟一个注释:

\begin{cmake}
command(argument1 "argument2" argument3) # comment
command2() #[[ multiline comment
\end{cmake}

但不能反着来:

\begin{cmake}
#[[ bracket
]] command()
\end{cmake}

CMake 列表文件中的所有内容要么是注释,要么是命令调用。CMake 语法真的很简单,在大多数情况下,这是一件好事。虽然有一些限制(不能使用表达式来递增计数器变量),但大部分情况下,因为 CMake 并不打算成为一种通用编程语言,这些限制可以接受。

CMake 提供了操作变量、控制执行流程、修改文件等的命令。为了简化操作,我们将在不同的示例中逐步介绍相关的命令。这些命令可以分为两组:

\begin{itemize}
\item
脚本命令:这些命令始终可用,改变命令处理器的状态,访问变量,并影响其他命令和环境。

\item
项目命令:这些命令在项目中可用,操纵项目状态和构建目标。
\end{itemize}

每个命令都依赖于语言的其它元素才能发挥作用:变量、条件语句,最重要的是,命令行参数。

现在,来看看如何使用它们。

\mySubsubsection{2.1.3}{命令参数}

CMake中的许多命令,需要以空格分隔的参数来配置行为。如图2.1所示,参数周围的引号使用方式相当奇特。虽然某些参数需要引号,但其他参数则不需要。背后的原因是什么?

在底层,CMake唯一识别的数据类型是字符串,这就是为什么每个命令都期望其参数为零个或多个字符串。CMake将评估每个参数为一个静态字符串,然后将它们传递给命令。评估意味着字符串插值,或者用另一个值替换字符串的一部分。这可能包括替换转义序列、展开变量引用(也称为变量插值),以及解包列表。

根据上下文,需要按需启用这种评估。为此,CMake提供了三种类型的参数:

\begin{itemize}
\item
括号参数

\item
引号参数

\item
非引号参数
\end{itemize}

CMake中的每种参数类型都有其独特性,并提供不同级别的评估。

\mySamllsection{括号参数}

括号参数不会计算,用于将多行字符串原封不动地作为单个参数传递给命令。所以这样的参数,将包括以制表符和换行符形式的空白。

括号参数的格式与注释相同,以 [=[ 开始,以 ]]=] 结束,并且开头和结尾标记中的等号数量必须匹配(只要匹配,省略等号也是可以的)。与注释的唯一区别是括号参数不能嵌套。

以下是与 message() 命令一起使用此类参数的示例,该命令将所有传递的参数输出到屏幕上:

\filename{ch02/01-arguments/bracket.cmake}

\begin{cmake}
message([[multiline
    bracket
    argument
]])
message([==[
    because we used two equal-signs "=="
    this command receives only a single argument
    even if it includes two square brackets in a row
    { "petsArray" = [["mouse","cat"],["dog"]] }
]==])
\end{cmake}

前面的示例中,可以看到不同形式的括号参数。第一个调用中在单独一行放置结束标签,如何在输出中引入一个空行:

\begin{shell}
$ cmake -P ch02/01-arguments/bracket.cmake
multiline
bracket
argument
  because we used two equal-signs "=="
  following is still a single argument:
  { "petsArray" = [["mouse","cat"],["dog"]] }
\end{shell}

当传递包含双括号(]])的文本时,第二种形式非常有用,不会解释为标记参数的结束。

这类括号参数的使用有限——包含较长的文本块,这些文本块包含显示给用户的消息。大多数情况下,需要更动态的内容,例如引号参数。

\mySamllsection{引号参数}

引号参数类似于常规的C++字符串——这些参数将多个字符(包括空白)组合在一起,并将展开转义序列。与C++字符串一样,以双引号字符 " 开头和结尾,因此要在输出字符串中包含引号字符,必须使用反斜杠进行转义,\verb|\|"。其他熟悉的转义序列也得到支持:\verb|\\| 表示字面上的反斜杠,\verb|\|t 是制表符,\verb|\|n 是换行符,而 \verb|\|r 是回车符。

这就是与C++字符串相似之处结束的地方。引号参数可以跨越多行,并且它们会展开变量引用。可以将它们视为内置的C语言中的 sprintf 函数或 C++20 中的 std::format 函数。要将变量引用插入到您的参数中,请将变量名包裹在如下标记中:\$\{name\}。我们将在本章的“使用变量”部分中更详细地讨论变量引用。

能猜到以下脚本的输出将有多少行吗?

\filename{ch02/01-arguments/quoted.cmake}

\begin{cmake}
message("1. escape sequence: \" \n in a quoted argument")
message("2. multi...
  line")
message("3. and a variable reference: ${CMAKE_VERSION}")
\end{cmake}

来看看实际效果:

\begin{shell}
$ cmake -P ch02/01-arguments/quoted.cmake
1. escape sequence: "
in a quoted argument
2. multi...
line
3. and a variable reference: 3.26.0
\end{shell}

没错——我们有一个转义的引号字符,一个换行转义序列,以及一个字面上的换行符。我们还访问了一个内置的 CMAKE\_VERSION 变量,可以看到它在最后一行。

现在,来看看CMake如何处理没有引号的参数。

\mySamllsection{非引号参数}

编程世界中,我们已经习惯了字符串以某种形式进行定界,例如使用单引号、双引号或反斜杠。CMake偏离了这一惯例,引入了非引号参数。我们可能会争论,省略定界符可以使代码更容易阅读。这是真的吗?

非引号参数会评估转义序列和变量引用,但请注意分号(;),因为在CMake中,分号视为列表定界符。如果参数包含分号,CMake会将它分割成多个参数。如果需要使用它们,请用反斜杠转义每个分号,\verb|\|;。

可能会发现这些参数难以处理,所以这里使用插图来帮助理解这些参数的分割情况:

\myGraphic{0.5}{content/chapter2/images/2.png}{图2.2:转义序列导致单独的标记解释为单个参数}

注意非引号参数。一些CMake命令需要特定数量的参数,并忽略超出部分。如果参数意外地分离了,将得到难以调试的错误。

非引号参数不能包含未转义的引号(“)、井号(\#)和反斜杠(\verb|\|)。 如果觉得这还记不住,那么括号(())只有在形成正确匹配成对时才允许。也就是说,将从一个开括号开始,并在关闭命令参数列表之前关闭它。

以下是一些示例,演示我们讨论过的规则:

\filename{ch02/01-arguments/unquoted.cmake}

\begin{cmake}
message(a\ single\ argument)
message(two arguments)
message(three;separated;arguments)
message(${CMAKE_VERSION})  # a variable reference
message(()()())            # matching parentheses
\end{cmake}

输出将是什么?一起来看看:

\begin{shell}
$ cmake -P ch02/01-arguments/unquoted.cmake
a single argument
twoarguments
threeseparatedarguments
3.16.3
()()()
\end{shell}

即使是像 message() 这样的简单命令,对于分离的非引号参数也非常挑剔。在明确转义的情况下,单个参数中的空格会正确打印。然而,twoarguments 和 threeseparatearguments 粘在一起,因为 message() 不会添加空格。

考虑到所有这些复杂性,什么时候使用非引号参数?一些CMake命令允许可选参数,这些参数前面有一个关键字,表示将提供可选参数。在这种情况下,使用非引号参数作为关键字可以使代码更加易读。例如:

\begin{cmake}
project(myProject VERSION 1.2.3)
\end{cmake}

此命令中,VERSION 关键字及其后面的参数 1.2.3 是可选的。为了提高可读性,两者都没有使用引号。注意,关键字区分大小写。

现在了解了如何处理 CMake 参数的复杂性和特性,我们准备好探讨下一个有趣的主题——在 CMake 中使用各种变量。

















































































