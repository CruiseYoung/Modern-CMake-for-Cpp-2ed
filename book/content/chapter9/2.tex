
当项目依赖于一个很流行的库时,操作系统很可能已经安装了这个库的包。我们只需要将其连接到项目的构建过程中。那么,该如何操作呢?需要找出系统中包的位置,以便CMake可以使用其文件。可以手动完成这个操作,但每个环境都有所不同。在一个系统上有效路径可能在另一个系统上无效。因此,在构建时,应该自动查找这些路径。有不同的方法可以实现这一点,但最好的方法是使用CMake内置的find\_package()命令,它了解如何找到许多常用的包。

如果我们的包不受支持,那么有两个选择:

\begin{itemize}
\item
可以编写一个名为find-module的插件来帮助find\_package()

\item
可以使用一种较老的方法,称为pkg-config
\end{itemize}

先从推荐选项开始。

\mySubsubsection{9.2.1.}{使用CMake的find\_package()查找包}

先从以下场景开始:想要改进网络通信或数据存储的方式。简单的纯文本文件或像JSON和XML这样的开放文本格式在大小方面太冗长。使用二进制格式会有所帮助,而像Google的Protobuf这样的知名库就是答案。

已经阅读了说明,并在系统上安装了所需的软件。现在怎么办?如何让CMake的find\_package()找到并使用这个库呢?

要执行此示例,必须安装想要使用的依赖项,find\_package()命令只查找已经存在于系统上的包。假定已经安装了所有内容,或者用户知道如何安装所需的软件。如果想处理其他情况,需要备用计划。

使用Protobuf时,情况相当简单:可以自己从官方仓库(\url{https://github.com/protocolbuffers/protobuf})下载、编译并安装库,或者使用操作系统的包管理器。如果按照第1章中提到的Docker镜像进行这些示例,所有依赖项就是已经安装好了的,不需要做任何事情。但如果想自己尝试安装,以下是在Debian Linux上安装Protobuf库和编译器的命令:

\begin{shell}
$ apt update
$ apt install protobuf-compiler libprotobuf-dev
\end{shell}

如今,许多项目选择支持CMake。它们会在安装过程中创建配置文件,并将其放入适当的系统目录,CMake基于创建的配置文件使用这些库。

如果想使用一个没有配置文件的库,请不要担心。CMake提供了一种机制以查找此类库,称为find模块。与基于配置文件查找库不同,find模块是CMake内置的功能,其中包含许多流行的库。当需要使用PackageName库时,其对应的find模块是一个可以被find\_package()命令加载的Find<PackageName>.cmake文件。

如果需要的库既没有配置文件也没有find模块,也还有其他选择:

\begin{itemize}
\item
为特定包编写自己的find模块,并将它们包含在项目中

\item
使用FindPkgConfig模块来利用传统的Unix包定义文件

\item
编写配置文件,并要求包维护者将其包含在内
\end{itemize}

可能认为没有准备好自己创建这样的合并请求,很可能不需要这样做。CMake自带了150多个find模块,可以找到如Boost、bzip2、curl、curses、GIF、GTK、iconv、ImageMagick、JPEG、Lua、OpenGL、OpenSSL、PNG、PostgreSQL、Qt、SDL、Threads、XML-RPC、X11和zlib等库,以及在此示例中将使用的Protobuf文件。完整的列表可以在CMake文档中找到(请参阅“扩展阅读”)。

find模块和配置文件都可以使用CMake的find\_package()命令查找。当使用该命令查找一个库时,首先检查CMake是否包含对应的find模块。如果CMake没有找到所需的find模块,会查找是否有满足要求的配置文件,即扫描常用安装包的路径(取决于操作系统),查找类似以下的文件:

\begin{itemize}
\item
<CamelCasePackageName>Config.cmake

\item
<kebab-case-package-name>-config.cmake
\end{itemize}

如果想将外部find模块添加到项目中,请设置CMAKE\_MODULE\_PATH变量。CMake将首先扫描此目录。

回到示例,目标很简单:我想展示可以有效地构建一个使用Protobuf的项目。别担心,不需要了解Protobuf来理解发生的事情。简单来说,Protobuf是一个将数据以特定二进制格式保存的库。这使得写入和从文件或网络读取C++对象变得容易。为了设置这个,我们使用一个.proto文件来给Protobuf数据结构:

\filename{ch09/01-find-package-variables/message.proto}

\begin{shell}
syntax = "proto3";
message Message {
    int32 id = 1;
}
\end{shell}

这段代码是一个简单的架构定义,包括一个32位整数。Protobuf包附带了一个二进制文件,可以将这些.proto文件编译成C++源文件和头文件,以便应用程序可以使用。需要将这个编译步骤添加到构建过程中,但稍后会回到这个话题。现在,来看看main.cpp文件是如何使用Protobuf生成的输出:

\filename{ch09/01-find-package-variables/main.cpp}

\begin{cpp}
#include "message.pb.h"
#include <fstream>
using namespace std;
int main()
{
    Message m;
    m.set_id(123);
    m.PrintDebugString();
    fstream fo("./hello.data", ios::binary | ios::out);
    m.SerializeToOstream(&fo);
    fo.close();
    return 0;
}
\end{cpp}

已经包含了预期的Protobuf生成的message.pb.h头文件。这个头文件将包含在message.proto中配置的Message对象的定义。在main()函数中,创建了一个简单的Message对象。将它的id字段设置为123作为一个随机示例,然后将其调试信息打印到标准输出。接下来,这个对象的二进制版本可写入文件流。这是像Protobuf这样的序列化库最基本的使用场景。

编译main.cpp之前,必须生成message.pb.h头文件。这是由protoc,即Protobuf编译器完成,将message.proto作为输入。管理这个过程听起来很复杂,但实际上并不复杂!

这就是CMake魔法发生的地方:

\filename{ch09/01-find-package-variables/CMakeLists.txt}

\begin{cmake}
cmake_minimum_required(VERSION 3.26.0)
project(FindPackageProtobufVariables CXX)
find_package(Protobuf REQUIRED)
protobuf_generate_cpp(GENERATED_SRC GENERATED_HEADER
                      message.proto)
add_executable(main main.cpp ${GENERATED_SRC} ${GENERATED_HEADER})
target_link_libraries(main PRIVATE ${Protobuf_LIBRARIES})
target_include_directories(main PRIVATE
    ${Protobuf_INCLUDE_DIRS} ${CMAKE_CURRENT_BINARY_DIR}
)
\end{cmake}

我们分解一下:

\begin{itemize}
\item
前两行很直接:设置项目并指定将使用C++语言。

\item
find\_package(Protobuf REQUIRED)告诉CMake找到Protobuf库(通过执行捆绑的FindProtobuf.cmake find模块)并为项目准备使用。因为我们使用了REQUIRED关键字,如果找不到库,构建将停止。

\item
protobuf\_generate\_cpp是Protobuf find模块中定义的自定义函数,自动化调用protoc编译器的过程。成功编译后,将生成的源文件路径存储在提供的前两个参数GENERATED\_SRC和GENERATED\_HEADER中。后续参数将视为要编译的文件列表(message.proto)。

\item
add\_executable使用main.cpp和Protobuf生成的文件创建可执行文件。

\item
target\_link\_libraries告诉CMake将Protobuf库链接到可执行文件。

\item
target\_include\_directories(将必要的INCLUDE\_DIRS提供的路径和CMAKE\_CURRENT\_BINARY\_DIR添加到包含路径中。后者告诉编译器在哪里找到message.pb.h头文件。
\end{itemize}

Protobuf find模块提供了以下功能:

\begin{itemize}
\item
找到Protobuf库及其编译器。

\item
提供了帮助函数来编译.proto文件。

\item
设置了包含和链接的路径变量。
\end{itemize}

虽然不是每个模块都像Protobuf那样提供方便的帮助函数,但大多数模块确实设置了一些关键的变量,这些变量对于管理项目中的依赖项很有用。无论使用内置的find模块还是配置文件,在成功找到包之后,可以期望设置以下全部或部分变量:

\begin{itemize}
\item
<PKG\_NAME>\_FOUND: 这表明是否成功找到了包。

\item
<PKG\_NAME>\_INCLUDE\_DIRS 或 <PKG\_NAME>\_INCLUDES: 这指向包的头文件所在的目录。

\item
<PKG\_NAME>\_LIBRARIES 或 <PKG\_NAME>\_LIBS: 这些是要链接的库的列表。

\item
<PKG\_NAME>\_DEFINITIONS: 包含包所需的编译器定义。
\end{itemize}

运行find\_package()后,可以立即检查<PKG\_NAME>\_FOUND变量,以查看CMake是否成功找到包。

如果包模块是为CMake 3.10或更高版本编写的,还可能会提供目标定义。这些目标将指定为IMPORTED目标,以区分其源自外部的依赖。

Protobuf是学习CMake中依赖项时的一个很好的探索示例,它定义了特定于模块的变量和IMPORTED目标。这样的目标允许我们编写更简洁的代码:

\filename{ch09/02-find-package-targets/CMakeLists.txt}

\begin{cmake}
cmake_minimum_required(VERSION 3.26.0)
project(FindPackageProtobufTargets CXX)
find_package(Protobuf REQUIRED)
protobuf_generate_cpp(GENERATED_SRC GENERATED_HEADER
    message.proto)
add_executable(main main.cpp ${GENERATED_SRC} ${GENERATED_HEADER})
target_link_libraries(main PRIVATE protobuf::libprotobuf)
target_include_directories(main PRIVATE
                                ${CMAKE_CURRENT_BINARY_DIR})
\end{cmake}

看看这段高亮代码与之前示例版本的对比:不是使用列出文件和目录的变量,而是使用IMPORTED目标是一个好主意。这种方法简化了列表文件。自动处理瞬态使用要求或传播属性,如下面使用protobuf::libprotobuf目标所示。

\begin{myNotic}{Note}
如果想确切知道一个特定的find模块提供了什么,最好的资源是在线文档。例如,可以在CMake官方网站的这个链接上找到Protobuf的详细信息:\url{https://cmake.org/cmake/help/latest/module/FindProtobuf.html}。
\end{myNotic}

为了保持简单,本节中的示例如果用户系统中没有找到Protobuf库将会直接失败。但真正健壮的解决方案,应该验证Protobuf\_FOUND变量,并为用户呈现一个清晰的诊断信息(以便可以安装),或者自动执行安装。我们将在本章后面学习如何做到这一点。

find\_package()命令有几个可选参数可以使用。虽然有一个更长的参数列表,但这里将重点关注关键的几个。该命令的基本格式如下:

\begin{shell}
find_package(<Name> [version] [EXACT] [QUIET] [REQUIRED])
\end{shell}

分解一下这些可选参数的含义:

\begin{itemize}
\item
version: 这指定了需要的包的最小版本,格式为major.minor.patch.tweak(例如1.22)。还可以指定一个范围,如1.22…1.40.1,使用三个点作为分隔符。

\item
EXACT: 与非范围[version]一起使用,告诉CMake需要确切的版本,而不是更新的版本。

\item
QUIET: 这将抑制关于包是否被找到的所有消息。

\item
REQUIRED: 如果找不到包,这将停止构建,并且即使使用了QUIET,也会显示诊断信息。
\end{itemize}

如果确信一个包存在于系统中,但find\_package()没有找到它,有一种方法可以进一步挖掘。从CMake 3.24开始,可以在调试模式下运行配置阶段以获取更多信息。使用以下命令:

\begin{shell}
cmake -B <build tree> -S <source tree> --debug-find-pkg=<pkg>
\end{shell}

使用这个命令时要小心。确保正确地输入包名,它对大小写敏感。

关于find\_package()命令的更多信息可以在文档页面这里找到:\url{https://cmake.org/cmake/help/latest/command/find_package.html}。

Find模块旨在作为一种非常方便的方式,为CMake提供有关已安装依赖项的信息。大多数流行的库在所有主要平台上都得到了CMake的广泛支持。但当我们想要使用一个尚未有专用find模块的第三方库时,应该怎么办呢?

\mySamllsection{编写自己的find模块}

极少数情况下,项目中真正想使用的库没有提供配置文件,CMake中也没有现成的find模块。这时,可以为该库编写一个自定义的find模块,并将其与项目一起分发。这种情况并不理想,但在照顾到项目用户的情况下,必须这样做。

我们可以尝试为libpqxx库编写一个自定义的find模块,这是一个PostgreSQL数据库的客户端。libpqxx在本书的Docker镜像中预安装,所以如果使用那个镜像,就不用担心了。Debian用户可以使用libpqxx-dev包来安装(其他操作系统可能需要不同的命令):

\begin{shell}
apt-get install libpqxx-dev
\end{shell}

我们将从在项目的源树中的cmake/module目录中创建一个名为FindPQXX.cmake的新文件开始。为了确保调用find\_package()时,CMake可以发现这个find模块,我们将其路径使用list(APPEND),添加到CMakeLists.txt中的CMAKE\_MODULE\_PATH变量中。注意,在搜索其他位置之前,CMake会首先检查CMAKE\_MODULE\_PATH中列出的目录以查找find模块。完整列表文件应该看起来像这样:

\filename{ch09/03-find-package-custom/CMakeLists.txt}

\begin{cmake}
cmake_minimum_required(VERSION 3.26.0)
project(FindPackageCustom CXX)
list(APPEND CMAKE_MODULE_PATH
            "${CMAKE_SOURCE_DIR}/cmake/module/")
find_package(PQXX REQUIRED)
add_executable(main main.cpp)
target_link_libraries(main PRIVATE PQXX::PQXX)
\end{cmake}

有了这些设置,可以继续编写实际的find模块。如果FindPQXX.cmake文件为空,即使使用REQUIRED调用find\_package(),CMake也不会引发错误。find模块的作者有责任设置正确的变量,并遵循最佳实践(例如:引发错误)。根据CMake的指南,这里有一些关键点需要注意:

\begin{itemize}
\item
当调用find\_package(<PKG\_NAME> REQUIRED)时,CMake会将<PKG\_NAME>\_FIND\_REQUIRED变量设置为1。find模块应该在找不到库时使用message(FATAL\_ERROR)。

\item
当使用find\_package(<PKG\_NAME> QUIET)时,CMake会将<PKG\_NAME>\_FIND\_QUIETLY设置为1。find模块应避免显示任何其他消息。

\item
CMake会将<PKG\_NAME>\_FIND\_VERSION变量设置为在列表文件中指定的版本。如果find模块无法定位正确的版本,应该引发一个FATAL\_ERROR。
\end{itemize}

当然,最好遵循上述规则,以便与其他find模块保持一致。

为了为PQXX创建一个优雅的find模块,可以遵循以下步骤:

\begin{enumerate}
\item
如果库和头文件的路径已经知道(由用户提供或从之前的运行的缓存中检索),使用这些路径来创建一个IMPORTED目标。如果这样做,可以停止。

\item
如果路径未知,首先找到底层依赖项(这种情况下是PostgreSQL)的库和头文件。

\item
接下来,搜索已知路径以定位PostgreSQL客户端库的二进制版本。

\item
同样,扫描已知路径以找到PostgreSQL客户端的头文件。

\item
最后,确认是否同时找到了库和头文件。如果是,创建一个IMPORTED目标。
\end{enumerate}

为了为PQXX创建一个健壮的find模块,我们将关注几个重要任务。首先,创建IMPORTED目标可以在两种情况下发生——用户指定库的路径或路径自动检测。为了保持代码整洁并避免重复,将编写一个函数来管理搜索过程的结果。

\mySamllsubsection{定义IMPORTED目标}

要设置一个IMPORTED目标,实际上只需要用IMPORTED关键字定义一个库。这允许在调用CMakeLists.txt列表文件时使用target\_link\_libraries()命令。我们需要指定库的类型,为了简化,将其标记为UNKNOWN。所以,我们不关心库是静态还是动态的,只是想向链接器传递一个参数。

接下来,需要为目标设置一些基本属性——即IMPORTED\_LOCATION和INTERFACE\_INCLUDE\_DIRECTORIES。使用函数提供的参数来设置这些属性,可以指定其他属性,如COMPILE\_DEFINITIONS,但对于PQXX来说没有必要。

然后,为了使find模块更高效,将找到的路径存储在缓存变量中。

这样,就不必在未来的运行中重复搜索了。值得注意的是,需要明确地将PQXX\_FOUND设置为缓存,使其全局可用,并允许用户的CMakeLists.txt引用它。

最后,将这些缓存变量标记为高级,除非激活高级选项,否则在CMake GUI中会隐藏它们。这是我们也将采用的常见最佳实践。

以下是这些操作的代码:

\filename{ch09/03-find-package-custom/cmake/module/FindPQXX.cmake}

\begin{cmake}
# Defining IMPORTED targets
function(define_imported_target library headers)
    add_library(PQXX::PQXX UNKNOWN IMPORTED)
    set_target_properties(PQXX::PQXX PROPERTIES
        IMPORTED_LOCATION ${library}
        INTERFACE_INCLUDE_DIRECTORIES ${headers}
    )
    set(PQXX_FOUND 1 CACHE INTERNAL "PQXX found" FORCE)
    set(PQXX_LIBRARIES ${library}
        CACHE STRING "Path to pqxx library" FORCE)
    set(PQXX_INCLUDES ${headers}
        CACHE STRING "Path to pqxx headers" FORCE)
    mark_as_advanced(FORCE PQXX_LIBRARIES)
    mark_as_advanced(FORCE PQXX_INCLUDES)
endfunction()
\end{cmake}

现在,我们将讨论如何使用自定义或之前存储的路径来进行快速设置。

\mySamllsubsection{接受用户提供的路径并重用缓存值}

考虑一种情况,即用户在非标准位置安装了PQXX,并通过命令行参数使用-D提供了所需的路径。如果是这样,我们立即调用我们之前定义的函数并使用return()停止搜索。假设用户已经提供了库,及其依赖项(如PostgreSQL)的准确路径:

\filename{ch09/03-find-package-custom/cmake/module/FindPQXX.cmake (continued)}

\begin{cmake}
...

# Accepting user-provided paths and reusing cached values
if (PQXX_LIBRARIES AND PQXX_INCLUDES)
    define_imported_target(${PQXX_LIBRARIES} ${PQXX_INCLUDES})
    return()
endif()
\end{cmake}

如果配置之前已经进行过,因为变量PQXX\_LIBRARIES和PQXX\_INCLUDES存储在缓存中,这个条件将成立。

现在,是处理PQXX依赖的其他库的时候了。

\mySamllsubsection{搜索嵌套依赖项}

为了使用PQXX,主机系统也必须安装了PostgreSQL。虽然在当前find模块中使用另一个find模块是可行的,但应该传递REQUIRED和QUIET标志,以确保嵌套搜索和主搜索之间的行为一致。为此,将设置两个辅助变量来存储需要传递的关键词,并根据CMake接收到的参数来对其进行填充:PQXX\_FIND\_QUIETLY和PQXX\_FIND\_REQUIRED。

\begin{cmake}
# Searching for nested dependencies
set(QUIET_ARG)
if(PQXX_FIND_QUIETLY)
    set(QUIET_ARG QUIET)
endif()

set(REQUIRED_ARG)
if(PQXX_FIND_REQUIRED)
    set(REQUIRED_ARG REQUIRED)
endif()
find_package(PostgreSQL ${QUIET_ARG} ${REQUIRED_ARG})
\end{cmake}

完成这一步后,我们将深入探讨如何准确地定位PQXX库在操作系统中的位置。

\mySamllsubsection{搜索库文件}

CMake提供了find\_library()命令来帮助查找库文件。这个命令将接受要查找的文件名和可能的路径列表,格式化为CMake的路径样式:

\begin{shell}
find_library(<VAR_NAME> NAMES <NAMES> PATHS <PATHS> <...>)
\end{shell}

<VAR\_NAME>将作为存储命令输出的变量的名称。如果找到匹配的文件,其路径将存储在<VAR\_NAME>变量中。否则,<VAR\_NAME>-NOTFOUND变量将设置为1。我们将使用PQXX\_LIBRARY\_PATH作为我们的VAR\_NAME,所以将得到PQXX\_LIBRARY\_PATH中的路径,或者PQXX\_LIBRARY\_PATH-NOTFOUND中的1。

PQXX库通常将其位置导出到\$ENV\{PQXX\_DIR\}环境变量,所以系统可能已经知道其位置。可以包含file(TO\_CMAKE\_PATH)这个格式化的路径:

\filename{ch09/03-find-package-custom/cmake/module/FindPQXX.cmake (continued)}

\begin{cmake}
...

# Searching for library files
file(TO_CMAKE_PATH "$ENV{PQXX_DIR}" _PQXX_DIR)
find_library(PQXX_LIBRARY_PATH NAMES libpqxx pqxx
    PATHS
        ${_PQXX_DIR}/lib/${CMAKE_LIBRARY_ARCHITECTURE}
        # (...) many other paths - removed for brevity
        /usr/lib
    NO_DEFAULT_PATH
)
\end{cmake}

NO\_DEFAULT\_PATH关键字指示CMake跳过其标准搜索路径列表。虽然不想这样做(默认路径通常正确),但使用NO\_DEFAULT\_PATH允许在必要时明确指定相应的搜索位置。

接下来,我们将查找库所需的头文件,库的用户可以包含这些头文件。

\mySamllsubsection{搜索头文件}

为了搜索所有已知的头文件,将使用find\_path()命令,其与find\_library()命令非常相似。主要区别在于find\_library()会自动添加系统特定的库扩展名,而使用find\_path()时,需要指定确切的名称。

此外,不要在这里混淆pqxx/pqxx。这是一个实际的头文件,但库创建者会故意省略其扩展名,以与C++ \#include指令保持一致。这使得它可以用尖括号使用,如下所示:\#include <pqxx/pqxx>。

以下是片段:

\filename{ch09/03-find-package-custom/cmake/module/FindPQXX.cmake (续)}

\begin{cmake}
...
# Searching for header files
find_path(PQXX_HEADER_PATH NAMES pqxx/pqxx
    PATHS
        ${_PQXX_DIR}/include
        # (...) many other paths - removed for brevity
        /usr/include
    NO_DEFAULT_PATH
)
\end{cmake}

接下来,我们将探讨如何完成搜索过程,处理缺失的路径,并调用定义导入目标的函数。

\mySamllsubsection{返回最终结果}

现在,检查是否设置了PQXX\_LIBRARY\_PATH-NOTFOUND或PQXX\_HEADER\_PATHNOTFOUND变量。可以手动打印诊断消息并停止构建,或者使用CMake的find\_package\_handle\_standard\_args()辅助函数。如果这个函数的路径变量正确填充,则将<PKG\_NAME>\_FOUND变量设置为1。它还提供适当的诊断消息(会尊重QUIET关键字),并在find\_package()调用中提供REQUIRED关键字时,如果路径变量未找到,则会以FATAL\_ERROR停止执行。

如果找到了库,将调用之前编写的函数来定义IMPORTED目标,并将路径存储在缓存中:

\filename{ch09/03-find-package-custom/cmake/module/FindPQXX.cmake (continued)}

\begin{cmake}
...

# Returning the final results
include(FindPackageHandleStandardArgs)
find_package_handle_standard_args(
    PQXX DEFAULT_MSG PQXX_LIBRARY_PATH PQXX_HEADER_PATH
)
if (PQXX_FOUND)
    define_imported_target(
        "${PQXX_LIBRARY_PATH};${POSTGRES_LIBRARIES}"
        "${PQXX_HEADER_PATH};${POSTGRES_INCLUDE_DIRECTORIES}"
    )
elseif(PQXX_FIND_REQUIRED)
    message(FATAL_ERROR "Required PQXX library not found")
endif()
\end{cmake}

就是这样!这个find模块将找到PQXX并创建适当的PQXX::PQXX目标。完整的文件可以在本书的示例存储库中找到。

对于那些得到良好支持,且很可能已经安装的库,这种方法非常有效。但如果处理的是较旧的、不太受支持的包怎么办?类Unix系统有一个名为pkg-config的工具,CMake也有一个有用的包装模块来支持它。

\mySubsubsection{9.2.2.}{使用FindPkgConfig发现遗留包}

管理依赖项并找出必要的编译标志,这和C++库本身一样古老。为了应对这个问题,已经开发了各种工具,从简单的机制到集成了构建系统和IDE的综合解决方案。PkgConfig(\url{freedesktop.org/wiki/Software/pkg-config})就是这样一个工具,它曾经非常流行,在类Unix系统中非常常见,也适用于macOS和Windows。

然而,更现代的解决方案正逐渐取代PkgConfig。那么,仍然需要考虑支持它吗?可能性不大,原因如下:

\begin{itemize}
\item
如果库没有提供.pc PkgConfig文件,为一种过时的工具编写定义文件就没有多大价值

\item
可以选择一个支持CMake的库的新版本(将在本章稍后讨论如何从互联网上下载依赖项)

\item
该包广泛使用,CMake的最新版本可能已经包含了它的find模块

\item
在线有社区创建的find模块可用,并且其许可证允许你使用它,也是另一个不错的选择

\item
能编写和维护自己的find模块
\end{itemize}

只有正在使用一个已经提供PkgConfig .pc文件的库版本,并且没有可用的配置模块或find模块时,才考虑使用PkgConfig。另外,创建自己的find模块不是一个可行的选项时,应该有充分的理由。如果确信不需要PkgConfig,可以跳过这个部分。

但并不是所有环境都可以快速更新到库的最新版本。许多公司仍在生产中使用遗留系统,这些系统不再接收最新包。如果在系统中有一个特定库的.pc文件,看起来就像这里显示的foobar文件一样:

\begin{shell}
prefix=/usr/local
exec_prefix=${prefix}
includedir=${prefix}/include
libdir=${exec_prefix}/lib
Name: foobar
Description: A foobar library
Version: 1.0.0
Cflags: -I${includedir}/foobar
Libs: -L${libdir} -lfoobar
\end{shell}

PkgConfig的格式很简单,许多熟悉这个工具的开发者出于习惯而使用,而不是学习更高级的系统如CMake。尽管很简单,PkgConfig可以检查特定库,及其版本是否可用,并且还可以获取库的链接标志和目录信息。

要与CMake一起使用它,需要找到你系统上的pkg-config工具,运行特定的命令,然后将结果存储起来,供编译器以后使用。每次使用PkgConfig时都进行这些步骤可能会觉得很多工作。幸运的是,CMake提供了一个FindPkgConfig find模块。如果找到了pkg-config,将设置PKG\_CONFIG\_FOUND。然后可以使用pkg\_check\_modules()来查找需要的包。

之前的章节中,已经熟悉了libpqxx,它提供了.pc文件,我们尝试使用PkgConfig来找到它。为了实际操作,我们写一个简单的main.cpp文件:

\filename{ch09/04-find-pkg-config/main.cpp}

\begin{cpp}
#include <pqxx/pqxx>
int main()
{
    // We're not actually connecting, but
    // just proving that pqxx is available.
    pqxx::nullconnection connection;
}
\end{cpp}

在列表文件中,通常从find\_package()函数开始,如果找不到库,就切换到PkgConfig。这种方法在环境更新时很有用,可以不修改代码就继续使用。为了保持这个示例简洁,我们省略了这一部分。

\filename{ch09/04-find-pkg-config/CMakeLists.txt}

\begin{cmake}
cmake_minimum_required(VERSION 3.26.0)
project(FindPkgConfig CXX)
find_package(PkgConfig REQUIRED)
pkg_check_modules(PQXX REQUIRED IMPORTED_TARGET libpqxx)
message("PQXX_FOUND: ${PQXX_FOUND}")
add_executable(main main.cpp)
target_link_libraries(main PRIVATE PkgConfig::PQXX)
\end{cmake}

分解一下发生了什么:

\begin{enumerate}
\item
使用find\_package()命令来定位PkgConfig。如果pkg-config缺失,则由于REQUIRED关键字,过程将停止。

\item
FindPkgConfig find模块中的pkg\_check\_modules(),自定义宏设置了一个名为PQXX的新IMPORTED目标。find模块寻找libpqxx依赖项,如果找不到,将再次因为REQUIRED关键字而失败。IMPORTED\_TARGET关键字至关重要;否则,需要手动定义目标。

\item
使用message()函数验证设置,显示PQXX\_FOUND。如果之前没有使用REQUIRED,这里需要检查变量是否设置,以激活其他备选方案。

\item
使用add\_executable()声明主可执行文件。

\item
最后,使用target\_link\_libraries()将PkgConfig::PQXX目标链接起来,这个目标是由pkg\_check\_modules()导入的。注意,PkgConfig::是一个固定的前缀,PQXX从传递给宏的第一个参数派生出来。
\end{enumerate}

使用这个选项比为没有CMake支持的依赖项创建find模块更快,但也有一些缺点。一它依赖于较旧的pkg-config工具,这可能在构建项目的操作系统中不可用。此外,这种方法创建了一个特殊案例,需要以与其它方法不同的方式进行维护。

我们讨论了如何与计算机上已安装的依赖项一起工作。然而,这只是故事的一部分。很多时候,项目将发送给可能没有所有所需依赖项的用户。让我们看看如何处理这种情况。























