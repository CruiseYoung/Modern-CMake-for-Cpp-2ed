
CMake 在管理依赖项方面表现出色,尤其是在系统尚未安装这些依赖项时。你可以采取几种方法。如果你使用的是 CMake 3.14 或更高版本,FetchContent 模块是管理依赖项的最佳选择。本质上,FetchContent 是另一个名为 ExternalProject 的模块的用户友好包装。它不仅简化了流程,还增加了一些额外功能。我们将在本章后面深入探讨 ExternalProject。现在,只需知道两者之间的主要区别在于执行顺序:

\begin{itemize}
\item
FetchContent在配置阶段引入依赖项。

\item
ExternalProject在构建阶段引入依赖项。
\end{itemize}

这个顺序很重要,因为在配置阶段由 FetchContent 定义的靶标将位于同一个命名空间中,因此可以轻松被我们的项目使用。我们可以将它们与其他靶标链接,就像是我们自己定义的一样。但在极少数情况下,这不是我们所希望的,这时就需要选择 ExternalProject。

让我们首先了解如何处理大多数情况。

\mySubsubsection{9.3.1.}{FetchContent}

FetchContent 模块非常有用,它提供以下功能:

\begin{itemize}
\item
管理外部项目的目录结构

\item
从 URL 下载源代码(如果需要,还可以从存档中提取)

\item
支持 Git、Subversion、Mercurial 和 CVS(并发版本系统)仓库

\item
在需要时获取更新

\item
使用 CMake、Make 或用户指定的工具配置和构建项目

\item
提供对其他目标的嵌套依赖
\end{itemize}

使用 FetchContent 模块涉及以下三个主要步骤:

\begin{enumerate}
\item
使用 include(FetchContent) 将模块添加到你的项目中。

\item
使用 FetchContent\_Declare() 命令配置依赖项。这将告诉 FetchContent 依赖项的位置以及应该使用哪个版本。

\item
使用 FetchContent\_MakeAvailable() 命令完成依赖项设置。这将下载、构建、安装,并将列表文件添加到你的主项目以供解析。
\end{enumerate}

你可能会想知道为什么步骤 2 和 3 是分开的。原因是为了允许在多层项目中进行配置覆盖。例如,考虑一个依赖于外部库 A 和 B 的项目。库 A 也依赖于 B,但其作者使用的是一个与父项目版本不同的旧版本(图 9.1):

\myGraphic{0.5}{content/chapter9/images/1.png}{图 9.1:层次化项目结构}

如果配置和下载发生在同一个命令中,即使新版本向后兼容,父项目也无法使用新版本,因为依赖项已经为旧版本配置了导入的靶标,这会引入库的目标名称和文件冲突。

为了指定所需的版本,最顶层的项目必须在库 A 完全设置之前调用 FetchContent\_Declare() 命令,并为 B 提供覆盖配置。随后在 A 中调用 FetchContent\_Declare() 将被忽略,因为 B 依赖项已经配置。

让我们看看 FetchContent\_Declare() 命令的签名:

\begin{shell}
FetchContent_Declare(<depName> <contentOptions>...)
\end{shell}

depName 是依赖项的唯一标识符,稍后将由 FetchContent\_MakeAvailable() 命令使用。

contentOptions 提供了依赖项的详细配置,这可能相当复杂。重要的是要意识到,在底层,FetchContent\_Declare() 使用较旧的 ExternalProject\_Add() 命令。事实上,提供给 FetchContent\_Declare 的大多数参数都会直接转发给内部调用。在详细解释所有参数之前,让我们先看一个从 GitHub 下载依赖项的工作示例。

\mySamllsection{基本的 YAML 读取器示例}

我编写了一个小程序,它从 YAML 文件中读取用户名并在欢迎消息中打印出来。YAML 是一种很好的、简单的格式,用于存储人类可读的配置,但对于机器来说解析起来相当复杂。我找到了一个整洁的小项目,名为 yaml-cpp,由 Jesse Beder 创建(\url{https://github.com/jbeder/yaml-cpp}),它解决了这个问题。

这个例子相当简单。它是一个打印欢迎 消息的问候程序。name 的默认值将是 Guest,但我们可以通过 YAML 配置文件指定不同的名称。以下是 C++ 代码:

\filename{ch09/05-fetch-content/main.cpp}

\begin{cpp}
#include <string>
#include <iostream>
#include "yaml-cpp/yaml.h"

using namespace std;
int main() {
    string name = "Guest";

    YAML::Node config = YAML::LoadFile("config.yaml");
    if (config["name"])
        name = config["name"].as<string>();

    cout << "Welcome " << name << endl;
    return 0;
}
\end{cpp}

这个例子的配置文件只有一行:

\filename{ch09/05-fetch-content/config.yaml}

\begin{shell}
name: Rafal
\end{shell}

我们将在其他部分重用这个示例,所以花点时间理解它是如何工作的。现在我们已经准备好了代码,让我们看看如何构建它并获取依赖:

\filename{ch09/05-fetch-content/CMakeLists.txt}

\begin{cmake}
cmake_minimum_required(VERSION 3.26.0)
project(ExternalProjectGit CXX)
add_executable(welcome main.cpp)
configure_file(config.yaml config.yaml COPYONLY)
include(FetchContent)
FetchContent_Declare(external-yaml-cpp
    GIT_REPOSITORY https://github.com/jbeder/yaml-cpp.git
    GIT_TAG 0.8.0
)
FetchContent_MakeAvailable(external-yaml-cpp)
target_link_libraries(welcome PRIVATE yaml-cpp::yaml-cpp)
\end{cmake}

我们可以显式访问由 yaml-cpp 库创建的目标。为了证明这一点,我们将使用 CMakePrintHelpers 帮助模块:

\begin{cmake}
include(CMakePrintHelpers)
cmake_print_properties(TARGETS yaml-cpp::yaml-cpp
                       PROPERTIES TYPE SOURCE_DIR)
\end{cmake}

当我们构建这样的项目时,配置阶段将打印以下输出:

\begin{shell}
Properties for TARGET yaml-cpp::yaml-cpp:
    yaml-cpp.TYPE = "STATIC_LIBRARY"
    yaml-cpp.SOURCE_DIR = "/tmp/b/_deps/external-yaml-cpp-src"
\end{shell}

这告诉我们由 external-yaml-cpp 依赖定义的目标存在;它是一个静态库,其源目录位于构建树内。这个打印输出对于实际项目来说不是必需的,但如果你不确定如何正确包含导入的目标,它可以帮助调试。

由于我们已经使用 configure\_file() 命令将 .yaml 文件复制到输出,我们可以运行程序:

\begin{shell}
~/examples/ch09/05-fetch-content$ /tmp/b/welcome
Welcome Rafal
\end{shell}

一切运行顺利!几乎不费吹灰之力,我们就引入了一个外部依赖并在我们的项目中使用了它。

如果我们需要多个依赖,我们应该编写多个对 FetchContent\_Declare() 命令的调用,每次选择一个唯一的标识符。但我们不需要多次调用 FetchContent\_MakeAvailable(),因为它支持多个标识符(这些标识符不区分大小写):

\begin{cmake}
FetchContent_MakeAvailable(lib-A lib-B lib-C)
\end{cmake}

现在,我们将学习如何编写依赖声明。

\mySamllsection{下载依赖项}

FetchContent\_Declare() 命令提供了广泛的选择,这些选择来自 ExternalProject 模块。本质上,你可以执行三个主要操作:

\begin{itemize}
\item
下载依赖项

\item
更新依赖项

\item
补丁依赖项
\end{itemize}

让我们首先看看最常见的场景:从互联网上获取文件。

CMake 支持多种下载源:

\begin{itemize}
\item
HTTP 服务器 (URL)

\item
Git

\item
Subversion

\item
Mercurial

\item
CVS
\end{itemize}

从列表顶部开始,我们将首先探索如何从 URL 下载依赖项,并自定义流程以满足我们的需求。

\mySamllsection{更新和打补丁}

我们可以提供一个 URL 列表,按顺序扫描,直到下载成功。CMake 将识别下载的文件是否是存档,并默认解压它。

基本声明:

\begin{shell}
FetchContent_Declare(dependency-id
                     URL <url1> [<url2>...]
)
\end{shell}

以下是一些额外的选项,用于进一步自定义此方法:

\begin{itemize}
\item
URL\_HASH <algo>=<hashValue>: 这会检查通过 生成的下载文件的校验和是否与提供的 <hashValue>匹配。推荐使用此选项以保证下载的完整性。以下算法受支持:: MD5, SHA1, SHA224, SHA256, SHA384, SHA512, SHA3\_224, SHA3\_256, SHA3\_384和SHA3\_512

\item
DOWNLOAD\_NO\_EXTRACT <bool>: 这会显式禁用下载后的解压。我们可以在后续步骤中通过访问 <DOWNLOADED\_FILE> 变量来使用下载文件的文件名。

\item
DOWNLOAD\_NO\_PROGRESS <bool>: 这会显式禁用下载进度的日志记录。

\item
TIMEOUT <seconds> 和 INACTIVITY\_TIMEOUT <seconds>: 这些选项设置超时,以在固定的总时间或非活动期后终止下载。

\item
HTTP\_USERNAME <username> 和 HTTP\_PASSWORD <password>: 这些选项配置 HTTP 认证。请注意不要硬编码凭据。

\item
HTTP\_HEADER <header1> [<header2>...]: 这会向你的 HTTP 请求添加额外的头部,这对于 AWS 或自定义令牌很有用。

\item
TLS\_VERIFY <bool>: 这会验证 SSL 证书。如果未设置,CMake 将从 CMAKE\_TLS\_VERIFY 变量中读取此设置,该变量默认设置为 false。跳过 TLS 验证是一种不安全的、不良的做法,尤其是在生产环境中应避免。

\item
TLS\_CAINFO <file>: 这提供了到权威文件的路径;如果未指定,CMake 将从 CMAKE\_TLS\_CAINFO 变量中读取此设置。如果你的公司颁发自签名 SSL 证书,这很有用。
\end{itemize}

大多数程序员会参考像 GitHub 这样的在线仓库来获取最新版本的库。以下是操作方法。

\mySamllsubsection{从 Git 下载}

要从 Git 下载依赖项,请确保宿主系统安装了 Git 1.6.5 或更高版本。以下选项对于从 Git 克隆项目至关重要:

\begin{shell}
FetchContent_Declare(dependency-id
                     GIT_REPOSITORY <url>
                     GIT_TAG <tag>
)
\end{shell}

<url>和<tag>应与 git 命令兼容。在生产环境中,建议使用特定的 git 哈希(而不是标签)以确保生成二进制文件的可追溯性,并避免不必要的 git fetch 操作。如果你更喜欢使用分支,请坚持使用 origin/main 这样的远程名称。这确保了本地克隆的正确同步。

其他选项包括:

\begin{itemize}
\item
GIT\_REMOTE\_NAME <name>: 这设置了远程名称(默认为 origin)。

\item
GIT\_SUBMODULES <module>...: 这指定要更新的子模块;从 3.16 版本开始,此值默认为 none(之前,所有子模块都会更新)。

\item
GIT\_SUBMODULES\_RECURSE 1: 这启用子模块的递归更新。

\item
GIT\_SHALLOW 1: 这执行浅克隆,由于跳过下载历史提交,因此速度更快。

\item
TLS\_VERIFY <bool>: 这会验证 SSL 证书。如果未设置,CMake 将从 CMAKE\_TLS\_VERIFY 变量中读取此设置,该变量默认设置为 false;跳过 TLS 验证是一种不安全的、不良的做法,尤其是在生产环境中应避免。
\end{itemize}

如果你的依赖项存储在 Subversion 中,你也可以使用 CMake 获取它。

\mySamllsubsection{从 Subversion 下载}

\begin{shell}
FetchContent_Declare(dependency-id
                     SVN_REPOSITORY <url>
                     SVN_REVISION -r<rev>
)
\end{shell}

此外,我们可以提供以下内容:

\begin{itemize}
\item
SVN\_USERNAME <user> 和 SVN\_PASSWORD <password>: 这些提供检出和更新的凭据。避免在项目中硬编码这些。

\item
SVN\_TRUST\_CERT <bool>: 这会跳过对 Subversion 服务器站点证书的验证。只有当服务器的网络路径及其完整性是可信的时候才使用此选项。
\end{itemize}

Subversion 与 CMake 配合使用非常简单。Mercurial 也是如此。

\mySamllsubsection{从 Mercurial 下载}

这种模式非常直接。我们需要提供两个参数就完成了:

\begin{shell}
FetchContent_Declare(dependency-id
                     HG_REPOSITORY <url>
                     HG_TAG <tag>
)
\end{shell}

最后,我们可以使用 CVS 来提供依赖项。

\mySamllsubsection{从 CVS 下载}

要从 CVS 检出模块,我们需要提供以下三个参数:

\begin{shell}
FetchContent_Declare(dependency-id
                     CVS_REPOSITORY <cvsroot>
                     CVS_MODULE <module>
                     CVS_TAG <tag>
)
\end{shell}

这样,我们就涵盖了 FetchContent\_Declare() 的所有下载选项。CMake 支持在成功下载后执行的额外步骤。

\mySamllsection{尽可能使用已安装的依赖项}

默认情况下,更新步骤将重新下载外部项目的文件,如果下载方法支持更新,例如,如果我们配置了指向主或 master 分支的 Git 依赖项。我们可以通过以下两种方式覆盖此行为:

\begin{itemize}
\item
提供在更新期间执行的定制命令,使用 UPDATE\_COMMAND <cmd>。

\item
完全禁用更新步骤(以允许在没有网络连接的情况下构建) – UPDATE\_DISCONNECTED <bool>。请注意,依赖项仍然会在第一次构建时下载。
\end{itemize}

另一方面,补丁是一个可选步骤,将在更新获取后执行。要启用它,我们需要指定要执行的确切命令,使用 PATCH\_COMMAND <cmd>。

CMake 文档警告,某些补丁可能比其他补丁更“粘性”。例如,在 Git 中,更改的文件在更新期间不会恢复到原始状态,我们需要小心避免对文件进行两次错误补丁。理想情况下,补丁命令应该是健壮且幂等的。

你可以链接更新和补丁命令:

\begin{shell}
FetchContent_Declare(dependency-id
                     GIT_REPOSITORY <url>
                     GIT_TAG <tag>
                     UPDATE_COMMAND <cmd>
                     PATCH_COMMAND <cmd>
)
\end{shell}

下载依赖项在它们尚未在系统上时很有帮助。但是,如果它们已经在了呢?我们如何使用本地版本?

\mySamllsection{尽可能使用已安装的依赖项}

从版本 3.24 开始,CMake 引入了一个特性,允许 FetchContent 在依赖项已经本地可用时跳过下载。要启用此功能,只需在你的声明中添加 FIND\_PACKAGE\_ARGS 关键字:

\begin{shell}
FetchContent_Declare(dependency-id
                     GIT_REPOSITORY <url>
                     GIT_TAG <tag>
                     FIND_PACKAGE_ARGS <args>
)
\end{shell}

如你所料,这个关键字指示 FetchContent 模块在启动任何下载之前使用 find\_package() 函数。如果本地找到包,它将使用该包,不会发生下载或构建。请注意,这个关键字应该是命令中的最后一个,因为它将消耗所有后续参数。

这里是更新先前示例的方法:

\filename{ch09/06-fetch-content-find-package/CMakeLists.txt}

\begin{cmake}
cmake_minimum_required(VERSION 3.26)
project(ExternalProjectGit CXX)

add_executable(welcome main.cpp)
configure_file(config.yaml config.yaml COPYONLY)

include(FetchContent)
FetchContent_Declare(external-yaml-cpp
    GIT_REPOSITORY    https://github.com/jbeder/yaml-cpp.git
    GIT_TAG           0.8.0
    FIND_PACKAGE_ARGS NAMES yaml-cpp
)
FetchContent_MakeAvailable(external-yaml-cpp)
target_link_libraries(welcome PRIVATE yaml-cpp::yaml-cpp)
include(CMakePrintHelpers)
cmake_print_properties(TARGETS yaml-cpp::yaml-cpp
                       PROPERTIES TYPE SOURCE_DIR
                       INTERFACE_INCLUDE_DIRECTORIES
)
\end{cmake}

我们做了两个关键更改:

\begin{enumerate}
\item
我们添加了 FIND\_PACKAGE\_ARGS 和 NAMES 关键字,以指定我们正在寻找 yaml-cpp 包。如果没有 NAMES,CMake 会默认使用 dependency-id,在本例中为 external-yaml-cpp。

\item
我们在打印属性中添加了 INTERFACE\_INCLUDE\_DIRECTORIES。这是一次性检查,我们可以手动验证我们是否在使用已安装的包,或者是否下载了一个新的包。
\end{enumerate}

在测试之前,请确保包实际上已经安装在你的系统上。如果没有安装,你可以使用以下命令安装它:

\begin{shell}
git clone https://github.com/jbeder/yaml-cpp.git
cmake -S yaml-cpp -B build-dir
cmake --build build-dir
cmake --install build-dir
\end{shell}

有了这个设置,我们现在可以构建我们的项目。如果一切顺利,你应该会看到来自 cmake\_print\_properties() 命令的调试输出。这将表明我们正在使用本地版本,如 INTERFACE\_INCLUDE\_DIRECTORIES 属性所示。请记住,这个输出是特定于你的环境的,你的结果可能会有所不同。

\begin{shell}
--
    Properties for TARGET yaml-cpp::yaml-cpp:
        yaml-cpp::yaml-cpp.TYPE = "STATIC_LIBRARY"
        yaml-cpp::yaml-cpp.INTERFACE_INCLUDE_DIRECTORIES =
                                                    "/usr/local/include
\end{shell}

如果你不使用 CMake 3.24,或者如果你想支持使用旧版本的用户,你可能会考虑手动运行 find\_package() 命令。这样,你只会下载尚未安装的包:

\begin{cmake}
find_package(yaml-cpp QUIET)
if (NOT TARGET yaml-cpp::yaml-cpp)
    # download missing dependency
endif()
\end{cmake}

无论你选择哪种方法,首先尝试使用本地版本,只有在依赖项找不到时才下载,这是一个体贴的方法,可以提供最佳的用户体验。

在 FetchContent 引入之前,CMake 有一个更简单的模块称为 ExternalProject。尽管 FetchContent 在大多数情况下是推荐的选择,但 ExternalProject 仍然有其自身的

\mySubsubsection{9.3.2.}{ExternalProject}

如前所述,在 FetchContent 引入 CMake 之前,有一个模块执行类似的功能:ExternalProject(在 3.0.0 版本中添加)。如你所料,它用于从在线仓库获取外部项目。多年来,该模块逐渐扩展以满足不同的需求,导致命令变得相当复杂:ExternalProject\_Add()。

ExternalProject 模块在构建阶段填充依赖项。这与 FetchContent 不同,后者在配置阶段执行。由于这个区别,ExternalProject 不能像 FetchContent 那样将目标导入项目。另一方面,ExternalProject 可以直接将依赖项安装到系统中,执行它们的测试,以及其他有趣的事情,例如覆盖配置和构建所使用的命令。

在某些情况下,这可能是必要的。由于使用这个遗留模块需要大量开销,将其视为一种好奇心。我们主要在这里介绍它,以展示当前方法是如何从它演变而来的。

ExternalProject 提供了一个 ExternalProject\_Add 命令来配置依赖项。

这里是一个例子:

\begin{cmake}
include(ExternalProject)
ExternalProject_Add(external-yaml-cpp
    GIT_REPOSITORY   https://github.com/jbeder/yaml-cpp.git
    GIT_TAG          0.8.0
    INSTALL_COMMAND  ""
    TEST_COMMAND     ""
)
\end{cmake}

如前所述,它与 FetchContent\_Declare 非常相似。你会注意到示例中有两个额外的关键字:INSTALL\_COMMAND 和 TEST\_COMMAND。在这个例子中,它们用于抑制依赖项的安装和测试,因为它们通常在构建期间执行。ExternalProject 执行许多步骤,这些步骤都可以进行深入的配置,并且它们按以下顺序执行:

\begin{enumerate}
\item
mkdir: 为外部项目创建一个子目录。

\item
download: 从仓库或 URL 下载项目文件。

\item
update: 如果 fetch 方法支持,则下载更新。

\item
patch : 执行一个修改下载文件的补丁命令。

\item
configure: 执行配置阶段。

\item
build: 为 CMake 项目执行构建阶段。

\item
install: 安装 CMake 项目。

\item
test: 执行测试。
\end{enumerate}

对于每个步骤(除了 mkdir),你可以通过添加<STEP>\_COMMAND 关键字来覆盖默认行为。还有许多其他选项——请参阅在线文档以获取完整的参考。如果你出于某种原因想使用这种方法而不是推荐的 FetchContent,可以通过在 CMake 内部执行 CMake 来应用一种丑陋的技巧,以导入目标。有关更多详细信息,请查看本书存储库中的 ch09/05-external-project 代码示例。

通常,我们会依赖库在系统中的可用性。如果它不可用,我们会求助于 FetchContent,这种方法特别适合于那些小且快速编译的依赖项。

然而,对于更重要的库,如 Qt,这种方法可能会很耗时。在这种情况下,提供预编译库的包管理器更适合用户的环境。虽然像 Apt 或 Conan 这样的工具提供了解决方案,但它们要么太系统特定,要么复杂,不适合本书涵盖。好消息是,大多数用户可以安装项目可能需要的依赖项,只要提供了清晰的安装说明。




